\documentclass[oneside,  10pt]{book}
\usepackage[spanish]{babel}
\usepackage[utf8]{inputenc}
\usepackage[T1]{fontenc}
\usepackage{lipsum} % Para generar texto de ejemplo
\usepackage{geometry}
\geometry{letterpaper, margin=1in}
\usepackage{setspace}
\usepackage{titlesec}
\usepackage{fancyhdr}
\usepackage{hyperref}

\usepackage{eso-pic}
\usepackage{graphicx}






% Configuración de los capítulos y secciones
\titleformat{\chapter}[display]
{\normalfont\bfseries\LARGE}
{\filleft}
{0pt}
{\filcenter}
[\vspace{2ex}\titlerule]


% Configuración del encabezado y pie de página
\pagestyle{fancy}
\fancyhf{}
\fancyhead[L]{\leftmark}
\fancyfoot[C]{\thepage}

% Espaciado entre líneas
\setstretch{1.15}

\begin{document}
%----------------------------------------------------------------------------------------
%	TITLE PAGE
%----------------------------------------------------------------------------------------

\begingroup
\thispagestyle{empty}
\AddToShipoutPicture*{{\includegraphics[scale=1.0]{graficas/fondo.pdf}}} % Image background

\vspace*{6cm}
\par\normalfont\fontsize{50}{50}


{\LARGE }\par % Book title
\vspace*{5cm}

\endgroup



	
	\frontmatter
	\title{La Santa Biblia}

	\date{}
	\maketitle
	
	\tableofcontents
	
	\mainmatter
	
	\part{Antiguo Testamento}

	
\section*{Introducción al Antiguo Testamento}

El \textbf{Antiguo Testamento} es la primera sección de la Biblia, sagrada tanto para el cristianismo como para el judaísmo (donde se le conoce como la \textbf{Tanaj}). Es una colección de libros que narran la historia, la fe y las leyes del pueblo de Israel, desde la creación del mundo hasta los siglos previos al nacimiento de Jesús. Estos textos fueron escritos en su mayoría en hebreo, con algunas secciones en arameo, y abarcan un período de más de mil años.

El Antiguo Testamento se divide en categorías temáticas principales:
\begin{itemize}
	\item \textbf{El Pentateuco (o la Torá)}: Relata el origen del mundo, la elección de Israel como pueblo de Dios y las leyes entregadas a Moisés.
	\begin{itemize}
		\item Génesis
		\item Éxodo
		\item Levítico
		\item Números
		\item Deuteronomio
	\end{itemize}
	
	\item \textbf{Los Libros Históricos}: Narran la historia de Israel desde su entrada en la Tierra Prometida hasta el exilio en Babilonia y el retorno.
	\begin{itemize}
		\item Josué
		\item Jueces
		\item Rut
		\item 1 Samuel
		\item 2 Samuel
		\item 1 Reyes
		\item 2 Reyes
		\item 1 Crónicas
		\item 2 Crónicas
		\item Esdras
		\item Nehemías
		\item Ester
	\end{itemize}
	
	\item \textbf{Los Libros Poéticos y Sapienciales}: Incluyen reflexiones sobre la vida, el sufrimiento, la sabiduría y la relación del ser humano con Dios.
	\begin{itemize}
		\item Job
		\item Salmos
		\item Proverbios
		\item Eclesiastés
		\item Cantar de los Cantares
	\end{itemize}
	
	\item \textbf{Los Libros Proféticos}: Contienen los mensajes de los profetas que llamaron al pueblo a la fidelidad a Dios y a la justicia, advirtiendo sobre las consecuencias de sus actos.
	\begin{itemize}
		\item \textbf{Profetas Mayores}:
		\begin{itemize}
			\item Isaías
			\item Jeremías
			\item Lamentaciones
			\item Ezequiel
			\item Daniel
		\end{itemize}
		\item \textbf{Profetas Menores}:
		\begin{itemize}
			\item Oseas
			\item Joel
			\item Amós
			\item Abdías
			\item Jonás
			\item Miqueas
			\item Nahúm
			\item Habacuc
			\item Sofonías
			\item Hageo
			\item Zacarías
			\item Malaquías
		\end{itemize}
	\end{itemize}
\end{itemize}

\subsection*{Importancia del Antiguo Testamento}

Este conjunto de libros no solo forma la base del judaísmo, sino que también es fundamental para el cristianismo, ya que presenta los antecedentes históricos y teológicos para la llegada de Jesús, considerado en el Nuevo Testamento como el cumplimiento de las promesas hechas en estos textos.

A través de relatos históricos, poesía, leyes y profecías, el Antiguo Testamento explora temas universales como la creación, el pecado, la redención y el pacto entre Dios y la humanidad.

	
	


%	Pentateuco (Ley o Torá)
	\include{genesis}
	\include{exodo}
%	\include{Levitico}
%	\include{Numeros}
%	\include{Deuteronomio}
%	
%%	Libros Históricos
%	\include{josue}
%	\include{Jueces}
%	\include{Rut}
%	\include{Samuel1}
%	\include{Samuel2}
%	\include{Reyes1}
%	\include{Reyes2}
%	\include{Cronicas1}
%	\include{Cronicas2}
%	\include{Esdras}
%	\include{Nehemias}
%	\include{Ester}
%	\include{job}
%	\chapter{Salmos Libro I}
\section*{Capítulo  1}
El justo y los pecadores 
1:1 Bienaventurado el varón que no anduvo en consejo de malos, 
Ni estuvo en camino de pecadores, 
Ni en silla de escarnecedores se ha sentado; 
1:2 Sino que en la ley de Jehová está su delicia, 
Y en su ley medita de día y de noche. 
1:3 Será como árbol plantado junto a corrientes de aguas, 
Que da su fruto en su tiempo, 
Y su hoja no cae; 
Y todo lo que hace, prosperará. 
1:4 No así los malos, 
Que son como el tamo que arrebata el viento. 
1:5 Por tanto, no se levantarán los malos en el juicio, 
Ni los pecadores en la congregación de los justos. 
1:6 Porque Jehová conoce el camino de los justos; 
Mas la senda de los malos perecerá. 
\section*{Capítulo 2}
El reino del ungido de Jehová 

2:1 ¿Por qué se amotinan las gentes, 
Y los pueblos piensan cosas vanas? 
2:2 Se levantarán los reyes de la tierra, 
Y príncipes consultarán unidos 
Contra Jehová y contra su ungido, diciendo: 
2:3 Rompamos sus ligaduras, 
Y echemos de nosotros sus cuerdas. 
2:4 El que mora en los cielos se reirá; 
El Señor se burlará de ellos. 
2:5 Luego hablará a ellos en su furor, 
Y los turbará con su ira. 
2:6 Pero yo he puesto mi rey 
Sobre Sion, mi santo monte. 
2:7 Yo publicaré el decreto; 
Jehová me ha dicho: Mi hijo eres tú; 
Yo te engendré hoy. 
2:8 Pídeme, y te daré por herencia las naciones, 
Y como posesión tuya los confines de la tierra. 
2:9 Los quebrantarás con vara de hierro; 
Como vasija de alfarero los desmenuzarás. 
2:10 Ahora, pues, oh reyes, sed prudentes; 
Admitid amonestación, jueces de la tierra. 
2:11 Servid a Jehová con temor, 
Y alegraos con temblor. 
2:12 Honrad al Hijo, para que no se enoje, y perezcáis en el camino; 
Pues se inflama de pronto su ira. 
Bienaventurados todos los que en él confían. 
\section*{Capítulo 3}
Oración matutina de confianza en Dios 
Salmo de David, cuando huía de delante de Absalón su hijo. 

3:1 ¡Oh Jehová, cuánto se han multiplicado mis adversarios! 
Muchos son los que se levantan contra mí. 
3:2 Muchos son los que dicen de mí: 
No hay para él salvación en Dios. Selah 
3:3 Mas tú, Jehová, eres escudo alrededor de mí; 
Mi gloria, y el que levanta mi cabeza. 
3:4 Con mi voz clamé a Jehová, 
Y él me respondió desde su monte santo. Selah 
3:5 Yo me acosté y dormí, 
Y desperté, porque Jehová me sustentaba. 
3:6 No temeré a diez millares de gente, 
Que pusieren sitio contra mí. 
3:7 Levántate, Jehová; sálvame, Dios mío; 
Porque tú heriste a todos mis enemigos en la mejilla; 
Los dientes de los perversos quebrantaste. 
3:8 La salvación es de Jehová; 
Sobre tu pueblo sea tu bendición. Selah 
\section*{Capítulo 4}
Oración vespertina de confianza en Dios 
Al músico principal; sobre Neginot. Salmo de David. 

4:1 Respóndeme cuando clamo, oh Dios de mi justicia. 
Cuando estaba en angustia, tú me hiciste ensanchar; 
Ten misericordia de mí, y oye mi oración. 
4:2 Hijos de los hombres, ¿hasta cuándo volveréis mi honra en infamia, 
Amaréis la vanidad, y buscaréis la mentira? Selah 
4:3 Sabed, pues, que Jehová ha escogido al piadoso para sí; 
Jehová oirá cuando yo a él clamare. 
4:4 Temblad, y no pequéis; 
Meditad en vuestro corazón estando en vuestra cama, y callad. Selah 
4:5 Ofreced sacrificios de justicia, 
Y confiad en Jehová. 
4:6 Muchos son los que dicen: ¿Quién nos mostrará el bien? 
Alza sobre nosotros, oh Jehová, la luz de tu rostro. 
4:7 Tú diste alegría a mi corazón 
Mayor que la de ellos cuando abundaba su grano y su mosto. 
4:8 En paz me acostaré, y asimismo dormiré; 
Porque solo tú, Jehová, me haces vivir confiado. 
\section*{Capítulo 5}
Plegaria pidiendo protección 
Al músico principal; sobre Nehilot. Salmo de David. 

5:1 Escucha, oh Jehová, mis palabras; 
Considera mi gemir. 
5:2 Está atento a la voz de mi clamor, Rey mío y Dios mío, 
Porque a ti oraré. 
5:3 Oh Jehová, de mañana oirás mi voz; 
De mañana me presentaré delante de ti, y esperaré. 
5:4 Porque tú no eres un Dios que se complace en la maldad; 
El malo no habitará junto a ti. 
5:5 Los insensatos no estarán delante de tus ojos; 
Aborreces a todos los que hacen iniquidad. 
5:6 Destruirás a los que hablan mentira; 
Al hombre sanguinario y engañador abominará Jehová. 
5:7 Mas yo por la abundancia de tu misericordia entraré en tu casa; 
Adoraré hacia tu santo templo en tu temor. 
5:8 Guíame, Jehová, en tu justicia, a causa de mis enemigos; 
Endereza delante de mí tu camino. 
5:9 Porque en la boca de ellos no hay sinceridad; 
Sus entrañas son maldad, 
Sepulcro abierto es su garganta, 
Con su lengua hablan lisonjas. 
5:10 Castígalos, oh Dios; 
Caigan por sus mismos consejos; 
Por la multitud de sus transgresiones échalos fuera, 
Porque se rebelaron contra ti. 
5:11 Pero alégrense todos los que en ti confían; 
Den voces de júbilo para siempre, porque tú los defiendes; 
En ti se regocijen los que aman tu nombre. 
5:12 Porque tú, oh Jehová, bendecirás al justo; 
Como con un escudo lo rodearás de tu favor. 
\section*{Capítulo 6}
Oración pidiendo misericordia en tiempo de prueba 
Al músico principal; en Neginot, sobre Seminit. Salmo de David. 

6:1 Jehová, no me reprendas en tu enojo, 
Ni me castigues con tu ira. 
6:2 Ten misericordia de mí, oh Jehová, porque estoy enfermo; 
Sáname, oh Jehová, porque mis huesos se estremecen. 
6:3 Mi alma también está muy turbada; 
Y tú, Jehová, ¿hasta cuándo? 
6:4 Vuélvete, oh Jehová, libra mi alma; 
Sálvame por tu misericordia. 
6:5 Porque en la muerte no hay memoria de ti; 
En el Seol, ¿quién te alabará? 
6:6 Me he consumido a fuerza de gemir; 
Todas las noches inundo de llanto mi lecho, 
Riego mi cama con mis lágrimas. 
6:7 Mis ojos están gastados de sufrir; 
Se han envejecido a causa de todos mis angustiadores. 
6:8 Apartaos de mí, todos los hacedores de iniquidad; 
Porque Jehová ha oído la voz de mi lloro. 
6:9 Jehová ha oído mi ruego; 
Ha recibido Jehová mi oración. 
6:10 Se avergonzarán y se turbarán mucho todos mis enemigos; 
Se volverán y serán avergonzados de repente. 
\section*{Capítulo 7}
Plegaria pidiendo vindicación 
Sigaión de David, que cantó a Jehová acerca de las palabras de Cus hijo de Benjamín. 

7:1 Jehová Dios mío, en ti he confiado; 
Sálvame de todos los que me persiguen, y líbrame, 
7:2 No sea que desgarren mi alma cual león, 
Y me destrocen sin que haya quien me libre. 
7:3 Jehová Dios mío, si yo he hecho esto, 
Si hay en mis manos iniquidad; 
7:4 Si he dado mal pago al que estaba en paz conmigo 
(Antes he libertado al que sin causa era mi enemigo), 
7:5 Persiga el enemigo mi alma, y alcáncela; 
Huelle en tierra mi vida, 
Y mi honra ponga en el polvo. Selah 
7:6 Levántate, oh Jehová, en tu ira; 
Alzate en contra de la furia de mis angustiadores, 
Y despierta en favor mío el juicio que mandaste. 
7:7 Te rodeará congregación de pueblos, 
Y sobre ella vuélvete a sentar en alto. 
7:8 Jehová juzgará a los pueblos; 
Júzgame, oh Jehová, conforme a mi justicia, 
Y conforme a mi integridad. 
7:9 Fenezca ahora la maldad de los inicuos, mas establece tú al justo; 
Porque el Dios justo prueba la mente y el corazón. 
7:10 Mi escudo está en Dios, 
Que salva a los rectos de corazón. 
7:11 Dios es juez justo, 
Y Dios está airado contra el impío todos los días. 
7:12 Si no se arrepiente, él afilará su espada; 
Armado tiene ya su arco, y lo ha preparado. 
7:13 Asimismo ha preparado armas de muerte, 
Y ha labrado saetas ardientes. 
7:14 He aquí, el impío concibió maldad, 
Se preñó de iniquidad, 
Y dio a luz engaño. 
7:15 Pozo ha cavado, y lo ha ahondado; 
Y en el hoyo que hizo caerá. 
7:16 Su iniquidad volverá sobre su cabeza, 
Y su agravio caerá sobre su propia coronilla. 
7:17 Alabaré a Jehová conforme a su justicia, 
Y cantaré al nombre de Jehová el Altísimo. 
\section*{Capítulo 8}
La gloria de Dios y la honra del hombre 
Al músico principal; sobre Gitit. Salmo de David. 

8:1 ¡Oh Jehová, Señor nuestro, 
Cuán glorioso es tu nombre en toda la tierra! 
Has puesto tu gloria sobre los cielos; 
8:2 De la boca de los niños y de los que maman, fundaste la fortaleza, 
A causa de tus enemigos, 
Para hacer callar al enemigo y al vengativo. 
8:3 Cuando veo tus cielos, obra de tus dedos, 
La luna y las estrellas que tú formaste, 
8:4 Digo: ¿Qué es el hombre, para que tengas de él memoria, 
Y el hijo del hombre, para que lo visites? 
8:5 Le has hecho poco menor que los ángeles, 
Y lo coronaste de gloria y de honra. 
8:6 Le hiciste señorear sobre las obras de tus manos; 
Todo lo pusiste debajo de sus pies: 
8:7 Ovejas y bueyes, todo ello, 
Y asimismo las bestias del campo, 
8:8 Las aves de los cielos y los peces del mar; 
Todo cuanto pasa por los senderos del mar. 
8:9 ¡Oh Jehová, Señor nuestro, 
Cuán grande es tu nombre en toda la tierra! 
\section*{Capítulo 9}
Acción de gracias por la justicia de Dios 
Al músico principal; sobre Mut-labén. Salmo de David. 

9:1 Te alabaré, oh Jehová, con todo mi corazón; 
Contaré todas tus maravillas. 
9:2 Me alegraré y me regocijaré en ti; 
Cantaré a tu nombre, oh Altísimo. 
9:3 Mis enemigos volvieron atrás; 
Cayeron y perecieron delante de ti. 
9:4 Porque has mantenido mi derecho y mi causa; 
Te has sentado en el trono juzgando con justicia. 
9:5 Reprendiste a las naciones, destruiste al malo, 
Borraste el nombre de ellos eternamente y para siempre. 
9:6 Los enemigos han perecido; han quedado desolados para siempre; 
Y las ciudades que derribaste, 
Su memoria pereció con ellas. 
9:7 Pero Jehová permanecerá para siempre; 
Ha dispuesto su trono para juicio. 
9:8 El juzgará al mundo con justicia, 
Y a los pueblos con rectitud. 
9:9 Jehová será refugio del pobre, 
Refugio para el tiempo de angustia. 
9:10 En ti confiarán los que conocen tu nombre, 
Por cuanto tú, oh Jehová, no desamparaste a los que te buscaron. 
9:11 Cantad a Jehová, que habita en Sion; 
Publicad entre los pueblos sus obras. 
9:12 Porque el que demanda la sangre se acordó de ellos; 
No se olvidó del clamor de los afligidos. 
9:13 Ten misericordia de mí, Jehová; 
Mira mi aflicción que padezco a causa de los que me aborrecen, 
Tú que me levantas de las puertas de la muerte, 
9:14 Para que cuente yo todas tus alabanzas 
En las puertas de la hija de Sion, 
Y me goce en tu salvación. 
9:15 Se hundieron las naciones en el hoyo que hicieron; 
En la red que escondieron fue tomado su pie. 
9:16 Jehová se ha hecho conocer en el juicio que ejecutó; 
En la obra de sus manos fue enlazado el malo. Higaion. Selah 
9:17 Los malos serán trasladados al Seol, 
Todas las gentes que se olvidan de Dios. 
9:18 Porque no para siempre será olvidado el menesteroso, 
Ni la esperanza de los pobres perecerá perpetuamente. 
9:19 Levántate, oh Jehová; no se fortalezca el hombre; 
Sean juzgadas las naciones delante de ti. 
9:20 Pon, oh Jehová, temor en ellos; 
Conozcan las naciones que no son sino hombres. Selah 
\section*{Capítulo 10}
Plegaria pidiendo la destrucción de los malvados 

10:1 ¿Por qué estás lejos, oh Jehová, 
Y te escondes en el tiempo de la tribulación? 
10:2 Con arrogancia el malo persigue al pobre; 
Será atrapado en los artificios que ha ideado. 
10:3 Porque el malo se jacta del deseo de su alma, 
Bendice al codicioso, y desprecia a Jehová. 
10:4 El malo, por la altivez de su rostro, no busca a Dios; 
No hay Dios en ninguno de sus pensamientos. 
10:5 Sus caminos son torcidos en todo tiempo; 
Tus juicios los tiene muy lejos de su vista; 
A todos sus adversarios desprecia. 
10:6 Dice en su corazón: No seré movido jamás; 
Nunca me alcanzará el infortunio. 
10:7 Llena está su boca de maldición, y de engaños y fraude; 
Debajo de su lengua hay vejación y maldad. 
10:8 Se sienta en acecho cerca de las aldeas; 
En escondrijos mata al inocente. 
Sus ojos están acechando al desvalido; 
10:9 Acecha en oculto, como el león desde su cueva; 
Acecha para arrebatar al pobre; 
Arrebata al pobre trayéndolo a su red. 
10:10 Se encoge, se agacha, 
Y caen en sus fuertes garras muchos desdichados. 
10:11 Dice en su corazón: Dios ha olvidado; 
Ha encubierto su rostro; nunca lo verá. 
10:12 Levántate, oh Jehová Dios, alza tu mano; 
No te olvides de los pobres. 
10:13 ¿Por qué desprecia el malo a Dios? 
En su corazón ha dicho: Tú no lo inquirirás. 
10:14 Tú lo has visto; porque miras el trabajo y la vejación, para dar la recompensa con tu mano; 
A ti se acoge el desvalido; 
Tú eres el amparo del huérfano. 
10:15 Quebranta tú el brazo del inicuo, 
Y persigue la maldad del malo hasta que no halles ninguna. 
10:16 Jehová es Rey eternamente y para siempre; 
De su tierra han perecido las naciones. 
10:17 El deseo de los humildes oíste, oh Jehová; 
Tú dispones su corazón, y haces atento tu oído, 
10:18 Para juzgar al huérfano y al oprimido, 
A fin de que no vuelva más a hacer violencia el hombre de la tierra. 
\section*{Capítulo 11}
El refugio del justo 
Al músico principal. Salmo de David. 

11:1 En Jehová he confiado; 
¿Cómo decís a mi alma, 
Que escape al monte cual ave? 
11:2 Porque he aquí, los malos tienden el arco, 
Disponen sus saetas sobre la cuerda, 
Para asaetear en oculto a los rectos de corazón. 
11:3 Si fueren destruidos los fundamentos, 
¿Qué ha de hacer el justo? 
11:4 Jehová está en su santo templo; 
Jehová tiene en el cielo su trono; 
Sus ojos ven, sus párpados examinan a los hijos de los hombres. 
11:5 Jehová prueba al justo; 
Pero al malo y al que ama la violencia, su alma los aborrece. 
11:6 Sobre los malos hará llover calamidades; 
Fuego, azufre y viento abrasador será la porción del cáliz de ellos. 
11:7 Porque Jehová es justo, y ama la justicia; 
El hombre recto mirará su rostro. 
\section*{Capítulo 12}
Oración pidiendo ayuda contra los malos 
Al músico principal; sobre Seminit. Salmo de David. 

12:1 Salva, oh Jehová, porque se acabaron los piadosos; 
Porque han desaparecido los fieles de entre los hijos de los hombres. 
12:2 Habla mentira cada uno con su prójimo; 
Hablan con labios lisonjeros, y con doblez de corazón. 
12:3 Jehová destruirá todos los labios lisonjeros, 
Y la lengua que habla jactanciosamente; 
12:4 A los que han dicho: Por nuestra lengua prevaleceremos; 
Nuestros labios son nuestros; ¿quién es señor de nosotros? 
12:5 Por la opresión de los pobres, por el gemido de los menesterosos, 
Ahora me levantaré, dice Jehová; 
Pondré en salvo al que por ello suspira. 
12:6 Las palabras de Jehová son palabras limpias, 
Como plata refinada en horno de tierra, 
Purificada siete veces. 
12:7 Tú, Jehová, los guardarás; 
De esta generación los preservarás para siempre. 
12:8 Cercando andan los malos, 
Cuando la vileza es exaltada entre los hijos de los hombres. 
\section*{Capítulo 13}
Plegaria pidiendo ayuda en la aflicción 
Al músico principal. Salmo de David. 

13:1 ¿Hasta cuándo, Jehová? ¿Me olvidarás para siempre? 
¿Hasta cuándo esconderás tu rostro de mí? 
13:2 ¿Hasta cuándo pondré consejos en mi alma, 
Con tristezas en mi corazón cada día? 
¿Hasta cuándo será enaltecido mi enemigo sobre mí? 
13:3 Mira, respóndeme, oh Jehová Dios mío; 
Alumbra mis ojos, para que no duerma de muerte; 
13:4 Para que no diga mi enemigo: Lo vencí. 
Mis enemigos se alegrarían, si yo resbalara. 
13:5 Mas yo en tu misericordia he confiado; 
Mi corazón se alegrará en tu salvación. 
13:6 Cantaré a Jehová, 
Porque me ha hecho bien. 
\section*{Capítulo 14}
Necedad y corrupción del hombre 
Al músico principal. Salmo de David. 

14:1 Dice el necio en su corazón: 
No hay Dios. 
Se han corrompido, hacen obras abominables; 
No hay quien haga el bien. 
14:2 Jehová miró desde los cielos sobre los hijos de los hombres, 
Para ver si había algún entendido, 
Que buscara a Dios. 
14:3 Todos se desviaron, a una se han corrompido; 
No hay quien haga lo bueno, no hay ni siquiera uno. 
14:4 ¿No tienen discernimiento todos los que hacen iniquidad, 
Que devoran a mi pueblo como si comiesen pan, 
Y a Jehová no invocan? 
14:5 Ellos temblaron de espanto; 
Porque Dios está con la generación de los justos. 
14:6 Del consejo del pobre se han burlado, 
Pero Jehová es su esperanza. 
14:7 ¡Oh, que de Sion saliera la salvación de Israel! 
Cuando Jehová hiciere volver a los cautivos de su pueblo, 
Se gozará Jacob, y se alegrará Israel. 
\section*{Capítulo 15}
Los que habitarán en el monte santo de Dios 
Salmo de David. 

15:1 Jehová, ¿quién habitará en tu tabernáculo? 
¿Quién morará en tu monte santo? 
15:2 El que anda en integridad y hace justicia, 
Y habla verdad en su corazón. 
15:3 El que no calumnia con su lengua, 
Ni hace mal a su prójimo, 
Ni admite reproche alguno contra su vecino. 
15:4 Aquel a cuyos ojos el vil es menospreciado, 
Pero honra a los que temen a Jehová. 
El que aun jurando en daño suyo, no por eso cambia; 
15:5 Quien su dinero no dio a usura, 
Ni contra el inocente admitió cohecho. 
El que hace estas cosas, no resbalará jamás. 
\section*{Capítulo 16}
Una herencia escogida 
Mictam de David. 

16:1 Guárdame, oh Dios, porque en ti he confiado. 
16:2 Oh alma mía, dijiste a Jehová: 
Tú eres mi Señor; 
No hay para mí bien fuera de ti. 
16:3 Para los santos que están en la tierra, 
Y para los íntegros, es toda mi complacencia. 
16:4 Se multiplicarán los dolores de aquellos que sirven diligentes a otro dios. 
No ofreceré yo sus libaciones de sangre, 
Ni en mis labios tomaré sus nombres. 
16:5 Jehová es la porción de mi herencia y de mi copa; 
Tú sustentas mi suerte. 
16:6 Las cuerdas me cayeron en lugares deleitosos, 
Y es hermosa la heredad que me ha tocado. 
16:7 Bendeciré a Jehová que me aconseja; 
Aun en las noches me enseña mi conciencia. 
16:8 A Jehová he puesto siempre delante de mí; 
Porque está a mi diestra, no seré conmovido. 
16:9 Se alegró por tanto mi corazón, y se gozó mi alma; 
Mi carne también reposará confiadamente; 
16:10 Porque no dejarás mi alma en el Seol, 
Ni permitirás que tu santo vea corrupción. 
16:11 Me mostrarás la senda de la vida; 
En tu presencia hay plenitud de gozo; 
Delicias a tu diestra para siempre. 
\section*{Capítulo 17}
Plegaria pidiendo protección contra los opresores 
Oración de David. 

17:1 Oye, oh Jehová, una causa justa; está atento a mi clamor. 
Escucha mi oración hecha de labios sin engaño. 
17:2 De tu presencia proceda mi vindicación; 
Vean tus ojos la rectitud. 
17:3 Tú has probado mi corazón, me has visitado de noche; 
Me has puesto a prueba, y nada inicuo hallaste; 
He resuelto que mi boca no haga transgresión. 
17:4 En cuanto a las obras humanas, por la palabra de tus labios 
Yo me he guardado de las sendas de los violentos. 
17:5 Sustenta mis pasos en tus caminos, 
Para que mis pies no resbalen. 
17:6 Yo te he invocado, por cuanto tú me oirás, oh Dios; 
Inclina a mí tu oído, escucha mi palabra. 
17:7 Muestra tus maravillosas misericordias, tú que salvas a los que se refugian a tu diestra, 
De los que se levantan contra ellos. 
17:8 Guárdame como a la niña de tus ojos; 
Escóndeme bajo la sombra de tus alas, 
17:9 De la vista de los malos que me oprimen, 
De mis enemigos que buscan mi vida. 
17:10 Envueltos están con su grosura; 
Con su boca hablan arrogantemente. 
17:11 Han cercado ahora nuestros pasos; 
Tienen puestos sus ojos para echarnos por tierra. 
17:12 Son como león que desea hacer presa, 
Y como leoncillo que está en su escondite. 
17:13 Levántate, oh Jehová; 
Sal a su encuentro, póstrales; 
Libra mi alma de los malos con tu espada, 
17:14 De los hombres con tu mano, oh Jehová, 
De los hombres mundanos, cuya porción la tienen en esta vida, 
Y cuyo vientre está lleno de tu tesoro. 
Sacian a sus hijos, 
Y aun sobra para sus pequeñuelos. 
17:15 En cuanto a mí, veré tu rostro en justicia; 
Estaré satisfecho cuando despierte a tu semejanza. 
\section*{Capítulo 18}
Acción de gracias por la victoria 

Al músico principal. Salmo de David, siervo de Jehová, el cual dirigió a Jehová las palabras de este cántico el día que le libró Jehová de mano de todos sus enemigos, y de mano de Saúl. Entonces dijo: 

18:1 Te amo, oh Jehová, fortaleza mía. 
18:2 Jehová, roca mía y castillo mío, y mi libertador; 
Dios mío, fortaleza mía, en él confiaré; 
Mi escudo, y la fuerza de mi salvación, mi alto refugio. 
18:3 Invocaré a Jehová, quien es digno de ser alabado, 
Y seré salvo de mis enemigos. 
18:4 Me rodearon ligaduras de muerte, 
Y torrentes de perversidad me atemorizaron. 
18:5 Ligaduras del Seol me rodearon, 
Me tendieron lazos de muerte. 
18:6 En mi angustia invoqué a Jehová, 
Y clamé a mi Dios. 
El oyó mi voz desde su templo, 
Y mi clamor llegó delante de él, a sus oídos. 
18:7 La tierra fue conmovida y tembló; 
Se conmovieron los cimientos de los montes, 
Y se estremecieron, porque se indignó él. 
18:8 Humo subió de su nariz, 
Y de su boca fuego consumidor; 
Carbones fueron por él encendidos. 
18:9 Inclinó los cielos, y descendió; 
Y había densas tinieblas debajo de sus pies. 
18:10 Cabalgó sobre un querubín, y voló; 
Voló sobre las alas del viento. 
18:11 Puso tinieblas por su escondedero, por cortina suya alrededor de sí; 
Oscuridad de aguas, nubes de los cielos. 
18:12 Por el resplandor de su presencia, sus nubes pasaron; 
Granizo y carbones ardientes. 
18:13 Tronó en los cielos Jehová, 
Y el Altísimo dio su voz; 
Granizo y carbones de fuego. 
18:14 Envió sus saetas, y los dispersó; 
Lanzó relámpagos, y los destruyó. 
18:15 Entonces aparecieron los abismos de las aguas, 
Y quedaron al descubierto los cimientos del mundo, 
A tu reprensión, oh Jehová, 
Por el soplo del aliento de tu nariz. 
18:16 Envió desde lo alto; me tomó, 
Me sacó de las muchas aguas. 
18:17 Me libró de mi poderoso enemigo, 
Y de los que me aborrecían; pues eran más fuertes que yo. 
18:18 Me asaltaron en el día de mi quebranto, 
Mas Jehová fue mi apoyo. 
18:19 Me sacó a lugar espacioso; 
Me libró, porque se agradó de mí. 
18:20 Jehová me ha premiado conforme a mi justicia; 
Conforme a la limpieza de mis manos me ha recompensado. 
18:21 Porque yo he guardado los caminos de Jehová, 
Y no me aparté impíamente de mi Dios. 
18:22 Pues todos sus juicios estuvieron delante de mí, 
Y no me he apartado de sus estatutos. 
18:23 Fui recto para con él, y me he guardado de mi maldad, 
18:24 Por lo cual me ha recompensado Jehová conforme a mi justicia; 
Conforme a la limpieza de mis manos delante de su vista. 
18:25 Con el misericordioso te mostrarás misericordioso, 
Y recto para con el hombre íntegro. 
18:26 Limpio te mostrarás para con el limpio, 
Y severo serás para con el perverso. 
18:27 Porque tú salvarás al pueblo afligido, 
Y humillarás los ojos altivos. 
18:28 Tú encenderás mi lámpara; 
Jehová mi Dios alumbrará mis tinieblas. 
18:29 Contigo desbarataré ejércitos, 
Y con mi Dios asaltaré muros. 
18:30 En cuanto a Dios, perfecto es su camino, 
Y acrisolada la palabra de Jehová; 
Escudo es a todos los que en él esperan. 
18:31 Porque ¿quién es Dios sino sólo Jehová? 
¿Y qué roca hay fuera de nuestro Dios? 
18:32 Dios es el que me ciñe de poder, 
Y quien hace perfecto mi camino; 
18:33 Quien hace mis pies como de ciervas, 
Y me hace estar firme sobre mis alturas; 
18:34 Quien adiestra mis manos para la batalla, 
Para entesar con mis brazos el arco de bronce. 
18:35 Me diste asimismo el escudo de tu salvación; 
Tu diestra me sustentó, 
Y tu benignidad me ha engrandecido. 
18:36 Ensanchaste mis pasos debajo de mí, 
Y mis pies no han resbalado. 
18:37 Perseguí a mis enemigos, y los alcancé, 
Y no volví hasta acabarlos. 
18:38 Los herí de modo que no se levantasen; 
Cayeron debajo de mis pies. 
18:39 Pues me ceñiste de fuerzas para la pelea; 
Has humillado a mis enemigos debajo de mí. 
18:40 Has hecho que mis enemigos me vuelvan las espaldas, 
Para que yo destruya a los que me aborrecen. 
18:41 Clamaron, y no hubo quien salvase; 
Aun a Jehová, pero no los oyó. 
18:42 Y los molí como polvo delante del viento; 
Los eché fuera como lodo de las calles. 
18:43 Me has librado de las contiendas del pueblo; 
Me has hecho cabeza de las naciones; 
Pueblo que yo no conocía me sirvió. 
18:44 Al oír de mí me obedecieron; 
Los hijos de extraños se sometieron a mí. 
18:45 Los extraños se debilitaron 
Y salieron temblando de sus encierros. 
18:46 Viva Jehová, y bendita sea mi roca, 
Y enaltecido sea el Dios de mi salvación; 
18:47 El Dios que venga mis agravios, 
Y somete pueblos debajo de mí; 
18:48 El que me libra de mis enemigos, 
Y aun me eleva sobre los que se levantan contra mí; 
Me libraste de varón violento. 
18:49 Por tanto yo te confesaré entre las naciones, oh Jehová, 
Y cantaré a tu nombre. 
18:50 Grandes triunfos da a su rey, 
Y hace misericordia a su ungido, 
A David y a su descendencia, para siempre. 
\section*{Capítulo 19}
Las obras y la palabra de Dios 
Al músico principal. Salmo de David. 

19:1 Los cielos cuentan la gloria de Dios, 
Y el firmamento anuncia la obra de sus manos. 
19:2 Un día emite palabra a otro día, 
Y una noche a otra noche declara sabiduría. 
19:3 No hay lenguaje, ni palabras, 
Ni es oída su voz. 
19:4 Por toda la tierra salió su voz, 
Y hasta el extremo del mundo sus palabras. 
En ellos puso tabernáculo para el sol; 
19:5 Y éste, como esposo que sale de su tálamo, 
Se alegra cual gigante para correr el camino. 
19:6 De un extremo de los cielos es su salida, 
Y su curso hasta el término de ellos; 
Y nada hay que se esconda de su calor. 
19:7 La ley de Jehová es perfecta, que convierte el alma; 
El testimonio de Jehová es fiel, que hace sabio al sencillo. 
19:8 Los mandamientos de Jehová son rectos, que alegran el corazón; 
El precepto de Jehová es puro, que alumbra los ojos. 
19:9 El temor de Jehová es limpio, que permanece para siempre; 
Los juicios de Jehová son verdad, todos justos. 
19:10 Deseables son más que el oro, y más que mucho oro afinado; 
Y dulces más que miel, y que la que destila del panal. 
19:11 Tu siervo es además amonestado con ellos; 
En guardarlos hay grande galardón. 
19:12 ¿Quién podrá entender sus propios errores? 
Líbrame de los que me son ocultos. 
19:13 Preserva también a tu siervo de las soberbias; 
Que no se enseñoreen de mí; 
Entonces seré íntegro, y estaré limpio de gran rebelión. 
19:14 Sean gratos los dichos de mi boca y la meditación de mi corazón delante de ti, 
Oh Jehová, roca mía, y redentor mío. 
\section*{Capítulo 20}
Oración pidiendo la victoria 
Al músico principal. Salmo de David. 

20:1 Jehová te oiga en el día de conflicto; 
El nombre del Dios de Jacob te defienda. 
20:2 Te envíe ayuda desde el santuario, 
Y desde Sion te sostenga. 
20:3 Haga memoria de todas tus ofrendas, 
Y acepte tu holocausto. Selah 
20:4 Te dé conforme al deseo de tu corazón, 
Y cumpla todo tu consejo. 
20:5 Nosotros nos alegraremos en tu salvación, 
Y alzaremos pendón en el nombre de nuestro Dios; 
Conceda Jehová todas tus peticiones. 
20:6 Ahora conozco que Jehová salva a su ungido; 
Lo oirá desde sus santos cielos 
Con la potencia salvadora de su diestra. 
20:7 Estos confían en carros, y aquéllos en caballos; 
Mas nosotros del nombre de Jehová nuestro Dios tendremos memoria. 
20:8 Ellos flaquean y caen, 
Mas nosotros nos levantamos, y estamos en pie. 
20:9 Salva, Jehová; 
Que el Rey nos oiga en el día que lo invoquemos. 
\section*{Capítulo 21}
Alabanza por haber sido librado del enemigo 
Al músico principal. Salmo de David. 

21:1 El rey se alegra en tu poder, oh Jehová; 
Y en tu salvación, ¡cómo se goza! 
21:2 Le has concedido el deseo de su corazón, 
Y no le negaste la petición de sus labios. Selah 
21:3 Porque le has salido al encuentro con bendiciones de bien; 
Corona de oro fino has puesto sobre su cabeza. 
21:4 Vida te demandó, y se la diste; 
Largura de días eternamente y para siempre. 
21:5 Grande es su gloria en tu salvación; 
Honra y majestad has puesto sobre él. 
21:6 Porque lo has bendecido para siempre; 
Lo llenaste de alegría con tu presencia. 
21:7 Por cuanto el rey confía en Jehová, 
Y en la misericordia del Altísimo, no será conmovido. 
21:8 Alcanzará tu mano a todos tus enemigos; 
Tu diestra alcanzará a los que te aborrecen. 
21:9 Los pondrás como horno de fuego en el tiempo de tu ira; 
Jehová los deshará en su ira, 
Y fuego los consumirá. 
21:10 Su fruto destruirás de la tierra, 
Y su descendencia de entre los hijos de los hombres. 
21:11 Porque intentaron el mal contra ti; 
Fraguaron maquinaciones, mas no prevalecerán, 
21:12 Pues tú los pondrás en fuga; 
En tus cuerdas dispondrás saetas contra sus rostros. 
21:13 Engrandécete, oh Jehová, en tu poder; 
Cantaremos y alabaremos tu poderío. 
\section*{Capítulo 22}
Un grito de angustia y un canto de alabanza 
Al músico principal; sobre Ajelet-sahar. Salmo de David. 

22:1 Dios mío, Dios mío, ¿por qué me has desamparado? 
¿Por qué estás tan lejos de mi salvación, y de las palabras de mi clamor? 
22:2 Dios mío, clamo de día, y no respondes; 
Y de noche, y no hay para mí reposo. 
22:3 Pero tú eres santo, 
Tú que habitas entre las alabanzas de Israel. 
22:4 En ti esperaron nuestros padres; 
Esperaron, y tú los libraste. 
22:5 Clamaron a ti, y fueron librados; 
Confiaron en ti, y no fueron avergonzados. 
22:6 Mas yo soy gusano, y no hombre; 
Oprobio de los hombres, y despreciado del pueblo. 
22:7 Todos los que me ven me escarnecen; 
Estiran la boca, menean la cabeza,  diciendo: 
22:8 Se encomendó a Jehová; líbrele él; 
Sálvele, puesto que en él se complacía. 
22:9 Pero tú eres el que me sacó del vientre; 
El que me hizo estar confiado desde que estaba a los pechos de mi madre. 
22:10 Sobre ti fui echado desde antes de nacer; 
Desde el vientre de mi madre, tú eres mi Dios. 
22:11 No te alejes de mí, porque la angustia está cerca; 
Porque no hay quien ayude. 
22:12 Me han rodeado muchos toros; 
Fuertes toros de Basán me han cercado. 
22:13 Abrieron sobre mí su boca 
Como león rapaz y rugiente. 
22:14 He sido derramado como aguas, 
Y todos mis huesos se descoyuntaron; 
Mi corazón fue como cera, 
Derritiéndose en medio de mis entrañas. 
22:15 Como un tiesto se secó mi vigor, 
Y mi lengua se pegó a mi paladar, 
Y me has puesto en el polvo de la muerte. 
22:16 Porque perros me han rodeado; 
Me ha cercado cuadrilla de malignos; 
Horadaron mis manos y mis pies. 
22:17 Contar puedo todos mis huesos; 
Entre tanto, ellos me miran y me observan. 
22:18 Repartieron entre sí mis vestidos, 
Y sobre mi ropa echaron suertes. 
22:19 Mas tú, Jehová, no te alejes; 
Fortaleza mía, apresúrate a socorrerme. 
22:20 Libra de la espada mi alma, 
Del poder del perro mi vida. 
22:21 Sálvame de la boca del león, 
Y líbrame de los cuernos de los búfalos. 
22:22 Anunciaré tu nombre a mis hermanos; 
En medio de la congregación te alabaré. 
22:23 Los que teméis a Jehová, alabadle; 
Glorificadle, descendencia toda de Jacob, 
Y temedle vosotros, descendencia toda de Israel. 
22:24 Porque no menospreció ni abominó la aflicción del afligido, 
Ni de él escondió su rostro; 
Sino que cuando clamó a él, le oyó. 
22:25 De ti será mi alabanza en la gran congregación; 
Mis votos pagaré delante de los que le temen. 
22:26 Comerán los humildes, y serán saciados; 
Alabarán a Jehová los que le buscan; 
Vivirá vuestro corazón para siempre. 
22:27 Se acordarán, y se volverán a Jehová todos los confines de la tierra, 
Y todas las familias de las naciones adorarán delante de ti. 
22:28 Porque de Jehová es el reino, 
Y él regirá las naciones. 
22:29 Comerán y adorarán todos los poderosos de la tierra; 
Se postrarán delante de él todos los que descienden al polvo, 
Aun el que no puede conservar la vida a su propia alma. 
22:30 La posteridad le servirá; 
Esto será contado de Jehová hasta la postrera generación. 
22:31 Vendrán, y anunciarán su justicia; 
A pueblo no nacido aún, anunciarán que él hizo esto. 
\section*{Capítulo 23}
Jehová es mi pastor 
Salmo de David. 

23:1 Jehová es mi pastor; nada me faltará. 
23:2 En lugares de delicados pastos me hará descansar; 
Junto a aguas de reposo me pastoreará. 
23:3 Confortará mi alma; 
Me guiará por sendas de justicia por amor de su nombre. 
23:4 Aunque ande en valle de sombra de muerte, 
No temeré mal alguno, porque tú estarás conmigo; 
Tu vara y tu cayado me infundirán aliento. 
23:5 Aderezas mesa delante de mí en presencia de mis angustiadores; 
Unges mi cabeza con aceite; mi copa está rebosando. 
23:6 Ciertamente el bien y la misericordia me seguirán todos los días de mi vida, 
Y en la casa de Jehová moraré por largos días. 
\section*{Capítulo 24}
El rey de gloria 
Salmo de David. 

24:1 De Jehová es la tierra y su plenitud; 
El mundo, y los que en él habitan. 
24:2 Porque él la fundó sobre los mares, 
Y la afirmó sobre los ríos. 
24:3 ¿Quién subirá al monte de Jehová? 
¿Y quién estará en su lugar santo? 
24:4 El limpio de manos y puro de corazón; 
El que no ha elevado su alma a cosas vanas, 
Ni jurado con engaño. 
24:5 El recibirá bendición de Jehová, 
Y justicia del Dios de salvación. 
24:6 Tal es la generación de los que le buscan, 
De los que buscan tu rostro, oh Dios de Jacob. Selah 
24:7 Alzad, oh puertas, vuestras cabezas, 
Y alzaos vosotras, puertas eternas, 
Y entrará el Rey de gloria. 
24:8 ¿Quién es este Rey de gloria? 
Jehová el fuerte y valiente, 
Jehová el poderoso en batalla. 
24:9 Alzad, oh puertas, vuestras cabezas, 
Y alzaos vosotras, puertas eternas, 
Y entrará el Rey de gloria. 
24:10 ¿Quién es este Rey de gloria? 
Jehová de los ejércitos, 
El es el Rey de la gloria. Selah 
\section*{Capítulo 25}
David implora dirección, perdón y protección 
Salmo de David. 

25:1 A ti, oh Jehová, levantaré mi alma. 
25:2 Dios mío, en ti confío; 
No sea yo avergonzado, 
No se alegren de mí mis enemigos. 
25:3 Ciertamente ninguno de cuantos esperan en ti será confundido; 
Serán avergonzados los que se rebelan sin causa. 
25:4 Muéstrame, oh Jehová, tus caminos; 
Enséñame tus sendas. 
25:5 Encamíname en tu verdad, y enséñame, 
Porque tú eres el Dios de mi salvación; 
En ti he esperado todo el día. 
25:6 Acuérdate, oh Jehová, de tus piedades y de tus misericordias, 
Que son perpetuas. 
25:7 De los pecados de mi juventud, y de mis rebeliones, no te acuerdes; 
Conforme a tu misericordia acuérdate de mí, 
Por tu bondad, oh Jehová. 
25:8 Bueno y recto es Jehová; 
Por tanto, él enseñará a los pecadores el camino. 
25:9 Encaminará a los humildes por el juicio, 
Y enseñará a los mansos su carrera. 
25:10 Todas las sendas de Jehová son misericordia y verdad, 
Para los que guardan su pacto y sus testimonios. 
25:11 Por amor de tu nombre, oh Jehová, 
Perdonarás también mi pecado, que es grande. 
25:12 ¿Quién es el hombre que teme a Jehová? 
El le enseñará el camino que ha de escoger. 
25:13 Gozará él de bienestar, 
Y su descendencia heredará la tierra. 
25:14 La comunión íntima de Jehová es con los que le temen, 
Y a ellos hará conocer su pacto. 
25:15 Mis ojos están siempre hacia Jehová, 
Porque él sacará mis pies de la red. 
25:16 Mírame, y ten misericordia de mí, 
Porque estoy solo y afligido. 
25:17 Las angustias de mi corazón se han aumentado; 
Sácame de mis congojas. 
25:18 Mira mi aflicción y mi trabajo, 
Y perdona todos mis pecados. 
25:19 Mira mis enemigos, cómo se han multiplicado, 
Y con odio violento me aborrecen. 
25:20 Guarda mi alma, y líbrame; 
No sea yo avergonzado, porque en ti confié. 
25:21 Integridad y rectitud me guarden, 
Porque en ti he esperado. 
25:22 Redime, oh Dios, a Israel 
De todas sus angustias. 
\section*{Capítulo 26}
Declaración de integridad 
Salmo de David. 

26:1 Júzgame, oh Jehová, porque yo en mi integridad he andado; 
He confiado asimismo en Jehová sin titubear. 
26:2 Escudríñame, oh Jehová, y pruébame; 
Examina mis íntimos pensamientos y mi corazón. 
26:3 Porque tu misericordia está delante de mis ojos, 
Y ando en tu verdad. 
26:4 No me he sentado con hombres hipócritas, 
Ni entré con los que andan simuladamente. 
26:5 Aborrecí la reunión de los malignos, 
Y con los impíos nunca me senté. 
26:6 Lavaré en inocencia mis manos, 
Y así andaré alrededor de tu altar, oh Jehová, 
26:7 Para exclamar con voz de acción de gracias, 
Y para contar todas tus maravillas. 
26:8 Jehová, la habitación de tu casa he amado, 
Y el lugar de la morada de tu gloria. 
26:9 No arrebates con los pecadores mi alma, 
Ni mi vida con hombres sanguinarios, 
26:10 En cuyas manos está el mal, 
Y su diestra está llena de sobornos. 
26:11 Mas yo andaré en mi integridad; 
Redímeme, y ten misericordia de mí. 
26:12 Mi pie ha estado en rectitud; 
En las congregaciones bendeciré a Jehová. 
\section*{Capítulo 27}
Jehová es mi luz y mi salvación 
Salmo de David. 

27:1 Jehová es mi luz y mi salvación; ¿de quién temeré? 
Jehová es la fortaleza de mi vida; ¿de quién he de atemorizarme? 
27:2 Cuando se juntaron contra mí los malignos, mis angustiadores y mis enemigos, 
Para comer mis carnes, ellos tropezaron y cayeron. 
27:3 Aunque un ejército acampe contra mí, 
No temerá mi corazón; 
Aunque contra mí se levante guerra, 
Yo estaré confiado. 
27:4 Una cosa he demandado a Jehová, ésta buscaré; 
Que esté yo en la casa de Jehová todos los días de mi vida, 
Para contemplar la hermosura de Jehová, y para inquirir en su templo. 
27:5 Porque él me esconderá en su tabernáculo en el día del mal; 
Me ocultará en lo reservado de su morada; 
Sobre una roca me pondrá en alto. 
27:6 Luego levantará mi cabeza sobre mis enemigos que me rodean, 
Y yo sacrificaré en su tabernáculo sacrificios de júbilo; 
Cantaré y entonaré alabanzas a Jehová. 
27:7 Oye, oh Jehová, mi voz con que a ti clamo; 
Ten misericordia de mí, y respóndeme. 
27:8 Mi corazón ha dicho de ti: Buscad mi rostro. 
Tu rostro buscaré, oh Jehová; 
27:9 No escondas tu rostro de mí. 
No apartes con ira a tu siervo; 
Mi ayuda has sido. 
No me dejes ni me desampares, Dios de mi salvación. 
27:10 Aunque mi padre y mi madre me dejaran, 
Con todo, Jehová me recogerá. 
27:11 Enséñame, oh Jehová, tu camino, 
Y guíame por senda de rectitud 
A causa de mis enemigos. 
27:12 No me entregues a la voluntad de mis enemigos; 
Porque se han levantado contra mí testigos falsos, y los que respiran crueldad. 
27:13 Hubiera yo desmayado, si no creyese que veré la bondad de Jehová 
En la tierra de los vivientes. 
27:14 Aguarda a Jehová; 
Esfuérzate, y aliéntese tu corazón; 
Sí, espera a Jehová. 
\section*{Capítulo 28}
Plegaria pidiendo ayuda, y alabanza por la respuesta 
Salmo de David. 

28:1 A ti clamaré, oh Jehová. 
Roca mía, no te desentiendas de mí, 
Para que no sea yo, dejándome tú, 
Semejante a los que descienden al sepulcro. 
28:2 Oye la voz de mis ruegos cuando clamo a ti, 
Cuando alzo mis manos hacia tu santo templo. 
28:3 No me arrebates juntamente con los malos, 
Y con los que hacen iniquidad, 
Los cuales hablan paz con sus prójimos, 
Pero la maldad está en su corazón. 
28:4 Dales conforme a su obra, y conforme a la perversidad de sus hechos; 
Dales su merecido conforme a la obra de sus manos. 
28:5 Por cuanto no atendieron a los hechos de Jehová, 
Ni a la obra de sus manos, 
El los derribará, y no los edificará. 
28:6 Bendito sea Jehová, 
Que oyó la voz de mis ruegos. 
28:7 Jehová es mi fortaleza y mi escudo; 
En él confió mi corazón, y fui ayudado, 
Por lo que se gozó mi corazón, 
Y con mi cántico le alabaré. 
28:8 Jehová es la fortaleza de su pueblo, 
Y el refugio salvador de su ungido. 
28:9 Salva a tu pueblo, y bendice a tu heredad; 
Y pastoréales y susténtales para siempre. 
\section*{Capítulo 29}
Poder y gloria de Jehová 
Salmo de David. 

29:1 Tributad a Jehová, oh hijos de los poderosos, 
Dad a Jehová la gloria y el poder. 
29:2 Dad a Jehová la gloria debida a su nombre; 
Adorad a Jehová en la hermosura de la santidad. 
29:3 Voz de Jehová sobre las aguas; 
Truena el Dios de gloria, 
Jehová sobre las muchas aguas. 
29:4 Voz de Jehová con potencia; 
Voz de Jehová con gloria. 
29:5 Voz de Jehová que quebranta los cedros; 
Quebrantó Jehová los cedros del Líbano. 
29:6 Los hizo saltar como becerros; 
Al Líbano y al Sirión como hijos de búfalos. 
29:7 Voz de Jehová que derrama llamas de fuego; 
29:8 Voz de Jehová que hace temblar el desierto; 
Hace temblar Jehová el desierto de Cades. 
29:9 Voz de Jehová que desgaja las encinas, 
Y desnuda los bosques; 
En su templo todo proclama su gloria. 
29:10 Jehová preside en el diluvio, 
Y se sienta Jehová como rey para siempre. 
29:11 Jehová dará poder a su pueblo; 
Jehová bendecirá a su pueblo con paz. 
\section*{Capítulo 30}
Acción de gracias por haber sido librado de la muerte 
Salmo cantado en la dedicación de la Casa. 
Salmo de David. 

30:1 Te glorificaré, oh Jehová, porque me has exaltado, 
Y no permitiste que mis enemigos se alegraran de mí. 
30:2 Jehová Dios mío, 
A ti clamé, y me sanaste. 
30:3 Oh Jehová, hiciste subir mi alma del Seol; 
Me diste vida, para que no descendiese a la sepultura. 
30:4 Cantad a Jehová, vosotros sus santos, 
Y celebrad la memoria de su santidad. 
30:5 Porque un momento será su ira, 
Pero su favor dura toda la vida. 
Por la noche durará el lloro, 
Y a la mañana vendrá la alegría. 
30:6 En mi prosperidad dije yo: 
No seré jamás conmovido, 
30:7 Porque tú, Jehová, con tu favor me afirmaste como monte fuerte. 
Escondiste tu rostro, fui turbado. 
30:8 A ti, oh Jehová, clamaré, 
Y al Señor suplicaré. 
30:9 ¿Qué provecho hay en mi muerte cuando descienda a la sepultura? 
¿Te alabará el polvo? ¿Anunciará tu verdad? 
30:10 Oye, oh Jehová, y ten misericordia de mí; 
Jehová, sé tú mi ayudador. 
30:11 Has cambiado mi lamento en baile; 
Desataste mi cilicio, y me ceñiste de alegría. 
30:12 Por tanto, a ti cantaré, gloria mía, y no estaré callado. 
Jehová Dios mío, te alabaré para siempre. 
\section*{Capítulo 31}
Declaración de confianza 
Al músico principal. Salmo de David. 
 
31:1 En ti, oh Jehová, he confiado; no sea yo confundido jamás; 
Líbrame en tu justicia. 
31:2 Inclina a mí tu oído, líbrame pronto; 
Sé tú mi roca fuerte, y fortaleza para salvarme. 
31:3 Porque tú eres mi roca y mi castillo; 
Por tu nombre me guiarás y me encaminarás. 
31:4 Sácame de la red que han escondido para mí, 
Pues tú eres mi refugio. 
31:5 En tu mano encomiendo mi espíritu;  
Tú me has redimido, oh Jehová, Dios de verdad. 
31:6 Aborrezco a los que esperan en vanidades ilusorias; 
Mas yo en Jehová he esperado. 
31:7 Me gozaré y alegraré en tu misericordia, 
Porque has visto mi aflicción; 
Has conocido mi alma en las angustias. 
31:8 No me entregaste en mano del enemigo; 
Pusiste mis pies en lugar espacioso. 
31:9 Ten misericordia de mí, oh Jehová, porque estoy en angustia; 
Se han consumido de tristeza mis ojos, mi alma también y mi cuerpo. 
31:10 Porque mi vida se va gastando de dolor, y mis años de suspirar; 
Se agotan mis fuerzas a causa de mi iniquidad, y mis huesos se han consumido. 
31:11 De todos mis enemigos soy objeto de oprobio, 
Y de mis vecinos mucho más, y el horror de mis conocidos; 
Los que me ven fuera huyen de mí. 
31:12 He sido olvidado de su corazón como un muerto; 
He venido a ser como un vaso quebrado. 
31:13 Porque oigo la calumnia de muchos; 
El miedo me asalta por todas partes, 
Mientras consultan juntos contra mí 
E idean quitarme la vida. 
31:14 Mas yo en ti confío, oh Jehová; 
Digo: Tú eres mi Dios. 
31:15 En tu mano están mis tiempos; 
Líbrame de la mano de mis enemigos y de mis perseguidores. 
31:16 Haz resplandecer tu rostro sobre tu siervo; 
Sálvame por tu misericordia. 
31:17 No sea yo avergonzado, oh Jehová, ya que te he invocado; 
Sean avergonzados los impíos, estén mudos en el Seol. 
31:18 Enmudezcan los labios mentirosos, 
Que hablan contra el justo cosas duras 
Con soberbia y menosprecio. 
31:19 ¡Cuán grande es tu bondad, que has guardado para los que te temen, 
Que has mostrado a los que esperan en ti, delante de los hijos de los hombres! 
31:20 En lo secreto de tu presencia los esconderás de la conspiración del hombre; 
Los pondrás en un tabernáculo a cubierto de contención de lenguas. 
31:21 Bendito sea Jehová, 
Porque ha hecho maravillosa su misericordia para conmigo en ciudad fortificada. 
31:22 Decía yo en mi premura: Cortado soy de delante de tus ojos; 
Pero tú oíste la voz de mis ruegos cuando a ti clamaba. 
31:23 Amad a Jehová, todos vosotros sus santos; 
A los fieles guarda Jehová, 
Y paga abundantemente al que procede con soberbia. 
31:24 Esforzaos todos vosotros los que esperáis en Jehová, 
Y tome aliento vuestro corazón. 
\section*{Capítulo 32}
La dicha del perdón 
Salmo de David. Masquil. 

32:1 Bienaventurado aquel cuya transgresión ha sido perdonada, y cubierto su pecado. 
32:2 Bienaventurado el hombre a quien Jehová no culpa de iniquidad, 
Y en cuyo espíritu no hay engaño. 
32:3 Mientras callé, se envejecieron mis huesos 
En mi gemir todo el día. 
32:4 Porque de día y de noche se agravó sobre mí tu mano; 
Se volvió mi verdor en sequedades de verano. Selah 
32:5 Mi pecado te declaré, y no encubrí mi iniquidad. 
Dije: Confesaré mis transgresiones a Jehová; 
Y tú perdonaste la maldad de mi pecado.  Selah 
32:6 Por esto orará a ti todo santo en el tiempo en que puedas ser hallado; 
Ciertamente en la inundación de muchas aguas no llegarán éstas a él. 
32:7 Tú eres mi refugio; me guardarás de la angustia; 
Con cánticos de liberación me rodearás. Selah 
32:8 Te haré entender, y te enseñaré el camino en que debes andar; 
Sobre ti fijaré mis ojos. 
32:9 No seáis como el caballo, o como el mulo, sin entendimiento, 
Que han de ser sujetados con cabestro y con freno, 
Porque si no, no se acercan a ti. 
32:10 Muchos dolores habrá para el impío; 
Mas al que espera en Jehová, le rodea la misericordia. 
32:11 Alegraos en Jehová y gozaos, justos; 
Y cantad con júbilo todos vosotros los rectos de corazón. 
\section*{Capítulo 33}
Alabanzas al Creador y Preservador 

33:1 Alegraos, oh justos, en Jehová; 
En los íntegros es hermosa la alabanza. 
33:2 Aclamad a Jehová con arpa; 
Cantadle con salterio y decacordio. 
33:3 Cantadle cántico nuevo; 
Hacedlo bien, tañendo con júbilo. 
33:4 Porque recta es la palabra de Jehová, 
Y toda su obra es hecha con fidelidad. 
33:5 El ama justicia y juicio; 
De la misericordia de Jehová está llena la tierra. 
33:6 Por la palabra de Jehová fueron hechos los cielos, 
Y todo el ejército de ellos por el aliento de su boca. 
33:7 El junta como montón las aguas del mar; 
El pone en depósitos los abismos. 
33:8 Tema a Jehová toda la tierra; 
Teman delante de él todos los habitantes del mundo. 
33:9 Porque él dijo, y fue hecho; 
El mandó, y existió. 
33:10 Jehová hace nulo el consejo de las naciones, 
Y frustra las maquinaciones de los pueblos. 
33:11 El consejo de Jehová permanecerá para siempre; 
Los pensamientos de su corazón por todas las generaciones. 
33:12 Bienaventurada la nación cuyo Dios es Jehová, 
El pueblo que él escogió como heredad para sí. 
33:13 Desde los cielos miró Jehová; 
Vio a todos los hijos de los hombres; 
33:14 Desde el lugar de su morada miró 
Sobre todos los moradores de la tierra. 
33:15 El formó el corazón de todos ellos; 
Atento está a todas sus obras. 
33:16 El rey no se salva por la multitud del ejército, 
Ni escapa el valiente por la mucha fuerza. 
33:17 Vano para salvarse es el caballo; 
La grandeza de su fuerza a nadie podrá librar. 
33:18 He aquí el ojo de Jehová sobre los que le temen, 
Sobre los que esperan en su misericordia, 
33:19 Para librar sus almas de la muerte, 
Y para darles vida en tiempo de hambre. 
33:20 Nuestra alma espera a Jehová; 
Nuestra ayuda y nuestro escudo es él. 
33:21 Por tanto, en él se alegrará nuestro corazón, 
Porque en su santo nombre hemos confiado. 
33:22 Sea tu misericordia, oh Jehová, sobre nosotros, 
Según esperamos en ti. 
\section*{Capítulo 34}
La protección divina 
Salmo de David, cuando mudó su semblante delante de Abimelec,  y él lo echó, y se fue. 
 
34:1 Bendeciré a Jehová en todo tiempo; 
Su alabanza estará de continuo en mi boca. 
34:2 En Jehová se gloriará mi alma; 
Lo oirán los mansos, y se alegrarán. 
34:3 Engrandeced a Jehová conmigo, 
Y exaltemos a una su nombre. 
34:4 Busqué a Jehová, y él me oyó, 
Y me libró de todos mis temores. 
34:5 Los que miraron a él fueron alumbrados, 
Y sus rostros no fueron avergonzados. 
34:6 Este pobre clamó, y le oyó Jehová, 
Y lo libró de todas sus angustias. 
34:7 El ángel de Jehová acampa alrededor de los que le temen, 
Y los defiende. 
34:8 Gustad, y ved que es bueno Jehová; 
Dichoso el hombre que confía en él. 
34:9 Temed a Jehová, vosotros sus santos, 
Pues nada falta a los que le temen. 
34:10 Los leoncillos necesitan, y tienen hambre; 
Pero los que buscan a Jehová no tendrán falta de ningún bien. 
34:11 Venid, hijos, oídme; 
El temor de Jehová os enseñaré. 
34:12 ¿Quién es el hombre que desea vida, 
Que desea muchos días para ver el bien? 
34:13 Guarda tu lengua del mal, 
Y tus labios de hablar engaño. 
34:14 Apártate del mal, y haz el bien; 
Busca la paz, y síguela. 
34:15 Los ojos de Jehová están sobre los justos, 
Y atentos sus oídos al clamor de ellos. 
34:16 La ira de Jehová contra los que hacen mal, 
Para cortar de la tierra la memoria de ellos. 
34:17 Claman los justos, y Jehová oye, 
Y los libra de todas sus angustias. 
34:18 Cercano está Jehová a los quebrantados de corazón; 
Y salva a los contritos de espíritu. 
34:19 Muchas son las aflicciones del justo, 
Pero de todas ellas le librará Jehová. 
34:20 El guarda todos sus huesos; 
Ni uno de ellos será quebrantado. 
34:21 Matará al malo la maldad, 
Y los que aborrecen al justo serán condenados. 
34:22 Jehová redime el alma de sus siervos, 
Y no serán condenados cuantos en él confían. 
\section*{Capítulo 35}
Plegaria pidiendo ser librado de los enemigos 
Salmo de David. 
 
35:1 Disputa, oh Jehová, con los que contra mí contienden; 
Pelea contra los que me combaten. 
35:2 Echa mano al escudo y al pavés, 
Y levántate en mi ayuda. 
35:3 Saca la lanza, cierra contra mis perseguidores; 
Di a mi alma: Yo soy tu salvación. 
35:4 Sean avergonzados y confundidos los que buscan mi vida; 
Sean vueltos atrás y avergonzados los que mi mal intentan. 
35:5 Sean como el tamo delante del viento, 
Y el ángel de Jehová los acose. 
35:6 Sea su camino tenebroso y resbaladizo, 
Y el ángel de Jehová los persiga. 
35:7 Porque sin causa escondieron para mí su red en un hoyo; 
Sin causa cavaron hoyo para mi alma. 
35:8 Véngale el quebrantamiento sin que lo sepa, 
Y la red que él escondió lo prenda; 
Con quebrantamiento caiga en ella. 
35:9 Entonces mi alma se alegrará en Jehová; 
Se regocijará en su salvación. 
35:10 Todos mis huesos dirán: Jehová, ¿quién como tú, 
Que libras al afligido del más fuerte que él, 
Y al pobre y menesteroso del que le despoja? 
35:11 Se levantan testigos malvados; 
De lo que no sé me preguntan; 
35:12 Me devuelven mal por bien, 
Para afligir a mi alma. 
35:13 Pero yo, cuando ellos enfermaron, me vestí de cilicio; 
Afligí con ayuno mi alma, 
Y mi oración se volvía a mi seno. 
35:14 Como por mi compañero, como por mi hermano andaba; 
Como el que trae luto por madre, enlutado me humillaba. 
35:15 Pero ellos se alegraron en mi adversidad, y se juntaron; 
Se juntaron contra mí gentes despreciables, y yo no lo entendía; 
Me despedazaban sin descanso; 
35:16 Como lisonjeros, escarnecedores y truhanes, 
Crujieron contra mí sus dientes. 
35:17 Señor, ¿hasta cuándo verás esto? 
Rescata mi alma de sus destrucciones, mi vida de los leones. 
35:18 Te confesaré en grande congregación; 
Te alabaré entre numeroso pueblo. 
35:19 No se alegren de mí los que sin causa son mis enemigos, 
Ni los que me aborrecen sin causa guiñen el ojo. 
35:20 Porque no hablan paz; 
Y contra los mansos de la tierra piensan palabras engañosas. 
35:21 Ensancharon contra mí su boca; 
Dijeron: ¡Ea, ea, nuestros ojos lo han visto! 
35:22 Tú lo has visto, oh Jehová; no calles; 
Señor, no te alejes de mí. 
35:23 Muévete y despierta para hacerme justicia, 
Dios mío y Señor mío, para defender mi causa. 
35:24 Júzgame conforme a tu justicia, Jehová Dios mío, 
Y no se alegren de mí. 
35:25 No digan en su corazón: ¡Ea, alma nuestra! 
No digan: ¡Le hemos devorado! 
35:26 Sean avergonzados y confundidos a una los que de mi mal se alegran; 
Vístanse de vergüenza y de confusión los que se engrandecen contra mí. 
35:27 Canten y alégrense los que están a favor de mi justa causa, 
Y digan siempre: Sea exaltado Jehová, 
Que ama la paz de su siervo. 
35:28 Y mi lengua hablará de tu justicia 
Y de tu alabanza todo el día. 
\section*{Capítulo 36}
La misericordia de Dios 
Al músico principal. Salmo de David, siervo de Jehová. 
 
36:1 La iniquidad del impío me dice al corazón: 
No hay temor de Dios delante de sus ojos. 
36:2 Se lisonjea, por tanto, en sus propios ojos, 
De que su iniquidad no será hallada y aborrecida. 
36:3 Las palabras de su boca son iniquidad y fraude; 
Ha dejado de ser cuerdo y de hacer el bien. 
36:4 Medita maldad sobre su cama; 
Está en camino no bueno, 
El mal no aborrece. 
36:5 Jehová, hasta los cielos llega tu misericordia, 
Y tu fidelidad alcanza hasta las nubes. 
36:6 Tu justicia es como los montes de Dios, 
Tus juicios, abismo grande. 
Oh Jehová, al hombre y al animal conservas. 
36:7 ¡Cuán preciosa, oh Dios, es tu misericordia! 
Por eso los hijos de los hombres se amparan bajo la sombra de tus alas. 
36:8 Serán completamente saciados de la grosura de tu casa, 
Y tú los abrevarás del torrente de tus delicias. 
36:9 Porque contigo está el manantial de la vida; 
En tu luz veremos la luz. 
36:10 Extiende tu misericordia a los que te conocen, 
Y tu justicia a los rectos de corazón. 
36:11 No venga pie de soberbia contra mí, 
Y mano de impíos no me mueva. 
36:12 Allí cayeron los hacedores de iniquidad; 
Fueron derribados, y no podrán levantarse. 
\section*{Capítulo 37}
El camino de los malos 
Salmo de David. 
 
37:1 No te impacientes a causa de los malignos, 
Ni tengas envidia de los que hacen iniquidad. 
37:2 Porque como hierba serán pronto cortados, 
Y como la hierba verde se secarán. 
37:3 Confía en Jehová, y haz el bien; 
Y habitarás en la tierra, y te apacentarás de la verdad. 
37:4 Deléitate asimismo en Jehová, 
Y él te concederá las peticiones de tu corazón. 
37:5 Encomienda a Jehová tu camino, 
Y confía en él; y él hará. 
37:6 Exhibirá tu justicia como la luz, 
Y tu derecho como el mediodía. 
37:7 Guarda silencio ante Jehová, y espera en él. 
No te alteres con motivo del que prospera en su camino, 
Por el hombre que hace maldades. 
37:8 Deja la ira, y desecha el enojo; 
No te excites en manera alguna a hacer lo malo. 
37:9 Porque los malignos serán destruidos, 
Pero los que esperan en Jehová, ellos heredarán la tierra. 
37:10 Pues de aquí a poco no existirá el malo; 
Observarás su lugar, y no estará allí. 
37:11 Pero los mansos heredarán la tierra, 
Y se recrearán con abundancia de paz. 
37:12 Maquina el impío contra el justo, 
Y cruje contra él sus dientes; 
37:13 El Señor se reirá de él; 
Porque ve que viene su día. 
37:14 Los impíos desenvainan espada y entesan su arco, 
Para derribar al pobre y al menesteroso, 
Para matar a los de recto proceder. 
37:15 Su espada entrará en su mismo corazón, 
Y su arco será quebrado. 
37:16 Mejor es lo poco del justo, 
Que las riquezas de muchos pecadores. 
37:17 Porque los brazos de los impíos serán quebrados; 
Mas el que sostiene a los justos es Jehová. 
37:18 Conoce Jehová los días de los perfectos, 
Y la heredad de ellos será para siempre. 
37:19 No serán avergonzados en el mal tiempo, 
Y en los días de hambre serán saciados. 
37:20 Mas los impíos perecerán, 
Y los enemigos de Jehová como la grasa de los carneros 
Serán consumidos; se disiparán como el humo. 
37:21 El impío toma prestado, y no paga; 
Mas el justo tiene misericordia, y da. 
37:22 Porque los benditos de él heredarán la tierra; 
Y los malditos de él serán destruidos. 
37:23 Por Jehová son ordenados los pasos del hombre, 
Y él aprueba su camino. 
37:24 Cuando el hombre cayere, no quedará postrado, 
Porque Jehová sostiene su mano. 
37:25 Joven fui, y he envejecido, 
Y no he visto justo desamparado, 
Ni su descendencia que mendigue pan. 
37:26 En todo tiempo tiene misericordia, y presta; 
Y su descendencia es para bendición. 
37:27 Apártate del mal, y haz el bien, 
Y vivirás para siempre. 
37:28 Porque Jehová ama la rectitud, 
Y no desampara a sus santos. 
Para siempre serán guardados; 
Mas la descendencia de los impíos será destruida. 
37:29 Los justos heredarán la tierra, 
Y vivirán para siempre sobre ella. 
37:30 La boca del justo habla sabiduría, 
Y su lengua habla justicia. 
37:31 La ley de su Dios está en su corazón; 
Por tanto, sus pies no resbalarán. 
37:32 Acecha el impío al justo, 
Y procura matarlo. 
37:33 Jehová no lo dejará en sus manos, 
Ni lo condenará cuando le juzgaren. 
37:34 Espera en Jehová, y guarda su camino, 
Y él te exaltará para heredar la tierra; 
Cuando sean destruidos los pecadores, lo verás. 
37:35 Vi yo al impío sumamente enaltecido, 
Y que se extendía como laurel verde. 
37:36 Pero él pasó, y he aquí ya no estaba; 
Lo busqué, y no fue hallado. 
37:37 Considera al íntegro, y mira al justo; 
Porque hay un final dichoso para el hombre de paz. 
37:38 Mas los transgresores serán todos a una destruidos; 
La posteridad de los impíos será extinguida. 
37:39 Pero la salvación de los justos es de Jehová, 
Y él es su fortaleza en el tiempo de la angustia. 
37:40 Jehová los ayudará y los librará; 
Los libertará de los impíos, y los salvará, 
Por cuanto en él esperaron. 
\section*{Capítulo 38}
Oración de un penitente 
Salmo de David, para recordar. 
 
38:1 Jehová, no me reprendas en tu furor, 
Ni me castigues en tu ira. 
38:2 Porque tus saetas cayeron sobre mí, 
Y sobre mí ha descendido tu mano. 
38:3 Nada hay sano en mi carne, a causa de tu ira; 
Ni hay paz en mis huesos, a causa de mi pecado. 
38:4 Porque mis iniquidades se han agravado sobre mi cabeza; 
Como carga pesada se han agravado sobre mí. 
38:5 Hieden y supuran mis llagas, 
A causa de mi locura. 
38:6 Estoy encorvado, estoy humillado en gran manera, 
Ando enlutado todo el día. 
38:7 Porque mis lomos están llenos de ardor, 
Y nada hay sano en mi carne. 
38:8 Estoy debilitado y molido en gran manera; 
Gimo a causa de la conmoción de mi corazón. 
38:9 Señor, delante de ti están todos mis deseos, 
Y mi suspiro no te es oculto. 
38:10 Mi corazón está acongojado, me ha dejado mi vigor, 
Y aun la luz de mis ojos me falta ya. 
38:11 Mis amigos y mis compañeros se mantienen lejos de mi plaga, 
Y mis cercanos se han alejado. 
38:12 Los que buscan mi vida arman lazos, 
Y los que procuran mi mal hablan iniquidades, 
Y meditan fraudes todo el día. 
38:13 Mas yo, como si fuera sordo, no oigo; 
Y soy como mudo que no abre la boca. 
38:14 Soy, pues, como un hombre que no oye, 
Y en cuya boca no hay reprensiones. 
38:15 Porque en ti, oh Jehová, he esperado; 
Tú responderás, Jehová Dios mío. 
38:16 Dije: No se alegren de mí; 
Cuando mi pie resbale, no se engrandezcan sobre mí. 
38:17 Pero yo estoy a punto de caer, 
Y mi dolor está delante de mí continuamente. 
38:18 Por tanto, confesaré mi maldad, 
Y me contristaré por mi pecado. 
38:19 Porque mis enemigos están vivos y fuertes, 
Y se han aumentado los que me aborrecen sin causa. 
38:20 Los que pagan mal por bien 
Me son contrarios, por seguir yo lo bueno. 
38:21 No me desampares, oh Jehová; 
Dios mío, no te alejes de mí. 
38:22 Apresúrate a ayudarme, 
Oh Señor, mi salvación. 
\section*{Capítulo 39}
El carácter transitorio de la vida 
Al músico principal; a Jedutún. Salmo de David. 
 
39:1 Yo dije: Atenderé a mis caminos, 
Para no pecar con mi lengua; 
Guardaré mi boca con freno, 
En tanto que el impío esté delante de mí. 
39:2 Enmudecí con silencio, me callé aun respecto de lo bueno; 
Y se agravó mi dolor. 
39:3 Se enardeció mi corazón dentro de mí; 
En mi meditación se encendió fuego, 
Y así proferí con mi lengua: 
39:4 Hazme saber, Jehová, mi fin, 
Y cuánta sea la medida de mis días; 
Sepa yo cuán frágil soy. 
39:5 He aquí, diste a mis días término corto, 
Y mi edad es como nada delante de ti; 
Ciertamente es completa vanidad todo hombre que vive. Selah 
39:6 Ciertamente como una sombra es el hombre; 
Ciertamente en vano se afana; 
Amontona riquezas, y no sabe quién las recogerá. 
39:7 Y ahora, Señor, ¿qué esperaré? 
Mi esperanza está en ti. 
39:8 Líbrame de todas mis transgresiones; 
No me pongas por escarnio del insensato. 
39:9 Enmudecí, no abrí mi boca, 
Porque tú lo hiciste. 
39:10 Quita de sobre mí tu plaga; 
Estoy consumido bajo los golpes de tu mano. 
39:11 Con castigos por el pecado corriges al hombre, 
Y deshaces como polilla lo más estimado de él; 
Ciertamente vanidad es todo hombre. Selah 
39:12 Oye mi oración, oh Jehová, y escucha mi clamor. 
No calles ante mis lágrimas; 
Porque forastero soy para ti, 
Y advenedizo, como todos mis padres. 
39:13 Déjame, y tomaré fuerzas, 
Antes que vaya y perezca. 
\section*{Capítulo 40}
Alabanza por la liberación divina 
Al músico principal. Salmo de David. 
 
40:1 Pacientemente esperé a Jehová, 
Y se inclinó a mí, y oyó mi clamor. 
40:2 Y me hizo sacar del pozo de la desesperación, del lodo cenagoso; 
Puso mis pies sobre peña, y enderezó mis pasos. 
40:3 Puso luego en mi boca cántico nuevo, alabanza a nuestro Dios. 
Verán esto muchos, y temerán, 
Y confiarán en Jehová. 
40:4 Bienaventurado el hombre que puso en Jehová su confianza, 
Y no mira a los soberbios, ni a los que se desvían tras la mentira. 
40:5 Has aumentado, oh Jehová Dios mío, tus maravillas; 
Y tus pensamientos para con nosotros, 
No es posible contarlos ante ti. 
Si yo anunciare y hablare de ellos, 
No pueden ser enumerados. 
40:6 Sacrificio y ofrenda no te agrada; 
Has abierto mis oídos; 
Holocausto y expiación no has demandado. 
40:7 Entonces dije: He aquí, vengo; 
En el rollo del libro está escrito de mí; 
40:8 El hacer tu voluntad, Dios mío, me ha agradado, 
Y tu ley está en medio de mi corazón. 
40:9 He anunciado justicia en grande congregación; 
He aquí, no refrené mis labios, 
Jehová, tú lo sabes. 
40:10 No encubrí tu justicia dentro de mi corazón; 
He publicado tu fidelidad y tu salvación; 
No oculté tu misericordia y tu verdad en grande asamblea. 
40:11 Jehová, no retengas de mí tus misericordias; 
Tu misericordia y tu verdad me guarden siempre. 
40:12 Porque me han rodeado males sin número; 
Me han alcanzado mis maldades, y no puedo levantar la vista. 
Se han aumentado más que los cabellos de mi cabeza, y mi corazón me falla. 
40:13 Quieras, oh Jehová, librarme; 
Jehová, apresúrate a socorrerme. 
40:14 Sean avergonzados y confundidos a una 
Los que buscan mi vida para destruirla. 
Vuelvan atrás y avergüéncense 
Los que mi mal desean; 
40:15 Sean asolados en pago de su afrenta 
Los que me dicen: ¡Ea, ea! 
40:16 Gócense y alégrense en ti todos los que te buscan, 
Y digan siempre los que aman tu salvación: 
Jehová sea enaltecido. 
40:17 Aunque afligido yo y necesitado, 
Jehová pensará en mí. 
Mi ayuda y mi libertador eres tú; 
Dios mío, no te tardes. 
\section*{Capítulo 41}
Oración pidiendo salud 
Al músico principal. Salmo de David. 
 
41:1 Bienaventurado el que piensa en el pobre; 
En el día malo lo librará Jehová. 
41:2 Jehová lo guardará, y le dará vida; 
Será bienaventurado en la tierra, 
Y no lo entregarás a la voluntad de sus enemigos. 
41:3 Jehová lo sustentará sobre el lecho del dolor; 
Mullirás toda su cama en su enfermedad. 
41:4 Yo dije: Jehová, ten misericordia de mí; 
Sana mi alma, porque contra ti he pecado. 
41:5 Mis enemigos dicen mal de mí, preguntando: 
¿Cuándo morirá, y perecerá su nombre? 
41:6 Y si vienen a verme, hablan mentira; 
Su corazón recoge para sí iniquidad, 
Y al salir fuera la divulgan. 
41:7 Reunidos murmuran contra mí todos los que me aborrecen; 
Contra mí piensan mal, diciendo de mí: 
41:8 Cosa pestilencial se ha apoderado de él; 
Y el que cayó en cama no volverá a levantarse. 
41:9 Aun el hombre de mi paz, en quien yo confiaba, el que de mi pan comía, 
Alzó contra mí el calcañar. 
41:10 Mas tú, Jehová, ten misericordia de mí, y hazme levantar, 
Y les daré el pago. 
41:11 En esto conoceré que te he agradado, 
Que mi enemigo no se huelgue de mí. 
41:12 En cuanto a mí, en mi integridad me has sustentado, 
Y me has hecho estar delante de ti para siempre. 
41:13 Bendito sea Jehová, el Dios de Israel, 
Por los siglos de los siglos. 
Amén y Amén.

%	\include{salmos2}
%	\include{Proverbios}
%	\include{Eclesiastes}
%	\include{Cantar}
%	\include{Isaias}
%	\include{Jeremias}
%	\include{Lamentaciones}
%	\include{Ezequiel}
%	\include{Daniel}
%	\include{Oseas}
%	\include{Joel}
%	\include{Amos}
%	\include{Abdias}
%	\include{Jonas}
%	\include{Miqueas}
%	\include{Nahum}
%	\include{Habacuc}
%	\include{Sofonias}
%	\include{Hageo}
%	\include{Zacarias}
%	\include{Malaquias}
%	
%\part{Nuevo Testamento}	
%
%%Evangelios y hechos de los apostoles
%
%\include{EvaMateo}
%\include{EvaMarcos}
%\include{EvaLucas}
%\include{EvaJuan}
%\include{hechos}
%
%%Epistolas de Pablo 
%
%\include{Romanos1}
%\include{corintios1}
%\include{corintios2}
%\include{galatas}	
%\include{Efesios}
%\include{Filipenses}
%\include{Colosenses}
%\include{Tesalonicenses1}
%\include{Tesalonicenses2}
%\include{Timoteo1}
%\include{Timoteo2}
%\include{Tito}
%\include{Filemon}
%
%%Cartas generales 
%\include{Hebreos}
%\include{Santiago}
%\include{Pedro1}
%\include{Pedro2}
%\include{Juan1}
%\include{Juan2}
%\include{Juan3}
%\include{Judas}
%\include{Apocalipsis}



	



\end{document}
