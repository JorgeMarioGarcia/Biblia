\documentclass[oneside, twocolumn,  12pt]{book}
\usepackage[spanish]{babel}
\usepackage[utf8]{inputenc}
\usepackage[T1]{fontenc}
\usepackage{lipsum} % Para generar texto de ejemplo
\usepackage{geometry}
\geometry{letterpaper, margin=1in}
\usepackage{setspace}
\usepackage{titlesec}
\usepackage{fancyhdr}
\usepackage{hyperref}

\usepackage{yfonts}




% Configuración de los capítulos y secciones
\titleformat{\chapter}[display]
{\normalfont\bfseries\LARGE}
{\filleft}
{0pt}
{\filcenter}
[\vspace{2ex}\titlerule]


% Configuración del encabezado y pie de página
\pagestyle{fancy}
\fancyhf{}
\fancyhead[L]{\leftmark}
\fancyfoot[C]{\thepage}

% Espaciado entre líneas
\setstretch{1.15}

\begin{document}
	
	\frontmatter
	\title{La Biblia}
	\author{Traducida al LaTeX}
	\date{}
	\maketitle
	
	\tableofcontents
	
	\mainmatter
	
	\part{Antiguo Testamento}
	\onecolumn
	
	LA SANTA BIBLIA, ANTIGUO TESTAMENTO, VERSIÓN DE CASIODORO DE REINA (1569)
	REVISADA POR CIPRIANO DE VALERA (1602), OTRAS REVISIONES: 1862, 1909 Y 1960
	
	Parte \# 1 (INCLUYE LA LEY), los 10 primeros libros del AT: Gn, Ex, Lv, Nm, Dt, Jos, Jue, Rt, 1 S y 2 S
	\twocolumn
	
	\include{genesis}
	\include{exodo}
	\include{levitico}
	\include{Numeros}
	\include{Deuteronomio}
\end{document}
