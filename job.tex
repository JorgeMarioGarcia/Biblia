\chapter{Job}
\section*{Capítulo 1 }
Las calamidades de Job  1:1 Hubo en tierra de Uz un varón llamado Job; y era este hombre perfecto y recto, temeroso de Dios y apartado del mal.  
1:2 Y le nacieron siete hijos y tres hijas.  
1:3 Su hacienda era siete mil ovejas, tres mil camellos, quinientas yuntas de bueyes, quinientas asnas, y muchísimos criados; y era aquel varón más grande que todos los orientales.  
1:4 E iban sus hijos y hacían banquetes en sus casas, cada uno en su día; y enviaban a llamar a sus tres hermanas para que comiesen y bebiesen con ellos.  
1:5 Y acontecía que habiendo pasado en turno los días del convite, Job enviaba y los santificaba, y se levantaba de mañana y ofrecía holocaustos conforme al número de todos ellos. Porque decía Job: Quizá habrán pecado mis hijos, y habrán blasfemado contra Dios en sus corazones. De esta manera hacía todos los días.  
1:6 Un día vinieron a presentarse delante de Jehová los hijos de Dios, entre los cuales vino también Satanás.  
1:7 Y dijo Jehová a Satanás: ¿De dónde vienes? Respondiendo Satanás a Jehová, dijo: De rodear la tierra y de andar por ella. 
1:8 Y Jehová dijo a Satanás: ¿No has considerado a mi siervo Job, que no hay otro como él en la tierra, varón perfecto y recto, temeroso de Dios y apartado del mal?  
1:9 Respondiendo Satanás a Jehová, dijo: ¿Acaso teme Job a Dios de balde?  
1:10 ¿No le has cercado alrededor a él y a su casa y a todo lo que tiene? Al trabajo de sus manos has dado bendición; por tanto, sus bienes han aumentado sobre la tierra.  
1:11 Pero extiende ahora tu mano y toca todo lo que tiene, y verás si no blasfema contra ti en tu misma presencia. 
1:12 Dijo Jehová a Satanás: He aquí, todo lo que tiene está en tu mano; solamente no pongas tu mano sobre él. Y salió Satanás de delante de Jehová.  
1:13 Y un día aconteció que sus hijos e hijas comían y bebían vino en casa de su hermano el primogénito,  
1:14 y vino un mensajero a Job, y le dijo: Estaban arando los bueyes, y las asnas paciendo cerca de ellos,  
1:15 y acometieron los sabeos y los tomaron, y mataron a los criados a filo de espada; solamente escapé yo para darte la noticia.  
1:16 Aún estaba éste hablando, cuando vino otro que dijo: Fuego de Dios cayó del cielo, que quemó las ovejas y a los pastores, y los consumió; solamente escapé yo para darte la noticia.  
1:17 Todavía estaba éste hablando, y vino otro que dijo: Los caldeos hicieron tres escuadrones, y arremetieron contra los camellos y se los llevaron, y mataron a los criados a filo de espada; y solamente escapé yo para darte la noticia.  
1:18 Entre tanto que éste hablaba, vino otro que dijo: Tus hijos y tus hijas estaban comiendo y bebiendo vino en casa de su hermano el primogénito;  
1:19 y un gran viento vino del lado del desierto y azotó las cuatro esquinas de la casa, la cual cayó sobre los jóvenes, y murieron; y solamente escapé yo para darte la noticia.  
1:20 Entonces Job se levantó, y rasgó su manto, y rasuró su cabeza, y se postró en tierra y adoró,  
1:21 y dijo: Desnudo salí del vientre de mi madre, y desnudo volveré allá. Jehová dio, y Jehová quitó; sea el nombre de Jehová bendito.  
1:22 En todo esto no pecó Job, ni atribuyó a Dios despropósito alguno.  
\section*{Capítulo 2 }

2:1 Aconteció que otro día vinieron los hijos de Dios para presentarse delante de Jehová, y Satanás vino también entre ellos presentándose delante de Jehová.  
2:2 Y dijo Jehová a Satanás: ¿De dónde vienes? Respondió Satanás a Jehová, y dijo: De rodear la tierra, y de andar por ella.  
2:3 Y Jehová dijo a Satanás: ¿No has considerado a mi siervo Job, que no hay otro como él en la tierra, varón perfecto y recto, temeroso de Dios y apartado del mal, y que todavía retiene su integridad, aun cuando tú me incitaste contra él para que lo arruinara sin causa?  
2:4 Respondiendo Satanás, dijo a Jehová: Piel por piel, todo lo que el hombre tiene dará por su vida.  
2:5 Pero extiende ahora tu mano, y toca su hueso y su carne, y verás si no blasfema contra ti en tu misma presencia.  
2:6 Y Jehová dijo a Satanás: He aquí, él está en tu mano; mas guarda su vida.  
2:7 Entonces salió Satanás de la presencia de Jehová, e hirió a Job con una sarna maligna desde la planta del pie hasta la coronilla de la cabeza.  
2:8 Y tomaba Job un tiesto para rascarse con él, y estaba sentado en medio de ceniza.  
2:9 Entonces le dijo su mujer: ¿Aún retienes tu integridad? Maldice a Dios, y muérete.  
2:10 Y él le dijo: Como suele hablar cualquiera de las mujeres fatuas, has hablado. ¿Qué? ¿Recibiremos de Dios el bien, y el mal no lo recibiremos? En todo esto no pecó Job con sus labios.  
2:11 Y tres amigos de Job, Elifaz temanita, Bildad suhita, y Zofar naamatita, luego que oyeron todo este mal que le había sobrevenido, vinieron cada uno de su lugar; porque habían convenido en venir juntos para condolerse de él y para consolarle.  
2:12 Los cuales, alzando los ojos desde lejos, no lo conocieron, y lloraron a gritos; y cada uno de ellos rasgó su manto, y los tres esparcieron polvo sobre sus cabezas hacia el cielo.  
2:13 Así se sentaron con él en tierra por siete días y siete noches, y ninguno le hablaba palabra, porque veían que su dolor era muy grande.  
\section*{Capítulo 3}
Job maldice el día en que nació  

3:1 Después de esto abrió Job su boca, y maldijo su día. 
3:2 Y exclamó Job, y dijo:  
3:3 Perezca el día en que yo nací,  
Y la noche en que se dijo: Varón es concebido.  
3:4 Sea aquel día sombrío,  
Y no cuide de él Dios desde arriba,  
Ni claridad sobre él resplandezca.  
3:5 Aféenlo tinieblas y sombra de muerte;  
Repose sobre él nublado  
Que lo haga horrible como día caliginoso.  
3:6 Ocupe aquella noche la oscuridad;  
No sea contada entre los días del año,  
Ni venga en el número de los meses.  
3:7 ¡Oh, que fuera aquella noche solitaria,  
Que no viniera canción alguna en ella!  
3:8 Maldíganla los que maldicen el día,  
Los que se aprestan para despertar a Leviatán.  
3:9 Oscurézcanse las estrellas de su alba;  
Espere la luz, y no venga,  
Ni vea los párpados de la mañana;  
3:10 Por cuanto no cerró las puertas del vientre donde yo estaba,  
Ni escondió de mis ojos la miseria.  
3:11 ¿Por qué no morí yo en la matriz,  
O expiré al salir del vientre?  
3:12 ¿Por qué me recibieron las rodillas?  
¿Y a qué los pechos para que mamase?  
3:13 Pues ahora estaría yo muerto, y reposaría;  
Dormiría, y entonces tendría descanso,  
3:14 Con los reyes y con los consejeros de la tierra,  
Que reedifican para sí ruinas;  
3:15 O con los príncipes que poseían el oro,  
Que llenaban de plata sus casas.  
3:16 ¿Por qué no fui escondido como abortivo,  
Como los pequeñitos que nunca vieron la luz?  
3:17 Allí los impíos dejan de perturbar,  
Y allí descansan los de agotadas fuerzas.  
3:18 Allí también reposan los cautivos;  
No oyen la voz del capataz.  
3:19 Allí están el chico y el grande,  
Y el siervo libre de su señor. 
3:20 ¿Por qué se da luz al trabajado,  
Y vida a los de ánimo amargado,  
3:21 Que esperan la muerte, y ella no llega, 
Aunque la buscan más que tesoros; 
3:22 Que se alegran sobremanera,  
Y se gozan cuando hallan el sepulcro?  
3:23 ¿Por qué se da vida al hombre que no sabe por donde ha de ir,  
Y a quien Dios ha encerrado?  
3:24 Pues antes que mi pan viene mi suspiro,  
Y mis gemidos corren como aguas.  
3:25 Porque el temor que me espantaba me ha venido,  
Y me ha acontecido lo que yo temía.  
3:26 No he tenido paz, no me aseguré, ni estuve reposado;  
No obstante, me vino turbación.  
\section*{Capítulo 4}

Elifaz reprende a Job  

4:1 Entonces respondió Elifaz temanita, y dijo:  
4:2 Si probáremos a hablarte, te será molesto;  
Pero ¿quién podrá detener las palabras?  
4:3 He aquí, tú enseñabas a muchos,  
Y fortalecías las manos débiles;  
4:4 Al que tropezaba enderezaban tus palabras,  
Y esforzabas las rodillas que decaían. 
4:5 Mas ahora que el mal ha venido sobre ti, te desalientas;  
Y cuando ha llegado hasta ti, te turbas.  
4:6 ¿No es tu temor a Dios tu confianza?  
¿No es tu esperanza la integridad de tus caminos? 
4:7 Recapacita ahora; ¿qué inocente se ha perdido?  
Y ¿en dónde han sido destruidos los rectos?  
4:8 Como yo he visto, los que aran iniquidad  
Y siembran injuria, la siegan.  
4:9 Perecen por el aliento de Dios,  
Y por el soplo de su ira son consumidos.  
4:10 Los rugidos del león, y los bramidos del rugiente,  
Y los dientes de los leoncillos son quebrantados. 
4:11 El león viejo perece por falta de presa,  
Y los hijos de la leona se dispersan.  
4:12 El asunto también me era a mí oculto;  
Mas mi oído ha percibido algo de ello.  
4:13 En imaginaciones de visiones nocturnas,  
Cuando el sueño cae sobre los hombres,  
4:14 Me sobrevino un espanto y un temblor,  
Que estremeció todos mis huesos;  
4:15 Y al pasar un espíritu por delante de mí,  
Hizo que se erizara el pelo de mi cuerpo.  
4:16 Paróse delante de mis ojos un fantasma,  
Cuyo rostro yo no conocí,  
Y quedo, oí que decía:  
4:17 ¿Será el hombre más justo que Dios?  
¿Será el varón más limpio que el que lo hizo?  
4:18 He aquí, en sus siervos no confía,  
Y notó necedad en sus ángeles;  
4:19 ¡Cuánto más en los que habitan en casas de barro,  
Cuyos cimientos están en el polvo,  
Y que serán quebrantados por la polilla!  
4:20 De la mañana a la tarde son destruidos,  
Y se pierden para siempre, sin haber quien repare en ello. 
4:21 Su hermosura, ¿no se pierde con ellos mismos?  
Y mueren sin haber adquirido sabiduría.  
\section*{Capítulo 5} 

5:1 Ahora, pues, da voces; ¿habrá quien te responda?  
¿Y a cuál de los santos te volverás?  
5:2 Es cierto que al necio lo mata la ira,  
Y al codicioso lo consume la envidia.  
5:3 Yo he visto al necio que echaba raíces,  
Y en la misma hora maldije su habitación.  
5:4 Sus hijos estarán lejos de la seguridad;  
En la puerta serán quebrantados,  
Y no habrá quien los libre.  
5:5 Su mies comerán los hambrientos,  
Y la sacarán de entre los espinos,  
Y los sedientos beberán su hacienda.  
5:6 Porque la aflicción no sale del polvo,  
Ni la molestia brota de la tierra.  
5:7 Pero como las chispas se levantan para volar por el aire,  
Así el hombre nace para la aflicción.  
5:8 Ciertamente yo buscaría a Dios,  
Y encomendaría a él mi causa;  
5:9 El cual hace cosas grandes e inescrutables,  
Y maravillas sin número;  
5:10 Que da la lluvia sobre la faz de la tierra,  
Y envía las aguas sobre los campos;  
5:11 Que pone a los humildes en altura,  
Y a los enlutados levanta a seguridad;  
5:12 Que frustra los pensamientos de los astutos,  
Para que sus manos no hagan nada;  
5:13 Que prende a los sabios en la astucia de ellos, 
Y frustra los designios de los perversos.  
5:14 De día tropiezan con tinieblas,  
Y a mediodía andan a tientas como de noche.  
5:15 Así libra de la espada al pobre, de la boca de los impíos,  
Y de la mano violenta;  
5:16 Pues es esperanza al menesteroso,  
Y la iniquidad cerrará su boca.  
5:17 He aquí, bienaventurado es el hombre a quien Dios castiga;  
Por tanto, no menosprecies la corrección del Todopoderoso. 
5:18 Porque él es quien hace la llaga, y él la vendará;  
El hiere, y sus manos curan.  
5:19 En seis tribulaciones te librará,  
Y en la séptima no te tocará el mal.  
5:20 En el hambre te salvará de la muerte,  
Y del poder de la espada en la guerra.  
5:21 Del azote de la lengua serás encubierto;  
No temerás la destrucción cuando viniere.  
5:22 De la destrucción y del hambre te reirás,  
Y no temerás de las fieras del campo;  
5:23 Pues aun con las piedras del campo tendrás tu pacto,  
Y las fieras del campo estarán en paz contigo.  
5:24 Sabrás que hay paz en tu tienda;  
Visitarás tu morada, y nada te faltará.  
5:25 Asimismo echarás de ver que tu descendencia es mucha,  
Y tu prole como la hierba de la tierra.  
5:26 Vendrás en la vejez a la sepultura,  
Como la gavilla de trigo que se recoge a su tiempo.  
5:27 He aquí lo que hemos inquirido, lo cual es así;  
Oyelo, y conócelo tú para tu provecho.  
\section*{Capítulo 6 }
Job reprocha la actitud de sus amigos  

6:1 Respondió entonces Job, y dijo:  
6:2 ¡Oh, que pesasen justamente mi queja y mi tormento,  
Y se alzasen igualmente en balanza!  
6:3 Porque pesarían ahora más que la arena del mar;  
Por eso mis palabras han sido precipitadas.  
6:4 Porque las saetas del Todopoderoso están en mí,  
Cuyo veneno bebe mi espíritu;  
Y terrores de Dios me combaten.  
6:5 ¿Acaso gime el asno montés junto a la hierba?  
¿Muge el buey junto a su pasto?  
6:6 ¿Se comerá lo desabrido sin sal?  
¿Habrá gusto en la clara del huevo?  
6:7 Las cosas que mi alma no quería tocar,  
Son ahora mi alimento.  
6:8 ¡Quién me diera que viniese mi petición,  
Y que me otorgase Dios lo que anhelo,  
6:9 Y que agradara a Dios quebrantarme;  
Que soltara su mano, y acabara conmigo! 
6:10 Sería aún mi consuelo,  
Si me asaltase con dolor sin dar más tregua,  
Que yo no he escondido las palabras del Santo.  
6:11 ¿Cuál es mi fuerza para esperar aún?  
¿Y cuál mi fin para que tenga aún paciencia? 
6:12 ¿Es mi fuerza la de las piedras,  
O es mi carne de bronce?  
6:13 ¿No es así que ni aun a mí mismo me puedo valer,  
Y que todo auxilio me ha faltado?  
6:14 El atribulado es consolado por su compañero;  
Aun aquel que abandona el temor del Omnipotente. 
6:15 Pero mis hermanos me traicionaron como un torrente;  
Pasan como corrientes impetuosas  
6:16 Que están escondidas por la helada,  
Y encubiertas por la nieve;  
6:17 Que al tiempo del calor son deshechas,  
Y al calentarse, desaparecen de su lugar;  
6:18 Se apartan de la senda de su rumbo,  
Van menguando, y se pierden.  
6:19 Miraron los caminantes de Temán,  
Los caminantes de Sabá esperaron en ellas;  
6:20 Pero fueron avergonzados por su esperanza;  
Porque vinieron hasta ellas, y se hallaron confusos.  
6:21 Ahora ciertamente como ellas sois vosotros;  
Pues habéis visto el tormento, y teméis.  
6:22 ¿Os he dicho yo: Traedme,  
Y pagad por mí de vuestra hacienda;  
6:23 Libradme de la mano del opresor,  
Y redimidme del poder de los violentos? 
6:24 Enseñadme, y yo callaré;  
Hacedme entender en qué he errado.  
6:25 ¡Cuán eficaces son las palabras rectas!  
Pero ¿qué reprende la censura vuestra?  
6:26 ¿Pensáis censurar palabras,  
Y los discursos de un desesperado, que son como el viento?  
6:27 También os arrojáis sobre el huérfano,  
Y caváis un hoyo para vuestro amigo.  
6:28 Ahora, pues, si queréis, miradme,  
Y ved si digo mentira delante de vosotros.  
6:29 Volved ahora, y no haya iniquidad;  
Volved aún a considerar mi justicia en esto.  
6:30 ¿Hay iniquidad en mi lengua?  
¿Acaso no puede mi paladar discernir las cosas inicuas?  
\section*{Capítulo 7 }
Job argumenta contra Dios  

7:1 ¿No es acaso brega la vida del hombre sobre la tierra,  
Y sus días como los días del jornalero?  
7:2 Como el siervo suspira por la sombra,  
Y como el jornalero espera el reposo de su trabajo, 
7:3 Así he recibido meses de calamidad,  
Y noches de trabajo me dieron por cuenta.  
7:4 Cuando estoy acostado, digo: ¿Cuándo me levantaré?  
Mas la noche es larga, y estoy lleno de inquietudes hasta el alba. 
7:5 Mi carne está vestida de gusanos, y de costras de polvo;  
Mi piel hendida y abominable.  
7:6 Y mis días fueron más veloces que la lanzadera del tejedor,  
Y fenecieron sin esperanza.  
7:7 Acuérdate que mi vida es un soplo,  
Y que mis ojos no volverán a ver el bien.  
7:8 Los ojos de los que me ven, no me verán más;  
Fijarás en mí tus ojos, y dejaré de ser.  
7:9 Como la nube se desvanece y se va,  
Así el que desciende al Seol no subirá;  
7:10 No volverá más a su casa,  
Ni su lugar le conocerá más.  
7:11 Por tanto, no refrenaré mi boca;  
Hablaré en la angustia de mi espíritu,  
Y me quejaré con la amargura de mi alma.  
7:12 ¿Soy yo el mar, o un monstruo marino,  
Para que me pongas guarda?  
7:13 Cuando digo: Me consolará mi lecho,  
Mi cama atenuará mis quejas; 
7:14 Entonces me asustas con sueños,  
Y me aterras con visiones.  
7:15 Y así mi alma tuvo por mejor la estrangulación,  
Y quiso la muerte más que mis huesos.  
7:16 Abomino de mi vida; no he de vivir para siempre;  
Déjame, pues, porque mis días son vanidad.  
7:17 ¿Qué es el hombre, para que lo engrandezcas,  
Y para que pongas sobre él tu corazón, 
7:18 Y lo visites todas las mañanas,  
Y todos los momentos lo pruebes?  
7:19 ¿Hasta cuándo no apartarás de mí tu mirada,  
Y no me soltarás siquiera hasta que trague mi saliva?  
7:20 Si he pecado, ¿qué puedo hacerte a ti, oh Guarda de los hombres?  
¿Por qué me pones por blanco tuyo,  
Hasta convertirme en una carga para mí mismo?  
7:21 ¿Y por qué no quitas mi rebelión, y perdonas mi iniquidad?  
Porque ahora dormiré en el polvo,  
Y si me buscares de mañana, ya no existiré.  
\section*{Capítulo 8 }
Bildad proclama la justicia de Dios  

8:1 Respondió Bildad suhita, y dijo:  
8:2 ¿Hasta cuándo hablarás tales cosas,  
Y las palabras de tu boca serán como viento impetuoso?  
8:3 ¿Acaso torcerá Dios el derecho,  
O pervertirá el Todopoderoso la justicia?  
8:4 Si tus hijos pecaron contra él,  
El los echó en el lugar de su pecado.  
8:5 Si tú de mañana buscares a Dios,  
Y rogares al Todopoderoso;  
8:6 Si fueres limpio y recto,  
Ciertamente luego se despertará por ti,  
Y hará próspera la morada de tu justicia.  
8:7 Y aunque tu principio haya sido pequeño,  
Tu postrer estado será muy grande.  
8:8 Porque pregunta ahora a las generaciones pasadas,  
Y disponte para inquirir a los padres de ellas;  
8:9 Pues nosotros somos de ayer, y nada sabemos,  
Siendo nuestros días sobre la tierra como sombra.  
8:10 ¿No te enseñarán ellos, te hablarán,  
Y de su corazón sacarán palabras?  
8:11 ¿Crece el junco sin lodo?  
¿Crece el prado sin agua?  
8:12 Aun en su verdor, y sin haber sido cortado,  
Con todo, se seca primero que toda hierba.  
8:13 Tales son los caminos de todos los que olvidan a Dios;  
Y la esperanza del impío perecerá;  
8:14 Porque su esperanza será cortada,  
Y su confianza es tela de araña.  
8:15 Se apoyará él en su casa, mas no permanecerá ella en pie;  
Se asirá de ella, mas no resistirá.  
8:16 A manera de un árbol está verde delante del sol,  
Y sus renuevos salen sobre su huerto;  
8:17 Se van entretejiendo sus raíces junto a una fuente,  
Y enlazándose hasta un lugar pedregoso.  
8:18 Si le arrancaren de su lugar,  
Este le negará entonces, diciendo: Nunca te vi.  
8:19 Ciertamente este será el gozo de su camino;  
Y del polvo mismo nacerán otros.  
8:20 He aquí, Dios no aborrece al perfecto,  
Ni apoya la mano de los malignos.  
8:21 Aún llenará tu boca de risa,  
Y tus labios de júbilo.  
8:22 Los que te aborrecen serán vestidos de confusión;  
Y la habitación de los impíos perecerá.  
\section*{Capítulo 9} 
Incapacidad de Job para responder a Dios  

9:1 Respondió Job, y dijo:  
9:2 Ciertamente yo sé que es así;  
¿Y cómo se justificará el hombre con Dios?  
9:3 Si quisiere contender con él,  
No le podrá responder a una cosa entre mil.  
9:4 El es sabio de corazón, y poderoso en fuerzas;  
¿Quién se endureció contra él, y le fue bien?  
9:5 El arranca los montes con su furor,  
Y no saben quién los trastornó;  
9:6 El remueve la tierra de su lugar,  
Y hace temblar sus columnas;  
9:7 El manda al sol, y no sale;  
Y sella las estrellas;  
9:8 El solo extendió los cielos,  
Y anda sobre las olas del mar;  
9:9 El hizo la Osa, el Orión y las Pléyades, 
Y los lugares secretos del sur; 
9:10 El hace cosas grandes e incomprensibles,  
Y maravillosas, sin número.  
9:11 He aquí que él pasará delante de mí, y yo no lo veré;  
Pasará, y no lo entenderé.  
9:12 He aquí, arrebatará; ¿quién le hará restituir?  
¿Quién le dirá: ¿Qué haces?  
9:13 Dios no volverá atrás su ira,  
Y debajo de él se abaten los que ayudan a los soberbios.  
9:14 ¿Cuánto menos le responderé yo,  
Y hablaré con él palabras escogidas?  
9:15 Aunque fuese yo justo, no respondería;  
Antes habría de rogar a mi juez.  
9:16 Si yo le invocara, y él me respondiese,  
Aún no creeré que haya escuchado mi voz.  
9:17 Porque me ha quebrantado con tempestad,  
Y ha aumentado mis heridas sin causa. 
9:18 No me ha concedido que tome aliento,  
Sino que me ha llenado de amarguras.  
9:19 Si habláremos de su potencia, por cierto es fuerte;  
Si de juicio, ¿quién me emplazará?  
9:20 Si yo me justificare, me condenaría mi boca;  
Si me dijere perfecto, esto me haría inicuo.  
9:21 Si fuese íntegro, no haría caso de mí mismo;  
Despreciaría mi vida.  
9:22 Una cosa resta que yo diga: 
Al perfecto y al impío él los consume.  
9:23 Si azote mata de repente,  
Se ríe del sufrimiento de los inocentes.  
9:24 La tierra es entregada en manos de los impíos,  
Y él cubre el rostro de sus jueces.  
Si no es él, ¿quién es? ¿Dónde está?  
9:25 Mis días han sido más ligeros que un correo;  
Huyeron, y no vieron el bien.  
9:26 Pasaron cual naves veloces;  
Como el águila que se arroja sobre la presa.  
9:27 Si yo dijere: Olvidaré mi queja,  
Dejaré mi triste semblante, y me esforzaré, 
9:28 Me turban todos mis dolores;  
Sé que no me tendrás por inocente.  
9:29 Yo soy impío;  
¿Para qué trabajaré en vano?  
9:30 Aunque me lave con aguas de nieve,  
Y limpie mis manos con la limpieza misma,  
9:31 Aún me hundirás en el hoyo,  
Y mis propios vestidos me abominarán.  
9:32 Porque no es hombre como yo, para que yo le responda,  
Y vengamos juntamente a juicio.  
9:33 No hay entre nosotros árbitro  
Que ponga su mano sobre nosotros dos. 
9:34 Quite de sobre mí su vara,  
Y su terror no me espante.  
9:35 Entonces hablaré, y no le temeré;  
Porque en este estado no estoy en mí.  
\section*{Capítulo 10 }
Job lamenta su condición  

10:1 Está mi alma hastiada de mi vida;  
Daré libre curso a mi queja,  
Hablaré con amargura de mi alma.  
10:2 Diré a Dios: No me condenes;  
Hazme entender por qué contiendes conmigo.  
10:3 ¿Te parece bien que oprimas,  
Que deseches la obra de tus manos,  
Y que favorezcas los designios de los impíos?  
10:4 ¿Tienes tú acaso ojos de carne?  
¿Ves tú como ve el hombre?  
10:5 ¿Son tus días como los días del hombre,  
O tus años como los tiempos humanos,  
10:6 Para que inquieras mi iniquidad,  
Y busques mi pecado,  
10:7 Aunque tú sabes que no soy impío,  
Y que no hay quien de tu mano me libre?  
10:8 Tus manos me hicieron y me formaron;  
¿Y luego te vuelves y me deshaces?  
10:9 Acuérdate que como a barro me diste forma;  
¿Y en polvo me has de volver?  
10:10 ¿No me vaciaste como leche,  
Y como queso me cuajaste?  
10:11 Me vestiste de piel y carne,  
Y me tejiste con huesos y nervios. 
10:12 Vida y misericordia me concediste,  
Y tu cuidado guardó mi espíritu.  
10:13 Estas cosas tienes guardadas en tu corazón;  
Yo sé que están cerca de ti.  
10:14 Si pequé, tú me has observado,  
Y no me tendrás por limpio de mi iniquidad. 
10:15 Si fuere malo, ¡ay de mí!  
Y si fuere justo, no levantaré mi cabeza,  
Estando hastiado de deshonra, y de verme afligido.  
10:16 Si mi cabeza se alzare, cual león tú me cazas;  
Y vuelves a hacer en mí maravillas.  
10:17 Renuevas contra mí tus pruebas,  
Y aumentas conmigo tu furor como tropas de relevo.  
10:18 ¿Por qué me sacaste de la matriz?  
Hubiera yo expirado, y ningún ojo me habría visto.  
10:19 Fuera como si nunca hubiera existido,  
Llevado del vientre a la sepultura.  
10:20 ¿No son pocos mis días?  
Cesa, pues, y déjame, para que me consuele un poco,  
10:21 Antes que vaya para no volver,  
A la tierra de tinieblas y de sombra de muerte;  
10:22 Tierra de oscuridad, lóbrega,  
Como sombra de muerte y sin orden,  
Y cuya luz es como densas tinieblas.  
\section*{Capítulo 11}
Zofar acusa de maldad a Job  

11:1 Respondió Zofar naamatita, y dijo:  
11:2 ¿Las muchas palabras no han de tener respuesta?  
¿Y el hombre que habla mucho será justificado?  
11:3 ¿Harán tus falacias callar a los hombres?  
¿Harás escarnio y no habrá quien te avergüence?  
11:4 Tú dices: Mi doctrina es pura,  
Y yo soy limpio delante de tus ojos.  
11:5 Mas ¡oh, quién diera que Dios hablara,  
Y abriera sus labios contigo,  
11:6 Y te declarara los secretos de la sabiduría,  
Que son de doble valor que las riquezas!  
Conocerías entonces que Dios te ha castigado menos de lo que tu iniquidad merece.  
11:7 ¿Descubrirás tú los secretos de Dios?  
¿Llegarás tú a la perfección del Todopoderoso?  
11:8 Es más alta que los cielos; ¿qué harás?  
Es más profunda que el Seol; ¿cómo la conocerás?  
11:9 Su dimensión es más extensa que la tierra,  
Y más ancha que el mar.  
11:10 Si él pasa, y aprisiona, y llama a juicio,  
¿Quién podrá contrarrestarle?  
11:11 Porque él conoce a los hombres vanos;  
Ve asimismo la iniquidad, ¿y no hará caso?  
11:12 El hombre vano se hará entendido,  
Cuando un pollino de asno montés nazca hombre.  
11:13 Si tú dispusieres tu corazón,  
Y extendieres a él tus manos;  
11:14 Si alguna iniquidad hubiere en tu mano, y la echares de ti,  
Y no consintieres que more en tu casa la injusticia,  
11:15 Entonces levantarás tu rostro limpio de mancha,  
Y serás fuerte, y nada temerás;  
11:16 Y olvidarás tu miseria,  
O te acordarás de ella como de aguas que pasaron.  
11:17 La vida te será más clara que el mediodía;  
Aunque oscureciere, será como la mañana.  
11:18 Tendrás confianza, porque hay esperanza;  
Mirarás alrededor, y dormirás seguro.  
11:19 Te acostarás, y no habrá quien te espante;  
Y muchos suplicarán tu favor.  
11:20 Pero los ojos de los malos se consumirán,  
Y no tendrán refugio; 
Y su esperanza será dar su último suspiro.  
\section*{Capítulo 12}
Job proclama el poder y la sabiduría de Dios  

12:1 Respondió entonces Job, diciendo:  
12:2 Ciertamente vosotros sois el pueblo,  
Y con vosotros morirá la sabiduría.  
12:3 También tengo yo entendimiento como vosotros;  
No soy yo menos que vosotros;  
¿Y quién habrá que no pueda decir otro tanto? 
12:4 Yo soy uno de quien su amigo se mofa,  
Que invoca a Dios, y él le responde;  
Con todo, el justo y perfecto es escarnecido.  
12:5 Aquel cuyos pies van a resbalar  
Es como una lámpara despreciada de aquel que está a sus anchas.  
12:6 Prosperan las tiendas de los ladrones,  
Y los que provocan a Dios viven seguros,  
En cuyas manos él ha puesto cuanto tienen.  
12:7 Y en efecto, pregunta ahora a las bestias, y ellas te enseñarán;  
A las aves de los cielos, y ellas te lo mostrarán;  
12:8 O habla a la tierra, y ella te enseñará;  
Los peces del mar te lo declararán también.  
12:9 ¿Qué cosa de todas estas no entiende  
Que la mano de Jehová la hizo?  
12:10 En su mano está el alma de todo viviente,  
Y el hálito de todo el género humano.  
12:11 Ciertamente el oído distingue las palabras,  
Y el paladar gusta las viandas.  
12:12 En los ancianos está la ciencia,  
Y en la larga edad la inteligencia.  
12:13 Con Dios está la sabiduría y el poder;  
Suyo es el consejo y la inteligencia.  
12:14 Si él derriba, no hay quien edifique;  
Encerrará al hombre, y no habrá quien le abra.  
12:15 Si él detiene las aguas, todo se seca;  
Si las envía, destruyen la tierra.  
12:16 Con él está el poder y la sabiduría;  
Suyo es el que yerra, y el que hace errar. 
12:17 El hace andar despojados de consejo a los consejeros,  
Y entontece a los jueces.  
12:18 El rompe las cadenas de los tiranos,  
Y les ata una soga a sus lomos.  
12:19 El lleva despojados a los príncipes,  
Y trastorna a los poderosos.  
12:20 Priva del habla a los que dicen verdad,  
Y quita a los ancianos el consejo. 
12:21 El derrama menosprecio sobre los príncipes,  
Y desata el cinto de los fuertes.  
12:22 El descubre las profundidades de las tinieblas,  
Y saca a luz la sombra de muerte.  
12:23 El multiplica las naciones, y él las destruye;  
Esparce a las naciones, y las vuelve a reunir.  
12:24 El quita el entendimiento a los jefes del pueblo de la tierra,  
Y los hace vagar como por un yermo sin camino.  
12:25 Van a tientas, como en tinieblas y sin luz,  
Y los hace errar como borrachos.  
\section*{Capítulo 13}
Job defiende su integridad  

13:1 He aquí que todas estas cosas han visto mis ojos,  
Y oído y entendido mis oídos.  
13:2 Como vosotros lo sabéis, lo sé yo;  
No soy menos que vosotros.  
13:3 Mas yo hablaría con el Todopoderoso,  
Y querría razonar con Dios.  
13:4 Porque ciertamente vosotros sois fraguadores de mentira;  
Sois todos vosotros médicos nulos.  
13:5 Ojalá callarais por completo,  
Porque esto os fuera sabiduría.  
13:6 Oíd ahora mi razonamiento,  
Y estad atentos a los argumentos de mis labios. 
13:7 ¿Hablaréis iniquidad por Dios?  
¿Hablaréis por él engaño?  
13:8 ¿Haréis acepción de personas a su favor?  
¿Contenderéis vosotros por Dios?  
13:9 ¿Sería bueno que él os escudriñase?  
¿Os burlaréis de él como quien se burla de algún hombre?  
13:10 El os reprochará de seguro,  
Si solapadamente hacéis acepción de personas.  
13:11 De cierto su alteza os habría de espantar,  
Y su pavor habría de caer sobre vosotros.  
13:12 Vuestras máximas son refranes de ceniza,  
Y vuestros baluartes son baluartes de lodo.  
13:13 Escuchadme, y hablaré yo,  
Y que me venga después lo que viniere.  
13:14 ¿Por qué quitaré yo mi carne con mis dientes,  
Y tomaré mi vida en mi mano?  
13:15 He aquí, aunque él me matare, en él esperaré;  
No obstante, defenderé delante de él mis caminos,  
13:16 Y él mismo será mi salvación,  
Porque no entrará en su presencia el impío.  
13:17 Oíd con atención mi razonamiento,  
Y mi declaración entre en vuestros oídos.  
13:18 He aquí ahora, si yo expusiere mi causa,  
Sé que seré justificado.  
13:19 ¿Quién es el que contenderá conmigo?  
Porque si ahora yo callara, moriría.  
13:20 A lo menos dos cosas no hagas conmigo;  
Entonces no me esconderé de tu rostro:  
13:21 Aparta de mí tu mano,  
Y no me asombre tu terror.  
13:22 Llama luego, y yo responderé;  
O yo hablaré, y respóndeme tú.  
13:23 ¿Cuántas iniquidades y pecados tengo yo?  
Hazme entender mi transgresión y mi pecado.  
13:24 ¿Por qué escondes tu rostro,  
Y me cuentas por tu enemigo?  
13:25 ¿A la hoja arrebatada has de quebrantar,  
Y a una paja seca has de perseguir?  
13:26 ¿Por qué escribes contra mí amarguras,  
Y me haces cargo de los pecados de mi juventud?  
13:27 Pones además mis pies en el cepo, y observas todos mis caminos,  
Trazando un límite para las plantas de mis pies. 
13:28 Y mi cuerpo se va gastando como de carcoma,  
Como vestido que roe la polilla. 
\section*{Capítulo 14}
Job discurre sobre la brevedad de la vida  

14:1 El hombre nacido de mujer,  
Corto de días, y hastiado de sinsabores,  
14:2 Sale como una flor y es cortado,  
Y huye como la sombra y no permanece.  
14:3 ¿Sobre éste abres tus ojos,  
Y me traes a juicio contigo?  
14:4 ¿Quién hará limpio a lo inmundo?  
Nadie.  
14:5 Ciertamente sus días están determinados,  
Y el número de sus meses está cerca de ti;  
Le pusiste límites, de los cuales no pasará.  
14:6 Si tú lo abandonares, él dejará de ser;  
Entre tanto deseará, como el jornalero, su día.  
14:7 Porque si el árbol fuere cortado, aún queda de él esperanza;  
Retoñará aún, y sus renuevos no faltarán.  
14:8 Si se envejeciere en la tierra su raíz,  
Y su tronco fuere muerto en el polvo,  
14:9 Al percibir el agua reverdecerá,  
Y hará copa como planta nueva.  
14:10 Mas el hombre morirá, y será cortado;  
Perecerá el hombre, ¿y dónde estará él?  
14:11 Como las aguas se van del mar,  
Y el río se agota y se seca,  
14:12 Así el hombre yace y no vuelve a levantarse;  
Hasta que no haya cielo, no despertarán,  
Ni se levantarán de su sueño.  
14:13 ¡Oh, quién me diera que me escondieses en el Seol,  
Que me encubrieses hasta apaciguarse tu ira,  
Que me pusieses plazo, y de mí te acordaras!  
14:14 Si el hombre muriere, ¿volverá a vivir?  
Todos los días de mi edad esperaré,  
Hasta que venga mi liberación.  
14:15 Entonces llamarás, y yo te responderé;  
Tendrás afecto a la hechura de tus manos.  
14:16 Pero ahora me cuentas los pasos,  
Y no das tregua a mi pecado;  
14:17 Tienes sellada en saco mi prevaricación,  
Y tienes cosida mi iniquidad.  
14:18 Ciertamente el monte que cae se deshace,  
Y las peñas son removidas de su lugar;  
14:19 Las piedras se desgastan con el agua impetuosa, que se lleva el polvo de la tierra;  
De igual manera haces tú perecer la esperanza del hombre.  
14:20 Para siempre serás más fuerte que él, y él se va;  
Demudarás su rostro, y le despedirás. 
14:21 Sus hijos tendrán honores, pero él no lo sabrá;  
O serán humillados, y no entenderá de ello.  
14:22 Mas su carne sobre él se dolerá,  
Y se entristecerá en él su alma. 
\section*{Capítulo 15 }
Elifaz reprende a Job  

15:1 Respondió Elifaz temanita, y dijo:  
15:2 ¿Proferirá el sabio vana sabiduría,  
Y llenará su vientre de viento solano?  
15:3 ¿Disputará con palabras inútiles,  
Y con razones sin provecho?  
15:4 Tú también disipas el temor,  
Y menoscabas la oración delante de Dios.  
15:5 Porque tu boca declaró tu iniquidad,  
Pues has escogido el hablar de los astutos.  
15:6 Tu boca te condenará, y no yo;  
Y tus labios testificarán contra ti.  
15:7 ¿Naciste tú primero que Adán?  
¿O fuiste formado antes que los collados?  
15:8 ¿Oíste tú el secreto de Dios,  
Y está limitada a ti la sabiduría?  
15:9 ¿Qué sabes tú que no sepamos?  
¿Qué entiendes tú que no se halle en nosotros? 
15:10 Cabezas canas y hombres muy ancianos hay entre nosotros,  
Mucho más avanzados en días que tu padre.  
15:11 ¿En tan poco tienes las consolaciones de Dios,  
Y las palabras que con dulzura se te dicen?  
15:12 ¿Por qué tu corazón te aleja,  
Y por qué guiñan tus ojos,  
15:13 Para que contra Dios vuelvas tu espíritu,  
Y saques tales palabras de tu boca?  
15:14 ¿Qué cosa es el hombre para que sea limpio,  
Y para que se justifique el nacido de mujer?  
15:15 He aquí, en sus santos no confía,  
Y ni aun los cielos son limpios delante de sus ojos;  
15:16 ¿Cuánto menos el hombre abominable y vil,  
Que bebe la iniquidad como agua?  
15:17 Escúchame; yo te mostraré,  
Y te contaré lo que he visto;  
15:18 Lo que los sabios nos contaron  
De sus padres, y no lo encubrieron;  
15:19 A quienes únicamente fue dada la tierra,  
Y no pasó extraño por en medio de ellos.  
15:20 Todos sus días, el impío es atormentado de dolor,  
Y el número de sus años está escondido para el violento. 
15:21 Estruendos espantosos hay en sus oídos;  
En la prosperidad el asolador vendrá sobre él.  
15:22 El no cree que volverá de las tinieblas,  
Y descubierto está para la espada.  
15:23 Vaga alrededor tras el pan, diciendo: ¿En dónde está?  
Sabe que le está preparado día de tinieblas.  
15:24 Tribulación y angustia le turbarán,  
Y se esforzarán contra él como un rey dispuesto para la batalla,  
15:25 Por cuanto él extendió su mano contra Dios,  
Y se portó con soberbia contra el Todopoderoso.  
15:26 Corrió contra él con cuello erguido, 
Con la espesa barrera de sus escudos.  
15:27 Porque la gordura cubrió su rostro,  
E hizo pliegues sobre sus ijares;  
15:28 Y habitó las ciudades asoladas,  
Las casas inhabitadas,  
Que estaban en ruinas.  
15:29 No prosperará, ni durarán sus riquezas,  
Ni extenderá por la tierra su hermosura.  
15:30 No escapará de las tinieblas;  
La llama secará sus ramas,  
Y con el aliento de su boca perecerá.  
15:31 No confíe el iluso en la vanidad,  
Porque ella será su recompensa.  
15:32 El será cortado antes de su tiempo,  
Y sus renuevos no reverdecerán.  
15:33 Perderá su agraz como la vid,  
Y derramará su flor como el olivo.  
15:34 Porque la congregación de los impíos será asolada,  
Y fuego consumirá las tiendas de soborno.  
15:35 Concibieron dolor, dieron a luz iniquidad,  
Y en sus entrañas traman engaño.  
\section*{Capítulo 16 }
Job se queja contra Dios  

16:1 Respondió Job, y dijo:  
16:2 Muchas veces he oído cosas como estas;  
Consoladores molestos sois todos vosotros.  
16:3 ¿Tendrán fin las palabras vacías?  
¿O qué te anima a responder?  
16:4 También yo podría hablar como vosotros,  
Si vuestra alma estuviera en lugar de la mía;  
Yo podría hilvanar contra vosotros palabras,  
Y sobre vosotros mover mi cabeza.  
16:5 Pero yo os alentaría con mis palabras,  
Y la consolación de mis labios apaciguaría vuestro dolor. 
16:6 Si hablo, mi dolor no cesa;  
Y si dejo de hablar, no se aparta de mí.  
16:7 Pero ahora tú me has fatigado;  
Has asolado toda mi compañía.  
16:8 Tú me has llenado de arrugas; testigo es mi flacura,  
Que se levanta contra mí para testificar en mi rostro.  
16:9 Su furor me despedazó, y me ha sido contrario;  
Crujió sus dientes contra mí;  
Contra mí aguzó sus ojos mi enemigo.  
16:10 Abrieron contra mí su boca;  
Hirieron mis mejillas con afrenta;  
Contra mí se juntaron todos.  
16:11 Me ha entregado Dios al mentiroso,  
Y en las manos de los impíos me hizo caer.  
16:12 Próspero estaba, y me desmenuzó;  
Me arrebató por la cerviz y me despedazó, 
Y me puso por blanco suyo.  
16:13 Me rodearon sus flecheros,  
Partió mis riñones, y no perdonó;  
Mi hiel derramó por tierra.  
16:14 Me quebrantó de quebranto en quebranto;  
Corrió contra mí como un gigante.  
16:15 Cosí cilicio sobre mi piel,  
Y puse mi cabeza en el polvo.  
16:16 Mi rostro está inflamado con el lloro,  
Y mis párpados entenebrecidos,  
16:17 A pesar de no haber iniquidad en mis manos,  
Y de haber sido mi oración pura.  
16:18 ¡Oh tierra! no cubras mi sangre,  
Y no haya lugar para mi clamor.  
16:19 Mas he aquí que en los cielos está mi testigo,  
Y mi testimonio en las alturas.  
16:20 Disputadores son mis amigos;  
Mas ante Dios derramaré mis lágrimas.  
16:21 ¡Ojalá pudiese disputar el hombre con Dios,  
Como con su prójimo!  
16:22 Mas los años contados vendrán,  
Y yo iré por el camino de donde no volveré.  
\section*{Capítulo 17}

17:1 Mi aliento se agota, se acortan mis días,  
Y me está preparado el sepulcro.  
17:2 No hay conmigo sino escarnecedores,  
En cuya amargura se detienen mis ojos.  
17:3 Dame fianza, oh Dios; sea mi protección cerca de ti.  
Porque ¿quién querría responder por mí?  
17:4 Porque a éstos has escondido de su corazón la inteligencia;  
Por tanto, no los exaltarás.  
17:5 Al que denuncia a sus amigos como presa,  
Los ojos de sus hijos desfallecerán.  
17:6 El me ha puesto por refrán de pueblos,  
Y delante de ellos he sido como tamboril.  
17:7 Mis ojos se oscurecieron por el dolor,  
Y mis pensamientos todos son como sombra.  
17:8 Los rectos se maravillarán de esto,  
Y el inocente se levantará contra el impío.  
17:9 No obstante, proseguirá el justo su camino,  
Y el limpio de manos aumentará la fuerza.  
17:10 Pero volved todos vosotros, y venid ahora,  
Y no hallaré entre vosotros sabio.  
17:11 Pasaron mis días, fueron arrancados mis pensamientos,  
Los designios de mi corazón.  
17:12 Pusieron la noche por día,  
Y la luz se acorta delante de las tinieblas.  
17:13 Si yo espero, el Seol es mi casa;  
Haré mi cama en las tinieblas.  
17:14 A la corrupción he dicho: Mi padre eres tú;  
A los gusanos: Mi madre y mi hermana.  
17:15 ¿Dónde, pues, estará ahora mi esperanza?  
Y mi esperanza, ¿quién la verá?  
17:16 A la profundidad del Seol descenderán,  
Y juntamente descansarán en el polvo. 
\section*{Capítulo 18 }
Bildad describe la suerte de los malos  

18:1 Respondió Bildad suhita, y dijo:  
18:2 ¿Cuándo pondréis fin a las palabras?  
Entended, y después hablemos.  
18:3 ¿Por qué somos tenidos por bestias,  
Y a vuestros ojos somos viles?  
18:4 Oh tú, que te despedazas en tu furor,  
¿Será abandonada la tierra por tu causa,  
Y serán removidas de su lugar las peñas? 
18:5 Ciertamente la luz de los impíos será apagada,  
Y no resplandecerá la centella de su fuego.  
18:6 La luz se oscurecerá en su tienda,  
Y se apagará sobre él su lámpara.  
18:7 Sus pasos vigorosos serán acortados,  
Y su mismo consejo lo precipitará.  
18:8 Porque red será echada a sus pies,  
Y sobre mallas andará.  
18:9 Lazo prenderá su calcañar;  
Se afirmará la trampa contra él.  
18:10 Su cuerda está escondida en la tierra,  
Y una trampa le aguarda en la senda.  
18:11 De todas partes lo asombrarán temores,  
Y le harán huir desconcertado.  
18:12 Serán gastadas de hambre sus fuerzas,  
Y a su lado estará preparado quebrantamiento.  
18:13 La enfermedad roerá su piel,  
Y a sus miembros devorará el primogénito de la muerte.  
18:14 Su confianza será arrancada de su tienda,  
Y al rey de los espantos será conducido.  
18:15 En su tienda morará como si no fuese suya;  
Piedra de azufre será esparcida sobre su morada.  
18:16 Abajo se secarán sus raíces,  
Y arriba serán cortadas sus ramas.  
18:17 Su memoria perecerá de la tierra,  
Y no tendrá nombre por las calles.  
18:18 De la luz será lanzado a las tinieblas,  
Y echado fuera del mundo.  
18:19 No tendrá hijo ni nieto en su pueblo,  
Ni quien le suceda en sus moradas.  
18:20 Sobre su día se espantarán los de occidente,  
Y pavor caerá sobre los de oriente.  
18:21 Ciertamente tales son las moradas del impío,  
Y este será el lugar del que no conoció a Dios.  
\section*{Capítulo 19}
Job confía en que Dios lo justificará  

19:1 Respondió entonces Job, y dijo:  
19:2 ¿Hasta cuándo angustiaréis mi alma,  
Y me moleréis con palabras?  
19:3 Ya me habéis vituperado diez veces;  
¿No os avergonzáis de injuriarme?  
19:4 Aun siendo verdad que yo haya errado,  
Sobre mí recaería mi error.  
19:5 Pero si vosotros os engrandecéis contra mí,  
Y contra mí alegáis mi oprobio,  
19:6 Sabed ahora que Dios me ha derribado,  
Y me ha envuelto en su red.  
19:7 He aquí, yo clamaré agravio, y no seré oído;  
Daré voces, y no habrá juicio.  
19:8 Cercó de vallado mi camino, y no pasaré;  
Y sobre mis veredas puso tinieblas.  
19:9 Me ha despojado de mi gloria,  
Y quitado la corona de mi cabeza.  
19:10 Me arruinó por todos lados, y perezco;  
Y ha hecho pasar mi esperanza como árbol arrancado.  
19:11 Hizo arder contra mí su furor,  
Y me contó para sí entre sus enemigos.  
19:12 Vinieron sus ejércitos a una, y se atrincheraron en mí,  
Y acamparon en derredor de mi tienda.  
19:13 Hizo alejar de mí a mis hermanos,  
Y mis conocidos como extraños se apartaron de mí.  
19:14 Mis parientes se detuvieron,  
Y mis conocidos se olvidaron de mí.  
19:15 Los moradores de mi casa y mis criadas me tuvieron por extraño;  
Forastero fui yo a sus ojos.  
19:16 Llamé a mi siervo, y no respondió;  
De mi propia boca le suplicaba.  
19:17 Mi aliento vino a ser extraño a mi mujer,  
Aunque por los hijos de mis entrañas le rogaba.  
19:18 Aun los muchachos me menospreciaron;  
Al levantarme, hablaban contra mí.  
19:19 Todos mis íntimos amigos me aborrecieron,  
Y los que yo amaba se volvieron contra mí.  
19:20 Mi piel y mi carne se pegaron a mis huesos,  
Y he escapado con sólo la piel de mis dientes.  
19:21 ¡Oh, vosotros mis amigos, tened compasión de mí, tened compasión de mí!  
Porque la mano de Dios me ha tocado.  
19:22 ¿Por qué me perseguís como Dios,  
Y ni aun de mi carne os saciáis?  
19:23 ¡Quién diese ahora que mis palabras fuesen escritas!  
¡Quién diese que se escribiesen en un libro;  
19:24 Que con cincel de hierro y con plomo  
Fuesen esculpidas en piedra para siempre!  
19:25 Yo sé que mi Redentor vive,  
Y al fin se levantará sobre el polvo;  
19:26 Y después de deshecha esta mi piel,  
En mi carne he de ver a Dios;  
19:27 Al cual veré por mí mismo,  
Y mis ojos lo verán, y no otro,  
Aunque mi corazón desfallece dentro de mí.  
19:28 Mas debierais decir: ¿Por qué le perseguimos?  
Ya que la raíz del asunto se halla en mí.  
19:29 Temed vosotros delante de la espada;  
Porque sobreviene el furor de la espada a causa de las injusticias,  
Para que sepáis que hay un juicio.  
\section*{Capítulo 20 }
Zofar describe las calamidades de los malos  

20:1 Respondió Zofar naamatita, y dijo:  
20:2 Por cierto mis pensamientos me hacen responder,  
Y por tanto me apresuro.  
20:3 La reprensión de mi censura he oído,  
Y me hace responder el espíritu de mi inteligencia.  
20:4 ¿No sabes esto, que así fue siempre,  
Desde el tiempo que fue puesto el hombre sobre la tierra,  
20:5 Que la alegría de los malos es breve,  
Y el gozo del impío por un momento?  
20:6 Aunque subiere su altivez hasta el cielo,  
Y su cabeza tocare en las nubes,  
20:7 Como su estiércol, perecerá para siempre;  
Los que le hubieren visto dirán: ¿Qué hay de él?  
20:8 Como sueño volará, y no será hallado,  
Y se disipará como visión nocturna.  
20:9 El ojo que le veía, nunca más le verá,  
Ni su lugar le conocerá más.  
20:10 Sus hijos solicitarán el favor de los pobres,  
Y sus manos devolverán lo que él robó. 
20:11 Sus huesos están llenos de su juventud,  
Mas con él en el polvo yacerán.  
20:12 Si el mal se endulzó en su boca,  
Si lo ocultaba debajo de su lengua,  
20:13 Si le parecía bien, y no lo dejaba,  
Sino que lo detenía en su paladar;  
20:14 Su comida se mudará en sus entrañas;  
Hiel de áspides será dentro de él.  
20:15 Devoró riquezas, pero las vomitará;  
De su vientre las sacará Dios.  
20:16 Veneno de áspides chupará;  
Lo matará lengua de víbora.  
20:17 No verá los arroyos, los ríos, 
Los torrentes de miel y de leche.  
20:18 Restituirá el trabajo conforme a los bienes que tomó,  
Y no los tragará ni gozará.  
20:19 Por cuanto quebrantó y desamparó a los pobres,  
Robó casas, y no las edificó;  
20:20 Por tanto, no tendrá sosiego en su vientre,  
Ni salvará nada de lo que codiciaba. 
20:21 No quedó nada que no comiese;  
Por tanto, su bienestar no será duradero. 
20:22 En el colmo de su abundancia padecerá estrechez;  
La mano de todos los malvados vendrá sobre él.  
20:23 Cuando se pusiere a llenar su vientre,  
Dios enviará sobre él el ardor de su ira,  
Y la hará llover sobre él y sobre su comida.  
20:24 Huirá de las armas de hierro,  
Y el arco de bronce le atravesará.  
20:25 La saeta le traspasará y saldrá de su cuerpo,  
Y la punta relumbrante saldrá por su hiel;  
Sobre él vendrán terrores.  
20:26 Todas las tinieblas están reservadas para sus tesoros;  
Fuego no atizado los consumirá;  
Devorará lo que quede en su tienda.  
20:27 Los cielos descubrirán su iniquidad,  
Y la tierra se levantará contra él.  
20:28 Los renuevos de su casa serán transportados;  
Serán esparcidos en el día de su furor.  
20:29 Esta es la porción que Dios prepara al hombre impío,  
Y la heredad que Dios le señala por su palabra.  
\section*{Capítulo 21}
Job afirma que los malos prosperan  

21:1 Entonces respondió Job, y dijo:  
21:2 Oíd atentamente mi palabra,  
Y sea esto el consuelo que me deis.  
21:3 Toleradme, y yo hablaré;  
Y después que haya hablado, escarneced.  
21:4 ¿Acaso me quejo yo de algún hombre?  
¿Y por qué no se ha de angustiar mi espíritu?  
21:5 Miradme, y espantaos,  
Y poned la mano sobre la boca.  
21:6 Aun yo mismo, cuando me acuerdo, me asombro,  
Y el temblor estremece mi carne.  
21:7 ¿Por qué viven los impíos,  
Y se envejecen, y aun crecen en riquezas?  
21:8 Su descendencia se robustece a su vista,  
Y sus renuevos están delante de sus ojos.  
21:9 Sus casas están a salvo de temor,  
Ni viene azote de Dios sobre ellos.  
21:10 Sus toros engendran, y no fallan;  
Paren sus vacas, y no malogran su cría.  
21:11 Salen sus pequeñuelos como manada,  
Y sus hijos andan saltando.  
21:12 Al son de tamboril y de cítara saltan,  
Y se regocijan al son de la flauta.  
21:13 Pasan sus días en prosperidad,  
Y en paz descienden al Seol.  
21:14 Dicen, pues, a Dios: Apártate de nosotros,  
Porque no queremos el conocimiento de tus caminos.  
21:15 ¿Quién es el Todopoderoso, para que le sirvamos?  
¿Y de qué nos aprovechará que oremos a él?  
21:16 He aquí que su bien no está en mano de ellos;  
El consejo de los impíos lejos esté de mí.  
21:17 ¡Oh, cuántas veces la lámpara de los impíos es apagada,  
Y viene sobre ellos su quebranto,  
Y Dios en su ira les reparte dolores!  
21:18 Serán como la paja delante del viento,  
Y como el tamo que arrebata el torbellino.  
21:19 Dios guardará para los hijos de ellos su violencia;  
Le dará su pago, para que conozca.  
21:20 Verán sus ojos su quebranto,  
Y beberá de la ira del Todopoderoso.  
21:21 Porque ¿qué deleite tendrá él de su casa después de sí,  
Siendo cortado el número de sus meses?  
21:22 ¿Enseñará alguien a Dios sabiduría,  
Juzgando él a los que están elevados?  
21:23 Este morirá en el vigor de su hermosura, todo quieto y pacífico;  
21:24 Sus vasijas estarán llenas de leche,  
Y sus huesos serán regados de tuétano.  
21:25 Y este otro morirá en amargura de ánimo,  
Y sin haber comido jamás con gusto.  
21:26 Igualmente yacerán ellos en el polvo,  
Y gusanos los cubrirán.  
21:27 He aquí, yo conozco vuestros pensamientos,  
Y las imaginaciones que contra mí forjáis.  
21:28 Porque decís: ¿Qué hay de la casa del príncipe,  
Y qué de la tienda de las moradas de los impíos?  
21:29 ¿No habéis preguntado a los que pasan por los caminos,  
Y no habéis conocido su respuesta,  
21:30 Que el malo es preservado en el día de la destrucción?  
Guardado será en el día de la ira.  
21:31 ¿Quién le denunciará en su cara su camino?  
Y de lo que él hizo, ¿quién le dará el pago?  
21:32 Porque llevado será a los sepulcros,  
Y sobre su túmulo estarán velando.  
21:33 Los terrones del valle le serán dulces;  
Tras de él será llevado todo hombre,  
Y antes de él han ido innumerables.  
21:34 ¿Cómo, pues, me consoláis en vano,  
Viniendo a parar vuestras respuestas en falacia? 

\section*{Capítulo 22 }
Elifaz acusa a Job de gran maldad  

22:1 Respondió Elifaz temanita, y dijo:  
22:2 ¿Traerá el hombre provecho a Dios?  
Al contrario, para sí mismo es provechoso el hombre sabio.  
22:3 ¿Tiene contentamiento el Omnipotente en que tú seas justificado,  
O provecho de que tú hagas perfectos tus caminos? 
22:4 ¿Acaso te castiga,  
O viene a juicio contigo, a causa de tu piedad?  
22:5 Por cierto tu malicia es grande,  
Y tus maldades no tienen fin.  
22:6 Porque sacaste prenda a tus hermanos sin causa,  
Y despojaste de sus ropas a los desnudos.  
22:7 No diste de beber agua al cansado,  
Y detuviste el pan al hambriento.  
22:8 Pero el hombre pudiente tuvo la tierra,  
Y habitó en ella el distinguido.  
22:9 A las viudas enviaste vacías,  
Y los brazos de los huérfanos fueron quebrados.  
22:10 Por tanto, hay lazos alrededor de ti,  
Y te turba espanto repentino;  
22:11 O tinieblas, para que no veas,  
Y abundancia de agua te cubre.  
22:12 ¿No está Dios en la altura de los cielos?  
Mira lo encumbrado de las estrellas, cuán elevadas están.  
22:13 ¿Y dirás tú: ¿Qué sabe Dios?  
¿Cómo juzgará a través de la oscuridad?  
22:14 Las nubes le rodearon, y no ve;  
Y por el circuito del cielo se pasea.  
22:15 ¿Quieres tú seguir la senda antigua  
Que pisaron los hombres perversos,  
22:16 Los cuales fueron cortados antes de tiempo,  
Cuyo fundamento fue como un río derramado?  
22:17 Decían a Dios: Apártate de nosotros.  
¿Y qué les había hecho el Omnipotente?  
22:18 Les había colmado de bienes sus casas.  
Pero sea el consejo de ellos lejos de mí.  
22:19 Verán los justos y se gozarán;  
Y el inocente los escarnecerá, diciendo:  
22:20 Fueron destruidos nuestros adversarios,  
Y el fuego consumió lo que de ellos quedó.  
22:21 Vuelve ahora en amistad con él, y tendrás paz;  
Y por ello te vendrá bien.  
22:22 Toma ahora la ley de su boca,  
Y pon sus palabras en tu corazón.  
22:23 Si te volvieres al Omnipotente, serás edificado;  
Alejarás de tu tienda la aflicción;  
22:24 Tendrás más oro que tierra,  
Y como piedras de arroyos oro de Ofir;  
22:25 El Todopoderoso será tu defensa,  
Y tendrás plata en abundancia.  
22:26 Porque entonces te deleitarás en el Omnipotente,  
Y alzarás a Dios tu rostro.  
22:27 Orarás a él, y él te oirá;  
Y tú pagarás tus votos.  
22:28 Determinarás asimismo una cosa, y te será firme,  
Y sobre tus caminos resplandecerá luz.  
22:29 Cuando fueren abatidos, dirás tú: Enaltecimiento habrá;  
Y Dios salvará al humilde de ojos.  
22:30 El libertará al inocente,  
Y por la limpieza de tus manos éste será librado.  
\section*{Capítulo 23}
Job desea abogar su causa delante de Dios  

23:1 Respondió Job, y dijo:  
23:2 Hoy también hablaré con amargura;  
Porque es más grave mi llaga que mi gemido.  
23:3 ¡Quién me diera el saber dónde hallar a Dios!  
Yo iría hasta su silla.  
23:4 Expondría mi causa delante de él,  
Y llenaría mi boca de argumentos.  
23:5 Yo sabría lo que él me respondiese,  
Y entendería lo que me dijera.  
23:6 ¿Contendería conmigo con grandeza de fuerza? 
No; antes él me atendería.  
23:7 Allí el justo razonaría con él;  
Y yo escaparía para siempre de mi juez.  
23:8 He aquí yo iré al oriente, y no lo hallaré;  
Y al occidente, y no lo percibiré;  
23:9 Si muestra su poder al norte, yo no lo veré;  
Al sur se esconderá, y no lo veré.  
23:10 Mas él conoce mi camino;  
Me probará, y saldré como oro.  
23:11 Mis pies han seguido sus pisadas;  
Guardé su camino, y no me aparté.  
23:12 Del mandamiento de sus labios nunca me separé;  
Guardé las palabras de su boca más que mi comida.  
23:13 Pero si él determina una cosa, ¿quién lo hará cambiar?  
Su alma deseó, e hizo.  
23:14 El, pues, acabará lo que ha determinado de mí;  
Y muchas cosas como estas hay en él.  
23:15 Por lo cual yo me espanto en su presencia;  
Cuando lo considero, tiemblo a causa de él.  
23:16 Dios ha enervado mi corazón, 
Y me ha turbado el Omnipotente.  
23:17 ¿Por qué no fui yo cortado delante de las tinieblas,  
Ni fue cubierto con oscuridad mi rostro?  
\section*{Capítulo 24}
Job se queja de que Dios es indiferente ante la maldad  

24:1 Puesto que no son ocultos los tiempos al Todopoderoso,  
¿Por qué los que le conocen no ven sus días?  
24:2 Traspasan los linderos,  
Roban los ganados, y los apacientan.  
24:3 Se llevan el asno de los huérfanos,  
Y toman en prenda el buey de la viuda.  
24:4 Hacen apartar del camino a los menesterosos,  
Y todos los pobres de la tierra se esconden.  
24:5 He aquí, como asnos monteses en el desierto,  
Salen a su obra madrugando para robar;  
El desierto es mantenimiento de sus hijos.  
24:6 En el campo siegan su pasto,  
Y los impíos vendimian la viña ajena.  
24:7 Al desnudo hacen dormir sin ropa,  
Sin tener cobertura contra el frío.  
24:8 Con las lluvias de los montes se mojan,  
Y abrazan las peñas por falta de abrigo.  
24:9 Quitan el pecho a los huérfanos,  
Y de sobre el pobre toman la prenda.  
24:10 Al desnudo hacen andar sin vestido,  
Y a los hambrientos quitan las gavillas.  
24:11 Dentro de sus paredes exprimen el aceite,  
Pisan los lagares, y mueren de sed.  
24:12 Desde la ciudad gimen los moribundos,  
Y claman las almas de los heridos de muerte,  
Pero Dios no atiende su oración.  
24:13 Ellos son los que, rebeldes a la luz,  
Nunca conocieron sus caminos,  
Ni estuvieron en sus veredas.  
24:14 A la luz se levanta el matador; mata al pobre y al necesitado, 
Y de noche es como ladrón.  
24:15 El ojo del adúltero está aguardando la noche,  
Diciendo: No me verá nadie;  
Y esconde su rostro.  
24:16 En las tinieblas minan las casas  
Que de día para sí señalaron;  
No conocen la luz.  
24:17 Porque la mañana es para todos ellos como sombra de muerte;  
Si son conocidos, terrores de sombra de muerte los toman.  
24:18 Huyen ligeros como corriente de aguas;  
Su porción es maldita en la tierra;  
No andarán por el camino de las viñas.  
24:19 La sequía y el calor arrebatan las aguas de la nieve; 
Así también el Seol a los pecadores.  
24:20 Los olvidará el seno materno; de ellos sentirán los gusanos dulzura;  
Nunca más habrá de ellos memoria,  
Y como un árbol los impíos serán quebrantados.  
24:21 A la mujer estéril, que no concebía, afligió,  
Y a la viuda nunca hizo bien.  
24:22 Pero a los fuertes adelantó con su poder;  
Una vez que se levante, ninguno está seguro de la vida.  
24:23 El les da seguridad y confianza;  
Sus ojos están sobre los caminos de ellos.  
24:24 Fueron exaltados un poco, mas desaparecen,  
Y son abatidos como todos los demás;  
Serán encerrados, y cortados como cabezas de espigas.  
24:25 Y si no, ¿quién me desmentirá ahora,  
O reducirá a nada mis palabras?  
\section*{Capítulo 25}
Bildad niega que el hombre pueda ser justificado delante de Dios  

25:1 Respondió Bildad suhita, y dijo:  
25:2 El señorío y el temor están con él;  
El hace paz en sus alturas.  
25:3 ¿Tienen sus ejércitos número?  
¿Sobre quién no está su luz?  
25:4 ¿Cómo, pues, se justificará el hombre para con Dios?  
¿Y cómo será limpio el que nace de mujer?  
25:5 He aquí que ni aun la misma luna será resplandeciente,  
Ni las estrellas son limpias delante de sus ojos; 
25:6 ¿Cuánto menos el hombre, que es un gusano,  
Y el hijo de hombre, también gusano?  
\section*{Capítulo 26 }
Job proclama la soberanía de Dios  

26:1 Respondió Job, y dijo:  
26:2 ¿En qué ayudaste al que no tiene poder?  
¿Cómo has amparado al brazo sin fuerza?  
26:3 ¿En qué aconsejaste al que no tiene ciencia,  
Y qué plenitud de inteligencia has dado a conocer?  
26:4 ¿A quién has anunciado palabras,  
Y de quién es el espíritu que de ti procede?  
26:5 Las sombras tiemblan en lo profundo,  
Los mares y cuanto en ellos mora.  
26:6 El Seol está descubierto delante de él, y el Abadón no tiene cobertura.  
26:7 El extiende el norte sobre vacío,  
Cuelga la tierra sobre nada.  
26:8 Ata las aguas en sus nubes,  
Y las nubes no se rompen debajo de ellas.  
26:9 El encubre la faz de su trono,  
Y sobre él extiende su nube.  
26:10 Puso límite a la superficie de las aguas,  
Hasta el fin de la luz y las tinieblas.  
26:11 Las columnas del cielo tiemblan,  
Y se espantan a su reprensión.  
26:12 El agita el mar con su poder,  
Y con su entendimiento hiere la arrogancia suya.  
26:13 Su espíritu adornó los cielos;  
Su mano creó la serpiente tortuosa.  
26:14 He aquí, estas cosas son sólo los bordes de sus caminos;  
¡Y cuán leve es el susurro que hemos oído de él!  
Pero el trueno de su poder, ¿quién lo puede comprender?  
\section*{Capítulo 27 }
Job describe el castigo de los malos  

27:1 Reasumió Job su discurso, y dijo:  
27:2 Vive Dios, que ha quitado mi derecho,  
Y el Omnipotente, que amargó el alma mía,  
27:3 Que todo el tiempo que mi alma esté en mí,  
Y haya hálito de Dios en mis narices,  
27:4 Mis labios no hablarán iniquidad,  
Ni mi lengua pronunciará engaño.  
27:5 Nunca tal acontezca que yo os justifique;  
Hasta que muera, no quitaré de mí mi integridad.  
27:6 Mi justicia tengo asida, y no la cederé;  
No me reprochará mi corazón en todos mis días.  
27:7 Sea como el impío mi enemigo,  
Y como el inicuo mi adversario.  
27:8 Porque ¿cuál es la esperanza del impío, por mucho que hubiere robado,  
Cuando Dios le quitare la vida?  
27:9 ¿Oirá Dios su clamor  
Cuando la tribulación viniere sobre él?  
27:10 ¿Se deleitará en el Omnipotente?  
¿Invocará a Dios en todo tiempo?  
27:11 Yo os enseñaré en cuanto a la mano de Dios;  
No esconderé lo que hay para con el Omnipotente.  
27:12 He aquí que todos vosotros lo habéis visto;  
¿Por qué, pues, os habéis hecho tan enteramente vanos?  
27:13 Esta es para con Dios la porción del hombre impío,  
Y la herencia que los violentos han de recibir del Omnipotente:  
27:14 Si sus hijos fueren multiplicados, serán para la espada;  
Y sus pequeños no se saciarán de pan.  
27:15 Los que de él quedaren, en muerte serán sepultados,  
Y no los llorarán sus viudas.  
27:16 Aunque amontone plata como polvo,  
Y prepare ropa como lodo;  
27:17 La habrá preparado él, mas el justo se vestirá,  
Y el inocente repartirá la plata.  
27:18 Edificó su casa como la polilla,  
Y como enramada que hizo el guarda.  
27:19 Rico se acuesta, pero por última vez;  
Abrirá sus ojos, y nada tendrá.  
27:20 Se apoderarán de él terrores como aguas;  
Torbellino lo arrebatará de noche.  
27:21 Le eleva el solano, y se va;  
Y tempestad lo arrebatará de su lugar.  
27:22 Dios, pues, descargará sobre él, y no perdonará;  
Hará él por huir de su mano.  
27:23 Batirán las manos sobre él,  
Y desde su lugar le silbarán.  
\section*{Capítulo 28} 
El hombre en busca de la sabiduría  

28:1 Ciertamente la plata tiene sus veneros,  
Y el oro lugar donde se refina.  
28:2 El hierro se saca del polvo,  
Y de la piedra se funde el cobre.  
28:3 A las tinieblas ponen término,  
Y examinan todo a la perfección,  
Las piedras que hay en oscuridad y en sombra de muerte. 
28:4 Abren minas lejos de lo habitado,  
En lugares olvidados, donde el pie no pasa.  
Son suspendidos y balanceados, lejos de los demás hombres.  
28:5 De la tierra nace el pan,  
Y debajo de ella está como convertida en fuego.  
28:6 Lugar hay cuyas piedras son zafiro,  
Y sus polvos de oro.  
28:7 Senda que nunca la conoció ave,  
Ni ojo de buitre la vio;  
28:8 Nunca la pisaron animales fieros,  
Ni león pasó por ella.  
28:9 En el pedernal puso su mano,  
Y trastornó de raíz los montes.  
28:10 De los peñascos cortó ríos,  
Y sus ojos vieron todo lo preciado.  
28:11 Detuvo los ríos en su nacimiento,  
E hizo salir a luz lo escondido.  
28:12 Mas ¿dónde se hallará la sabiduría?  
¿Dónde está el lugar de la inteligencia?  
28:13 No conoce su valor el hombre,  
Ni se halla en la tierra de los vivientes.  
28:14 El abismo dice: No está en mí;  
Y el mar dijo: Ni conmigo.  
28:15 No se dará por oro,  
Ni su precio será a peso de plata.  
28:16 No puede ser apreciada con oro de Ofir,  
Ni con ónice precioso, ni con zafiro.  
28:17 El oro no se le igualará, ni el diamante,  
Ni se cambiará por alhajas de oro fino.  
28:18 No se hará mención de coral ni de perlas;  
La sabiduría es mejor que las piedras preciosas.  
28:19 No se igualará con ella topacio de Etiopía;  
No se podrá apreciar con oro fino.  
28:20 ¿De dónde, pues, vendrá la sabiduría?  
¿Y dónde está el lugar de la inteligencia?  
28:21 Porque encubierta está a los ojos de todo viviente,  
Y a toda ave del cielo es oculta.  
28:22 El Abadón y la muerte dijeron:  
Su fama hemos oído con nuestros oídos.  
28:23 Dios entiende el camino de ella,  
Y conoce su lugar.  
28:24 Porque él mira hasta los fines de la tierra,  
Y ve cuanto hay bajo los cielos.  
28:25 Al dar peso al viento,  
Y poner las aguas por medida;  
28:26 Cuando él dio ley a la lluvia,  
Y camino al relámpago de los truenos,  
28:27 Entonces la veía él, y la manifestaba;  
La preparó y la descubrió también.  
28:28 Y dijo al hombre:  
He aquí que el temor del Señor es la sabiduría, 
Y el apartarse del mal, la inteligencia.  
\section*{Capítulo 29}
Job recuerda su felicidad anterior  

29:1 Volvió Job a reanudar su discurso, y dijo:  
29:2 ¡Quién me volviese como en los meses pasados,  
Como en los días en que Dios me guardaba,  
29:3 Cuando hacía resplandecer sobre mi cabeza su lámpara,  
A cuya luz yo caminaba en la oscuridad;  
29:4 Como fui en los días de mi juventud,  
Cuando el favor de Dios velaba sobre mi tienda; 
29:5 Cuando aún estaba conmigo el Omnipotente,  
Y mis hijos alrededor de mí;  
29:6 Cuando lavaba yo mis pasos con leche,  
Y la piedra me derramaba ríos de aceite!  
29:7 Cuando yo salía a la puerta a juicio,  
Y en la plaza hacía preparar mi asiento,  
29:8 Los jóvenes me veían, y se escondían;  
Y los ancianos se levantaban, y estaban de pie.  
29:9 Los príncipes detenían sus palabras;  
Ponían la mano sobre su boca. 
29:10 La voz de los principales se apagaba,  
Y su lengua se pegaba a su paladar.  
29:11 Los oídos que me oían me llamaban bienaventurado,  
Y los ojos que me veían me daban testimonio,  
29:12 Porque yo libraba al pobre que clamaba,  
Y al huérfano que carecía de ayudador.  
29:13 La bendición del que se iba a perder venía sobre mí,  
Y al corazón de la viuda yo daba alegría.  
29:14 Me vestía de justicia, y ella me cubría;  
Como manto y diadema era mi rectitud. 
29:15 Yo era ojos al ciego,  
Y pies al cojo.  
29:16 A los menesterosos era padre,  
Y de la causa que no entendía, me informaba con diligencia;  
29:17 Y quebrantaba los colmillos del inicuo,  
Y de sus dientes hacía soltar la presa.  
29:18 Decía yo: En mi nido moriré,  
Y como arena multiplicaré mis días.  
29:19 Mi raíz estaba abierta junto a las aguas,  
Y en mis ramas permanecía el rocío.  
29:20 Mi honra se renovaba en mí,  
Y mi arco se fortalecía en mi mano.  
29:21 Me oían, y esperaban,  
Y callaban a mi consejo.  
29:22 Tras mi palabra no replicaban, 
Y mi razón destilaba sobre ellos.  
29:23 Me esperaban como a la lluvia,  
Y abrían su boca como a la lluvia tardía.  
29:24 Si me reía con ellos, no lo creían;  
Y no abatían la luz de mi rostro.  
29:25 Calificaba yo el camino de ellos, y me sentaba entre ellos como el jefe;  
Y moraba como rey en el ejército,  
Como el que consuela a los que lloran.  
\section*{Capítulo 30}
Job lamenta su desdicha actual  

30:1 Pero ahora se ríen de mí los más jóvenes que yo,  
A cuyos padres yo desdeñara poner con los perros de mi ganado.  
30:2 ¿Y de qué me serviría ni aun la fuerza de sus manos?  
No tienen fuerza alguna.  
30:3 Por causa de la pobreza y del hambre andaban solos;  
Huían a la soledad, a lugar tenebroso, asolado y desierto.  
30:4 Recogían malvas entre los arbustos,  
Y raíces de enebro para calentarse.  
30:5 Eran arrojados de entre las gentes,  
Y todos les daban grita como tras el ladrón.  
30:6 Habitaban en las barrancas de los arroyos,  
En las cavernas de la tierra, y en las rocas.  
30:7 Bramaban entre las matas,  
Y se reunían debajo de los espinos.  
30:8 Hijos de viles, y hombres sin nombre,  
Más bajos que la misma tierra.  
30:9 Y ahora yo soy objeto de su burla,  
Y les sirvo de refrán.  
30:10 Me abominan, se alejan de mí,  
Y aun de mi rostro no detuvieron su saliva.  
30:11 Porque Dios desató su cuerda, y me afligió, 
Por eso se desenfrenaron delante de mi rostro.  
30:12 A la mano derecha se levantó el populacho;  
Empujaron mis pies,  
Y prepararon contra mí caminos de perdición.  
30:13 Mi senda desbarataron,  
Se aprovecharon de mi quebrantamiento,  
Y contra ellos no hubo ayudador.  
30:14 Vinieron como por portillo ancho,  
Se revolvieron sobre mi calamidad.  
30:15 Se han revuelto turbaciones sobre mí;  
Combatieron como viento mi honor,  
Y mi prosperidad pasó como nube.  
30:16 Y ahora mi alma está derramada en mí;  
Días de aflicción se apoderan de mí.  
30:17 La noche taladra mis huesos,  
Y los dolores que me roen no reposan.  
30:18 La violencia deforma mi vestidura; me ciñe como el cuello de mi túnica.  
30:19 El me derribó en el lodo,  
Y soy semejante al polvo y a la ceniza.  
30:20 Clamo a ti, y no me oyes;  
Me presento, y no me atiendes.  
30:21 Te has vuelto cruel para mí;  
Con el poder de tu mano me persigues.  
30:22 Me alzaste sobre el viento, me hiciste cabalgar en él,  
Y disolviste mi sustancia.  
30:23 Porque yo sé que me conduces a la muerte,  
Y a la casa determinada a todo viviente.  
30:24 Mas él no extenderá la mano contra el sepulcro;  
¿Clamarán los sepultados cuando él los quebrantare?  
30:25 ¿No lloré yo al afligido?  
Y mi alma, ¿no se entristeció sobre el menesteroso?  
30:26 Cuando esperaba yo el bien, entonces vino el mal;  
Y cuando esperaba luz, vino la oscuridad.  
30:27 Mis entrañas se agitan, y no reposan;  
Días de aflicción me han sobrecogido.  
30:28 Ando ennegrecido, y no por el sol;  
Me he levantado en la congregación, y clamado.  
30:29 He venido a ser hermano de chacales,  
Y compañero de avestruces.  
30:30 Mi piel se ha ennegrecido y se me cae,  
Y mis huesos arden de calor.  
30:31 Se ha cambiado mi arpa en luto,  
Y mi flauta en voz de lamentadores.  
\section*{Capítulo 31}
Job afirma su integridad  

31:1 Hice pacto con mis ojos;  
¿Cómo, pues, había yo de mirar a una virgen?  
31:2 Porque ¿qué galardón me daría de arriba Dios,  
Y qué heredad el Omnipotente desde las alturas?  
31:3 ¿No hay quebrantamiento para el impío,  
Y extrañamiento para los que hacen iniquidad?  
31:4 ¿No ve él mis caminos,  
Y cuenta todos mis pasos?  
31:5 Si anduve con mentira,  
Y si mi pie se apresuró a engaño,  
31:6 Péseme Dios en balanzas de justicia,  
Y conocerá mi integridad.  
31:7 Si mis pasos se apartaron del camino,  
Si mi corazón se fue tras mis ojos,  
Y si algo se pegó a mis manos,  
31:8 Siembre yo, y otro coma,  
Y sea arrancada mi siembra.  
31:9 Si fue mi corazón engañado acerca de mujer,  
Y si estuve acechando a la puerta de mi prójimo, 
31:10 Muela para otro mi mujer,  
Y sobre ella otros se encorven.  
31:11 Porque es maldad e iniquidad  
Que han de castigar los jueces.  
31:12 Porque es fuego que devoraría hasta el Abadón,  
Y consumiría toda mi hacienda.  
31:13 Si hubiera tenido en poco el derecho de mi siervo y de mi sierva,  
Cuando ellos contendían conmigo,  
31:14 ¿Qué haría yo cuando Dios se levantase?  
Y cuando él preguntara, ¿qué le respondería yo?  
31:15 El que en el vientre me hizo a mí, ¿no lo hizo a él?  
¿Y no nos dispuso uno mismo en la matriz?  
31:16 Si estorbé el contento de los pobres,  
E hice desfallecer los ojos de la viuda;  
31:17 Si comí mi bocado solo,  
Y no comió de él el huérfano  
31:18 (Porque desde mi juventud creció conmigo como con un padre,  
Y desde el vientre de mi madre fui guía de la viuda);  
31:19 Si he visto que pereciera alguno sin vestido,  
Y al menesteroso sin abrigo;  
31:20 Si no me bendijeron sus lomos,  
Y del vellón de mis ovejas se calentaron;  
31:21 Si alcé contra el huérfano mi mano,  
Aunque viese que me ayudaran en la puerta;  
31:22 Mi espalda se caiga de mi hombro,  
Y el hueso de mi brazo sea quebrado.  
31:23 Porque temí el castigo de Dios,  
Contra cuya majestad yo no tendría poder.  
31:24 Si puse en el oro mi esperanza,  
Y dije al oro: Mi confianza eres tú;  
31:25 Si me alegré de que mis riquezas se multiplicasen,  
Y de que mi mano hallase mucho;  
31:26 Si he mirado al sol cuando resplandecía,  
O a la luna cuando iba hermosa,  
31:27 Y mi corazón se engañó en secreto,  
Y mi boca besó mi mano;  
31:28 Esto también sería maldad juzgada;  
Porque habría negado al Dios soberano.  
31:29 Si me alegré en el quebrantamiento del que me aborrecía,  
Y me regocijé cuando le halló el mal  
31:30 (Ni aun entregué al pecado mi lengua,  
Pidiendo maldición para su alma);  
31:31 Si mis siervos no decían:  
¿Quién no se ha saciado de su carne?  
31:32 (El forastero no pasaba fuera la noche;  
Mis puertas abría al caminante);  
31:33 Si encubrí como hombre mis transgresiones,  
Escondiendo en mi seno mi iniquidad,  
31:34 Porque tuve temor de la gran multitud,  
Y el menosprecio de las familias me atemorizó,  
Y callé, y no salí de mi puerta;  
31:35 ¡Quién me diera quien me oyese!  
He aquí mi confianza es que el Omnipotente testificará por mí,  
Aunque mi adversario me forme proceso.  
31:36 Ciertamente yo lo llevaría sobre mi hombro,  
Y me lo ceñiría como una corona.  
31:37 Yo le contaría el número de mis pasos,  
Y como príncipe me presentaría ante él.  
31:38 Si mi tierra clama contra mí,  
Y lloran todos sus surcos;  
31:39 Si comí su sustancia sin dinero,  
O afligí el alma de sus dueños,  
31:40 En lugar de trigo me nazcan abrojos,  
Y espinos en lugar de cebada.  
Aquí terminan las palabras de Job.  
\section*{Capítulo 32 }
Eliú justifica su derecho de contestar a Job  

32:1 Cesaron estos tres varones de responder a Job, por cuanto él era justo a sus propios ojos.  
32:2 Entonces Eliú hijo de Baraquel buzita, de la familia de Ram, se encendió en ira contra Job; se encendió en ira, por cuanto se justificaba a sí mismo más que a Dios.  
32:3 Asimismo se encendió en ira contra sus tres amigos, porque no hallaban qué responder, aunque habían condenado a Job.  
32:4 Y Eliú había esperado a Job en la disputa, porque los otros eran más viejos que él.  
32:5 Pero viendo Eliú que no había respuesta en la boca de aquellos tres varones, se encendió en ira.  
32:6 Y respondió Eliú hijo de Baraquel buzita, y dijo:  
Yo soy joven, y vosotros ancianos;  
Por tanto, he tenido miedo, y he temido declararos mi opinión.  
32:7 Yo decía: Los días hablarán,  
Y la muchedumbre de años declarará sabiduría.  
32:8 Ciertamente espíritu hay en el hombre,  
Y el soplo del Omnipotente le hace que entienda.  
32:9 No son los sabios los de mucha edad,  
Ni los ancianos entienden el derecho.  
32:10 Por tanto, yo dije: Escuchadme;  
Declararé yo también mi sabiduría.  
32:11 He aquí yo he esperado a vuestras razones,  
He escuchado vuestros argumentos,  
En tanto que buscabais palabras.  
32:12 Os he prestado atención,  
Y he aquí que no hay de vosotros quien redarguya a Job,  
Y responda a sus razones.  
32:13 Para que no digáis: Nosotros hemos hallado sabiduría;  
Lo vence Dios, no el hombre.  
32:14 Ahora bien, Job no dirigió contra mí sus palabras,  
Ni yo le responderé con vuestras razones.  
32:15 Se espantaron, no respondieron más;  
Se les fueron los razonamientos.  
32:16 Yo, pues, he esperado, pero no hablaban;  
Más bien callaron y no respondieron más.  
32:17 Por eso yo también responderé mi parte;  
También yo declararé mi juicio. 
32:18 Porque lleno estoy de palabras, 
Y me apremia el espíritu dentro de mí.  
32:19 De cierto mi corazón está como el vino que no tiene respiradero,  
Y se rompe como odres nuevos.  
32:20 Hablaré, pues, y respiraré;  
Abriré mis labios, y responderé.  
32:21 No haré ahora acepción de personas,  
Ni usaré con nadie de títulos lisonjeros.  
32:22 Porque no sé hablar lisonjas;  
De otra manera, en breve mi Hacedor me consumiría.  
\section*{Capítulo 33 }
Eliú censura a Job  

33:1 Por tanto, Job, oye ahora mis razones,  
Y escucha todas mis palabras.  
33:2 He aquí yo abriré ahora mi boca,  
Y mi lengua hablará en mi garganta.  
33:3 Mis razones declararán la rectitud de mi corazón,  
Y lo que saben mis labios, lo hablarán con sinceridad.  
33:4 El espíritu de Dios me hizo,  
Y el soplo del Omnipotente me dio vida.  
33:5 Respóndeme si puedes;  
Ordena tus palabras, ponte en pie.  
33:6 Heme aquí a mí en lugar de Dios, conforme a tu dicho;  
De barro fui yo también formado.  
33:7 He aquí, mi terror no te espantará,  
Ni mi mano se agravará sobre ti.  
33:8 De cierto tú dijiste a oídos míos,  
Y yo oí la voz de tus palabras que decían: 
33:9 Yo soy limpio y sin defecto;  
Soy inocente, y no hay maldad en mí. 
33:10 He aquí que él buscó reproches contra mí,  
Y me tiene por su enemigo;  
33:11 Puso mis pies en el cepo,  
Y vigiló todas mis sendas.  
33:12 He aquí, en esto no has hablado justamente;  
Yo te responderé que mayor es Dios que el hombre.  
33:13 ¿Por qué contiendes contra él?  
Porque él no da cuenta de ninguna de sus razones.  
33:14 Sin embargo, en una o en dos maneras habla Dios;  
Pero el hombre no entiende.  
33:15 Por sueño, en visión nocturna,  
Cuando el sueño cae sobre los hombres, 
Cuando se adormecen sobre el lecho, 
33:16 Entonces revela al oído de los hombres,  
Y les señala su consejo,  
33:17 Para quitar al hombre de su obra,  
Y apartar del varón la soberbia.  
33:18 Detendrá su alma del sepulcro,  
Y su vida de que perezca a espada.  
33:19 También sobre su cama es castigado  
Con dolor fuerte en todos sus huesos,  
33:20 Que le hace que su vida aborrezca el pan,  
Y su alma la comida suave.  
33:21 Su carne desfallece, de manera que no se ve,  
Y sus huesos, que antes no se veían, aparecen.  
33:22 Su alma se acerca al sepulcro,  
Y su vida a los que causan la muerte.  
33:23 Si tuviese cerca de él  
Algún elocuente mediador muy escogido,  
Que anuncie al hombre su deber;  
33:24 Que le diga que Dios tuvo de él misericordia,  
Que lo libró de descender al sepulcro,  
Que halló redención;  
33:25 Su carne será más tierna que la del niño,  
Volverá a los días de su juventud.  
33:26 Orará a Dios, y éste le amará,  
Y verá su faz con júbilo;  
Y restaurará al hombre su justicia.  
33:27 El mira sobre los hombres; y al que dijere:  
Pequé, y pervertí lo recto,  
Y no me ha aprovechado,  
33:28 Dios redimirá su alma para que no pase al sepulcro,  
Y su vida se verá en luz.  
33:29 He aquí, todas estas cosas hace Dios  
Dos y tres veces con el hombre,  
33:30 Para apartar su alma del sepulcro,  
Y para iluminarlo con la luz de los vivientes.  
33:31 Escucha, Job, y óyeme;  
Calla, y yo hablaré.  
33:32 Si tienes razones, respóndeme;  
Habla, porque yo te quiero justificar.  
33:33 Y si no, óyeme tú a mí;  
Calla, y te enseñaré sabiduría. 
\section*{Capítulo 34 }
Eliú justifica a Dios  

34:1 Además Eliú dijo:  
34:2 Oíd, sabios, mis palabras;  
Y vosotros, doctos, estadme atentos.  
34:3 Porque el oído prueba las palabras,  
Como el paladar gusta lo que uno come.  
34:4 Escojamos para nosotros el juicio,  
Conozcamos entre nosotros cuál sea lo bueno.  
34:5 Porque Job ha dicho: Yo soy justo,  
Y Dios me ha quitado mi derecho.  
34:6 ¿He de mentir yo contra mi razón?  
Dolorosa es mi herida sin haber hecho yo transgresión.  
34:7 ¿Qué hombre hay como Job,  
Que bebe el escarnio como agua,  
34:8 Y va en compañía con los que hacen iniquidad,  
Y anda con los hombres malos?  
34:9 Porque ha dicho: De nada servirá al hombre  
El conformar su voluntad a Dios.  
34:10 Por tanto, varones de inteligencia, oídme:  
Lejos esté de Dios la impiedad,  
Y del Omnipotente la iniquidad.  
34:11 Porque él pagará al hombre según su obra,  
Y le retribuirá conforme a su camino. 
34:12 Sí, por cierto, Dios no hará injusticia,  
Y el Omnipotente no pervertirá el derecho.  
34:13 ¿Quién visitó por él la tierra?  
¿Y quién puso en orden todo el mundo?  
34:14 Si él pusiese sobre el hombre su corazón,  
Y recogiese así su espíritu y su aliento,  
34:15 Toda carne perecería juntamente,  
Y el hombre volvería al polvo.  
34:16 Si, pues, hay en ti entendimiento, oye esto;  
Escucha la voz de mis palabras.  
34:17 ¿Gobernará el que aborrece juicio?  
¿Y condenarás tú al que es tan justo?  
34:18 ¿Se dirá al rey: Perverso;  
Y a los príncipes: Impíos?  
34:19 ¿Cuánto menos a aquel que no hace acepción de personas de príncipes.  
Ni respeta más al rico que al pobre,  
Porque todos son obra de sus manos?  
34:20 En un momento morirán,  
Y a medianoche se alborotarán los pueblos, y pasarán,  
Y sin mano será quitado el poderoso.  
34:21 Porque sus ojos están sobre los caminos del hombre,  
Y ve todos sus pasos.  
34:22 No hay tinieblas ni sombra de muerte  
Donde se escondan los que hacen maldad.  
34:23 No carga, pues, él al hombre más de lo justo,  
Para que vaya con Dios a juicio. 
34:24 El quebrantará a los fuertes sin indagación,  
Y hará estar a otros en su lugar.  
34:25 Por tanto, él hará notorias las obras de ellos,  
Cuando los trastorne en la noche, y sean quebrantados.  
34:26 Como a malos los herirá  
En lugar donde sean vistos;  
34:27 Por cuanto así se apartaron de él,  
Y no consideraron ninguno de sus caminos,  
34:28 Haciendo venir delante de él el clamor del pobre,  
Y que oiga el clamor de los necesitados.  
34:29 Si él diere reposo, ¿quién inquietará?  
Si escondiere el rostro, ¿quién lo mirará?  
Esto sobre una nación, y lo mismo sobre un hombre;  
34:30 Haciendo que no reine el hombre impío  
Para vejaciones del pueblo.  
34:31 De seguro conviene que se diga a Dios:  
He llevado ya castigo, no ofenderé ya más;  
34:32 Enséñame tú lo que yo no veo;  
Si hice mal, no lo haré más.  
34:33 ¿Ha de ser eso según tu parecer?  
El te retribuirá, ora rehúses, ora aceptes, y no yo;  
Di, si no, lo que tú sabes.  
34:34 Los hombres inteligentes dirán conmigo,  
Y el hombre sabio que me oiga: 
34:35 Que Job no habla con sabiduría,  
Y que sus palabras no son con entendimiento.  
34:36 Deseo yo que Job sea probado ampliamente,  
A causa de sus respuestas semejantes a las de los hombres inicuos.  
34:37 Porque a su pecado añadió rebeldía;  
Bate palmas contra nosotros,  
Y contra Dios multiplica sus palabras.  
\section*{Capítulo 35 }

35:1 Prosiguió Eliú en su razonamiento, y dijo:  
35:2 ¿Piensas que es cosa recta lo que has dicho:  
Más justo soy yo que Dios?  
35:3 Porque dijiste: ¿Qué ventaja sacaré de ello?  
¿O qué provecho tendré de no haber pecado?  
35:4 Yo te responderé razones, 
Y a tus compañeros contigo.  
35:5 Mira a los cielos, y ve, 
Y considera que las nubes son más altas que tú. 
35:6 Si pecares, ¿qué habrás logrado contra él?  
Y si tus rebeliones se multiplicaren, ¿qué le harás tú?  
35:7 Si fueres justo, ¿qué le darás a él?  
¿O qué recibirá de tu mano?  
35:8 Al hombre como tú dañará tu impiedad,  
Y al hijo de hombre aprovechará tu justicia. 
35:9 A causa de la multitud de las violencias claman,  
Y se lamentan por el poderío de los grandes.  
35:10 Y ninguno dice: ¿Dónde está Dios mi Hacedor,  
Que da cánticos en la noche,  
35:11 Que nos enseña más que a las bestias de la tierra,  
Y nos hace sabios más que a las aves del cielo?  
35:12 Allí clamarán, y él no oirá,  
Por la soberbia de los malos.  
35:13 Ciertamente Dios no oirá la vanidad,  
Ni la mirará el Omnipotente.  
35:14 ¿Cuánto menos cuando dices que no haces caso de él?  
La causa está delante de él; por tanto, aguárdale.  
35:15 Mas ahora, porque en su ira no castiga,  
Ni inquiere con rigor,  
35:16 Por eso Job abre su boca vanamente,  
Y multiplica palabras sin sabiduría.  
\section*{Capítulo 36}
Eliú exalta la grandeza de Dios  

36:1 Añadió Eliú y dijo:  
36:2 Espérame un poco, y te enseñaré;  
Porque todavía tengo razones en defensa de Dios. 
36:3 Tomaré mi saber desde lejos,  
Y atribuiré justicia a mi Hacedor.  
36:4 Porque de cierto no son mentira mis palabras;  
Contigo está el que es íntegro en sus conceptos.  
36:5 He aquí que Dios es grande, pero no desestima a nadie;  
Es poderoso en fuerza de sabiduría.  
36:6 No otorgará vida al impío,  
Pero a los afligidos dará su derecho.  
36:7 No apartará de los justos sus ojos;  
Antes bien con los reyes los pondrá en trono para siempre,  
Y serán exaltados.  
36:8 Y si estuvieren prendidos en grillos,  
Y aprisionados en las cuerdas de aflicción,  
36:9 El les dará a conocer la obra de ellos,  
Y que prevalecieron sus rebeliones. 
36:10 Despierta además el oído de ellos para la corrección,  
Y les dice que se conviertan de la iniquidad.  
36:11 Si oyeren, y le sirvieren,  
Acabarán sus días en bienestar,  
Y sus años en dicha.  
36:12 Pero si no oyeren, serán pasados a espada,  
Y perecerán sin sabiduría.  
36:13 Mas los hipócritas de corazón atesoran para sí la ira,  
Y no clamarán cuando él los atare.  
36:14 Fallecerá el alma de ellos en su juventud,  
Y su vida entre los sodomitas.  
36:15 Al pobre librará de su pobreza,  
Y en la aflicción despertará su oído.  
36:16 Asimismo te apartará de la boca de la angustia  
A lugar espacioso, libre de todo apuro,  
Y te preparará mesa llena de grosura.  
36:17 Mas tú has llenado el juicio del impío,  
En vez de sustentar el juicio y la justicia.  
36:18 Por lo cual teme, no sea que en su ira te quite con golpe,  
El cual no puedas apartar de ti con gran rescate. 
36:19 ¿Hará él estima de tus riquezas, del oro,  
O de todas las fuerzas del poder?  
36:20 No anheles la noche,  
En que los pueblos desaparecen de su lugar.  
36:21 Guárdate, no te vuelvas a la iniquidad;  
Pues ésta escogiste más bien que la aflicción.  
36:22 He aquí que Dios es excelso en su poder;  
¿Qué enseñador semejante a él?  
36:23 ¿Quién le ha prescrito su camino?  
¿Y quién le dirá: Has hecho mal?  
36:24 Acuérdate de engrandecer su obra,  
La cual contemplan los hombres.  
36:25 Los hombres todos la ven;  
La mira el hombre de lejos.  
36:26 He aquí, Dios es grande, y nosotros no le conocemos,  
Ni se puede seguir la huella de sus años.  
36:27 El atrae las gotas de las aguas,  
Al transformarse el vapor en lluvia,  
36:28 La cual destilan las nubes,  
Goteando en abundancia sobre los hombres.  
36:29 ¿Quién podrá comprender la extensión de las nubes,  
Y el sonido estrepitoso de su morada?  
36:30 He aquí que sobre él extiende su luz,  
Y cobija con ella las profundidades del mar.  
36:31 Bien que por esos medios castiga a los pueblos,  
A la multitud él da sustento.  
36:32 Con las nubes encubre la luz,  
Y le manda no brillar, interponiendo aquéllas.  
36:33 El trueno declara su indignación,  
Y la tempestad proclama su ira contra la iniquidad.  
\section*{Capítulo 37 }

37:1 Por eso también se estremece mi corazón,  
Y salta de su lugar.  
37:2 Oíd atentamente el estrépito de su voz,  
Y el sonido que sale de su boca.  
37:3 Debajo de todos los cielos lo dirige,  
Y su luz hasta los fines de la tierra.  
37:4 Después de ella brama el sonido,  
Truena él con voz majestuosa;  
Y aunque sea oída su voz, no los detiene.  
37:5 Truena Dios maravillosamente con su voz;  
El hace grandes cosas, que nosotros no entendemos.  
37:6 Porque a la nieve dice: Desciende a la tierra;  
También a la llovizna, y a los aguaceros torrenciales.  
37:7 Así hace retirarse a todo hombre,  
Para que los hombres todos reconozcan su obra.  
37:8 Las bestias entran en su escondrijo,  
Y se están en sus moradas.  
37:9 Del sur viene el torbellino,  
Y el frío de los vientos del norte.  
37:10 Por el soplo de Dios se da el hielo,  
Y las anchas aguas se congelan.  
37:11 Regando también llega a disipar la densa nube,  
Y con su luz esparce la niebla.  
37:12 Asimismo por sus designios se revuelven las nubes en derredor,  
Para hacer sobre la faz del mundo,  
En la tierra, lo que él les mande.  
37:13 Unas veces por azote, otras por causa de su tierra,  
Otras por misericordia las hará venir.  
37:14 Escucha esto, Job;  
Detente, y considera las maravillas de Dios.  
37:15 ¿Sabes tú cómo Dios las pone en concierto,  
Y hace resplandecer la luz de su nube?  
37:16 ¿Has conocido tú las diferencias de las nubes,  
Las maravillas del Perfecto en sabiduría?  
37:17 ¿Por qué están calientes tus vestidos  
Cuando él sosiega la tierra con el viento del sur?  
37:18 ¿Extendiste tú con él los cielos,  
Firmes como un espejo fundido? 
37:19 Muéstranos qué le hemos de decir; 
Porque nosotros no podemos ordenar las ideas a causa de las tinieblas.  
37:20 ¿Será preciso contarle cuando yo hablare?  
Por más que el hombre razone, quedará como abismado.  
37:21 Mas ahora ya no se puede mirar la luz esplendente en los cielos,  
Luego que pasa el viento y los limpia,  
37:22 Viniendo de la parte del norte la dorada claridad.  
En Dios hay una majestad terrible.  
37:23 El es Todopoderoso, al cual no alcanzamos, grande en poder;  
Y en juicio y en multitud de justicia no afligirá.  
37:24 Lo temerán por tanto los hombres;  
El no estima a ninguno que cree en su propio corazón ser sabio.  
\section*{Capítulo 38 }
Jehová convence a Job de su ignorancia  

38:1 Entonces respondió Jehová a Job desde un torbellino, y dijo:  
38:2 ¿Quién es ése que oscurece el consejo  
Con palabras sin sabiduría?  
38:3 Ahora ciñe como varón tus lomos;  
Yo te preguntaré, y tú me contestarás.  
38:4 ¿Dónde estabas tú cuando yo fundaba la tierra?  
Házmelo saber, si tienes inteligencia.  
38:5 ¿Quién ordenó sus medidas, si lo sabes?  
¿O quién extendió sobre ella cordel?  
38:6 ¿Sobre qué están fundadas sus bases?  
¿O quién puso su piedra angular,  
38:7 Cuando alababan todas las estrellas del alba,  
Y se regocijaban todos los hijos de Dios?  
38:8 ¿Quién encerró con puertas el mar,  
Cuando se derramaba saliéndose de su seno, 
38:9 Cuando puse yo nubes por vestidura suya,  
Y por su faja oscuridad,  
38:10 Y establecí sobre él mi decreto,  
Le puse puertas y cerrojo,  
38:11 Y dije: Hasta aquí llegarás, y no pasarás adelante,  
Y ahí parará el orgullo de tus olas? 
38:12 ¿Has mandado tú a la mañana en tus días?  
¿Has mostrado al alba su lugar,  
38:13 Para que ocupe los fines de la tierra,  
Y para que sean sacudidos de ella los impíos?  
38:14 Ella muda luego de aspecto como barro bajo el sello,  
Y viene a estar como con vestidura;  
38:15 Mas la luz de los impíos es quitada de ellos,  
Y el brazo enaltecido es quebrantado.  
38:16 ¿Has entrado tú hasta las fuentes del mar,  
Y has andado escudriñando el abismo?  
38:17 ¿Te han sido descubiertas las puertas de la muerte,  
Y has visto las puertas de la sombra de muerte?  
38:18 ¿Has considerado tú hasta las anchuras de la tierra?  
Declara si sabes todo esto.  
38:19 ¿Por dónde va el camino a la habitación de la luz,  
Y dónde está el lugar de las tinieblas,  
38:20 Para que las lleves a sus límites,  
Y entiendas las sendas de su casa?  
38:21 ¡Tú lo sabes! Pues entonces ya habías nacido,  
Y es grande el número de tus días.  
38:22 ¿Has entrado tú en los tesoros de la nieve,  
O has visto los tesoros del granizo,  
38:23 Que tengo reservados para el tiempo de angustia,  
Para el día de la guerra y de la batalla?  
38:24 ¿Por qué camino se reparte la luz,  
Y se esparce el viento solano sobre la tierra? 
38:25 ¿Quién repartió conducto al turbión,  
Y camino a los relámpagos y truenos,  
38:26 Haciendo llover sobre la tierra deshabitada, 
Sobre el desierto, donde no hay hombre,  
38:27 Para saciar la tierra desierta e inculta,  
Y para hacer brotar la tierna hierba?  
38:28 ¿Tiene la lluvia padre?  
¿O quién engendró las gotas del rocío?  
38:29 ¿De qué vientre salió el hielo?  
Y la escarcha del cielo, ¿quién la engendró?  
38:30 Las aguas se endurecen a manera de piedra,  
Y se congela la faz del abismo.  
38:31 ¿Podrás tú atar los lazos de las Pléyades,  
O desatarás las ligaduras de Orión? 
38:32 ¿Sacarás tú a su tiempo las constelaciones de los cielos,  
O guiarás a la Osa Mayor con sus hijos?  
38:33 ¿Supiste tú las ordenanzas de los cielos?  
¿Dispondrás tú de su potestad en la tierra?  
38:34 ¿Alzarás tú a las nubes tu voz,  
Para que te cubra muchedumbre de aguas?  
38:35 ¿Enviarás tú los relámpagos, para que ellos vayan?  
¿Y te dirán ellos: Henos aquí?  
38:36 ¿Quién puso la sabiduría en el corazón?  
¿O quién dio al espíritu inteligencia?  
38:37 ¿Quién puso por cuenta los cielos con sabiduría?  
Y los odres de los cielos, ¿quién los hace inclinar,  
38:38 Cuando el polvo se ha convertido en dureza,  
Y los terrones se han pegado unos con otros?  
38:39 ¿Cazarás tú la presa para el león?  
¿Saciarás el hambre de los leoncillos,  
38:40 Cuando están echados en las cuevas,  
O se están en sus guaridas para acechar?  
38:41 ¿Quién prepara al cuervo su alimento,  
Cuando sus polluelos claman a Dios,  
Y andan errantes por falta de comida?  
\section*{Capítulo 39 }

39:1 ¿Sabes tú el tiempo en que paren las cabras monteses?  
¿O miraste tú las ciervas cuando están pariendo? 
39:2 ¿Contaste tú los meses de su preñez,  
Y sabes el tiempo cuando han de parir?  
39:3 Se encorvan, hacen salir sus hijos,  
Pasan sus dolores.  
39:4 Sus hijos se fortalecen, crecen con el pasto;  
Salen, y no vuelven a ellas. 
39:5 ¿Quién echó libre al asno montés,  
Y quién soltó sus ataduras?  
39:6 Al cual yo puse casa en la soledad,  
Y sus moradas en lugares estériles.  
39:7 Se burla de la multitud de la ciudad;  
No oye las voces del arriero.  
39:8 Lo oculto de los montes es su pasto,  
Y anda buscando toda cosa verde.  
39:9 ¿Querrá el búfalo servirte a ti,  
O quedar en tu pesebre?  
39:10 ¿Atarás tú al búfalo con coyunda para el surco?  
¿Labrará los valles en pos de ti?  
39:11 ¿Confiarás tú en él, por ser grande su fuerza,  
Y le fiarás tu labor? 
39:12 ¿Fiarás de él para que recoja tu semilla,  
Y la junte en tu era?  
39:13 ¿Diste tú hermosas alas al pavo real,  
o alas y plumas al avestruz?  
39:14 El cual desampara en la tierra sus huevos,  
Y sobre el polvo los calienta,  
39:15 Y olvida que el pie los puede pisar,  
Y que puede quebrarlos la bestia del campo.  
39:16 Se endurece para con sus hijos, como si no fuesen suyos,  
No temiendo que su trabajo haya sido en vano;  
39:17 Porque le privó Dios de sabiduría,  
Y no le dio inteligencia.  
39:18 Luego que se levanta en alto, 
Se burla del caballo y de su jinete.  
39:19 ¿Diste tú al caballo la fuerza?  
¿Vestiste tú su cuello de crines ondulantes? 
39:20 ¿Le intimidarás tú como a langosta?  
El resoplido de su nariz es formidable.  
39:21 Escarba la tierra, se alegra en su fuerza,  
Sale al encuentro de las armas;  
39:22 Hace burla del espanto, y no teme,  
Ni vuelve el rostro delante de la espada.  
39:23 Contra él suenan la aljaba,  
El hierro de la lanza y de la jabalina;  
39:24 Y él con ímpetu y furor escarba la tierra,  
Sin importarle el sonido de la trompeta;  
39:25 Antes como que dice entre los clarines: ¡Ea!  
Y desde lejos huele la batalla,  
El grito de los capitanes, y el vocerío.  
39:26 ¿Vuela el gavilán por tu sabiduría,  
Y extiende hacia el sur sus alas?  
39:27 ¿Se remonta el águila por tu mandamiento,  
Y pone en alto su nido?  
39:28 Ella habita y mora en la peña,  
En la cumbre del peñasco y de la roca.  
39:29 Desde allí acecha la presa;  
Sus ojos observan de muy lejos.  
39:30 Sus polluelos chupan la sangre;  
Y donde hubiere cadáveres, allí está ella.  
\section*{Capítulo 40} 

40:1 Además respondió Jehová a Job, y dijo:  
40:2 ¿Es sabiduría contender con el Omnipotente? 
El que disputa con Dios, responda a esto.  
40:3 Entonces respondió Job a Jehová, y dijo:  
40:4 He aquí que yo soy vil; ¿qué te responderé? 
Mi mano pongo sobre mi boca. 
40:5 Una vez hablé, mas no responderé; Aun dos veces, mas no volveré a hablar.  
Manifestaciones del poder de Dios 
40:6 Respondió Jehová a Job desde el torbellino, y dijo:  
40:7 Cíñete ahora como varón tus lomos;  
Yo te preguntaré, y tú me responderás.  
40:8 ¿Invalidarás tú también mi juicio?  
¿Me condenarás a mí, para justificarte tú?  
40:9 ¿Tienes tú un brazo como el de Dios?  
¿Y truenas con voz como la suya?  
40:10 Adórnate ahora de majestad y de alteza,  
Y vístete de honra y de hermosura.  
40:11 Derrama el ardor de tu ira;  
Mira a todo altivo, y abátelo.  
40:12 Mira a todo soberbio, y humíllalo,  
Y quebranta a los impíos en su sitio.  
40:13 Encúbrelos a todos en el polvo,  
Encierra sus rostros en la oscuridad;  
40:14 Y yo también te confesaré  
Que podrá salvarte tu diestra.  
40:15 He aquí ahora behemot, el cual hice como a ti;  
Hierba come como buey.  
40:16 He aquí ahora que su fuerza está en sus lomos,  
Y su vigor en los músculos de su vientre.  
40:17 Su cola mueve como un cedro,  
Y los nervios de sus muslos están entretejidos.  
40:18 Sus huesos son fuertes como bronce,  
Y sus miembros como barras de hierro. 
40:19 El es el principio de los caminos de Dios;  
El que lo hizo, puede hacer que su espada a él se acerque. 
40:20 Ciertamente los montes producen hierba para él;  
Y toda bestia del campo retoza allá.  
40:21 Se echará debajo de las sombras,  
En lo oculto de las cañas y de los lugares húmedos.  
40:22 Los árboles sombríos lo cubren con su sombra;  
Los sauces del arroyo lo rodean. 
40:23 He aquí, sale de madre el río, pero él no se inmuta;  
Tranquilo está, aunque todo un Jordán se estrelle contra su boca. 
40:24 ¿Lo tomará alguno cuando está vigilante,  
Y horadará su nariz?  
\section*{Capítulo 41 }

41:1 ¿Sacarás tú al leviatán con anzuelo,  
O con cuerda que le eches en su lengua?  
41:2 ¿Pondrás tú soga en sus narices,  
Y horadarás con garfio su quijada?  
41:3 ¿Multiplicará él ruegos para contigo?  
¿Te hablará él lisonjas?  
41:4 ¿Hará pacto contigo  
Para que lo tomes por siervo perpetuo?  
41:5 ¿Jugarás con él como con pájaro,  
O lo atarás para tus niñas?  
41:6 ¿Harán de él banquete los compañeros?  
¿Lo repartirán entre los mercaderes?  
41:7 ¿Cortarás tú con cuchillo su piel,  
O con arpón de pescadores su cabeza?  
41:8 Pon tu mano sobre él;  
Te acordarás de la batalla, y nunca más volverás.  
41:9 He aquí que la esperanza acerca de él será burlada,  
Porque aun a su sola vista se desmayarán.  
41:10 Nadie hay tan osado que lo despierte;  
¿Quién, pues, podrá estar delante de mí?  
41:11 ¿Quién me ha dado a mí primero, para que yo restituya? 
Todo lo que hay debajo del cielo es mío.  
41:12 No guardaré silencio sobre sus miembros,  
Ni sobre sus fuerzas y la gracia de su disposición.  
41:13 ¿Quién descubrirá la delantera de su vestidura?  
¿Quién se acercará a él con su freno doble?  
41:14 ¿Quién abrirá las puertas de su rostro?  
Las hileras de sus dientes espantan.  
41:15 La gloria de su vestido son escudos fuertes,  
Cerrados entre sí estrechamente. 
41:16 El uno se junta con el otro,  
Que viento no entra entre ellos.  
41:17 Pegado está el uno con el otro;  
Están trabados entre sí, que no se pueden apartar.  
41:18 Con sus estornudos enciende lumbre,  
Y sus ojos son como los párpados del alba.  
41:19 De su boca salen hachones de fuego;  
Centellas de fuego proceden.  
41:20 De sus narices sale humo,  
Como de una olla o caldero que hierve. 
41:21 Su aliento enciende los carbones,  
Y de su boca sale llama.  
41:22 En su cerviz está la fuerza,  
Y delante de él se esparce el desaliento.  
41:23 Las partes más flojas de su carne están endurecidas;  
Están en él firmes, y no se mueven.  
41:24 Su corazón es firme como una piedra,  
Y fuerte como la muela de abajo.  
41:25 De su grandeza tienen temor los fuertes,  
Y a causa de su desfallecimiento hacen por purificarse. 
41:26 Cuando alguno lo alcanzare,  
Ni espada, ni lanza, ni dardo, ni coselete durará.  
41:27 Estima como paja el hierro,  
Y el bronce como leño podrido.  
41:28 Saeta no le hace huir;  
Las piedras de honda le son como paja.  
41:29 Tiene toda arma por hojarasca,  
Y del blandir de la jabalina se burla.  
41:30 Por debajo tiene agudas conchas;  
Imprime su agudez en el suelo.  
41:31 Hace hervir como una olla el mar profundo,  
Y lo vuelve como una olla de ungüento.  
41:32 En pos de sí hace resplandecer la senda, 
Que parece que el abismo es cano.  
41:33 No hay sobre la tierra quien se le parezca;  
Animal hecho exento de temor.  
41:34 Menosprecia toda cosa alta;  
Es rey sobre todos los soberbios.  
\section*{Capítulo 42 }

Confesión y justificación de Job  
42:1 Respondió Job a Jehová, y dijo:  
42:2 Yo conozco que todo lo puedes,  
Y que no hay pensamiento que se esconda de ti.  
42:3 ¿Quién es el que oscurece el consejo sin entendimiento?  
Por tanto, yo hablaba lo que no entendía;  
Cosas demasiado maravillosas para mí, que yo no comprendía.  
42:4 Oye, te ruego, y hablaré;  
Te preguntaré, y tú me enseñarás. 
42:5 De oídas te había oído;  
Mas ahora mis ojos te ven.  
42:6 Por tanto me aborrezco,  
Y me arrepiento en polvo y ceniza.  
42:7 Y aconteció que después que habló Jehová estas palabras a Job, Jehová dijo a Elifaz temanita: Mi ira se encendió contra ti y tus dos compañeros; porque no habéis hablado de mí lo recto, como mi siervo Job.  
42:8 Ahora, pues, tomaos siete becerros y siete carneros, e id a mi siervo Job, y ofreced holocausto por vosotros, y mi siervo Job orará por vosotros; porque de cierto a él atenderé para no trataros afrentosamente, por cuanto no habéis hablado de mí con rectitud, como mi siervo Job.  
42:9 Fueron, pues, Elifaz temanita, Bildad suhita y Zofar naamatita, e hicieron como Jehová les dijo; y Jehová aceptó la oración de Job.  
Restauración de la prosperidad de Job  
42:10 Y quitó Jehová la aflicción de Job, cuando él hubo orado por sus amigos; y aumentó al doble todas las cosas que habían sido de Job. 
42:11 Y vinieron a él todos sus hermanos y todas sus hermanas, y todos los que antes le habían conocido, y comieron con él pan en su casa, y se condolieron de él, y le consolaron de todo aquel mal que Jehová había traído sobre él; y cada uno de ellos le dio una pieza de dinero y un anillo de oro.  
42:12 Y bendijo Jehová el postrer estado de Job más que el primero; porque tuvo catorce mil ovejas, seis mil camellos, mil yuntas de bueyes y mil asnas,  
42:13 y tuvo siete hijos y tres hijas.  
42:14 Llamó el nombre de la primera, Jemima, el de la segunda, Cesia, y el de la tercera, Keren-hapuc.  
42:15 Y no había mujeres tan hermosas como las hijas de Job en toda la tierra; y les dio su padre herencia entre sus hermanos.  
42:16 Después de esto vivió Job ciento cuarenta años, y vio a sus hijos, y a los hijos de sus hijos, hasta la cuarta generación.  
42:17 Y murió Job viejo y lleno de días.