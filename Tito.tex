\chapter{La Epístola Del Apóstol San Pablo a Tito}



\section*{Capítulo 1 }
Salutación  
1:1 Pablo, siervo de Dios y apóstol de Jesucristo, conforme a la fe de los escogidos de Dios y el conocimiento de la verdad que es según la piedad,  
1:2 en la esperanza de la vida eterna, la cual Dios, que no miente, prometió desde antes del principio de los siglos,  
1:3 y a su debido tiempo manifestó su palabra por medio de la predicación que me fue encomendada por mandato de Dios nuestro Salvador,  
1:4 a Tito, verdadero hijo en la común fe: Gracia, misericordia y paz, de Dios Padre y del Señor Jesucristo nuestro Salvador.  
Requisitos de ancianos y obispos  
1:5 Por esta causa te dejé en Creta, para que corrigieses lo deficiente, y establecieses ancianos en cada ciudad, así como yo te mandé;  
1:6 el que fuere irreprensible, marido de una sola mujer, y tenga hijos creyentes que no estén acusados de disolución ni de rebeldía.  
1:7 Porque es necesario que el obispo sea irreprensible, como administrador de Dios; no soberbio, no iracundo, no dado al vino, no pendenciero, no codicioso de ganancias deshonestas,  
1:8 sino hospedador, amante de lo bueno, sobrio, justo, santo, dueño de sí mismo,  
1:9 retenedor de la palabra fiel tal como ha sido enseñada, para que también pueda exhortar con sana enseñanza y convencer a los que contradicen. 
1:10 Porque hay aún muchos contumaces, habladores de vanidades y engañadores, mayormente los de la circuncisión,  
1:11 a los cuales es preciso tapar la boca; que trastornan casas enteras, enseñando por ganancia deshonesta lo que no conviene.  
1:12 Uno de ellos, su propio profeta, dijo: Los cretenses, siempre mentirosos, malas bestias, glotones ociosos.  
1:13 Este testimonio es verdadero; por tanto, repréndelos duramente, para que sean sanos en la fe,  
1:14 no atendiendo a fábulas judaicas, ni a mandamientos de hombres que se apartan de la verdad.  
1:15 Todas las cosas son puras para los puros, mas para los corrompidos e incrédulos nada les es puro; pues hasta su mente y su conciencia están corrompidas.  
1:16 Profesan conocer a Dios, pero con los hechos lo niegan, siendo abominables y rebeldes, reprobados en cuanto a toda buena obra.  
\section*{Capítulo 2}
Enseñanza de la sana doctrina  

2:1 Pero tú habla lo que está de acuerdo con la sana doctrina.  
2:2 Que los ancianos sean sobrios, serios, prudentes, sanos en la fe, en el amor, en la paciencia.  
2:3 Las ancianas asimismo sean reverentes en su porte; no calumniadoras, no esclavas del vino, maestras del bien;  
2:4 que enseñen a las mujeres jóvenes a amar a sus maridos y a sus hijos,  
2:5 a ser prudentes, castas, cuidadosas de su casa, buenas, sujetas a sus maridos, para que la palabra de Dios no sea blasfemada.  
2:6 Exhorta asimismo a los jóvenes a que sean prudentes;  
2:7 presentándote tú en todo como ejemplo de buenas obras; en la enseñanza mostrando integridad, seriedad,  
2:8 palabra sana e irreprochable, de modo que el adversario se avergüence, y no tenga nada malo que decir de vosotros.  
2:9 Exhorta a los siervos a que se sujeten a sus amos, que agraden en todo, que no sean respondones;  
2:10 no defraudando, sino mostrándose fieles en todo, para que en todo adornen la doctrina de Dios nuestro Salvador.  
2:11 Porque la gracia de Dios se ha manifestado para salvación a todos los hombres,  
2:12 enseñándonos que, renunciando a la impiedad y a los deseos mundanos, vivamos en este siglo sobria, justa y piadosamente,  
2:13 aguardando la esperanza bienaventurada y la manifestación gloriosa de nuestro gran Dios y Salvador Jesucristo,  
2:14 quien se dio a sí mismo por nosotros para redimirnos de toda iniquidad y purificar para sí un pueblo propio, celoso de buenas obras.  
2:15 Esto habla, y exhorta y reprende con toda autoridad. Nadie te menosprecie.  
\section*{Capítulo 3}
Justificados por gracia  

3:1 Recuérdales que se sujeten a los gobernantes y autoridades, que obedezcan, que estén dispuestos a toda buena obra.  
3:2 Que a nadie difamen, que no sean pendencieros, sino amables, mostrando toda mansedumbre para con todos los hombres.  
3:3 Porque nosotros también éramos en otro tiempo insensatos, rebeldes, extraviados, esclavos de concupiscencias y deleites diversos, viviendo en malicia y envidia, aborrecibles, y aborreciéndonos unos a otros.  
3:4 Pero cuando se manifestó la bondad de Dios nuestro Salvador, y su amor para con los hombres,  
3:5 nos salvó, no por obras de justicia que nosotros hubiéramos hecho, sino por su misericordia, por el lavamiento de la regeneración y por la renovación en el Espíritu Santo,  
3:6 el cual derramó en nosotros abundantemente por Jesucristo nuestro Salvador,  
3:7 para que justificados por su gracia, viniésemos a ser herederos conforme a la esperanza de la vida eterna.  
3:8 Palabra fiel es esta, y en estas cosas quiero que insistas con firmeza, para que los que creen en Dios procuren ocuparse en buenas obras. Estas cosas son buenas y útiles a los hombres.  
3:9 Pero evita las cuestiones necias, y genealogías, y contenciones, y discusiones acerca de la ley; porque son vanas y sin provecho.  
3:10 Al hombre que cause divisiones, después de una y otra amonestación deséchalo,  
3:11 sabiendo que el tal se ha pervertido, y peca y está condenado por su propio juicio.  
Instrucciones personales  
3:12 Cuando envíe a ti a Artemas o a Tíquico, apresúrate a venir a mí en Nicópolis, porque allí he determinado pasar el invierno.  
3:13 A Zenas intérprete de la ley, y a Apolos, encamínales con solicitud, de modo que nada les falte.  
3:14 Y aprendan también los nuestros a ocuparse en buenas obras para los casos de necesidad, para que no sean sin fruto.  
Salutaciones y bendición final  
3:15 Todos los que están conmigo te saludan. Saluda a los que nos aman en la fe. La gracia sea con todos vosotros. Amén. 