\chapter{Habacuc}
\section*{Capítulo 1 }
Habacuc se queja de injusticia  
1:1 La profecía que vio el profeta Habacuc.  
1:2 ¿Hasta cuándo, oh Jehová, clamaré, y no oirás; y daré voces a ti a causa de la violencia, y no salvarás?  
1:3 ¿Por qué me haces ver iniquidad, y haces que vea molestia? Destrucción y violencia están delante de mí, y pleito y contienda se levantan.  
1:4 Por lo cual la ley es debilitada, y el juicio no sale según la verdad; por cuanto el impío asedia al justo, por eso sale torcida la justicia.  
Los caldeos castigarán a Judá  
1:5 Mirad entre las naciones, y ved, y asombraos; porque haré una obra en vuestros días, que aun cuando se os contare, no la creeréis. 
1:6 Porque he aquí, yo levanto a los caldeos, nación cruel y presurosa, que camina por la anchura de la tierra para poseer las moradas ajenas.  
1:7 Formidable es y terrible; de ella misma procede su justicia y su dignidad.  
1:8 Sus caballos serán más ligeros que leopardos, y más feroces que lobos nocturnos, y sus jinetes se multiplicarán; vendrán de lejos sus jinetes, y volarán como águilas que se apresuran a devorar.  
1:9 Toda ella vendrá a la presa; el terror va delante de ella, y recogerá cautivos como arena.  
1:10 Escarnecerá a los reyes, y de los príncipes hará burla; se reirá de toda fortaleza, y levantará terraplén y la tomará.  
1:11 Luego pasará como el huracán, y ofenderá atribuyendo su fuerza a su dios.  
Protesta de Habacuc  
1:12 ¿No eres tú desde el principio, oh Jehová, Dios mío, Santo mío? No moriremos. Oh Jehová, para juicio lo pusiste; y tú, oh Roca, lo fundaste para castigar.  
1:13 Muy limpio eres de ojos para ver el mal, ni puedes ver el agravio; ¿por qué ves a los menospreciadores, y callas cuando destruye el impío al más justo que él,  
1:14 y haces que sean los hombres como los peces del mar, como reptiles que no tienen quien los gobierne?  
1:15 Sacará a todos con anzuelo, los recogerá con su red, y los juntará en sus mallas; por lo cual se alegrará y se regocijará.  
1:16 Por esto hará sacrificios a su red, y ofrecerá sahumerios a sus mallas; porque con ellas engordó su porción, y engrasó su comida.  
1:17 ¿Vaciará por eso su red, y no tendrá piedad de aniquilar naciones continuamente?  
\section*{Capítulo 2 }
Jehová responde a Habacuc  

2:1 Sobre mi guarda estaré, y sobre la fortaleza afirmaré el pie, y velaré para ver lo que se me dirá, y qué he de responder tocante a mi queja.  
2:2 Y Jehová me respondió, y dijo: Escribe la visión, y declárala en tablas, para que corra el que leyere en ella.  
2:3 Aunque la visión tardará aún por un tiempo, mas se apresura hacia el fin, y no mentirá; aunque tardare, espéralo, porque sin duda vendrá, no tardará. 
2:4 He aquí que aquel cuya alma no es recta, se enorgullece; mas el justo por su fe vivirá. 
2:5 Y también, el que es dado al vino es traicionero, hombre soberbio, que no permanecerá; ensanchó como el Seol su alma, y es como la muerte, que no se saciará; antes reunió para sí todas las gentes, y juntó para sí todos los pueblos.  
Ayes contra los injustos  
2:6 ¿No han de levantar todos éstos refrán sobre él, y sarcasmos contra él? Dirán: ¡Ay del que multiplicó lo que no era suyo! ¿Hasta cuándo había de acumular sobre sí prenda tras prenda?  
2:7 ¿No se levantarán de repente tus deudores, y se despertarán los que te harán temblar, y serás despojo para ellos?  
2:8 Por cuanto tú has despojado a muchas naciones, todos los otros pueblos te despojarán, a causa de la sangre de los hombres, y de los robos de la tierra, de las ciudades y de todos los que habitan en ellas.  
2:9 ¡Ay del que codicia injusta ganancia para su casa, para poner en alto su nido, para escaparse del poder del mal!  
2:10 Tomaste consejo vergonzoso para tu casa, asolaste muchos pueblos, y has pecado contra tu vida.  
2:11 Porque la piedra clamará desde el muro, y la tabla del enmaderado le responderá.  
2:12 ¡Ay del que edifica la ciudad con sangre, y del que funda una ciudad con iniquidad!  
2:13 ¿No es esto de Jehová de los ejércitos? Los pueblos, pues, trabajarán para el fuego, y las naciones se fatigarán en vano.  
2:14 Porque la tierra será llena del conocimiento de la gloria de Jehová, como las aguas cubren el mar. 
2:15 ¡Ay del que da de beber a su prójimo! ¡Ay de ti, que le acercas tu hiel, y le embriagas para mirar su desnudez!  
2:16 Te has llenado de deshonra más que de honra; bebe tú también, y serás descubierto; el cáliz de la mano derecha de Jehová vendrá hasta ti, y vómito de afrenta sobre tu gloria.  
2:17 Porque la rapiña del Líbano caerá sobre ti, y la destrucción de las fieras te quebrantará, a causa de la sangre de los hombres, y del robo de la tierra, de las ciudades y de todos los que en ellas habitaban.  
2:18 ¿De qué sirve la escultura que esculpió el que la hizo? ¿la estatua de fundición que enseña mentira, para que haciendo imágenes mudas confíe el hacedor en su obra?  
2:19 ¡Ay del que dice al palo: Despiértate; y a la piedra muda: Levántate! ¿Podrá él enseñar? He aquí está cubierto de oro y plata, y no hay espíritu dentro de él.  
2:20 Mas Jehová está en su santo templo; calle delante de él toda la tierra.  
\section*{Capítulo 3}
Oración de Habacuc  

3:1 Oración del profeta Habacuc, sobre Sigionot.  
3:2 Oh Jehová, he oído tu palabra, y temí.  
Oh Jehová, aviva tu obra en medio de los tiempos,  
En medio de los tiempos hazla conocer;  
En la ira acuérdate de la misericordia.  
3:3 Dios vendrá de Temán,  
Y el Santo desde el monte de Parán. Selah  
Su gloria cubrió los cielos,  
Y la tierra se llenó de su alabanza.  
3:4 Y el resplandor fue como la luz;  
Rayos brillantes salían de su mano,  
Y allí estaba escondido su poder.  
3:5 Delante de su rostro iba mortandad,  
Y a sus pies salían carbones encendidos.  
3:6 Se levantó, y midió la tierra;  
Miró, e hizo temblar las gentes;  
Los montes antiguos fueron desmenuzados,  
Los collados antiguos se humillaron.  
Sus caminos son eternos.  
3:7 He visto las tiendas de Cusán en aflicción;  
Las tiendas de la tierra de Madián temblaron.  
3:8 ¿Te airaste, oh Jehová, contra los ríos?  
¿Contra los ríos te airaste?  
¿Fue tu ira contra el mar  
Cuando montaste en tus caballos,  
Y en tus carros de victoria?  
3:9 Se descubrió enteramente tu arco;  
Los juramentos a las tribus fueron palabra segura. Selah  
Hendiste la tierra con ríos.  
3:10 Te vieron y tuvieron temor los montes;  
Pasó la inundación de las aguas;  
El abismo dio su voz,  
A lo alto alzó sus manos.  
3:11 El sol y la luna se pararon en su lugar;  
A la luz de tus saetas anduvieron,  
Y al resplandor de tu fulgente lanza.  
3:12 Con ira hollaste la tierra,  
Con furor trillaste las naciones.  
3:13 Saliste para socorrer a tu pueblo,  
Para socorrer a tu ungido.  
Traspasaste la cabeza de la casa del impío,  
Descubriendo el cimiento hasta la roca. Selah  
3:14 Horadaste con sus propios dardos las cabezas de sus guerreros,  
Que como tempestad acometieron para dispersarme,  
Cuyo regocijo era como para devorar al pobre encubiertamente.  
3:15 Caminaste en el mar con tus caballos,  
Sobre la mole de las grandes aguas.  
3:16 Oí, y se conmovieron mis entrañas;  
A la voz temblaron mis labios;  
Pudrición entró en mis huesos, y dentro de mí me estremecí; 
Si bien estaré quieto en el día de la angustia,  
Cuando suba al pueblo el que lo invadirá con sus tropas.  
3:17 Aunque la higuera no florezca,  
Ni en las vides haya frutos,  
Aunque falte el producto del olivo,  
Y los labrados  
2 no den mantenimiento,  
Y las ovejas sean quitadas de la majada,  
Y no haya vacas en los corrales;  
3:18 Con todo, yo me alegraré en Jehová,  
Y me gozaré en el Dios de mi salvación.  
3:19 Jehová el Señor es mi fortaleza,  
El cual hace mis pies como de ciervas, 
Y en mis alturas me hace andar. 
Al jefe de los cantores, sobre mis instrumentos de cuerdas.
