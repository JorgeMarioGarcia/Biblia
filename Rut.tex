\chapter{Rut}

\section*{Capítulo 1}

Rut y Noemí  
1:1 Aconteció en los días que gobernaban los jueces, que hubo hambre en la tierra. Y un varón de Belén de Judá fue a morar en los campos de Moab, él y su mujer, y dos hijos suyos.  
1:2 El nombre de aquel varón era Elimelec, y el de su mujer, Noemí; y los nombres de sus hijos eran Mahlón y Quelión, efrateos de Belén de Judá. Llegaron, pues, a los campos de Moab, y se quedaron allí.  
1:3 Y murió Elimelec, marido de Noemí, y quedó ella con sus dos hijos,  
1:4 los cuales tomaron para sí mujeres moabitas; el nombre de una era Orfa, y el nombre de la otra, Rut; y habitaron allí unos diez años.  
1:5 Y murieron también los dos, Mahlón y Quelión, quedando así la mujer desamparada de sus dos hijos y de su marido.  
1:6 Entonces se levantó con sus nueras, y regresó de los campos de Moab; porque oyó en el campo de Moab que Jehová había visitado a su pueblo para darles pan.  
1:7 Salió, pues, del lugar donde había estado, y con ella sus dos nueras, y comenzaron a caminar para volverse a la tierra de Judá.  
1:8 Y Noemí dijo a sus dos nueras: Andad, volveos cada una a la casa de su madre; Jehová haga con vosotras misericordia, como la habéis hecho con los muertos y conmigo.  
1:9 Os conceda Jehová que halléis descanso, cada una en casa de su marido. Luego las besó, y ellas alzaron su voz y lloraron,  
1:10 y le dijeron: Ciertamente nosotras iremos contigo a tu pueblo.  
1:11 Y Noemí respondió: Volveos, hijas mías; ¿para qué habéis de ir conmigo? ¿Tengo yo más hijos en el vientre, que puedan ser vuestros maridos?  
1:12 Volveos, hijas mías, e idos; porque yo ya soy vieja para tener marido. Y aunque dijese: Esperanza tengo, y esta noche estuviese con marido, y aun diese a luz hijos,  
1:13 ¿habíais vosotras de esperarlos hasta que fuesen grandes? ¿Habíais de quedaros sin casar por amor a ellos? No, hijas mías; que mayor amargura tengo yo que vosotras, pues la mano de Jehová ha salido contra mí.  
1:14 Y ellas alzaron otra vez su voz y lloraron; y Orfa besó a su suegra, mas Rut se quedó con ella.  
1:15 Y Noemí dijo: He aquí tu cuñada se ha vuelto a su pueblo y a sus dioses; vuélvete tú tras ella.  
1:16 Respondió Rut: No me ruegues que te deje, y me aparte de ti; porque a dondequiera que tú fueres, iré yo, y dondequiera que vivieres, viviré. Tu pueblo será mi pueblo, y tu Dios mi Dios.  
1:17 Donde tú murieres, moriré yo, y allí seré sepultada; así me haga Jehová, y aun me añada, que sólo la muerte hará separación entre nosotras dos. 
1:18 Y viendo Noemí que estaba tan resuelta a ir con ella, no dijo más.  
1:19 Anduvieron, pues, ellas dos hasta que llegaron a Belén; y aconteció que habiendo entrado en Belén, toda la ciudad se conmovió por causa de ellas, y decían: ¿No es ésta Noemí?  
1:20 Y ella les respondía: No me llaméis Noemí, sino llamadme Mara; porque en grande amargura me ha puesto el Todopoderoso.  
1:21 Yo me fui llena, pero Jehová me ha vuelto con las manos vacías. ¿Por qué me llamaréis Noemí, ya que Jehová ha dado testimonio contra mí, y el Todopoderoso me ha afligido?  
1:22 Así volvió Noemí, y Rut la moabita su nuera con ella; volvió de los campos de Moab, y llegaron a Belén al comienzo de la siega de la cebada.  
\section*{Capítulo 2}
Rut recoge espigas en el campo de Booz  

2:1 Tenía Noemí un pariente de su marido, hombre rico de la familia de Elimelec, el cual se llamaba Booz.  
2:2 Y Rut la moabita dijo a Noemí: Te ruego que me dejes ir al campo, y recogeré espigas en pos de aquel a cuyos ojos hallare gracia. Y ella le respondió: Vé, hija mía.  
2:3 Fue, pues, y llegando, espigó en el campo en pos de los segadores; y aconteció que aquella parte del campo era de Booz, el cual era de la familia de Elimelec.  
2:4 Y he aquí que Booz vino de Belén, y dijo a los segadores: Jehová sea con vosotros. Y ellos respondieron: Jehová te bendiga.  
2:5 Y Booz dijo a su criado el mayordomo de los segadores: ¿De quién es esta joven?  
2:6 Y el criado, mayordomo de los segadores, respondió y dijo: Es la joven moabita que volvió con Noemí de los campos de Moab;  
2:7 y ha dicho: Te ruego que me dejes recoger y juntar tras los segadores entre las gavillas. Entró, pues, y está desde por la mañana hasta ahora, sin descansar ni aun por un momento.  
2:8 Entonces Booz dijo a Rut: Oye, hija mía, no vayas a espigar a otro campo, ni pases de aquí; y aquí estarás junto a mis criadas.  
2:9 Mira bien el campo que sieguen, y síguelas; porque yo he mandado a los criados que no te molesten. Y cuando tengas sed, ve a las vasijas, y bebe del agua que sacan los criados.  
2:10 Ella entonces bajando su rostro se inclinó a tierra, y le dijo: ¿Por qué he hallado gracia en tus ojos para que me reconozcas, siendo yo extranjera?  
2:11 Y respondiendo Booz, le dijo: He sabido todo lo que has hecho con tu suegra después de la muerte de tu marido, y que dejando a tu padre y a tu madre y la tierra donde naciste, has venido a un pueblo que no conociste antes.  
2:12 Jehová recompense tu obra, y tu remuneración sea cumplida de parte de Jehová Dios de Israel, bajo cuyas alas has venido a refugiarte.  
2:13 Y ella dijo: Señor mío, halle yo gracia delante de tus ojos; porque me has consolado, y porque has hablado al corazón de tu sierva, aunque no soy ni como una de tus criadas.  
2:14 Y Booz le dijo a la hora de comer: Ven aquí, y come del pan, y moja tu bocado en el vinagre. Y ella se sentó junto a los segadores, y él le dio del potaje, y comió hasta que se sació, y le sobró.  
2:15 Luego se levantó para espigar. Y Booz mandó a sus criados, diciendo: Que recoja también espigas entre las gavillas, y no la avergoncéis;  
2:16 y dejaréis también caer para ella algo de los manojos, y lo dejaréis para que lo recoja, y no la reprendáis.  
2:17 Espigó, pues, en el campo hasta la noche, y desgranó lo que había recogido, y fue como un efa  de cebada.  
2:18 Y lo tomó, y se fue a la ciudad; y su suegra vio lo que había recogido. Sacó también luego lo que le había sobrado después de haber quedado saciada, y se lo dio.  
2:19 Y le dijo su suegra: ¿Dónde has espigado hoy? ¿y dónde has trabajado? Bendito sea el que te ha reconocido. Y contó ella a su suegra con quién había trabajado, y dijo: El nombre del varón con quien hoy he trabajado es Booz.  
2:20 Y dijo Noemí a su nuera: Sea él bendito de Jehová, pues que no ha rehusado a los vivos la benevolencia que tuvo para con los que han muerto. Después le dijo Noemí: Nuestro pariente es aquel varón, y uno de los que pueden redimirnos.  
2:21 Y Rut la moabita dijo: Además de esto me ha dicho: Júntate con mis criadas, hasta que hayan acabado toda mi siega.  
2:22 Y Noemí respondió a Rut su nuera: Mejor es, hija mía, que salgas con sus criadas, y que no te encuentren en otro campo.  
2:23 Estuvo, pues, junto con las criadas de Booz espigando, hasta que se acabó la siega de la cebada y la del trigo; y vivía con su suegra.  
\section*{Capítulo 3}
Rut y Booz en la era  

3:1 Después le dijo su suegra Noemí: Hija mía, ¿no he de buscar hogar para ti, para que te vaya bien?  
3:2 ¿No es Booz nuestro pariente, con cuyas criadas tú has estado? He aquí que él avienta esta noche la parva de las cebadas.  
3:3 Te lavarás, pues, y te ungirás, y vistiéndote tus vestidos, irás a la era; mas no te darás a conocer al varón hasta que él haya acabado de comer y de beber.  
3:4 Y cuando él se acueste, notarás el lugar donde se acuesta, e irás y descubrirás sus pies, y te acostarás allí; y él te dirá lo que hayas de hacer.  
3:5 Y ella respondió: Haré todo lo que tú me mandes.  
3:6 Descendió, pues, a la era, e hizo todo lo que su suegra le había mandado.  
3:7 Y cuando Booz hubo comido y bebido, y su corazón estuvo contento, se retiró a dormir a un lado del montón. Entonces ella vino calladamente, y le descubrió los pies y se acostó.  
3:8 Y aconteció que a la medianoche se estremeció aquel hombre, y se volvió; y he aquí, una mujer estaba acostada a sus pies.  
3:9 Entonces él dijo: ¿Quién eres? Y ella respondió: Yo soy Rut tu sierva; extiende el borde de tu capa sobre tu sierva, por cuanto eres pariente cercano.  
3:10 Y él dijo: Bendita seas tú de Jehová, hija mía; has hecho mejor tu postrera bondad que la primera, no yendo en busca de los jóvenes, sean pobres o ricos.  
3:11 Ahora pues, no temas, hija mía; yo haré contigo lo que tú digas, pues toda la gente de mi pueblo sabe que eres mujer virtuosa.  
3:12 Y ahora, aunque es cierto que yo soy pariente cercano, con todo eso hay pariente más cercano que yo.  
3:13 Pasa aquí la noche, y cuando sea de día, si él te redimiere, bien, redímate; mas si él no te quisiere redimir, yo te redimiré, vive Jehová. Descansa, pues, hasta la mañana.  
3:14 Y después que durmió a sus pies hasta la mañana, se levantó antes que los hombres pudieran reconocerse unos a otros; porque él dijo: No se sepa que vino mujer a la era.  
3:15 Después le dijo: Quítate el manto que traes sobre ti, y tenlo. Y teniéndolo ella, él midió seis medidas  de cebada, y se las puso encima; y ella se fue a la ciudad.  
3:16 Y cuando llegó a donde estaba su suegra, ésta le dijo: ¿Qué hay, hija mía? Y le contó ella todo lo que con aquel varón le había acontecido.  
3:17 Y dijo: Estas seis medidas  de cebada me dio, diciéndome: A fin de que no vayas a tu suegra con las manos vacías.  
3:18 Entonces Noemí dijo: Espérate, hija mía, hasta que sepas cómo se resuelve el asunto; porque aquel hombre no descansará hasta que concluya el asunto hoy.  
\section*{Capítulo 4}
Booz se casa con Rut  

4:1 Booz subió a la puerta y se sentó allí; y he aquí pasaba aquel pariente de quien Booz había hablado, y le dijo: Eh, fulano, ven acá y siéntate. Y él vino y se sentó.  
4:2 Entonces él tomó a diez varones de los ancianos de la ciudad, y dijo: Sentaos aquí. Y ellos se sentaron.  
4:3 Luego dijo al pariente: Noemí, que ha vuelto del campo de Moab, vende una parte de las tierras que tuvo nuestro hermano Elimelec.  
4:4 Y yo decidí hacértelo saber, y decirte que la compres en presencia de los que están aquí sentados, y de los ancianos de mi pueblo. Si tú quieres redimir, redime; y si no quieres redimir, decláramelo para que yo lo sepa; porque no hay otro que redima sino tú, y yo después de ti. Y él respondió: Yo redimiré.  
4:5 Entonces replicó Booz: El mismo día que compres las tierras de mano de Noemí, debes tomar también a Rut la moabita, mujer del difunto, para que restaures el nombre del muerto sobre su posesión.  
4:6 Y respondió el pariente: No puedo redimir para mí, no sea que dañe mi heredad. Redime tú, usando de mi derecho, porque yo no podré redimir.  
4:7 Había ya desde hacía tiempo esta costumbre en Israel tocante a la redención y al contrato, que para la confirmación de cualquier negocio, el uno se quitaba el zapato y lo daba a su compañero; y esto servía de testimonio en Israel.  
4:8 Entonces el pariente dijo a Booz: Tómalo tú. Y se quitó el zapato. 
4:9 Y Booz dijo a los ancianos y a todo el pueblo: Vosotros sois testigos hoy, de que he adquirido de mano de Noemí todo lo que fue de Elimelec, y todo lo que fue de Quelión y de Mahlón.  
4:10 Y que también tomo por mi mujer a Rut la moabita, mujer de Mahlón, para restaurar el nombre del difunto sobre su heredad, para que el nombre del muerto no se borre de entre sus hermanos y de la puerta de su lugar. Vosotros sois testigos hoy.  
4:11 Y dijeron todos los del pueblo que estaban a la puerta con los ancianos: Testigos somos. Jehová haga a la mujer que entra en tu casa como a Raquel y a Lea, las cuales edificaron la casa de Israel; y tú seas ilustre en Efrata, y seas de renombre en Belén.  
4:12 Y sea tu casa como la casa de Fares, el que Tamar dio a luz a Judá, por la descendencia que de esa joven te dé Jehová.  
4:13 Booz, pues, tomó a Rut, y ella fue su mujer; y se llegó a ella, y Jehová le dio que concibiese y diese a luz un hijo.  
4:14 Y las mujeres decían a Noemí: Loado sea Jehová, que hizo que no te faltase hoy pariente, cuyo nombre será celebrado en Israel;  
4:15 el cual será restaurador de tu alma, y sustentará tu vejez; pues tu nuera, que te ama, lo ha dado a luz; y ella es de más valor para ti que siete hijos.  
4:16 Y tomando Noemí el hijo, lo puso en su regazo, y fue su aya.  
4:17 Y le dieron nombre las vecinas, diciendo: Le ha nacido un hijo a Noemí; y lo llamaron Obed. Este es padre de Isaí, padre de David.  
4:18 Estas son las generaciones de Fares: Fares engendró a Hezrón,  
4:19 Hezrón engendró a Ram, y Ram engendró a Aminadab,  
4:20 Aminadab engendró a Naasón, y Naasón engendró a Salmón,  
4:21 Salmón engendró a Booz, y Booz engendró a Obed,  
4:22 Obed engendró a Isaí, e Isaí engendró a David.
