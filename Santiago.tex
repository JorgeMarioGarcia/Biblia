\chapter{La Epístola Universal De Santiago}


\section*{Capítulo 1 }
Salutación  
1:1 Santiago, siervo de Dios y del Señor Jesucristo, a las doce tribus que están en la dispersión: Salud.  
La sabiduría que viene de Dios  
1:2 Hermanos míos, tened por sumo gozo cuando os halléis en diversas pruebas,  
1:3 sabiendo que la prueba de vuestra fe produce paciencia.  
1:4 Mas tenga la paciencia su obra completa, para que seáis perfectos y cabales, sin que os falte cosa alguna.  
1:5 Y si alguno de vosotros tiene falta de sabiduría, pídala a Dios, el cual da a todos abundantemente y sin reproche, y le será dada.  
1:6 Pero pida con fe, no dudando nada; porque el que duda es semejante a la onda del mar, que es arrastrada por el viento y echada de una parte a otra.  
1:7 No piense, pues, quien tal haga, que recibirá cosa alguna del Señor.  
1:8 El hombre de doble ánimo es inconstante en todos sus caminos.  
1:9 El hermano que es de humilde condición, gloríese en su exaltación;  
1:10 pero el que es rico, en su humillación; porque él pasará como la flor de la hierba.  
1:11 Porque cuando sale el sol con calor abrasador, la hierba se seca, su flor se cae, y perece su hermosa apariencia; así también se marchitará el rico en todas sus empresas.  
Soportando las pruebas  
1:12 Bienaventurado el varón que soporta la tentación; porque cuando haya resistido la prueba, recibirá la corona de vida, que Dios ha prometido a los que le aman.  
1:13 Cuando alguno es tentado, no diga que es tentado de parte de Dios; porque Dios no puede ser tentado por el mal, ni él tienta a nadie;  
1:14 sino que cada uno es tentado, cuando de su propia concupiscencia es atraído y seducido. 
1:15 Entonces la concupiscencia, después que ha concebido, da a luz el pecado; y el pecado, siendo consumado, da a luz la muerte.  
1:16 Amados hermanos míos, no erréis.  
1:17 Toda buena dádiva y todo don perfecto desciende de lo alto, del Padre de las luces, en el cual no hay mudanza, ni sombra de variación.  
1:18 El, de su voluntad, nos hizo nacer por la palabra de verdad, para que seamos primicias de sus criaturas.  
Hacedores de la palabra  
1:19 Por esto, mis amados hermanos, todo hombre sea pronto para oír, tardo para hablar, tardo para airarse;  
1:20 porque la ira del hombre no obra la justicia de Dios.  
1:21 Por lo cual, desechando toda inmundicia y abundancia de malicia, recibid con mansedumbre la palabra implantada, la cual puede salvar vuestras almas.  
1:22 Pero sed hacedores de la palabra, y no tan solamente oidores, engañándoos a vosotros mismos.  
1:23 Porque si alguno es oidor de la palabra pero no hacedor de ella, éste es semejante al hombre que considera en un espejo su rostro natural.  
1:24 Porque él se considera a sí mismo, y se va, y luego olvida cómo era.  
1:25 Mas el que mira atentamente en la perfecta ley, la de la libertad, y persevera en ella, no siendo oidor olvidadizo, sino hacedor de la obra, éste será bienaventurado en lo que hace.  
1:26 Si alguno se cree religioso entre vosotros, y no refrena su lengua, sino que engaña su corazón, la religión del tal es vana.  
1:27 La religión pura y sin mácula delante de Dios el Padre es esta: Visitar a los huérfanos y a las viudas en sus tribulaciones, y guardarse sin mancha del mundo.  
\section*{Capítulo 2}
Amonestación contra la parcialidad  

2:1 Hermanos míos, que vuestra fe en nuestro glorioso Señor Jesucristo sea sin acepción de personas.  
2:2 Porque si en vuestra congregación entra un hombre con anillo de oro y con ropa espléndida, y también entra un pobre con vestido andrajoso,  
2:3 y miráis con agrado al que trae la ropa espléndida y le decís: Siéntate tú aquí en buen lugar; y decís al pobre: Estate tú allí en pie, o siéntate aquí bajo mi estrado;  
2:4 ¿no hacéis distinciones entre vosotros mismos, y venís a ser jueces con malos pensamientos?  
2:5 Hermanos míos amados, oíd: ¿No ha elegido Dios a los pobres de este mundo, para que sean ricos en fe y herederos del reino que ha prometido a los que le aman?  
2:6 Pero vosotros habéis afrentado al pobre. ¿No os oprimen los ricos, y no son ellos los mismos que os arrastran a los tribunales?  
2:7 ¿No blasfeman ellos el buen nombre que fue invocado sobre vosotros?  
2:8 Si en verdad cumplís la ley real, conforme a la Escritura: Amarás a tu prójimo como a ti mismo, bien hacéis;  
2:9 pero si hacéis acepción de personas, cometéis pecado, y quedáis convictos por la ley como transgresores.  
2:10 Porque cualquiera que guardare toda la ley, pero ofendiere en un punto, se hace culpable de todos.  
2:11 Porque el que dijo: No cometerás adulterio, también ha dicho: No matarás. Ahora bien, si no cometes adulterio, pero matas, ya te has hecho transgresor de la ley.  
2:12 Así hablad, y así haced, como los que habéis de ser juzgados por la ley de la libertad.  
2:13 Porque juicio sin misericordia se hará con aquel que no hiciere misericordia; y la misericordia triunfa sobre el juicio.  
La fe sin obras es muerta  
2:14 Hermanos míos, ¿de qué aprovechará si alguno dice que tiene fe, y no tiene obras? ¿Podrá la fe salvarle?  
2:15 Y si un hermano o una hermana están desnudos, y tienen necesidad del mantenimiento de cada día,  
2:16 y alguno de vosotros les dice: Id en paz, calentaos y saciaos, pero no les dais las cosas que son necesarias para el cuerpo, ¿de qué aprovecha?  
2:17 Así también la fe, si no tiene obras, es muerta en sí misma.  
2:18 Pero alguno dirá: Tú tienes fe, y yo tengo obras. Muéstrame tu fe sin tus obras, y yo te mostraré mi fe por mis obras.  
2:19 Tú crees que Dios es uno; bien haces. También los demonios creen, y tiemblan.  
2:20 ¿Mas quieres saber, hombre vano, que la fe sin obras es muerta?  
2:21 ¿No fue justificado por las obras Abraham nuestro padre, cuando ofreció a su hijo Isaac sobre el altar? 
2:22 ¿No ves que la fe actuó juntamente con sus obras, y que la fe se perfeccionó por las obras?  
2:23 Y se cumplió la Escritura que dice: Abraham creyó a Dios, y le fue contado por justicia, y fue llamado amigo de Dios. 
2:24 Vosotros veis, pues, que el hombre es justificado por las obras, y no solamente por la fe.  
2:25 Asimismo también Rahab la ramera, ¿no fue justificada por obras, cuando recibió a los mensajeros y los envió por otro camino? 
2:26 Porque como el cuerpo sin espíritu está muerto, así también la fe sin obras está muerta.  
\section*{Capítulo 3 }
La lengua  

3:1 Hermanos míos, no os hagáis maestros muchos de vosotros, sabiendo que recibiremos mayor condenación.  
3:2 Porque todos ofendemos muchas veces. Si alguno no ofende en palabra, éste es varón perfecto, capaz también de refrenar todo el cuerpo.  
3:3 He aquí nosotros ponemos freno en la boca de los caballos para que nos obedezcan, y dirigimos así todo su cuerpo.  
3:4 Mirad también las naves; aunque tan grandes, y llevadas de impetuosos vientos, son gobernadas con un muy pequeño timón por donde el que las gobierna quiere.  
3:5 Así también la lengua es un miembro pequeño, pero se jacta de grandes cosas. He aquí, ¡cuán grande bosque enciende un pequeño fuego!  
3:6 Y la lengua es un fuego, un mundo de maldad. La lengua está puesta entre nuestros miembros, y contamina todo el cuerpo, e inflama la rueda de la creación, y ella misma es inflamada por el infierno.  
3:7 Porque toda naturaleza de bestias, y de aves, y de serpientes, y de seres del mar, se doma y ha sido domada por la naturaleza humana;  
3:8 pero ningún hombre puede domar la lengua, que es un mal que no puede ser refrenado, llena de veneno mortal.  
3:9 Con ella bendecimos al Dios y Padre, y con ella maldecimos a los hombres, que están hechos a la semejanza de Dios. 
3:10 De una misma boca proceden bendición y maldición. Hermanos míos, esto no debe ser así.  
3:11 ¿Acaso alguna fuente echa por una misma abertura agua dulce y amarga?  
3:12 Hermanos míos, ¿puede acaso la higuera producir aceitunas, o la vid higos? Así también ninguna fuente puede dar agua salada y dulce.  
La sabiduría de lo alto  
3:13 ¿Quién es sabio y entendido entre vosotros? Muestre por la buena conducta sus obras en sabia mansedumbre.  
3:14 Pero si tenéis celos amargos y contención en vuestro corazón, no os jactéis, ni mintáis contra la verdad;  
3:15 porque esta sabiduría no es la que desciende de lo alto, sino terrenal, animal, diabólica.  
3:16 Porque donde hay celos y contención, allí hay perturbación y toda obra perversa.  
3:17 Pero la sabiduría que es de lo alto es primeramente pura, después pacífica, amable, benigna, llena de misericordia y de buenos frutos, sin incertidumbre ni hipocresía.  
3:18 Y el fruto de justicia se siembra en paz para aquellos que hacen la paz.  
\section*{Capítulo 4}
La amistad con el mundo  

4:1 ¿De dónde vienen las guerras y los pleitos entre vosotros? ¿No es de vuestras pasiones, las cuales combaten en vuestros miembros?  
4:2 Codiciáis, y no tenéis; matáis y ardéis de envidia, y no podéis alcanzar; combatís y lucháis, pero no tenéis lo que deseáis, porque no pedís.  
4:3 Pedís, y no recibís, porque pedís mal, para gastar en vuestros deleites.  
4:4 ¡Oh almas adúlteras! ¿No sabéis que la amistad del mundo es enemistad contra Dios? Cualquiera, pues, que quiera ser amigo del mundo, se constituye enemigo de Dios.  
4:5 ¿O pensáis que la Escritura dice en vano: El Espíritu que él ha hecho morar en nosotros nos anhela celosamente?  
4:6 Pero él da mayor gracia. Por esto dice: Dios resiste a los soberbios, y da gracia a los humildes. 
4:7 Someteos, pues, a Dios; resistid al diablo, y huirá de vosotros.  
4:8 Acercaos a Dios, y él se acercará a vosotros. Pecadores, limpiad las manos; y vosotros los de doble ánimo, purificad vuestros corazones.  
4:9 Afligíos, y lamentad, y llorad. Vuestra risa se convierta en lloro, y vuestro gozo en tristeza.  
4:10 Humillaos delante del Señor, y él os exaltará.  
Juzgando al hermano  
4:11 Hermanos, no murmuréis los unos de los otros. El que murmura del hermano y juzga a su hermano, murmura de la ley y juzga a la ley; pero si tú juzgas a la ley, no eres hacedor de la ley, sino juez.  
4:12 Uno solo es el dador de la ley, que puede salvar y perder; pero tú, ¿quién eres para que juzgues a otro?  
No os gloriéis del día de mañana  
4:13 ¡Vamos ahora! los que decís: Hoy y mañana iremos a tal ciudad, y estaremos allá un año, y traficaremos, y ganaremos;  
4:14 cuando no sabéis lo que será mañana. Porque ¿qué es vuestra vida? Ciertamente es neblina que se aparece por un poco de tiempo, y luego se desvanece.  
4:15 En lugar de lo cual deberíais decir: Si el Señor quiere, viviremos y haremos esto o aquello.  
4:16 Pero ahora os jactáis en vuestras soberbias. Toda jactancia semejante es mala;  
4:17 y al que sabe hacer lo bueno, y no lo hace, le es pecado.  
\section*{Capítulo 5 }
Contra los ricos opresores  

5:1 ¡Vamos ahora, ricos! Llorad y aullad por las miserias que os vendrán.  
5:2 Vuestras riquezas están podridas, y vuestras ropas están comidas de polilla.  
5:3 Vuestro oro y plata están enmohecidos; y su moho testificará contra vosotros, y devorará del todo vuestras carnes como fuego. Habéis acumulado tesoros para los días postreros. 
5:4 He aquí, clama el jornal de los obreros que han cosechado vuestras tierras, el cual por engaño no les ha sido pagado por vosotros; y los clamores de los que habían segado han entrado en los oídos del Señor de los ejércitos. 
5:5 Habéis vivido en deleites sobre la tierra, y sido disolutos; habéis engordado vuestros corazones como en día de matanza.  
5:6 Habéis condenado y dado muerte al justo, y él no os hace resistencia.  
Sed pacientes y orad  
5:7 Por tanto, hermanos, tened paciencia hasta la venida del Señor. Mirad cómo el labrador espera el precioso fruto de la tierra, aguardando con paciencia hasta que reciba la lluvia temprana y la tardía.  
5:8 Tened también vosotros paciencia, y afirmad vuestros corazones; porque la venida del Señor se acerca.  
5:9 Hermanos, no os quejéis unos contra otros, para que no seáis condenados; he aquí, el juez está delante de la puerta.  
5:10 Hermanos míos, tomad como ejemplo de aflicción y de paciencia a los profetas que hablaron en nombre del Señor.  
5:11 He aquí, tenemos por bienaventurados a los que sufren. Habéis oído de la paciencia de Job, y habéis visto el fin del Señor, que el Señor es muy misericordioso y compasivo. 
5:12 Pero sobre todo, hermanos míos, no juréis, ni por el cielo, ni por la tierra, ni por ningún otro juramento; sino que vuestro sí sea sí, y vuestro no sea no, para que no caigáis en condenación. 
5:13 ¿Está alguno entre vosotros afligido? Haga oración. ¿Está alguno alegre? Cante alabanzas.  
5:14 ¿Está alguno enfermo entre vosotros? Llame a los ancianos de la iglesia, y oren por él, ungiéndole con aceite en el nombre del Señor.  
5:15 Y la oración de fe salvará al enfermo, y el Señor lo levantará; y si hubiere cometido pecados, le serán perdonados.  
5:16 Confesaos vuestras ofensas unos a otros, y orad unos por otros, para que seáis sanados. La oración eficaz del justo puede mucho.  
5:17 Elías era hombre sujeto a pasiones semejantes a las nuestras, y oró fervientemente para que no lloviese, y no llovió sobre la tierra por tres años y seis meses. 
5:18 Y otra vez oró, y el cielo dio lluvia, y la tierra produjo su fruto. 
5:19 Hermanos, si alguno de entre vosotros se ha extraviado de la verdad, y alguno le hace volver,  
5:20 sepa que el que haga volver al pecador del error de su camino, salvará de muerte un alma, y cubrirá multitud de pecados.