\chapter{Primera Epístola Universal De San Juan Apóstol}


\section*{Capítulo 1}
La palabra de vida  
1:1 Lo que era desde el principio, lo que hemos oído, lo que hemos visto con nuestros ojos, lo que hemos contemplado, y palparon nuestras manos tocante al Verbo de vida  
1:2 (porque la vida fue manifestada, y la hemos visto, y testificamos, y os anunciamos la vida eterna, la cual estaba con el Padre, y se nos manifestó);  
1:3 lo que hemos visto y oído, eso os anunciamos, para que también vosotros tengáis comunión con nosotros; y nuestra comunión verdaderamente es con el Padre, y con su Hijo Jesucristo.  
1:4 Estas cosas os escribimos, para que vuestro gozo sea cumplido.  
Dios es luz  
1:5 Este es el mensaje que hemos oído de él, y os anunciamos: Dios es luz, y no hay ningunas tinieblas en él.  
1:6 Si decimos que tenemos comunión con él, y andamos en tinieblas, mentimos, y no practicamos la verdad;  
1:7 pero si andamos en luz, como él está en luz, tenemos comunión unos con otros, y la sangre de Jesucristo su Hijo nos limpia de todo pecado.  
1:8 Si decimos que no tenemos pecado, nos engañamos a nosotros mismos, y la verdad no está en nosotros.  
1:9 Si confesamos nuestros pecados, él es fiel y justo para perdonar nuestros pecados, y limpiarnos de toda maldad.  
1:10 Si decimos que no hemos pecado, le hacemos a él mentiroso, y su palabra no está en nosotros.  
\section*{Capítulo 2}
Cristo, nuestro abogado  

2:1 Hijitos míos, estas cosas os escribo para que no pequéis; y si alguno hubiere pecado, abogado tenemos para con el Padre, a Jesucristo el justo.  
2:2 Y él es la propiciación por nuestros pecados; y no solamente por los nuestros, sino también por los de todo el mundo.  
2:3 Y en esto sabemos que nosotros le conocemos, si guardamos sus mandamientos.  
2:4 El que dice: Yo le conozco, y no guarda sus mandamientos, el tal es mentiroso, y la verdad no está en él;  
2:5 pero el que guarda su palabra, en éste verdaderamente el amor de Dios se ha perfeccionado; por esto sabemos que estamos en él.  
2:6 El que dice que permanece en él, debe andar como él anduvo.  
El nuevo mandamiento  
2:7 Hermanos, no os escribo mandamiento nuevo, sino el mandamiento antiguo que habéis tenido desde el principio; este mandamiento antiguo es la palabra que habéis oído desde el principio.  
2:8 Sin embargo, os escribo un mandamiento nuevo, que es verdadero en él y en vosotros, porque las tinieblas van pasando, y la luz verdadera ya alumbra.  
2:9 El que dice que está en la luz, y aborrece a su hermano, está todavía en tinieblas.  
2:10 El que ama a su hermano, permanece en la luz, y en él no hay tropiezo.  
2:11 Pero el que aborrece a su hermano está en tinieblas, y anda en tinieblas, y no sabe a dónde va, porque las tinieblas le han cegado los ojos. 
2:12 Os escribo a vosotros, hijitos, porque vuestros pecados os han sido perdonados por su nombre.  
2:13 Os escribo a vosotros, padres, porque conocéis al que es desde el principio. Os escribo a vosotros, jóvenes, porque habéis vencido al maligno. Os escribo a vosotros, hijitos, porque habéis conocido al Padre.  
2:14 Os he escrito a vosotros, padres, porque habéis conocido al que es desde el principio. Os he escrito a vosotros, jóvenes, porque sois fuertes, y la palabra de Dios permanece en vosotros, y habéis vencido al maligno.  
2:15 No améis al mundo, ni las cosas que están en el mundo. Si alguno ama al mundo, el amor del Padre no está en él.  
2:16 Porque todo lo que hay en el mundo, los deseos de la carne, los deseos de los ojos, y la vanagloria de la vida, no proviene del Padre, sino del mundo.  
2:17 Y el mundo pasa, y sus deseos; pero el que hace la voluntad de Dios permanece para siempre.  
El anticristo  
2:18 Hijitos, ya es el último tiempo; y según vosotros oísteis que el anticristo viene, así ahora han surgido muchos anticristos; por esto conocemos que es el último tiempo.  
2:19 Salieron de nosotros, pero no eran de nosotros; porque si hubiesen sido de nosotros, habrían permanecido con nosotros; pero salieron para que se manifestase que no todos son de nosotros.  
2:20 Pero vosotros tenéis la unción del Santo, y conocéis todas las cosas.  
2:21 No os he escrito como si ignoraseis la verdad, sino porque la conocéis, y porque ninguna mentira procede de la verdad.  
2:22 ¿Quién es el mentiroso, sino el que niega que Jesús es el Cristo? Este es anticristo, el que niega al Padre y al Hijo.  
2:23 Todo aquel que niega al Hijo, tampoco tiene al Padre. El que confiesa al Hijo, tiene también al Padre.  
2:24 Lo que habéis oído desde el principio, permanezca en vosotros. Si lo que habéis oído desde el principio permanece en vosotros, también vosotros permaneceréis en el Hijo y en el Padre.  
2:25 Y esta es la promesa que él nos hizo, la vida eterna.  
2:26 Os he escrito esto sobre los que os engañan.  
2:27 Pero la unción que vosotros recibisteis de él permanece en vosotros, y no tenéis necesidad de que nadie os enseñe; así como la unción misma os enseña todas las cosas, y es verdadera, y no es mentira, según ella os ha enseñado, permaneced en él.  
2:28 Y ahora, hijitos, permaneced en él, para que cuando se manifieste, tengamos confianza, para que en su venida no nos alejemos de él avergonzados.  
2:29 Si sabéis que él es justo, sabed también que todo el que hace justicia es nacido de él.  
\section*{Capítulo 3 }
Hijos de Dios  

3:1 Mirad cuál amor nos ha dado el Padre, para que seamos llamados hijos de Dios; por esto el mundo no nos conoce, porque no le conoció a él.  
3:2 Amados, ahora somos hijos de Dios, y aún no se ha manifestado lo que hemos de ser; pero sabemos que cuando él se manifieste, seremos semejantes a él, porque le veremos tal como él es.  
3:3 Y todo aquel que tiene esta esperanza en él, se purifica a sí mismo, así como él es puro.  
3:4 Todo aquel que comete pecado, infringe también la ley; pues el pecado es infracción de la ley.  
3:5 Y sabéis que él apareció para quitar nuestros pecados, y no hay pecado en él.  
3:6 Todo aquel que permanece en él, no peca; todo aquel que peca, no le ha visto, ni le ha conocido.  
3:7 Hijitos, nadie os engañe; el que hace justicia es justo, como él es justo.  
3:8 El que practica el pecado es del diablo; porque el diablo peca desde el principio. Para esto apareció el Hijo de Dios, para deshacer las obras del diablo.  
3:9 Todo aquel que es nacido de Dios, no practica el pecado, porque la simiente de Dios permanece en él; y no puede pecar, porque es nacido de Dios.  
3:10 En esto se manifiestan los hijos de Dios, y los hijos del diablo: todo aquel que no hace justicia, y que no ama a su hermano, no es de Dios.  
3:11 Porque este es el mensaje que habéis oído desde el principio: Que nos amemos unos a otros. 
3:12 No como Caín, que era del maligno y mató a su hermano. ¿Y por qué causa le mató? Porque sus obras eran malas, y las de su hermano justas.  
3:13 Hermanos míos, no os extrañéis si el mundo os aborrece.  
3:14 Nosotros sabemos que hemos pasado de muerte a vida, en que amamos a los hermanos. El que no ama a su hermano, permanece en muerte.  
3:15 Todo aquel que aborrece a su hermano es homicida; y sabéis que ningún homicida tiene vida eterna permanente en él.  
3:16 En esto hemos conocido el amor, en que él puso su vida por nosotros; también nosotros debemos poner nuestras vidas por los hermanos.  
3:17 Pero el que tiene bienes de este mundo y ve a su hermano tener necesidad, y cierra contra él su corazón, ¿cómo mora el amor de Dios en él?  
3:18 Hijitos míos, no amemos de palabra ni de lengua, sino de hecho y en verdad.  
3:19 Y en esto conocemos que somos de la verdad, y aseguraremos nuestros corazones delante de él;  
3:20 pues si nuestro corazón nos reprende, mayor que nuestro corazón es Dios, y él sabe todas las cosas.  
3:21 Amados, si nuestro corazón no nos reprende, confianza tenemos en Dios;  
3:22 y cualquiera cosa que pidiéremos la recibiremos de él, porque guardamos sus mandamientos, y hacemos las cosas que son agradables delante de él.  
3:23 Y este es su mandamiento: Que creamos en el nombre de su Hijo Jesucristo, y nos amemos unos a otros como nos lo ha mandado. 
3:24 Y el que guarda sus mandamientos, permanece en Dios, y Dios en él. Y en esto sabemos que él permanece en nosotros, por el Espíritu que nos ha dado.  
\section*{Capítulo 4 }
El Espíritu de Dios y el espíritu del anticristo  

4:1 Amados, no creáis a todo espíritu, sino probad los espíritus si son de Dios; porque muchos falsos profetas han salido por el mundo.  
4:2 En esto conoced el Espíritu de Dios: Todo espíritu que confiesa que Jesucristo ha venido en carne, es de Dios;  
4:3 y todo espíritu que no confiesa que Jesucristo ha venido en carne, no es de Dios; y este es el espíritu del anticristo, el cual vosotros habéis oído que viene, y que ahora ya está en el mundo.  
4:4 Hijitos, vosotros sois de Dios, y los habéis vencido; porque mayor es el que está en vosotros, que el que está en el mundo.  
4:5 Ellos son del mundo; por eso hablan del mundo, y el mundo los oye.  
4:6 Nosotros somos de Dios; el que conoce a Dios, nos oye; el que no es de Dios, no nos oye. En esto conocemos el espíritu de verdad y el espíritu de error.  
Dios es amor  
4:7 Amados, amémonos unos a otros; porque el amor es de Dios. Todo aquel que ama, es nacido de Dios, y conoce a Dios.  
4:8 El que no ama, no ha conocido a Dios; porque Dios es amor.  
4:9 En esto se mostró el amor de Dios para con nosotros, en que Dios envió a su Hijo unigénito al mundo, para que vivamos por él.  
4:10 En esto consiste el amor: no en que nosotros hayamos amado a Dios, sino en que él nos amó a nosotros, y envió a su Hijo en propiciación por nuestros pecados.  
4:11 Amados, si Dios nos ha amado así, debemos también nosotros amarnos unos a otros.  
4:12 Nadie ha visto jamás a Dios. Si nos amamos unos a otros, Dios permanece en nosotros, y su amor se ha perfeccionado en nosotros.  
4:13 En esto conocemos que permanecemos en él, y él en nosotros, en que nos ha dado de su Espíritu.  
4:14 Y nosotros hemos visto y testificamos que el Padre ha enviado al Hijo, el Salvador del mundo.  
4:15 Todo aquel que confiese que Jesús es el Hijo de Dios, Dios permanece en él, y él en Dios.  
4:16 Y nosotros hemos conocido y creído el amor que Dios tiene para con nosotros. Dios es amor; y el que permanece en amor, permanece en Dios, y Dios en él.  
4:17 En esto se ha perfeccionado el amor en nosotros, para que tengamos confianza en el día del juicio; pues como él es, así somos nosotros en este mundo.  
4:18 En el amor no hay temor, sino que el perfecto amor echa fuera el temor; porque el temor lleva en sí castigo. De donde el que teme, no ha sido perfeccionado en el amor.  
4:19 Nosotros le amamos a él, porque él nos amó primero.  
4:20 Si alguno dice: Yo amo a Dios, y aborrece a su hermano, es mentiroso. Pues el que no ama a su hermano a quien ha visto, ¿cómo puede amar a Dios a quien no ha visto?  
4:21 Y nosotros tenemos este mandamiento de él: El que ama a Dios, ame también a su hermano.  
\section*{Capítulo 5 }
La fe que vence al mundo  

5:1 Todo aquel que cree que Jesús es el Cristo, es nacido de Dios; y todo aquel que ama al que engendró, ama también al que ha sido engendrado por él.  
5:2 En esto conocemos que amamos a los hijos de Dios, cuando amamos a Dios, y guardamos sus mandamientos.  
5:3 Pues este es el amor a Dios, que guardemos sus mandamientos; y sus mandamientos no son gravosos.  
5:4 Porque todo lo que es nacido de Dios vence al mundo; y esta es la victoria que ha vencido al mundo, nuestra fe.  
5:5 ¿Quién es el que vence al mundo, sino el que cree que Jesús es el Hijo de Dios?  
El testimonio del Espíritu  
5:6 Este es Jesucristo, que vino mediante agua y sangre; no mediante agua solamente, sino mediante agua y sangre. Y el Espíritu es el que da testimonio; porque el Espíritu es la verdad.  
5:7 Porque tres son los que dan testimonio en el cielo: el Padre, el Verbo y el Espíritu Santo; y estos tres son uno.  
5:8 Y tres son los que dan testimonio en la tierra: el Espíritu, el agua y la sangre; y estos tres concuerdan.  
5:9 Si recibimos el testimonio de los hombres, mayor es el testimonio de Dios; porque este es el testimonio con que Dios ha testificado acerca de su Hijo.  
5:10 El que cree en el Hijo de Dios, tiene el testimonio en sí mismo; el que no cree a Dios, le ha hecho mentiroso, porque no ha creído en el testimonio que Dios ha dado acerca de su Hijo.  
5:11 Y este es el testimonio: que Dios nos ha dado vida eterna; y esta vida está en su Hijo. 
5:12 El que tiene al Hijo, tiene la vida; el que no tiene al Hijo de Dios no tiene la vida.  
El conocimiento de la vida eterna  
5:13 Estas cosas os he escrito a vosotros que creéis en el nombre del Hijo de Dios, para que sepáis que tenéis vida eterna, y para que creáis en el nombre del Hijo de Dios.  
5:14 Y esta es la confianza que tenemos en él, que si pedimos alguna cosa conforme a su voluntad, él nos oye.  
5:15 Y si sabemos que él nos oye en cualquiera cosa que pidamos, sabemos que tenemos las peticiones que le hayamos hecho.  
5:16 Si alguno viere a su hermano cometer pecado que no sea de muerte, pedirá, y Dios le dará vida; esto es para los que cometen pecado que no sea de muerte. Hay pecado de muerte, por el cual yo no digo que se pida.  
5:17 Toda injusticia es pecado; pero hay pecado no de muerte.  
5:18 Sabemos que todo aquel que ha nacido de Dios, no practica el pecado, pues Aquel que fue engendrado por Dios le guarda, y el maligno no le toca.  
5:19 Sabemos que somos de Dios, y el mundo entero está bajo el maligno.  
5:20 Pero sabemos que el Hijo de Dios ha venido, y nos ha dado entendimiento para conocer al que es verdadero; y estamos en el verdadero, en su Hijo Jesucristo. Este es el verdadero Dios, y la vida eterna.  
5:21 Hijitos, guardaos de los ídolos. Amén.