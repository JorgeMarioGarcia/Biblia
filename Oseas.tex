\chapter{Oseas}
\section*{Capítulo 1 }
La esposa infiel de Oseas, y sus hijos 
1:1 Palabra de Jehová que vino a Oseas hijo de Beeri, en días de Uzías,  Jotam, Acaz y Ezequías, reyes de Judá, y en días de Jeroboam hijo de Joás, rey de Israel. 
1:2 El principio de la palabra de Jehová por medio de Oseas. Dijo Jehová a Oseas: Ve, tómate una mujer fornicaria, e hijos de fornicación; porque la tierra fornica apartándose de Jehová. 
1:3 Fue, pues, y tomó a Gomer hija de Diblaim, la cual concibió y le dio a luz un hijo. 
1:4 Y le dijo Jehová: Ponle por nombre Jezreel; porque de aquí a poco yo castigaré a la casa de Jehú por causa de la sangre de Jezreel, y haré cesar el reino de la casa de Israel. 
1:5 Y en aquel día quebraré yo el arco de Israel en el valle de Jezreel. 
1:6 Concibió ella otra vez, y dio a luz una hija. Y le dijo Dios: Ponle por nombre Lo-ruhama, porque no me compadeceré más de la casa de Israel, sino que los quitaré del todo. 
1:7 Mas de la casa de Judá tendré misericordia, y los salvaré por Jehová su Dios; y no los salvaré con arco, ni con espada, ni con batalla, ni con caballos ni jinetes. 
1:8 Después de haber destetado a Lo-ruhama, concibió y dio a luz un hijo. 
1:9 Y dijo Dios: Ponle por nombre Lo-ammi, porque vosotros no sois mi pueblo, ni yo seré vuestro Dios. 
1:10 Con todo, será el número de los hijos de Israel como la arena del mar, que no se puede medir ni contar. Y en el lugar en donde les fue dicho: Vosotros no sois pueblo mío, les será dicho: Sois hijos del Dios viviente. 
1:11 Y se congregarán los hijos de Judá y de Israel, y nombrarán un solo jefe, y subirán de la tierra; porque el día de Jezreel será grande. 
\section*{Capítulo 2 }
	El amor de Jehová hacia su pueblo infiel 
	
	2:1 Decid a vuestros hermanos: Ammi; y a vuestras hermanas: Ruhama. 
	2:2 Contended con vuestra madre, contended; porque ella no es mi mujer, ni yo su marido; aparte, pues, sus fornicaciones de su rostro, y sus adulterios de entre sus pechos; 
	2:3 no sea que yo la despoje y desnude, la ponga como el día en que nació, la haga como un desierto, la deje como tierra seca, y la mate de sed. 
	2:4 Ni tendré misericordia de sus hijos, porque son hijos de prostitución. 
	2:5 Porque su madre se prostituyó; la que los dio a luz se deshonró, porque dijo: Iré tras mis amantes, que me dan mi pan y mi agua, mi lana y mi lino, mi aceite y mi bebida. 
	2:6 Por tanto, he aquí yo rodearé de espinos su camino, y la cercaré con seto, y no hallará sus caminos. 
	2:7 Seguirá a sus amantes, y no los alcanzará; los buscará, y no los hallará. Entonces dirá: Iré y me volveré a mi primer marido; porque mejor me iba entonces que ahora. 
	2:8 Y ella no reconoció que yo le daba el trigo, el vino y el aceite, y que le multipliqué la plata y el oro que ofrecían a Baal. 
	2:9 Por tanto, yo volveré y tomaré mi trigo a su tiempo, y mi vino a su sazón, y quitaré mi lana y mi lino que había dado para cubrir su desnudez. 
	2:10 Y ahora descubriré yo su locura delante de los ojos de sus amantes, y nadie la librará de mi mano. 
	2:11 Haré cesar todo su gozo, sus fiestas, sus nuevas lunas y sus días de reposo, y todas sus festividades. 
	2:12 Y haré talar sus vides y sus higueras, de las cuales dijo: Mi salario son, salario que me han dado mis amantes. Y las reduciré a un matorral, y las comerán las bestias del campo. 
	2:13 Y la castigaré por los días en que incensaba a los baales, y se adornaba de sus zarcillos y de sus joyeles, y se iba tras sus amantes y se olvidaba de mí, dice Jehová. 
	2:14 Pero he aquí que yo la atraeré y la llevaré al desierto, y hablaré a su corazón. 
	2:15 Y le daré sus viñas desde allí, y el valle de Acor por puerta de esperanza; y allí cantará como en los tiempos de su juventud, y como en el día de su subida de la tierra de Egipto. 
	2:16 En aquel tiempo, dice Jehová, me llamarás Ishi, y nunca más me llamarás Baali. 
	2:17 Porque quitaré de su boca los nombres de los baales, y nunca más se mencionarán sus nombres. 
	2:18 En aquel tiempo haré para ti pacto con las bestias del campo, con las aves del cielo y con las serpientes de la tierra; y quitaré de la tierra arco y espada y guerra, y te haré dormir segura. 
	2:19 Y te desposaré conmigo para siempre; te desposaré conmigo en justicia, juicio, benignidad y misericordia. 
	2:20 Y te desposaré conmigo en fidelidad, y conocerás a Jehová. 
	2:21 En aquel tiempo responderé, dice Jehová, yo responderé a los cielos, y ellos responderán a la tierra. 
	2:22 Y la tierra responderá al trigo, al vino y al aceite, y ellos responderán a Jezreel. 
	2:23 Y la sembraré para mí en la tierra, y tendré misericordia de Lo-ruhama; y diré a Lo-ammi: Tú eres pueblo mío, y él dirá: Dios mío. 
	\section*{Capítulo 3 }
		Oseas y la adúltera 
		
		3:1 Me dijo otra vez Jehová: Ve, ama a una mujer amada de su compañero, aunque adúltera, como el amor de Jehová para con los hijos de Israel, los cuales miran a dioses ajenos, y aman tortas de pasas. 
		3:2 La compré entonces para mí por quince siclos de plata y un homer y medio de cebada. 
		3:3 Y le dije: Tú serás mía durante muchos días; no fornicarás, ni tomarás otro varón; lo mismo haré yo contigo. 
		3:4 Porque muchos días estarán los hijos de Israel sin rey, sin príncipe, sin sacrificio, sin estatua, sin efod y sin terafines. 
		3:5 Después volverán los hijos de Israel, y buscarán a Jehová su Dios, y a David su rey; y temerán a Jehová y a su bondad en el fin de los días. 
		\section*{Capítulo 4 }
			Controversia de Jehová con Israel 
			
			4:1 Oíd palabra de Jehová, hijos de Israel, porque Jehová contiende con los moradores de la tierra; porque no hay verdad, ni misericordia, ni conocimiento de Dios en la tierra. 
			4:2 Perjurar, mentir, matar, hurtar y adulterar prevalecen, y homicidio tras homicidio se suceden. 
			4:3 Por lo cual se enlutará la tierra, y se extenuará todo morador de ella, con las bestias del campo y las aves del cielo; y aun los peces del mar morirán. 
			4:4 Ciertamente hombre no contienda ni reprenda a hombre, porque tu pueblo es como los que resisten al sacerdote. 
			4:5 Caerás por tanto en el día, y caerá también contigo el profeta de noche; y a tu madre destruiré. 
			4:6 Mi pueblo fue destruido, porque le faltó conocimiento. Por cuanto desechaste el conocimiento, yo te echaré del sacerdocio; y porque olvidaste la ley de tu Dios, también yo me olvidaré de tus hijos. 
			4:7 Conforme a su grandeza, así pecaron contra mí; también yo cambiaré su honra en afrenta. 
			4:8 Del pecado de mi pueblo comen, y en su maldad levantan su alma. 
			4:9 Y será el pueblo como el sacerdote; le castigaré por su conducta, y le pagaré conforme a sus obras. 
			4:10 Comerán, pero no se saciarán; fornicarán, mas no se multiplicarán, porque dejaron de servir a Jehová. 
			4:11 Fornicación, vino y mosto quitan el juicio. 
			4:12 Mi pueblo a su ídolo de madera pregunta, y el leño le responde; porque espíritu de fornicaciones lo hizo errar, y dejaron a su Dios para fornicar. 
			4:13 Sobre las cimas de los montes sacrificaron, e incensaron sobre los collados, debajo de las encinas, álamos y olmos que tuviesen buena sombra; por tanto, vuestras hijas fornicarán, y adulterarán vuestras nueras. 
			4:14 No castigaré a vuestras hijas cuando forniquen, ni a vuestras nueras cuando adulteren; porque ellos mismos se van con rameras, y con malas mujeres sacrifican; por tanto, el pueblo sin entendimiento caerá. 
			4:15 Si fornicas tú, Israel, a lo menos no peque Judá; y no entréis en Gilgal, ni subáis a Bet-avén, ni juréis: Vive Jehová. 
			4:16 Porque como novilla indómita se apartó Israel; ¿los apacentará ahora Jehová como a corderos en lugar espacioso? 
			4:17 Efraín es dado a ídolos; déjalo. 
			4:18 Su bebida se corrompió; fornicaron sin cesar; sus príncipes amaron lo que avergüenza. 
			4:19 El viento los ató en sus alas, y de sus sacrificios serán avergonzados. 
			\section*{Capítulo 5}
				Castigo de la apostasía de Israel 
				
				5:1 Sacerdotes, oíd esto, y estad atentos, casa de Israel, y casa del rey, escuchad; porque para vosotros es el juicio, pues habéis sido lazo en Mizpa, y red tendida sobre Tabor. 
				5:2 Y haciendo víctimas han bajado hasta lo profundo; por tanto, yo castigaré a todos ellos. 
				5:3 Yo conozco a Efraín, e Israel no me es desconocido; porque ahora, oh Efraín, te has prostituido, y se ha contaminado Israel. 
				5:4 No piensan en convertirse a su Dios, porque espíritu de fornicación está en medio de ellos, y no conocen a Jehová. 
				5:5 La soberbia de Israel le desmentirá en su cara; Israel y Efraín tropezarán en su pecado, y Judá tropezará también con ellos. 
				5:6 Con sus ovejas y con sus vacas andarán buscando a Jehová, y no le hallarán; se apartó de ellos. 
				5:7 Contra Jehová prevaricaron, porque han engendrado hijos extraños; ahora en un solo mes serán consumidos ellos y sus heredades. 
				5:8 Tocad bocina en Gabaa, trompeta en Ramá: sonad alarma en Bet-avén; tiembla, oh Benjamín. 
				5:9 Efraín será asolado en el día del castigo; en las tribus de Israel hice conocer la verdad. 
				5:10 Los príncipes de Judá fueron como los que traspasan los linderos; derramaré sobre ellos como agua mi ira. 
				5:11 Efraín es vejado, quebrantado en juicio, porque quiso andar en pos de vanidades. 
				5:12 Yo, pues, seré como polilla a Efraín, y como carcoma a la casa de Judá. 
				5:13 Y verá Efraín su enfermedad, y Judá su llaga; irá entonces Efraín a Asiria, y enviará al rey Jareb; mas él no os podrá sanar, ni os curará la llaga. 
				5:14 Porque yo seré como león a Efraín, y como cachorro de león a la casa de Judá; yo, yo arrebataré, y me iré; tomaré, y no habrá quien liberte. 
				Insinceridad del arrepentimiento de Israel 
				5:15 Andaré y volveré a mi lugar, hasta que reconozcan su pecado y busquen mi rostro. En su angustia me buscarán. 
				\section*{Capítulo 6 }
					
6:1 Venid y volvamos a Jehová; porque él arrebató, y nos curará; hirió, y nos vendará. 
					6:2 Nos dará vida después de dos días; en el tercer día nos resucitará, y viviremos delante de él. 
					6:3 Y conoceremos, y proseguiremos en conocer a Jehová; como el alba está dispuesta su salida, y vendrá a nosotros como la lluvia, como la lluvia tardía y temprana a la tierra. 
					6:4 ¿Qué haré a ti, Efraín? ¿Qué haré a ti, oh Judá? La piedad vuestra es como nube de la mañana, y como el rocío de la madrugada, que se desvanece. 
					6:5 Por esta causa los corté por medio de los profetas, con las palabras de mi boca los maté; y tus juicios serán como luz que sale. 
					6:6 Porque misericordia quiero, y no sacrificio, y conocimiento de Dios más que holocaustos. 
					6:7 Mas ellos, cual Adán, traspasaron el pacto; allí prevaricaron contra mí. 
					6:8 Galaad, ciudad de hacedores de iniquidad, manchada de sangre. 
					6:9 Y como ladrones que esperan a algún hombre, así una compañía de sacerdotes mata en el camino hacia Siquem; así cometieron abominación. 
					6:10 En la casa de Israel he visto inmundicia; allí fornicó Efraín, y se contaminó Israel. 
					6:11 Para ti también, oh Judá, está preparada una siega, cuando yo haga volver el cautiverio de mi pueblo. 
\section*{Capítulo 7 }
						Iniquidad y rebelión de Israel 
						
						7:1 Mientras curaba yo a Israel, se descubrió la iniquidad de Efraín, y las maldades de Samaria; porque hicieron engaño; y entra el ladrón, y el salteador despoja por fuera. 
						7:2 Y no consideran en su corazón que tengo en memoria toda su maldad; ahora les rodearán sus obras; delante de mí están. 
						7:3 Con su maldad alegran al rey, y a los príncipes con sus mentiras. 
						7:4 Todos ellos son adúlteros; son como horno encendido por el hornero, que cesa de avivar el fuego después que está hecha la masa, hasta que se haya leudado. 
						7:5 En el día de nuestro rey los príncipes lo hicieron enfermar con copas de vino; extendió su mano con los escarnecedores. 
						7:6 Aplicaron su corazón, semejante a un horno, a sus artificios; toda la noche duerme su hornero; a la mañana está encendido como llama de fuego. 
						7:7 Todos ellos arden como un horno, y devoraron a sus jueces; cayeron todos sus reyes; no hay entre ellos quien a mí clame. 
						7:8 Efraín se ha mezclado con los demás pueblos; Efraín fue torta no volteada. 
						7:9 Devoraron extraños su fuerza, y él no lo supo; y aun canas le han cubierto, y él no lo supo. 
						7:10 Y la soberbia de Israel testificará contra él en su cara; y no se volvieron a Jehová su Dios, ni lo buscaron con todo esto. 
						7:11 Efraín fue como paloma incauta, sin entendimiento; llamarán a Egipto, acudirán a Asiria. 
						7:12 Cuando fueren, tenderé sobre ellos mi red; les haré caer como aves del cielo; les castigaré conforme a lo que se ha anunciado en sus congregaciones. 
						7:13 ¡Ay de ellos! porque se apartaron de mí; destrucción vendrá sobre ellos, porque contra mí se rebelaron; yo los redimí, y ellos hablaron mentiras contra mí. 
						7:14 Y no clamaron a mí con su corazón cuando gritaban sobre sus camas; para el trigo y el mosto se congregaron, se rebelaron contra mí. 
						7:15 Y aunque yo los enseñé y fortalecí sus brazos, contra mí pensaron mal. 
						7:16 Volvieron, pero no al Altísimo; fueron como arco engañoso; cayeron sus príncipes a espada por la soberbia de su lengua; esto será su escarnio en la tierra de Egipto. 
\section*{Capítulo 8 }
							Reprensión de la idolatría de Israel 
							
							8:1 Pon a tu boca trompeta. Como águila viene contra la casa de Jehová, porque traspasaron mi pacto, y se rebelaron contra mi ley. 
							8:2 A mí clamará Israel: Dios mío, te hemos conocido. 
							8:3 Israel desechó el bien; enemigo lo perseguirá. 
							8:4 Ellos establecieron reyes, pero no escogidos por mí; constituyeron príncipes, mas yo no lo supe; de su plata y de su oro hicieron ídolos para sí, para ser ellos mismos destruidos. 
							8:5 Tu becerro, oh Samaria, te hizo alejarte; se encendió mi enojo contra ellos, hasta que no pudieron alcanzar purificación. 
							8:6 Porque de Israel es también éste, y artífice lo hizo; no es Dios; por lo que será deshecho en pedazos el becerro de Samaria. 
							8:7 Porque sembraron viento, y torbellino segarán; no tendrán mies, ni su espiga hará harina; y si la hiciere, extraños la comerán. 
							8:8 Devorado será Israel; pronto será entre las naciones como vasija que no se estima. 
							8:9 Porque ellos subieron a Asiria, como asno montés para sí solo; Efraín con salario alquiló amantes. 
							8:10 Aunque alquilen entre las naciones, ahora las juntaré, y serán afligidos un poco de tiempo por la carga del rey y de los príncipes. 
							8:11 Porque multiplicó Efraín altares para pecar, tuvo altares para pecar. 
							8:12 Le escribí las grandezas de mi ley, y fueron tenidas por cosa extraña. 
							8:13 En los sacrificios de mis ofrendas sacrificaron carne, y comieron; no los quiso Jehová; ahora se acordará de su iniquidad, y castigará su pecado; ellos volverán a Egipto. 
							8:14 Olvidó, pues, Israel a su Hacedor, y edificó templos, y Judá multiplicó ciudades fortificadas; mas yo meteré fuego en sus ciudades, el cual consumirá sus palacios. 
							\section*{Capítulo 9 }
								Castigo de la persistente infidelidad de Israel 
								
								9:1 No te alegres, oh Israel, hasta saltar de gozo como los pueblos, pues has fornicado apartándote de tu Dios; amaste salario de ramera en todas las eras de trigo. 
								9:2 La era y el lagar no los mantendrán, y les fallará el mosto. 
								9:3 No quedarán en la tierra de Jehová, sino que volverá Efraín a Egipto y a Asiria, donde comerán vianda inmunda. 
								9:4 No harán libaciones a Jehová, ni sus sacrificios le serán gratos; como pan de enlutados les serán a ellos; todos los que coman de él serán inmundos. Será, pues, el pan de ellos para sí mismos; ese pan no entrará en la casa de Jehová. 
								9:5 ¿Qué haréis en el día de la solemnidad, y en el día de la fiesta de Jehová? 
								9:6 Porque he aquí se fueron ellos a causa de la destrucción. Egipto los recogerá, Menfis los enterrará. La ortiga conquistará lo deseable de su plata, y espino crecerá en sus moradas. 
								9:7 Vinieron los días del castigo, vinieron los días de la retribución; e Israel lo conocerá. Necio es el profeta, insensato es el varón de espíritu, a causa de la multitud de tu maldad, y grande odio. 
								9:8 Atalaya es Efraín para con mi Dios; el profeta es lazo de cazador en todos sus caminos, odio en la casa de su Dios. 
								9:9 Llegaron hasta lo más bajo en su corrupción, como en los días de Gabaa; ahora se acordará de su iniquidad, castigará su pecado. 
								9:10 Como uvas en el desierto hallé a Israel; como la fruta temprana de la higuera en su principio vi a vuestros padres. Ellos acudieron a Baal-peor, se apartaron para vergüenza, y se hicieron abominables como aquello que amaron. 
								9:11 La gloria de Efraín volará cual ave, de modo que no habrá nacimientos, ni embarazos, ni concepciones. 
								9:12 Y si llegaren a grandes sus hijos, los quitaré de entre los hombres, porque ¡ay de ellos también, cuando de ellos me aparte! 
								9:13 Efraín, según veo, es semejante a Tiro, situado en lugar delicioso; pero Efraín sacará sus hijos a la matanza. 
								9:14 Dales, oh Jehová, lo que les has de dar; dales matriz que aborte, y pechos enjutos. 
								9:15 Toda la maldad de ellos fue en Gilgal; allí, pues, les tomé aversión; por la perversidad de sus obras los echaré de mi casa; no los amaré más; todos sus príncipes son desleales. 
								9:16 Efraín fue herido, su raíz está seca, no dará más fruto; aunque engendren, yo mataré lo deseable de su vientre. 
								9:17 Mi Dios los desechará, porque ellos no le oyeron; y andarán errantes entre las naciones. 
								\section*{Capítulo 10 }
									
									10:1 Israel es una frondosa viña, que da abundante fruto para sí mismo; conforme a la abundancia de su fruto multiplicó también los altares, conforme a la bondad de su tierra aumentaron sus ídolos. 
									10:2 Está dividido su corazón. Ahora serán hallados culpables; Jehová demolerá sus altares, destruirá sus ídolos. 
									10:3 Seguramente dirán ahora: No tenemos rey, porque no temimos a Jehová; ¿y qué haría el rey por nosotros? 
									10:4 Han hablado palabras jurando en vano al hacer pacto; por tanto, el juicio florecerá como ajenjo en los surcos del campo. 
									10:5 Por las becerras de Bet-avén serán atemorizados los moradores de Samaria; porque su pueblo lamentará a causa del becerro, y sus sacerdotes que en él se regocijaban por su gloria, la cual será disipada. 
									10:6 Aun será él llevado a Asiria como presente al rey Jareb; Efraín será avergonzado, e Israel se avergonzará de su consejo. 
									10:7 De Samaria fue cortado su rey como espuma sobre la superficie de las aguas. 
									10:8 Y los lugares altos de Avén serán destruidos, el pecado de Israel; crecerá sobre sus altares espino y cardo. Y dirán a los montes: Cubridnos; y a los collados: Caed sobre nosotros. 
									10:9 Desde los días de Gabaa has pecado, oh Israel; allí estuvieron; no los tomó la batalla en Gabaa contra los inicuos. 
									10:10 Y los castigaré cuando lo desee; y pueblos se juntarán sobre ellos cuando sean atados por su doble crimen. 
									10:11 Efraín es novilla domada, que le gusta trillar, mas yo pasaré sobre su lozana cerviz; haré llevar yugo a Efraín; arará Judá, quebrará sus terrones Jacob. 
									10:12 Sembrad para vosotros en justicia, segad para vosotros en misericordia; haced para vosotros barbecho; porque es el tiempo de buscar a Jehová, hasta que venga y os enseñe justicia. 
									10:13 Habéis arado impiedad, y segasteis iniquidad; comeréis fruto de mentira, porque confiaste en tu camino y en la multitud de tus valientes. 
									10:14 Por tanto, en tus pueblos se levantará alboroto, y todas tus fortalezas serán destruidas, como destruyó Salmán a Bet-arbel en el día de la batalla, cuando la madre fue destrozada con los hijos. 
									10:15 Así hará a vosotros Bet-el, por causa de vuestra gran maldad; a la mañana será del todo cortado el rey de Israel. 
									\section*{Capítulo 11 }
										Dios se compadece de su pueblo obstinado 
										
										11:1 Cuando Israel era muchacho, yo lo amé, y de Egipto llamé a mi hijo. 
										11:2 Cuanto más yo los llamaba, tanto más se alejaban de mí; a los baales sacrificaban, y a los ídolos ofrecían sahumerios. 
										11:3 Yo con todo eso enseñaba a andar al mismo Efraín, tomándole de los brazos; y no conoció que yo le cuidaba. 
										11:4 Con cuerdas humanas los atraje, con cuerdas de amor; y fui para ellos como los que alzan el yugo de sobre su cerviz, y puse delante de ellos la comida. 
										11:5 No volverá a tierra de Egipto, sino que el asirio mismo será su rey, porque no se quisieron convertir. 
										11:6 Caerá espada sobre sus ciudades, y consumirá sus aldeas; las consumirá a causa de sus propios consejos. 
										11:7 Entre tanto, mi pueblo está adherido a la rebelión contra mí; aunque me llaman el Altísimo, ninguno absolutamente me quiere enaltecer. 
										11:8 ¿Cómo podré abandonarte, oh Efraín? ¿Te entregaré yo, Israel? ¿Cómo podré yo hacerte como Adma, o ponerte como a Zeboim? Mi corazón se conmueve dentro de mí, se inflama toda mi compasión. 
										11:9 No ejecutaré el ardor de mi ira, ni volveré para destruir a Efraín; porque Dios soy, y no hombre, el Santo en medio de ti; y no entraré en la ciudad. 
										11:10 En pos de Jehová caminarán; él rugirá como león; rugirá, y los hijos vendrán temblando desde el occidente. 
										11:11 Como ave acudirán velozmente de Egipto, y de la tierra de Asiria como paloma; y los haré habitar en sus casas, dice Jehová. 
										11:12 Me rodeó Efraín de mentira, y la casa de Israel de engaño. Judá aún gobierna con Dios, y es fiel con los santos. 
										\section*{Capítulo 12 }
											Efraín reprendido por su falsedad y opresión 
											
											12:1 Efraín se apacienta de viento, y sigue al solano; mentira y destrucción aumenta continuamente; porque hicieron pacto con los asirios, y el aceite se lleva a Egipto. 
											12:2 Pleito tiene Jehová con Judá para castigar a Jacob conforme a sus caminos; le pagará conforme a sus obras. 
											12:3 En el seno materno tomó por el calcañar a su hermano, y con su poder venció al ángel. 
											12:4 Venció al ángel, y prevaleció; lloró, y le rogó; en Bet-el le halló, y allí habló con nosotros. 
											12:5 Mas Jehová es Dios de los ejércitos; Jehová es su nombre. 
											12:6 Tú, pues, vuélvete a tu Dios; guarda misericordia y juicio, y en tu Dios confía siempre. 
											12:7 Mercader que tiene en su mano peso falso, amador de opresión, 
											12:8 Efraín dijo: Ciertamente he enriquecido, he hallado riquezas para mí; nadie hallará iniquidad en mí, ni pecado en todos mis trabajos. 
											12:9 Pero yo soy Jehová tu Dios desde la tierra de Egipto; aún te haré morar en tiendas, como en los días de la fiesta. 
											12:10 Y he hablado a los profetas, y aumenté la profecía, y por medio de los profetas usé parábolas. 
											12:11 ¿Es Galaad iniquidad? Ciertamente vanidad han sido; en Gilgal sacrificaron bueyes, y sus altares son como montones en los surcos del campo. 
											12:12 Pero Jacob huyó a tierra de Aram, Israel sirvió para adquirir mujer, y por adquirir mujer fue pastor. 
											12:13 Y por un profeta Jehová hizo subir a Israel de Egipto, y por un profeta fue guardado. 
											12:14 Efraín ha provocado a Dios con amarguras; por tanto, hará recaer sobre él la sangre que ha derramado, y su Señor le pagará su oprobio. 
											\section*{Capítulo 13 }
												Destrucción total de Efraín predicha 
												
												13:1 Cuando Efraín hablaba, hubo temor; fue exaltado en Israel; mas pecó en Baal, y murió. 
												13:2 Y ahora añadieron a su pecado, y de su plata se han hecho según su entendimiento imágenes de fundición, ídolos, toda obra de artífices, acerca de los cuales dicen a los hombres que sacrifican, que besen los becerros. 
												13:3 Por tanto, serán como la niebla de la mañana, y como el rocío de la madrugada que se pasa; como el tamo que la tempestad arroja de la era, y como el humo que sale de la chimenea. 
												13:4 Mas yo soy Jehová tu Dios desde la tierra de Egipto; no conocerás, pues, otro dios fuera de mí, ni otro salvador sino a mí. 
												13:5 Yo te conocí en el desierto, en tierra seca. 
												13:6 En sus pastos se saciaron, y repletos, se ensoberbeció su corazón; por esta causa se olvidaron de mí. 
												13:7 Por tanto, yo seré para ellos como león; como un leopardo en el camino los acecharé. 
												13:8 Como osa que ha perdido los hijos los encontraré, y desgarraré las fibras de su corazón, y allí los devoraré como león; fiera del campo los despedazará. 
												13:9 Te perdiste, oh Israel, mas en mí está tu ayuda. 
												13:10 ¿Dónde está tu rey, para que te guarde con todas tus ciudades; y tus jueces, de los cuales dijiste: Dame rey y príncipes? 
												13:11 Te di rey en mi furor, y te lo quité en mi ira. 
												13:12 Atada está la maldad de Efraín; su pecado está guardado. 
												13:13 Dolores de mujer que da a luz le vendrán; es un hijo no sabio, porque ya hace tiempo que no debiera detenerse al punto mismo de nacer. 
												13:14 De la mano del Seol los redimiré, los libraré de la muerte. Oh muerte, yo seré tu muerte; y seré tu destrucción, oh Seol; la compasión será escondida de mi vista. 
												13:15 Aunque él fructifique entre los hermanos, vendrá el solano, viento de Jehová; se levantará desde el desierto, y se secará su manantial, y se agotará su fuente; él saqueará el tesoro de todas sus preciosas alhajas. 
												13:16 Samaria será asolada, porque se rebeló contra su Dios; caerán a espada; sus niños serán estrellados, y sus mujeres encintas serán abiertas. 
												\section*{Capítulo 14 }
													Súplica a Israel para que vuelva a Jehová 
													
													14:1 Vuelve, oh Israel, a Jehová tu Dios; porque por tu pecado has caído. 
													14:2 Llevad con vosotros palabras de súplica, y volved a Jehová, y decidle: Quita toda iniquidad, y acepta el bien, y te ofreceremos la ofrenda de nuestros labios. 
													14:3 No nos librará el asirio; no montaremos en caballos, ni nunca más diremos a la obra de nuestras manos: Dioses nuestros; porque en ti el huérfano alcanzará misericordia. 
													14:4 Yo sanaré su rebelión, los amaré de pura gracia; porque mi ira se apartó de ellos. 
													14:5 Yo seré a Israel como rocío; él florecerá como lirio, y extenderá sus raíces como el Líbano. 
													14:6 Se extenderán sus ramas, y será su gloria como la del olivo, y perfumará como el Líbano. 
													14:7 Volverán y se sentarán bajo su sombra; serán vivificados como trigo, y florecerán como la vid; su olor será como de vino del Líbano. 
													14:8 Efraín dirá: ¿Qué más tendré ya con los ídolos? Yo lo oiré, y miraré; yo seré a él como la haya verde; de mí será hallado tu fruto. 
													14:9 ¿Quién es sabio para que entienda esto, y prudente para que lo sepa? Porque los caminos de Jehová son rectos, y los justos andarán por ellos; mas los rebeldes caerán en ellos.