\chapter{Segunda Epístola Universal De San Pedro Apóstol}

\section*{Capítulo 1}
Salutación  
1:1 Simón Pedro, siervo y apóstol de Jesucristo, a los que habéis alcanzado, por la justicia de nuestro Dios y Salvador Jesucristo, una fe igualmente preciosa que la nuestra:  
1:2 Gracia y paz os sean multiplicadas, en el conocimiento de Dios y de nuestro Señor Jesús.  
Partícipes de la naturaleza divina  
1:3 Como todas las cosas que pertenecen a la vida y a la piedad nos han sido dadas por su divino poder, mediante el conocimiento de aquel que nos llamó por su gloria y excelencia,  
1:4 por medio de las cuales nos ha dado preciosas y grandísimas promesas, para que por ellas llegaseis a ser participantes de la naturaleza divina, habiendo huido de la corrupción que hay en el mundo a causa de la concupiscencia;  
1:5 vosotros también, poniendo toda diligencia por esto mismo, añadid a vuestra fe virtud; a la virtud, conocimiento;  
1:6 al conocimiento, dominio propio; al dominio propio, paciencia; a la paciencia, piedad;  
1:7 a la piedad, afecto fraternal; y al afecto fraternal, amor.  
1:8 Porque si estas cosas están en vosotros, y abundan, no os dejarán estar ociosos ni sin fruto en cuanto al conocimiento de nuestro Señor Jesucristo.  
1:9 Pero el que no tiene estas cosas tiene la vista muy corta; es ciego, habiendo olvidado la purificación de sus antiguos pecados.  
1:10 Por lo cual, hermanos, tanto más procurad hacer firme vuestra vocación y elección; porque haciendo estas cosas, no caeréis jamás.  
1:11 Porque de esta manera os será otorgada amplia y generosa entrada en el reino eterno de nuestro Señor y Salvador Jesucristo.  
1:12 Por esto, yo no dejaré de recordaros siempre estas cosas, aunque vosotros las sepáis, y estéis confirmados en la verdad presente.  
1:13 Pues tengo por justo, en tanto que estoy en este cuerpo, el despertaros con amonestación;  
1:14 sabiendo que en breve debo abandonar el cuerpo, como nuestro Señor Jesucristo me ha declarado.  
1:15 También yo procuraré con diligencia que después de mi partida vosotros podáis en todo momento tener memoria de estas cosas.  
Testigos presenciales de la gloria de Cristo  
1:16 Porque no os hemos dado a conocer el poder y la venida de nuestro Señor Jesucristo siguiendo fábulas artificiosas, sino como habiendo visto con nuestros propios ojos su majestad.  
1:17 Pues cuando él recibió de Dios Padre honra y gloria, le fue enviada desde la magnífica gloria una voz que decía: Este es mi Hijo amado, en el cual tengo complacencia.  
1:18 Y nosotros oímos esta voz enviada del cielo, cuando estábamos con él en el monte santo. 
1:19 Tenemos también la palabra profética más segura, a la cual hacéis bien en estar atentos como a una antorcha que alumbra en lugar oscuro, hasta que el día esclarezca y el lucero de la mañana salga en vuestros corazones;  
1:20 entendiendo primero esto, que ninguna profecía de la Escritura es de interpretación privada,  
1:21 porque nunca la profecía fue traída por voluntad humana, sino que los santos hombres de Dios hablaron siendo inspirados por el Espíritu Santo.  
\section*{Capítulo 2}
Falsos profetas y falsos maestros  

2:1 Pero hubo también falsos profetas entre el pueblo, como habrá entre vosotros falsos maestros, que introducirán encubiertamente herejías destructoras, y aun negarán al Señor que los rescató, atrayendo sobre sí mismos destrucción repentina.  
2:2 Y muchos seguirán sus disoluciones, por causa de los cuales el camino de la verdad será blasfemado,  
2:3 y por avaricia harán mercadería de vosotros con palabras fingidas. Sobre los tales ya de largo tiempo la condenación no se tarda, y su perdición no se duerme.  
2:4 Porque si Dios no perdonó a los ángeles que pecaron, sino que arrojándolos al infierno los entregó a prisiones de oscuridad, para ser reservados al juicio;  
2:5 y si no perdonó al mundo antiguo, sino que guardó a Noé, pregonero de justicia, con otras siete personas, trayendo el diluvio sobre el mundo de los impíos; 
2:6 y si condenó por destrucción a las ciudades de Sodoma y de Gomorra, reduciéndolas a ceniza y poniéndolas de ejemplo a los que habían de vivir impíamente,  
2:7 y libró al justo Lot, abrumado por la nefanda conducta de los malvados
2:8 (porque este justo, que moraba entre ellos, afligía cada día su alma justa, viendo y oyendo los hechos inicuos de ellos),  
2:9 sabe el Señor librar de tentación a los piadosos, y reservar a los injustos para ser castigados en el día del juicio;  
2:10 y mayormente a aquellos que, siguiendo la carne, andan en concupiscencia e inmundicia, y desprecian el señorío. Atrevidos y contumaces, no temen decir mal de las potestades superiores,  
2:11 mientras que los ángeles, que son mayores en fuerza y en potencia, no pronuncian juicio de maldición contra ellas delante del Señor.  
2:12 Pero éstos, hablando mal de cosas que no entienden, como animales irracionales, nacidos para presa y destrucción, perecerán en su propia perdición,  
2:13 recibiendo el galardón de su injusticia, ya que tienen por delicia el gozar de deleites cada día. Estos son inmundicias y manchas, quienes aun mientras comen con vosotros, se recrean en sus errores.  
2:14 Tienen los ojos llenos de adulterio, no se sacian de pecar, seducen a las almas inconstantes, tienen el corazón habituado a la codicia, y son hijos de maldición.  
2:15 Han dejado el camino recto, y se han extraviado siguiendo el camino de Balaam hijo de Beor, el cual amó el premio de la maldad,  
2:16 y fue reprendido por su iniquidad; pues una muda bestia de carga, hablando con voz de hombre, refrenó la locura del profeta. 
2:17 Estos son fuentes sin agua, y nubes empujadas por la tormenta; para los cuales la más densa oscuridad está reservada para siempre.  
2:18 Pues hablando palabras infladas y vanas, seducen con concupiscencias de la carne y disoluciones a los que verdaderamente habían huido de los que viven en error.  
2:19 Les prometen libertad, y son ellos mismos esclavos de corrupción. Porque el que es vencido por alguno es hecho esclavo del que lo venció.  
2:20 Ciertamente, si habiéndose ellos escapado de las contaminaciones del mundo, por el conocimiento del Señor y Salvador Jesucristo, enredándose otra vez en ellas son vencidos, su postrer estado viene a ser peor que el primero.  
2:21 Porque mejor les hubiera sido no haber conocido el camino de la justicia, que después de haberlo conocido, volverse atrás del santo mandamiento que les fue dado.  
2:22 Pero les ha acontecido lo del verdadero proverbio: El perro vuelve a su vómito, y la puerca lavada a revolcarse en el cieno.  
\section*{Capítulo 3 }
El día del Señor vendrá  

3:1 Amados, esta es la segunda carta que os escribo, y en ambas despierto con exhortación vuestro limpio entendimiento,  
3:2 para que tengáis memoria de las palabras que antes han sido dichas por los santos profetas, y del mandamiento del Señor y Salvador dado por vuestros apóstoles;  
3:3 sabiendo primero esto, que en los postreros días vendrán burladores, andando según sus propias concupiscencias, 
3:4 y diciendo: ¿Dónde está la promesa de su advenimiento? Porque desde el día en que los padres durmieron, todas las cosas permanecen así como desde el principio de la creación.  
3:5 Estos ignoran voluntariamente, que en el tiempo antiguo fueron hechos por la palabra de Dios los cielos, y también la tierra, que proviene del agua y por el agua subsiste, 
3:6 por lo cual el mundo de entonces pereció anegado en agua; 
3:7 pero los cielos y la tierra que existen ahora, están reservados por la misma palabra, guardados para el fuego en el día del juicio y de la perdición de los hombres impíos.  
3:8 Mas, oh amados, no ignoréis esto: que para con el Señor un día es como mil años, y mil años como un día. 
3:9 El Señor no retarda su promesa, según algunos la tienen por tardanza, sino que es paciente para con nosotros, no queriendo que ninguno perezca, sino que todos procedan al arrepentimiento.  
3:10 Pero el día del Señor vendrá como ladrón en la noche; en el cual los cielos pasarán con grande estruendo, y los elementos ardiendo serán deshechos, y la tierra y las obras que en ella hay serán quemadas.  
3:11 Puesto que todas estas cosas han de ser deshechas, ¡cómo no debéis vosotros andar en santa y piadosa manera de vivir,  
3:12 esperando y apresurándoos para la venida del día de Dios, en el cual los cielos, encendiéndose, serán deshechos, y los elementos, siendo quemados, se fundirán!  
3:13 Pero nosotros esperamos, según sus promesas, cielos nuevos y tierra nueva, en los cuales mora la justicia. 
3:14 Por lo cual, oh amados, estando en espera de estas cosas, procurad con diligencia ser hallados por él sin mancha e irreprensibles, en paz.  
3:15 Y tened entendido que la paciencia de nuestro Señor es para salvación; como también nuestro amado hermano Pablo, según la sabiduría que le ha sido dada, os ha escrito,  
3:16 casi en todas sus epístolas, hablando en ellas de estas cosas; entre las cuales hay algunas difíciles de entender, las cuales los indoctos e inconstantes tuercen, como también las otras Escrituras, para su propia perdición.  
3:17 Así que vosotros, oh amados, sabiéndolo de antemano, guardaos, no sea que arrastrados por el error de los inicuos, caigáis de vuestra firmeza.  
3:18 Antes bien, creced en la gracia y el conocimiento de nuestro Señor y Salvador Jesucristo. A él sea gloria ahora y hasta el día de la eternidad. Amén.
