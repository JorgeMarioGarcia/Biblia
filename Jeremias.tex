\chapter{Jeremías}

\section*{Capítulo 1 }
Llamamiento y misión de Jeremías  1:1 Las palabras de Jeremías hijo de Hilcías, de los sacerdotes que estuvieron en Anatot, en tierra de Benjamín. 
1:2 Palabra de Jehová que le vino en los días de Josías hijo de Amón, rey de Judá, en el año decimotercero de su reinado. 
1:3 Le vino también en días de Joacim hijo de Josías, rey de Judá, hasta el fin del año undécimo de Sedequías hijo de Josías, rey de Judá, hasta la cautividad de Jerusalén en el mes quinto. 
1:4 Vino, pues, palabra de Jehová a mí, diciendo: 
1:5 Antes que te formase en el vientre te conocí, y antes que nacieses te santifiqué, te di por profeta a las naciones. 
1:6 Y yo dije: ¡Ah! ¡ah, Señor Jehová! He aquí, no sé hablar, porque soy niño. 
1:7 Y me dijo Jehová: No digas: Soy un niño; porque a todo lo que te envíe irás tú, y dirás todo lo que te mande. 
1:8 No temas delante de ellos, porque contigo estoy para librarte, dice Jehová. 
1:9 Y extendió Jehová su mano y tocó mi boca, y me dijo Jehová: He aquí he puesto mis palabras en tu boca. 
1:10 Mira que te he puesto en este día sobre naciones y sobre reinos, para arrancar y para destruir, para arruinar y para derribar, para edificar y para plantar. 
1:11 La palabra de Jehová vino a mí, diciendo: ¿Qué ves tú, Jeremías? Y dije: Veo una vara de almendro. 
1:12 Y me dijo Jehová: Bien has visto; porque yo apresuro mi palabra para ponerla por obra. 
1:13 Vino a mí la palabra de Jehová por segunda vez, diciendo: ¿Qué ves tú? Y dije: Veo una olla que hierve; y su faz está hacia el norte. 
1:14 Me dijo Jehová: Del norte se soltará el mal sobre todos los moradores de esta tierra. 
1:15 Porque he aquí que yo convoco a todas las familias de los reinos del norte, dice Jehová; y vendrán, y pondrá cada uno su campamento a la entrada de las puertas de Jerusalén, y junto a todos sus muros en derredor, y contra todas las ciudades de Judá. 
1:16 Y a causa de toda su maldad, proferiré mis juicios contra los que me dejaron, e incensaron a dioses extraños, y la obra de sus manos adoraron. 
1:17 Tú, pues, ciñe tus lomos, levántate, y háblales todo cuanto te mande; no temas delante de ellos, para que no te haga yo quebrantar delante de ellos. 
1:18 Porque he aquí que yo te he puesto en este día como ciudad fortificada, como columna de hierro, y como muro de bronce contra toda esta tierra, contra los reyes de Judá, sus príncipes, sus sacerdotes, y el pueblo de la tierra. 
1:19 Y pelearán contra ti, pero no te vencerán; porque yo estoy contigo, dice Jehová, para librarte. 
\section*{Capítulo 2 }
Jehová y la apostasía de Israel 
 
2:1 Vino a mí palabra de Jehová, diciendo: 
2:2 Anda y clama a los oídos de Jerusalén, diciendo: Así dice Jehová: Me he acordado de ti, de la fidelidad de tu juventud, del amor de tu desposorio, cuando andabas en pos de mí en el desierto, en tierra no sembrada. 
2:3 Santo era Israel a Jehová, primicias de sus nuevos frutos. Todos los que le devoraban eran culpables; mal venía sobre ellos, dice Jehová. 
2:4 Oíd la palabra de Jehová, casa de Jacob, y todas las familias de la casa de Israel. 
2:5 Así dijo Jehová: ¿Qué maldad hallaron en mí vuestros padres, que se alejaron de mí, y se fueron tras la vanidad y se hicieron vanos? 
2:6 Y no dijeron: ¿Dónde está Jehová, que nos hizo subir de la tierra de Egipto, que nos condujo por el desierto, por una tierra desierta y despoblada, por tierra seca y de sombra de muerte, por una tierra por la cual no pasó varón, ni allí habitó hombre? 
2:7 Y os introduje en tierra de abundancia, para que comieseis su fruto y su bien; pero entrasteis y contaminasteis mi tierra, e hicisteis abominable mi heredad. 
2:8 Los sacerdotes no dijeron: ¿Dónde está Jehová? y los que tenían la ley no me conocieron; y los pastores se rebelaron contra mí, y los profetas profetizaron en nombre de Baal, y anduvieron tras lo que no aprovecha. 
2:9 Por tanto, contenderé aún con vosotros, dijo Jehová, y con los hijos de vuestros hijos pleitearé. 
2:10 Porque pasad a las costas de Quitim y mirad; y enviad a Cedar, y considerad cuidadosamente, y ved si se ha hecho cosa semejante a esta. 
2:11 ¿Acaso alguna nación ha cambiado sus dioses, aunque ellos no son dioses? Sin embargo, mi pueblo ha trocado su gloria por lo que no aprovecha. 
2:12 Espantaos, cielos, sobre esto, y horrorizaos; desolaos en gran manera, dijo Jehová. 
2:13 Porque dos males ha hecho mi pueblo: me dejaron a mí, fuente de agua viva, y cavaron para sí cisternas, cisternas rotas que no retienen agua. 
2:14 ¿Es Israel siervo? ¿es esclavo? ¿Por qué ha venido a ser presa? 
2:15 Los cachorros del león rugieron contra él, alzaron su voz, y asolaron su tierra; quemadas están sus ciudades, sin morador. 
2:16 Aun los hijos de Menfis y de Tafnes te quebrantaron la coronilla. 
2:17 ¿No te acarreó esto el haber dejado a Jehová tu Dios, cuando te conducía por el camino? 
2:18 Ahora, pues, ¿qué tienes tú en el camino de Egipto, para que bebas agua del Nilo? ¿Y qué tienes tú en el camino de Asiria, para que bebas agua del Eufrates? 
2:19 Tu maldad te castigará, y tus rebeldías te condenarán; sabe, pues, y ve cuán malo y amargo es el haber dejado tú a Jehová tu Dios, y faltar mi temor en ti, dice el Señor, Jehová de los ejércitos. 
2:20 Porque desde muy atrás rompiste tu yugo y tus ataduras, y dijiste: No serviré. Con todo eso, sobre todo collado alto y debajo de todo árbol frondoso te echabas como ramera. 
2:21 Te planté de vid escogida, simiente verdadera toda ella; ¿cómo, pues, te me has vuelto sarmiento de vid extraña? 
2:22 Aunque te laves con lejía, y amontones jabón sobre ti, la mancha de tu pecado permanecerá aún delante de mí, dijo Jehová el Señor. 
2:23 ¿Cómo puedes decir: No soy inmunda, nunca anduve tras los baales? Mira tu proceder en el valle, conoce lo que has hecho, dromedaria ligera que tuerce su camino, 
2:24 asna montés acostumbrada al desierto, que en su ardor olfatea el viento. De su lujuria, ¿quién la detendrá? Todos los que la buscaren no se fatigarán, porque en el tiempo de su celo la hallarán. 
2:25 Guarda tus pies de andar descalzos, y tu garganta de la sed. Mas dijiste: No hay remedio en ninguna manera, porque a extraños he amado, y tras ellos he de ir. 
2:26 Como se avergüenza el ladrón cuando es descubierto, así se avergonzará la casa de Israel, ellos, sus reyes, sus príncipes, sus sacerdotes y sus profetas, 
2:27 que dicen a un leño: Mi padre eres tú; y a una piedra: Tú me has engendrado. Porque me volvieron la cerviz, y no el rostro; y en el tiempo de su calamidad dicen: Levántate, y líbranos. 
2:28 ¿Y dónde están tus dioses que hiciste para ti? Levántense ellos, a ver si te podrán librar en el tiempo de tu aflicción; porque según el número de tus ciudades, oh Judá, fueron tus dioses. 
2:29 ¿Por qué porfías conmigo? Todos vosotros prevaricasteis contra mí, dice Jehová. 
2:30 En vano he azotado a vuestros hijos; no han recibido corrección. Vuestra espada devoró a vuestros profetas como león destrozador. 
2:31 ¡Oh generación! atended vosotros a la palabra de Jehová. ¿He sido yo un desierto para Israel, o tierra de tinieblas? ¿Por qué ha dicho mi pueblo: Somos libres; nunca más vendremos a ti? 
2:32 ¿Se olvida la virgen de su atavío, o la desposada de sus galas? Pero mi pueblo se ha olvidado de mí por innumerables días. 
2:33 ¿Por qué adornas tu camino para hallar amor? Aun a las malvadas enseñaste tus caminos. 
2:34 Aun en tus faldas se halló la sangre de los pobres, de los inocentes. No los hallaste en ningún delito; sin embargo, en todas estas cosas dices: 
2:35 Soy inocente, de cierto su ira se apartó de mí. He aquí yo entraré en juicio contigo, porque dijiste: No he pecado. 
2:36 ¿Para qué discurres tanto, cambiando tus caminos? También serás avergonzada de Egipto, como fuiste avergonzada de Asiria. 
2:37 También de allí saldrás con tus manos sobre tu cabeza, porque Jehová desechó a aquellos en quienes tú confiabas, y no prosperarás por ellos. 
\section*{Capítulo 3 }
 
3:1 Dicen: Si alguno dejare a su mujer, y yéndose ésta de él se juntare a otro hombre, ¿volverá a ella más? ¿No será tal tierra del todo amancillada? Tú, pues, has fornicado con muchos amigos; mas ¡vuélvete a mí! dice Jehová. 
3:2 Alza tus ojos a las alturas, y ve en qué lugar no te hayas prostituido. Junto a los caminos te sentabas para ellos como árabe en el desierto, y con tus fornicaciones y con tu maldad has contaminado la tierra. 
3:3 Por esta causa las aguas han sido detenidas, y faltó la lluvia tardía; y has tenido frente de ramera, y no quisiste tener vergüenza. 
3:4 A lo menos desde ahora, ¿no me llamarás a mí, Padre mío, guiador de mi juventud? 
3:5 ¿Guardará su enojo para siempre? ¿Eternamente lo guardará? He aquí que has hablado y hecho cuantas maldades pudiste. 
Jehová exhorta a Israel y a Judá al arrepentimiento 
3:6 Me dijo Jehová en días del rey Josías: ¿Has visto lo que ha hecho la rebelde Israel? Ella se va sobre todo monte alto y debajo de todo árbol frondoso, y allí fornica. 
3:7 Y dije: Después de hacer todo esto, se volverá a mí; pero no se volvió, y lo vio su hermana la rebelde Judá. 
3:8 Ella vio que por haber fornicado la rebelde Israel, yo la había despedido y dado carta de repudio; pero no tuvo temor la rebelde Judá su hermana, sino que también fue ella y fornicó. 
3:9 Y sucedió que por juzgar ella cosa liviana su fornicación, la tierra fue contaminada, y adulteró con la piedra y con el leño. 
3:10 Con todo esto, su hermana la rebelde Judá no se volvió a mí de todo corazón, sino fingidamente, dice Jehová. 
3:11 Y me dijo Jehová: Ha resultado justa la rebelde Israel en comparación con la desleal Judá. 
3:12 Ve y clama estas palabras hacia el norte, y di: Vuélvete, oh rebelde Israel, dice Jehová; no haré caer mi ira sobre ti, porque misericordioso soy yo, dice Jehová, no guardaré para siempre el enojo. 
3:13 Reconoce, pues, tu maldad, porque contra Jehová tu Dios has prevaricado, y fornicaste con los extraños debajo de todo árbol frondoso, y no oíste mi voz, dice Jehová. 
3:14 Convertíos, hijos rebeldes, dice Jehová, porque yo soy vuestro esposo; y os tomaré uno de cada ciudad, y dos de cada familia, y os introduciré en Sion; 
3:15 y os daré pastores según mi corazón, que os apacienten con ciencia y con inteligencia. 
3:16 Y acontecerá que cuando os multipliquéis y crezcáis en la tierra, en esos días, dice Jehová, no se dirá más: Arca del pacto de Jehová; ni vendrá al pensamiento, ni se acordarán de ella, ni la echarán de menos, ni se hará otra. 
3:17 En aquel tiempo llamarán a Jerusalén: Trono de Jehová, y todas las naciones vendrán a ella en el nombre de Jehová en Jerusalén; ni andarán más tras la dureza de su malvado corazón. 
3:18 En aquellos tiempos irán de la casa de Judá a la casa de Israel, y vendrán juntamente de la tierra del norte a la tierra que hice heredar a vuestros padres. 
3:19 Yo preguntaba: ¿Cómo os pondré por hijos, y os daré la tierra deseable, la rica heredad de las naciones? Y dije: Me llamaréis: Padre mío, y no os apartaréis de en pos de mí. 
3:20 Pero como la esposa infiel abandona a su compañero, así prevaricasteis contra mí, oh casa de Israel, dice Jehová. 
3:21 Voz fue oída sobre las alturas, llanto de los ruegos de los hijos de Israel; porque han torcido su camino, de Jehová su Dios se han olvidado. 
3:22 Convertíos, hijos rebeldes, y sanaré vuestras rebeliones. He aquí nosotros venimos a ti, porque tú eres Jehová nuestro Dios. 
3:23 Ciertamente vanidad son los collados, y el bullicio sobre los montes; ciertamente en Jehová nuestro Dios está la salvación de Israel. 
3:24 Confusión consumió el trabajo de nuestros padres desde nuestra juventud; sus ovejas, sus vacas, sus hijos y sus hijas. 
3:25 Yacemos en nuestra confusión, y nuestra afrenta nos cubre; porque pecamos contra Jehová nuestro Dios, nosotros y nuestros padres, desde nuestra juventud y hasta este día, y no hemos escuchado la voz de Jehová nuestro Dios. 
\section*{Capítulo 4 }
 
4:1 Si te volvieres, oh Israel, dice Jehová, vuélvete a mí. Y si quitares de delante de mí tus abominaciones, y no anduvieres de acá para allá, 
4:2 y jurares: Vive Jehová, en verdad, en juicio y en justicia, entonces las naciones serán benditas en él, y en él se gloriarán. 
4:3 Porque así dice Jehová a todo varón de Judá y de Jerusalén: Arad campo para vosotros, y no sembréis entre espinos. 
4:4 Circuncidaos a Jehová, y quitad el prepucio de vuestro corazón, varones de Judá y moradores de Jerusalén; no sea que mi ira salga como fuego, y se encienda y no haya quien la apague, por la maldad de vuestras obras. 
Judá es amenazada de invasión 
4:5 Anunciad en Judá, y proclamad en Jerusalén, y decid: Tocad trompeta en la tierra; pregonad, juntaos, y decid: Reuníos, y entrémonos en las ciudades fortificadas. 
4:6 Alzad bandera en Sion, huid, no os detengáis; porque yo hago venir mal del norte, y quebrantamiento grande. 
4:7 El león sube de la espesura, y el destruidor de naciones está en marcha, y ha salido de su lugar para poner tu tierra en desolación; tus ciudades quedarán asoladas y sin morador. 
4:8 Por esto vestíos de cilicio, endechad y aullad; porque la ira de Jehová no se ha apartado de nosotros. 
4:9 En aquel día, dice Jehová, desfallecerá el corazón del rey y el corazón de los príncipes, y los sacerdotes estarán atónitos, y se maravillarán los profetas. 
4:10 Y dije: ¡Ay, ay, Jehová Dios! Verdaderamente en gran manera has engañado a este pueblo y a Jerusalén, diciendo: Paz tendréis; pues la espada ha venido hasta el alma. 
4:11 En aquel tiempo se dirá a este pueblo y a Jerusalén: Viento seco de las alturas del desierto vino a la hija de mi pueblo, no para aventar, ni para limpiar. 
4:12 Viento más vehemente que este vendrá a mí; y ahora yo pronunciaré juicios contra ellos. 
4:13 He aquí que subirá como nube, y su carro como torbellino; más ligeros son sus caballos que las águilas. ¡Ay de nosotros, porque entregados somos a despojo! 
4:14 Lava tu corazón de maldad, oh Jerusalén, para que seas salva. ¿Hasta cuándo permitirás en medio de ti los pensamientos de iniquidad? 
4:15 Porque una voz trae las nuevas desde Dan, y hace oír la calamidad desde el monte de Efraín. 
4:16 Decid a las naciones: He aquí, haced oír sobre Jerusalén: Guardas vienen de tierra lejana, y lanzarán su voz contra las ciudades de Judá. 
4:17 Como guardas de campo estuvieron en derredor de ella, porque se rebeló contra mí, dice Jehová. 
4:18 Tu camino y tus obras te hicieron esto; esta es tu maldad, por lo cual amargura penetrará hasta tu corazón. 
4:19 ¡Mis entrañas, mis entrañas! Me duelen las fibras de mi corazón; mi corazón se agita dentro de mí; no callaré; porque sonido de trompeta has oído, oh alma mía, pregón de guerra. 
4:20 Quebrantamiento sobre quebrantamiento es anunciado; porque toda la tierra es destruida; de repente son destruidas mis tiendas, en un momento mis cortinas. 
4:21 ¿Hasta cuándo he de ver bandera, he de oír sonido de trompeta? 
4:22 Porque mi pueblo es necio, no me conocieron; son hijos ignorantes y no son entendidos; sabios para hacer el mal, pero hacer el bien no supieron. 
4:23 Miré a la tierra, y he aquí que estaba asolada y vacía; y a los cielos, y no había en ellos luz. 
4:24 Miré a los montes, y he aquí que temblaban, y todos los collados fueron destruidos. 
4:25 Miré, y no había hombre, y todas las aves del cielo se habían ido. 
4:26 Miré, y he aquí el campo fértil era un desierto, y todas sus ciudades eran asoladas delante de Jehová, delante del ardor de su ira. 
4:27 Porque así dijo Jehová: Toda la tierra será asolada; pero no la destruiré del todo. 
4:28 Por esto se enlutará la tierra, y los cielos arriba se oscurecerán, porque hablé, lo pensé, y no me arrepentí, ni desistiré de ello. 
4:29 Al estruendo de la gente de a caballo y de los flecheros huyó toda la ciudad; entraron en las espesuras de los bosques, y subieron a los peñascos; todas las ciudades fueron abandonadas, y no quedó en ellas morador alguno. 
4:30 Y tú, destruida, ¿qué harás? Aunque te vistas de grana, aunque te adornes con atavíos de oro, aunque pintes con antimonio tus ojos, en vano te engalanas; te menospreciarán tus amantes, buscarán tu vida. 
4:31 Porque oí una voz como de mujer que está de parto, angustia como de primeriza; voz de la hija de Sion que lamenta y extiende sus manos, diciendo: ¡Ay ahora de mí! que mi alma desmaya a causa de los asesinos. 
\section*{Capítulo 5 }
Impiedad de Jerusalén y de Judá 
 
5:1 Recorred las calles de Jerusalén, y mirad ahora, e informaos; buscad en sus plazas a ver si halláis hombre, si hay alguno que haga justicia, que busque verdad; y yo la perdonaré. 
5:2 Aunque digan: Vive Jehová, juran falsamente. 
5:3 Oh Jehová, ¿no miran tus ojos a la verdad? Los azotaste, y no les dolió; los consumiste, y no quisieron recibir corrección; endurecieron sus rostros más que la piedra, no quisieron convertirse. 
5:4 Pero yo dije: Ciertamente éstos son pobres, han enloquecido, pues no conocen el camino de Jehová, el juicio de su Dios. 
5:5 Iré a los grandes, y les hablaré; porque ellos conocen el camino de Jehová, el juicio de su Dios. Pero ellos también quebraron el yugo, rompieron las coyundas. 
5:6 Por tanto, el león de la selva los matará, los destruirá el lobo del desierto, el leopardo acechará sus ciudades; cualquiera que de ellas saliere será arrebatado; porque sus rebeliones se han multiplicado, se han aumentado sus deslealtades. 
5:7 ¿Cómo te he de perdonar por esto? Sus hijos me dejaron, y juraron por lo que no es Dios. Los sacié, y adulteraron, y en casa de rameras se juntaron en compañías. 
5:8 Como caballos bien alimentados, cada cual relinchaba tras la mujer de su prójimo. 
5:9 ¿No había de castigar esto? dijo Jehová. De una nación como esta, ¿no se había de vengar mi alma? 
5:10 Escalad sus muros y destruid, pero no del todo; quitad las almenas de sus muros, porque no son de Jehová. 
5:11 Porque resueltamente se rebelaron contra mí la casa de Israel y la casa de Judá, dice Jehová. 
5:12 Negaron a Jehová, y dijeron: El no es, y no vendrá mal sobre nosotros, ni veremos espada ni hambre; 
5:13 antes los profetas serán como viento, porque no hay en ellos palabra; así se hará a ellos. 
5:14 Por tanto, así ha dicho Jehová Dios de los ejércitos: Porque dijeron esta palabra, he aquí yo pongo mis palabras en tu boca por fuego, y a este pueblo por leña, y los consumirá. 
5:15 He aquí yo traigo sobre vosotros gente de lejos, oh casa de Israel, dice Jehová; gente robusta, gente antigua, gente cuya lengua ignorarás, y no entenderás lo que hablare. 
5:16 Su aljaba como sepulcro abierto, todos valientes. 
5:17 Y comerá tu mies y tu pan, comerá a tus hijos y a tus hijas; comerá tus ovejas y tus vacas, comerá tus viñas y tus higueras, y a espada convertirá en nada tus ciudades fortificadas en que confías. 
5:18 No obstante, en aquellos días, dice Jehová, no os destruiré del todo. 
5:19 Y cuando dijeren: ¿Por qué Jehová el Dios nuestro hizo con nosotros todas estas cosas?, entonces les dirás: De la manera que me dejasteis a mí, y servisteis a dioses ajenos en vuestra tierra, así serviréis a extraños en tierra ajena. 
5:20 Anunciad esto en la casa de Jacob, y haced que esto se oiga en Judá, diciendo: 
5:21 Oíd ahora esto, pueblo necio y sin corazón, que tiene ojos y no ve, que tiene oídos y no oye: 
5:22 ¿A mí no me temeréis? dice Jehová. ¿No os amedrentaréis ante mí, que puse arena por término al mar, por ordenación eterna la cual no quebrantará? Se levantarán tempestades, mas no prevalecerán; bramarán sus ondas, mas no lo pasarán. 
5:23 No obstante, este pueblo tiene corazón falso y rebelde; se apartaron y se fueron. 
5:24 Y no dijeron en su corazón: Temamos ahora a Jehová Dios nuestro, que da lluvia temprana y tardía en su tiempo, y nos guarda los tiempos establecidos de la siega. 
5:25 Vuestras iniquidades han estorbado estas cosas, y vuestros pecados apartaron de vosotros el bien. 
5:26 Porque fueron hallados en mi pueblo impíos; acechaban como quien pone lazos, pusieron trampa para cazar hombres. 
5:27 Como jaula llena de pájaros, así están sus casas llenas de engaño; así se hicieron grandes y ricos. 
5:28 Se engordaron y se pusieron lustrosos, y sobrepasaron los hechos del malo; no juzgaron la causa, la causa del huérfano; con todo, se hicieron prósperos, y la causa de los pobres no juzgaron. 
5:29 ¿No castigaré esto? dice Jehová; ¿y de tal gente no se vengará mi alma? 
5:30 Cosa espantosa y fea es hecha en la tierra; 
5:31 los profetas profetizaron mentira, y los sacerdotes dirigían por manos de ellos; y mi pueblo así lo quiso. ¿Qué, pues, haréis cuando llegue el fin? 
\section*{Capítulo 6 }
El juicio contra Jerusalén y Judá 
 
6:1 Huid, hijos de Benjamín, de en medio de Jerusalén, y tocad bocina en Tecoa, y alzad por señal humo sobre Bet-haquerem; porque del norte se ha visto mal, y quebrantamiento grande. 
6:2 Destruiré a la bella y delicada hija de Sion. 
6:3 Contra ella vendrán pastores y sus rebaños; junto a ella plantarán sus tiendas alrededor; cada uno apacentará en su lugar. 
6:4 Anunciad guerra contra ella; levantaos y asaltémosla a mediodía. ¡Ay de nosotros! que va cayendo ya el día, que las sombras de la tarde se han extendido. 
6:5 Levantaos y asaltemos de noche, y destruyamos sus palacios. 
6:6 Porque así dijo Jehová de los ejércitos: Cortad árboles, y levantad vallado contra Jerusalén; esta es la ciudad que ha de ser castigada; toda ella está llena de violencia. 
6:7 Como la fuente nunca cesa de manar sus aguas, así ella nunca cesa de manar su maldad; injusticia y robo se oyen en ella; continuamente en mi presencia, enfermedad y herida. 
6:8 Corrígete, Jerusalén, para que no se aparte mi alma de ti, para que no te convierta en desierto, en tierra inhabitada. 
6:9 Así dijo Jehová de los ejércitos: Del todo rebuscarán como a vid el resto de Israel; vuelve tu mano como vendimiador entre los sarmientos. 
6:10 ¿A quién hablaré y amonestaré, para que oigan? He aquí que sus oídos son incircuncisos, y no pueden escuchar; he aquí que la palabra de Jehová les es cosa vergonzosa, no la aman. 
6:11 Por tanto, estoy lleno de la ira de Jehová, estoy cansado de contenerme; la derramaré sobre los niños en la calle, y sobre la reunión de los jóvenes igualmente; porque será preso tanto el marido como la mujer, tanto el viejo como el muy anciano. 
6:12 Y sus casas serán traspasadas a otros, sus heredades y también sus mujeres; porque extenderé mi mano sobre los moradores de la tierra, dice Jehová. 
6:13 Porque desde el más chico de ellos hasta el más grande, cada uno sigue la avaricia; y desde el profeta hasta el sacerdote, todos son engañadores. 
6:14 Y curan la herida de mi pueblo con liviandad, diciendo: Paz, paz; y no hay paz. 
6:15 ¿Se han avergonzado de haber hecho abominación? Ciertamente no se han avergonzado, ni aun saben tener vergüenza; por tanto, caerán entre los que caigan; cuando los castigue caerán, dice Jehová. 
6:16 Así dijo Jehová: Paraos en los caminos, y mirad, y preguntad por las sendas antiguas, cuál sea el buen camino, y andad por él, y hallaréis descanso para vuestra alma. Mas dijeron: No andaremos. 
6:17 Puse también sobre vosotros atalayas, que dijesen: Escuchad al sonido de la trompeta. Y dijeron ellos: No escucharemos. 
6:18 Por tanto, oíd, naciones, y entended, oh congregación, lo que sucederá. 
6:19 Oye, tierra: He aquí yo traigo mal sobre este pueblo, el fruto de sus pensamientos; porque no escucharon mis palabras, y aborrecieron mi ley. 
6:20 ¿Para qué a mí este incienso de Sabá, y la buena caña olorosa de tierra lejana? Vuestros holocaustos no son aceptables, ni vuestros sacrificios me agradan. 
6:21 Por tanto, Jehová dice esto: He aquí yo pongo a este pueblo tropiezos, y caerán en ellos los padres y los hijos juntamente; el vecino y su compañero perecerán. 
6:22 Así ha dicho Jehová: He aquí que viene pueblo de la tierra del norte, y una nación grande se levantará de los confines de la tierra. 
6:23 Arco y jabalina empuñarán; crueles son, y no tendrán misericordia; su estruendo brama como el mar, y montarán a caballo como hombres dispuestos para la guerra, contra ti, oh hija de Sion. 
6:24 Su fama oímos, y nuestras manos se descoyuntaron; se apoderó de nosotros angustia, dolor como de mujer que está de parto. 
6:25 No salgas al campo, ni andes por el camino; porque espada de enemigo y temor hay por todas partes. 
6:26 Hija de mi pueblo, cíñete de cilicio, y revuélcate en ceniza; ponte luto como por hijo único, llanto de amarguras; porque pronto vendrá sobre nosotros el destruidor. 
6:27 Por fortaleza te he puesto en mi pueblo, por torre; conocerás, pues, y examinarás el camino de ellos. 
6:28 Todos ellos son rebeldes, porfiados, andan chismeando; son bronce y hierro; todos ellos son corruptores. 
6:29 Se quemó el fuelle, por el fuego se ha consumido el plomo; en vano fundió el fundidor, pues la escoria no se ha arrancado. 
6:30 Plata desechada los llamarán, porque Jehová los desechó. 
\section*{Capítulo 7} 
Mejorad vuestros caminos y vuestras obras 
 
7:1 Palabra de Jehová que vino a Jeremías, diciendo: 
7:2 Ponte a la puerta de la casa de Jehová, y proclama allí esta palabra, y di: Oíd palabra de Jehová, todo Judá, los que entráis por estas puertas para adorar a Jehová. 
7:3 Así ha dicho Jehová de los ejércitos, Dios de Israel: Mejorad vuestros caminos y vuestras obras, y os haré morar en este lugar. 
7:4 No fiéis en palabras de mentira, diciendo: Templo de Jehová, templo de Jehová, templo de Jehová es este. 
7:5 Pero si mejorareis cumplidamente vuestros caminos y vuestras obras; si con verdad hiciereis justicia entre el hombre y su prójimo, 
7:6 y no oprimiereis al extranjero, al huérfano y a la viuda, ni en este lugar derramareis la sangre inocente, ni anduviereis en pos de dioses ajenos para mal vuestro, 
7:7 os haré morar en este lugar, en la tierra que di a vuestros padres para siempre. 
7:8 He aquí, vosotros confiáis en palabras de mentira, que no aprovechan. 
7:9 Hurtando, matando, adulterando, jurando en falso, e incensando a Baal, y andando tras dioses extraños que no conocisteis, 
7:10 ¿vendréis y os pondréis delante de mí en esta casa sobre la cual es invocado mi nombre, y diréis: Librados somos; para seguir haciendo todas estas abominaciones? 
7:11 ¿Es cueva de ladrones delante de vuestros ojos esta casa sobre la cual es invocado mi nombre? He aquí que también yo lo veo, dice Jehová. 
7:12 Andad ahora a mi lugar en Silo, donde hice morar mi nombre al principio, y ved lo que le hice por la maldad de mi pueblo Israel. 
7:13 Ahora, pues, por cuanto vosotros habéis hecho todas estas obras, dice Jehová, y aunque os hablé desde temprano y sin cesar, no oísteis, y os llamé, y no respondisteis; 
7:14 haré también a esta casa sobre la cual es invocado mi nombre, en la que vosotros confiáis, y a este lugar que di a vosotros y a vuestros padres, como hice a Silo. 
7:15 Os echaré de mi presencia, como eché a todos vuestros hermanos, a toda la generación de Efraín. 
7:16 Tú, pues, no ores por este pueblo, ni levantes por ellos clamor ni oración, ni me ruegues; porque no te oiré. 
7:17 ¿No ves lo que éstos hacen en las ciudades de Judá y en las calles de Jerusalén? 
7:18 Los hijos recogen la leña, los padres encienden el fuego, y las mujeres amasan la masa, para hacer tortas a la reina del cielo y para hacer ofrendas a dioses ajenos, para provocarme a ira. 
7:19 ¿Me provocarán ellos a ira? dice Jehová. ¿No obran más bien ellos mismos su propia confusión? 
7:20 Por tanto, así ha dicho Jehová el Señor: He aquí que mi furor y mi ira se derramarán sobre este lugar, sobre los hombres, sobre los animales, sobre los árboles del campo y sobre los frutos de la tierra; se encenderán, y no se apagarán. 
Castigo de la rebelión de Judá 
7:21 Así ha dicho Jehová de los ejércitos, Dios de Israel: Añadid vuestros holocaustos sobre vuestros sacrificios, y comed la carne. 
7:22 Porque no hablé yo con vuestros padres, ni nada les mandé acerca de holocaustos y de víctimas el día que los saqué de la tierra de Egipto. 
7:23 Mas esto les mandé, diciendo: Escuchad mi voz, y seré a vosotros por Dios, y vosotros me seréis por pueblo; y andad en todo camino que os mande, para que os vaya bien. 
7:24 Y no oyeron ni inclinaron su oído; antes caminaron en sus propios consejos, en la dureza de su corazón malvado, y fueron hacia atrás y no hacia adelante, 
7:25 desde el día que vuestros padres salieron de la tierra de Egipto hasta hoy. Y os envié todos los profetas mis siervos, enviándolos desde temprano y sin cesar; 
7:26 pero no me oyeron ni inclinaron su oído, sino que endurecieron su cerviz, e hicieron peor que sus padres. 
7:27 Tú, pues, les dirás todas estas palabras, pero no te oirán; los llamarás, y no te responderán. 
7:28 Les dirás, por tanto: Esta es la nación que no escuchó la voz de Jehová su Dios, ni admitió corrección; pereció la verdad, y de la boca de ellos fue cortada. 
7:29 Corta tu cabello, y arrójalo, y levanta llanto sobre las alturas; porque Jehová ha aborrecido y dejado la generación objeto de su ira. 
7:30 Porque los hijos de Judá han hecho lo malo ante mis ojos, dice Jehová; pusieron sus abominaciones en la casa sobre la cual fue invocado mi nombre, amancillándola. 
7:31 Y han edificado los lugares altos de Tofet, que está en el valle del hijo de Hinom,  para quemar al fuego a sus hijos y a sus hijas, cosa que yo no les mandé, ni subió en mi corazón. 
7:32 Por tanto, he aquí vendrán días, ha dicho Jehová, en que no se diga más, Tofet, ni valle del hijo de Hinom, sino Valle de la Matanza; y serán enterrados en Tofet, por no haber lugar. 
7:33 Y serán los cuerpos muertos de este pueblo para comida de las aves del cielo y de las bestias de la tierra; y no habrá quien las espante. 
7:34 Y haré cesar de las ciudades de Judá, y de las calles de Jerusalén, la voz de gozo y la voz de alegría, la voz del esposo y la voz de la esposa; porque la tierra será desolada. 
\section*{Capítulo 8 }
 
8:1 En aquel tiempo, dice Jehová, sacarán los huesos de los reyes de Judá, y los huesos de sus príncipes, y los huesos de los sacerdotes, y los huesos de los profetas, y los huesos de los moradores de Jerusalén, fuera de sus sepulcros; 
8:2 y los esparcirán al sol y a la luna y a todo el ejército del cielo, a quienes amaron y a quienes sirvieron, en pos de quienes anduvieron, a quienes preguntaron, y ante quienes se postraron. No serán recogidos ni enterrados; serán como estiércol sobre la faz de la tierra. 
8:3 Y escogerá la muerte antes que la vida todo el resto que quede de esta mala generación, en todos los lugares adonde arroje yo a los que queden, dice Jehová de los ejércitos. 
8:4 Les dirás asimismo: Así ha dicho Jehová: El que cae, ¿no se levanta? El que se desvía, ¿no vuelve al camino? 
8:5 ¿Por qué es este pueblo de Jerusalén rebelde con rebeldía perpetua? Abrazaron el engaño, y no han querido volverse. 
8:6 Escuché y oí; no hablan rectamente, no hay hombre que se arrepienta de su mal, diciendo: ¿Qué he hecho? Cada cual se volvió a su propia carrera, como caballo que arremete con ímpetu a la batalla. 
8:7 Aun la cigüeña en el cielo conoce su tiempo, y la tórtola y la grulla y la golondrina guardan el tiempo de su venida; pero mi pueblo no conoce el juicio de Jehová. 
8:8 ¿Cómo decís: Nosotros somos sabios, y la ley de Jehová está con nosotros? Ciertamente la ha cambiado en mentira la pluma mentirosa de los escribas. 
8:9 Los sabios se avergonzaron, se espantaron y fueron consternados; he aquí que aborrecieron la palabra de Jehová; ¿y qué sabiduría tienen? 
8:10 Por tanto, daré a otros sus mujeres, y sus campos a quienes los conquisten; porque desde el más pequeño hasta el más grande cada uno sigue la avaricia; desde el profeta hasta el sacerdote todos hacen engaño. 
8:11 Y curaron la herida de la hija de mi pueblo con liviandad, diciendo: Paz, paz; y no hay paz. 
8:12 ¿Se han avergonzado de haber hecho abominación? Ciertamente no se han avergonzado en lo más mínimo, ni supieron avergonzarse; caerán, por tanto, entre los que caigan; cuando los castigue caerán, dice Jehová 
8:13 Los cortaré del todo, dice Jehová. No quedarán uvas en la vid, ni higos en la higuera, y se caerá la hoja; y lo que les he dado pasará de ellos. 
8:14 ¿Por qué nos estamos sentados? Reuníos, y entremos en las ciudades fortificadas, y perezcamos allí; porque Jehová nuestro Dios nos ha destinado a perecer, y nos ha dado a beber aguas de hiel, porque pecamos contra Jehová. 
8:15 Esperamos paz, y no hubo bien; día de curación, y he aquí turbación. 
8:16 Desde Dan se oyó el bufido de sus caballos; al sonido de los relinchos de sus corceles tembló toda la tierra; y vinieron y devoraron la tierra y su abundancia, a la ciudad y a los moradores de ella. 
8:17 Porque he aquí que yo envío sobre vosotros serpientes, áspides contra los cuales no hay encantamiento, y os morderán, dice Jehová. 
Lamento sobre Judá y Jerusalén 
8:18 A causa de mi fuerte dolor, mi corazón desfallece en mí. 
8:19 He aquí voz del clamor de la hija de mi pueblo, que viene de la tierra lejana: ¿No está Jehová en Sion? ¿No está en ella su Rey? ¿Por qué me hicieron airar con sus imágenes de talla, con vanidades ajenas? 
8:20 Pasó la siega, terminó el verano, y nosotros no hemos sido salvos. 
8:21 Quebrantado estoy por el quebrantamiento de la hija de mi pueblo; entenebrecido estoy, espanto me ha arrebatado. 
8:22 ¿No hay bálsamo en Galaad? ¿No hay allí médico? ¿Por qué, pues, no hubo medicina para la hija de mi pueblo? 
\section*{Capítulo 9 }
 
9:1 ¡Oh, si mi cabeza se hiciese aguas, y mis ojos fuentes de lágrimas, para que llore día y noche los muertos de la hija de mi pueblo! 
9:2 ¡Oh, quién me diese en el desierto un albergue de caminantes, para que dejase a mi pueblo, y de ellos me apartase! Porque todos ellos son adúlteros, congregación de prevaricadores. 
9:3 Hicieron que su lengua lanzara mentira como un arco, y no se fortalecieron para la verdad en la tierra; porque de mal en mal procedieron, y me han desconocido, dice Jehová. 
9:4 Guárdese cada uno de su compañero, y en ningún hermano tenga confianza; porque todo hermano engaña con falacia, y todo compañero anda calumniando. 
9:5 Y cada uno engaña a su compañero, y ninguno habla verdad; acostumbraron su lengua a hablar mentira, se ocupan de actuar perversamente. 
9:6 Su morada está en medio del engaño; por muy engañadores no quisieron conocerme, dice Jehová. 
9:7 Por tanto, así ha dicho Jehová de los ejércitos: He aquí que yo los refinaré y los probaré; porque ¿qué más he de hacer por la hija de mi pueblo? 
9:8 Saeta afilada es la lengua de ellos; engaño habla; con su boca dice paz a su amigo, y dentro de sí pone sus asechanzas. 
9:9 ¿No los he de castigar por estas cosas? dice Jehová. De tal nación, ¿no se vengará mi alma? 
9:10 Por los montes levantaré lloro y lamentación, y llanto por los pastizales del desierto; porque fueron desolados hasta no quedar quien pase, ni oírse bramido de ganado; desde las aves del cielo hasta las bestias de la tierra huyeron, y se fueron. 
9:11 Reduciré a Jerusalén a un montón de ruinas, morada de chacales; y convertiré las ciudades de Judá en desolación en que no quede morador. 
Amenaza de ruina y exilio 
9:12 ¿Quién es varón sabio que entienda esto? ¿y a quién habló la boca de Jehová, para que pueda declararlo? ¿Por qué causa la tierra ha perecido, ha sido asolada como desierto, hasta no haber quien pase? 
9:13 Dijo Jehová: Porque dejaron mi ley, la cual di delante de ellos, y no obedecieron a mi voz, ni caminaron conforme a ella; 
9:14 antes se fueron tras la imaginación de su corazón, y en pos de los baales, según les enseñaron sus padres. 
9:15 Por tanto, así ha dicho Jehová de los ejércitos, Dios de Israel: He aquí que a este pueblo yo les daré a comer ajenjo, y les daré a beber aguas de hiel. 
9:16 Y los esparciré entre naciones que ni ellos ni sus padres conocieron; y enviaré espada en pos de ellos, hasta que los acabe. 
9:17 Así dice Jehová de los ejércitos: Considerad, y llamad plañideras que vengan; buscad a las hábiles en su oficio; 
9:18 y dense prisa, y levanten llanto por nosotros, y desháganse nuestros ojos en lágrimas, y nuestros párpados se destilen en aguas. 
9:19 Porque de Sion fue oída voz de endecha: ¡Cómo hemos sido destruidos! En gran manera hemos sido avergonzados, porque abandonamos la tierra, porque han destruido nuestras moradas. 
9:20 Oíd, pues, oh mujeres, palabra de Jehová, y vuestro oído reciba la palabra de su boca: Enseñad endechas a vuestras hijas, y lamentación cada una a su amiga. 
9:21 Porque la muerte ha subido por nuestras ventanas, ha entrado en nuestros palacios, para exterminar a los niños de las calles, a los jóvenes de las plazas. 
9:22 Habla: Así ha dicho Jehová: Los cuerpos de los hombres muertos caerán como estiércol sobre la faz del campo, y como manojo tras el segador, que no hay quien lo recoja. 
El conocimiento de Dios es la gloria del hombre 
9:23 Así dijo Jehová: No se alabe el sabio en su sabiduría, ni en su valentía se alabe el valiente, ni el rico se alabe en sus riquezas. 
9:24 Mas alábese en esto el que se hubiere de alabar: en entenderme y conocerme, que yo soy Jehová, que hago misericordia, juicio y justicia en la tierra; porque estas cosas quiero, dice Jehová. 
9:25 He aquí que vienen días, dice Jehová, en que castigaré a todo circuncidado, y a todo incircunciso; 
9:26 a Egipto y a Judá, a Edom y a los hijos de Amón y de Moab, y a todos los arrinconados en el postrer rincón, los que moran en el desierto; porque todas las naciones son incircuncisas, y toda la casa de Israel es incircuncisa de corazón. 
\section*{Capítulo 10 }
Los falsos dioses y el Dios verdadero 
 
10:1 Oíd la palabra que Jehová ha hablado sobre vosotros, oh casa de Israel. 
10:2 Así dijo Jehová: No aprendáis el camino de las naciones, ni de las señales del cielo tengáis temor, aunque las naciones las teman. 
10:3 Porque las costumbres de los pueblos son vanidad; porque leño del bosque cortaron, obra de manos de artífice con buril. 
10:4 Con plata y oro lo adornan; con clavos y martillo lo afirman para que no se mueva. 
10:5 Derechos están como palmera, y no hablan; son llevados, porque no pueden andar. No tengáis temor de ellos, porque ni pueden hacer mal, ni para hacer bien tienen poder. 
10:6 No hay semejante a ti, oh Jehová; grande eres tú, y grande tu nombre en poderío. 
10:7 ¿Quién no te temerá, oh Rey de las naciones? Porque a ti es debido el temor; porque entre todos los sabios de las naciones y en todos sus reinos, no hay semejante a ti. 
10:8 Todos se infatuarán y entontecerán. Enseñanza de vanidades es el leño. 
10:9 Traerán plata batida de Tarsis y oro de Ufaz, obra del artífice, y de manos del fundidor; los vestirán de azul y de púrpura, obra de peritos es todo. 
10:10 Mas Jehová es el Dios verdadero; él es Dios vivo y Rey eterno; a su ira tiembla la tierra, y las naciones no pueden sufrir su indignación. 
10:11 Les diréis así: Los dioses que no hicieron los cielos ni la tierra, desaparezcan de la tierra y de debajo de los cielos. 
10:12 El que hizo la tierra con su poder, el que puso en orden el mundo con su saber, y extendió los cielos con su sabiduría; 
10:13 a su voz se produce muchedumbre de aguas en el cielo, y hace subir las nubes de lo postrero de la tierra; hace los relámpagos con la lluvia, y saca el viento de sus depósitos. 
10:14 Todo hombre se embrutece, y le falta ciencia; se avergüenza de su ídolo todo fundidor, porque mentirosa es su obra de fundición, y no hay espíritu en ella. 
10:15 Vanidad son, obra vana; al tiempo de su castigo perecerán. 
10:16 No es así la porción de Jacob; porque él es el Hacedor de todo, e Israel es la vara de su heredad; Jehová de los ejércitos es su nombre. 
Asolamiento de Judá 
10:17 Recoge de las tierras tus mercaderías, la que moras en lugar fortificado. 
10:18 Porque así ha dicho Jehová: He aquí que esta vez arrojaré con honda los moradores de la tierra, y los afligiré, para que lo sientan. 
10:19 ¡Ay de mí, por mi quebrantamiento! mi llaga es muy dolorosa. Pero dije: Ciertamente enfermedad mía es esta, y debo sufrirla. 
10:20 Mi tienda está destruida, y todas mis cuerdas están rotas; mis hijos me han abandonado y perecieron; no hay ya más quien levante mi tienda, ni quien cuelgue mis cortinas. 
10:21 Porque los pastores se infatuaron, y no buscaron a Jehová; por tanto, no prosperaron, y todo su ganado se esparció. 
10:22 He aquí que voz de rumor viene, y alboroto grande de la tierra del norte, para convertir en soledad todas las ciudades de Judá, en morada de chacales. 
10:23 Conozco, oh Jehová, que el hombre no es señor de su camino, ni del hombre que camina es el ordenar sus pasos. 
10:24 Castígame, oh Jehová, mas con juicio; no con tu furor, para que no me aniquiles. 
10:25 Derrama tu enojo sobre los pueblos que no te conocen, y sobre las naciones que no invocan tu nombre; porque se comieron a Jacob, lo devoraron, le han consumido, y han asolado su morada. 
\section*{Capítulo 11 }
El pacto violado 
 
11:1 Palabra que vino de Jehová a Jeremías, diciendo: 
11:2 Oíd las palabras de este pacto, y hablad a todo varón de Judá, y a todo morador de Jerusalén. 
11:3 Y les dirás tú: Así dijo Jehová Dios de Israel: Maldito el varón que no obedeciere las palabras de este pacto, 
11:4 el cual mandé a vuestros padres el día que los saqué de la tierra de Egipto, del horno de hierro, diciéndoles: Oíd mi voz, y cumplid mis palabras, conforme a todo lo que os mando; y me seréis por pueblo, y yo seré a vosotros por Dios; 
11:5 para que confirme el juramento que hice a vuestros padres, que les daría la tierra que fluye leche y miel, como en este día. Y respondí y dije: Amén, oh Jehová. 
11:6 Y Jehová me dijo: Pregona todas estas palabras en las ciudades de Judá y en las calles de Jerusalén, diciendo: Oíd las palabras de este pacto, y ponedlas por obra. 
11:7 Porque solemnemente protesté a vuestros padres el día que les hice subir de la tierra de Egipto, amonestándoles desde temprano y sin cesar hasta el día de hoy, diciendo: Oíd mi voz. 
11:8 Pero no oyeron, ni inclinaron su oído, antes se fueron cada uno tras la imaginación de su malvado corazón; por tanto, traeré sobre ellos todas las palabras de este pacto, el cual mandé que cumpliesen, y no lo cumplieron. 
11:9 Y me dijo Jehová: Conspiración se ha hallado entre los varones de Judá, y entre los moradores de Jerusalén. 
11:10 Se han vuelto a las maldades de sus primeros padres, los cuales no quisieron escuchar mis palabras, y se fueron tras dioses ajenos para servirles; la casa de Israel y la casa de Judá invalidaron mi pacto, el cual había yo concertado con sus padres. 
11:11 Por tanto, así ha dicho Jehová: He aquí yo traigo sobre ellos mal del que no podrán salir; y clamarán a mí, y no los oiré. 
11:12 E irán las ciudades de Judá y los moradores de Jerusalén, y clamarán a los dioses a quienes queman ellos incienso, los cuales no los podrán salvar en el tiempo de su mal. 
11:13 Porque según el número de tus ciudades fueron tus dioses, oh Judá; y según el número de tus calles, oh Jerusalén, pusiste los altares de ignominia, altares para ofrecer incienso a Baal. 
11:14 Tú, pues, no ores por este pueblo, ni levantes por ellos clamor ni oración; porque yo no oiré en el día que en su aflicción clamen a mí. 
11:15 ¿Qué derecho tiene mi amada en mi casa, habiendo hecho muchas abominaciones? ¿Crees que los sacrificios y las carnes santificadas de las víctimas pueden evitarte el castigo? ¿Puedes gloriarte de eso? 
11:16 Olivo verde, hermoso en su fruto y en su parecer, llamó Jehová tu nombre. A la voz de recio estrépito hizo encender fuego sobre él, y quebraron sus ramas. 
11:17 Porque Jehová de los ejércitos que te plantó ha pronunciado mal contra ti, a causa de la maldad que la casa de Israel y la casa de Judá han hecho, provocándome a ira con incensar a Baal. 
Complot contra Jeremías 
11:18 Y Jehová me lo hizo saber, y lo conocí; entonces me hiciste ver sus obras. 
11:19 Y yo era como cordero inocente que llevan a degollar, pues no entendía que maquinaban designios contra mí, diciendo: Destruyamos el árbol con su fruto, y cortémoslo de la tierra de los vivientes, para que no haya más memoria de su nombre. 
11:20 Pero, oh Jehová de los ejércitos, que juzgas con justicia, que escudriñas la mente y el corazón, vea yo tu venganza de ellos; porque ante ti he expuesto mi causa. 
11:21 Por tanto, así ha dicho Jehová acerca de los varones de Anatot que buscan tu vida, diciendo: No profetices en nombre de Jehová, para que no mueras a nuestras manos; 
11:22 así, pues, ha dicho Jehová de los ejércitos: He aquí que yo los castigaré; los jóvenes morirán a espada, sus hijos y sus hijas morirán de hambre, 
11:23 y no quedará remanente de ellos, pues yo traeré mal sobre los varones de Anatot, el año de su castigo. 
\section*{Capítulo 12 }
Queja de Jeremías y respuesta de Dios 
 
12:1 Justo eres tú, oh Jehová, para que yo dispute contigo; sin embargo, alegaré mi causa ante ti. ¿Por qué es prosperado el camino de los impíos, y tienen bien todos los que se portan deslealmente? 
12:2 Los plantaste, y echaron raíces; crecieron y dieron fruto; cercano estás tú en sus bocas, pero lejos de sus corazones. 
12:3 Pero tú, oh Jehová, me conoces; me viste, y probaste mi corazón para contigo; arrebátalos como a ovejas para el degolladero, y señálalos para el día de la matanza. 
12:4 ¿Hasta cuándo estará desierta la tierra, y marchita la hierba de todo el campo? Por la maldad de los que en ella moran, faltaron los ganados y las aves; porque dijeron: No verá Dios nuestro fin. 
12:5 Si corriste con los de a pie, y te cansaron, ¿cómo contenderás con los caballos? Y si en la tierra de paz no estabas seguro, ¿cómo harás en la espesura del Jordán? 
12:6 Porque aun tus hermanos y la casa de tu padre, aun ellos se levantaron contra ti, aun ellos dieron grito en pos de ti. No los creas cuando bien te hablen. 
12:7 He dejado mi casa, desamparé mi heredad, he entregado lo que amaba mi alma en mano de sus enemigos. 
12:8 Mi heredad fue para mí como león en la selva; contra mí dio su rugido; por tanto, la aborrecí. 
12:9 ¿Es mi heredad para mí como ave de rapiña de muchos colores? ¿No están contra ella aves de rapiña en derredor? Venid, reuníos, vosotras todas las fieras del campo, venid a devorarla. 
12:10 Muchos pastores han destruido mi viña, hollaron mi heredad, convirtieron en desierto y soledad mi heredad preciosa. 
12:11 Fue puesta en asolamiento, y lloró sobre mí desolada; fue asolada toda la tierra, porque no hubo hombre que reflexionase. 
12:12 Sobre todas las alturas del desierto vinieron destruidores; porque la espada de Jehová devorará desde un extremo de la tierra hasta el otro; no habrá paz para ninguna carne. 
12:13 Sembraron trigo, y segaron espinos; tuvieron la heredad, mas no aprovecharon nada; se avergonzarán de sus frutos, a causa de la ardiente ira de Jehová. 
12:14 Así dijo Jehová contra todos mis malos vecinos, que tocan la heredad que hice poseer a mi pueblo Israel: He aquí que yo los arrancaré de su tierra, y arrancaré de en medio de ellos a la casa de Judá. 
12:15 Y después que los haya arrancado, volveré y tendré misericordia de ellos, y los haré volver cada uno a su heredad y cada cual a su tierra. 
12:16 Y si cuidadosamente aprendieren los caminos de mi pueblo, para jurar en mi nombre, diciendo: Vive Jehová, así como enseñaron a mi pueblo a jurar por Baal, ellos serán prosperados en medio de mi pueblo. 
12:17 Mas si no oyeren, arrancaré esa nación, sacándola de raíz y destruyéndola, dice Jehová. 
\section*{Capítulo 13 }
La señal del cinto podrido 
 
13:1 Así me dijo Jehová: Ve y cómprate un cinto de lino, y cíñelo sobre tus lomos, y no lo metas en agua. 
13:2 Y compré el cinto conforme a la palabra de Jehová, y lo puse sobre mis lomos. 
13:3 Vino a mí segunda vez palabra de Jehová, diciendo: 
13:4 Toma el cinto que compraste, que está sobre tus lomos, y levántate y vete al Eufrates, y escóndelo allá en la hendidura de una peña. 
13:5 Fui, pues, y lo escondí junto al Eufrates, como Jehová me mandó. 
13:6 Y sucedió que después de muchos días me dijo Jehová: Levántate y vete al Eufrates, y toma de allí el cinto que te mandé esconder allá. 
13:7 Entonces fui al Eufrates, y cavé, y tomé el cinto del lugar donde lo había escondido; y he aquí que el cinto se había podrido; para ninguna cosa era bueno. 
13:8 Y vino a mí palabra de Jehová, diciendo: 
13:9 Así ha dicho Jehová: Así haré podrir la soberbia de Judá, y la mucha soberbia de Jerusalén. 
13:10 Este pueblo malo, que no quiere oír mis palabras, que anda en las imaginaciones de su corazón, y que va en pos de dioses ajenos para servirles, y para postrarse ante ellos, vendrá a ser como este cinto, que para ninguna cosa es bueno. 
13:11 Porque como el cinto se junta a los lomos del hombre, así hice juntar a mí toda la casa de Israel y toda la casa de Judá, dice Jehová, para que me fuesen por pueblo y por fama, por alabanza y por honra; pero no escucharon. 
La señal de las tinajas llenas 
13:12 Les dirás, pues, esta palabra: Así ha dicho Jehová, Dios de Israel: Toda tinaja se llenará de vino. Y ellos te dirán: ¿No sabemos que toda tinaja se llenará de vino? 
13:13 Entonces les dirás: Así ha dicho Jehová: He aquí que yo lleno de embriaguez a todos los moradores de esta tierra, y a los reyes de la estirpe de David que se sientan sobre su trono, a los sacerdotes y profetas, y a todos los moradores de Jerusalén; 
13:14 y los quebrantaré el uno contra el otro, los padres con los hijos igualmente, dice Jehová; no perdonaré, ni tendré piedad ni misericordia, para no destruirlos. 
Judá será llevada en cautiverio 
13:15 Escuchad y oíd; no os envanezcáis, pues Jehová ha hablado. 
13:16 Dad gloria a Jehová Dios vuestro, antes que haga venir tinieblas, y antes que vuestros pies tropiecen en montes de oscuridad, y esperéis luz, y os la vuelva en sombra de muerte y tinieblas. 
13:17 Mas si no oyereis esto, en secreto llorará mi alma a causa de vuestra soberbia; y llorando amargamente se desharán mis ojos en lágrimas, porque el rebaño de Jehová fue hecho cautivo. 
13:18 Di al rey y a la reina: Humillaos, sentaos en tierra; porque la corona de vuestra gloria ha caído de vuestras cabezas. 
13:19 Las ciudades del Neguev fueron cerradas, y no hubo quien las abriese; toda Judá fue transportada, llevada en cautiverio fue toda ella. 
13:20 Alzad vuestros ojos, y ved a los que vienen del norte. ¿Dónde está el rebaño que te fue dado, tu hermosa grey? 
13:21 ¿Qué dirás cuando él ponga como cabeza sobre ti a aquellos a quienes tú enseñaste a ser tus amigos? ¿No te darán dolores como de mujer que está de parto? 
13:22 Si dijeres en tu corazón: ¿Por qué me ha sobrevenido esto? Por la enormidad de tu maldad fueron descubiertas tus faldas, fueron desnudados tus calcañares. 
13:23 ¿Mudará el etíope su piel, y el leopardo sus manchas? Así también, ¿podréis vosotros hacer bien, estando habituados a hacer mal? 
13:24 Por tanto, yo los esparciré al viento del desierto, como tamo que pasa. 
13:25 Esta es tu suerte, la porción que yo he medido para ti, dice Jehová, porque te olvidaste de mí y confiaste en la mentira. 
13:26 Yo, pues, descubriré también tus faldas delante de tu rostro, y se manifestará tu ignominia, 
13:27 tus adulterios, tus relinchos, la maldad de tu fornicación sobre los collados; en el campo vi tus abominaciones. ¡Ay de ti, Jerusalén! ¿No serás al fin limpia? ¿Cuánto tardarás tú en purificarte? 
\section*{Capítulo 14 }
Mensaje con motivo de la sequía 
 
14:1 Palabra de Jehová que vino a Jeremías, con motivo de la sequía. 
14:2 Se enlutó Judá, y sus puertas se despoblaron; se sentaron tristes en tierra, y subió el clamor de Jerusalén. 
14:3 Los nobles enviaron sus criados al agua; vinieron a las lagunas, y no hallaron agua; volvieron con sus vasijas vacías; se avergonzaron, se confundieron, y cubrieron sus cabezas. 
14:4 Porque se resquebrajó la tierra por no haber llovido en el país, están confusos los labradores, cubrieron sus cabezas. 
14:5 Aun las ciervas en los campos parían y dejaban la cría, porque no había hierba. 
14:6 Y los asnos monteses se ponían en las alturas, aspiraban el viento como chacales; sus ojos se ofuscaron porque no había hierba. 
14:7 Aunque nuestras iniquidades testifican contra nosotros, oh Jehová, actúa por amor de tu nombre; porque nuestras rebeliones se han multiplicado, contra ti hemos pecado. 
14:8 Oh esperanza de Israel, Guardador suyo en el tiempo de la aflicción, ¿por qué te has hecho como forastero en la tierra, y como caminante que se retira para pasar la noche? 
14:9 ¿Por qué eres como hombre atónito, y como valiente que no puede librar? Sin embargo, tú estás entre nosotros, oh Jehová, y sobre nosotros es invocado tu nombre; no nos desampares. 
14:10 Así ha dicho Jehová acerca de este pueblo: Se deleitaron en vagar, y no dieron reposo a sus pies; por tanto, Jehová no se agrada de ellos; se acordará ahora de su maldad, y castigará sus pecados. 
14:11 Me dijo Jehová: No ruegues por este pueblo para bien. 
14:12 Cuando ayunen, yo no oiré su clamor, y cuando ofrezcan holocausto y ofrenda no lo aceptaré, sino que los consumiré con espada, con hambre y con pestilencia. 
14:13 Y yo dije: ¡Ah! ¡ah, Señor Jehová! He aquí que los profetas les dicen: No veréis espada, ni habrá hambre entre vosotros, sino que en este lugar os daré paz verdadera. 
14:14 Me dijo entonces Jehová: Falsamente profetizan los profetas en mi nombre; no los envié, ni les mandé, ni les hablé; visión mentirosa, adivinación, vanidad y engaño de su corazón os profetizan. 
14:15 Por tanto, así ha dicho Jehová sobre los profetas que profetizan en mi nombre, los cuales yo no envié, y que dicen: Ni espada ni hambre habrá en esta tierra; con espada y con hambre serán consumidos esos profetas. 
14:16 Y el pueblo a quien profetizan será echado en las calles de Jerusalén por hambre y por espada, y no habrá quien los entierre a ellos, a sus mujeres, a sus hijos y a sus hijas; y sobre ellos derramaré su maldad. 
14:17 Les dirás, pues, esta palabra: Derramen mis ojos lágrimas noche y día, y no cesen; porque de gran quebrantamiento es quebrantada la virgen hija de mi pueblo, de plaga muy dolorosa. 
14:18 Si salgo al campo, he aquí muertos a espada; y si entro en la ciudad, he aquí enfermos de hambre; porque tanto el profeta como el sacerdote anduvieron vagando en la tierra, y no entendieron. 
14:19 ¿Has desechado enteramente a Judá? ¿Ha aborrecido tu alma a Sion? ¿Por qué nos hiciste herir sin que haya remedio? Esperamos paz, y no hubo bien; tiempo de curación, y he aquí turbación. 
14:20 Reconocemos, oh Jehová, nuestra impiedad, la iniquidad de nuestros padres; porque contra ti hemos pecado. 
14:21 Por amor de tu nombre no nos deseches, ni deshonres tu glorioso trono; acuérdate, no invalides tu pacto con nosotros. 
14:22 ¿Hay entre los ídolos de las naciones quien haga llover? ¿y darán los cielos lluvias? ¿No eres tú, Jehová, nuestro Dios? En ti, pues, esperamos, pues tú hiciste todas estas cosas. 
\section*{Capítulo 15} 
La implacable ira de Dios contra Judá 
 
15:1 Me dijo Jehová: Si Moisés y Samuel se pusieran delante de mí, no estaría mi voluntad con este pueblo; échalos de mi presencia, y salgan. 
15:2 Y si te preguntaren: ¿A dónde saldremos? les dirás: Así ha dicho Jehová: El que a muerte, a muerte; el que a espada, a espada; el que a hambre, a hambre; y el que a cautiverio, a cautiverio. 
15:3 Y enviaré sobre ellos cuatro géneros de castigo, dice Jehová: espada para matar, y perros para despedazar, y aves del cielo y bestias de la tierra para devorar y destruir. 
15:4 Y los entregaré para terror a todos los reinos de la tierra, a causa de Manasés hijo de Ezequías, rey de Judá, por lo que hizo en Jerusalén. 
15:5 Porque ¿quién tendrá compasión de ti, oh Jerusalén? ¿Quién se entristecerá por tu causa, o quién vendrá a preguntar por tu paz? 
15:6 Tú me dejaste, dice Jehová; te volviste atrás; por tanto, yo extenderé sobre ti mi mano y te destruiré; estoy cansado de arrepentirme. 
15:7 Aunque los aventé con aventador hasta las puertas de la tierra, y dejé sin hijos a mi pueblo y lo desbaraté, no se volvieron de sus caminos. 
15:8 Sus viudas se me multiplicaron más que la arena del mar; traje contra ellos destruidor a mediodía sobre la madre y sobre los hijos; hice que de repente cayesen terrores sobre la ciudad. 
15:9 Languideció la que dio a luz siete; se llenó de dolor su alma, su sol se puso siendo aún de día; fue avergonzada y llena de confusión; y lo que de ella quede, lo entregaré a la espada delante de sus enemigos, dice Jehová. 
15:10 ¡Ay de mí, madre mía, que me engendraste hombre de contienda y hombre de discordia para toda la tierra! Nunca he dado ni tomado en préstamo, y todos me maldicen. 
15:11 ¡Sea así, oh Jehová, si no te he rogado por su bien, si no he suplicado ante ti en favor del enemigo en tiempo de aflicción y en época de angustia! 
15:12 ¿Puede alguno quebrar el hierro, el hierro del norte y el bronce? 
15:13 Tus riquezas y tus tesoros entregaré a la rapiña sin ningún precio, por todos tus pecados, y en todo tu territorio. 
15:14 Y te haré servir a tus enemigos en tierra que no conoces; porque fuego se ha encendido en mi furor, y arderá sobre vosotros. 
Jehová reanima a Jeremías 
15:15 Tú lo sabes, oh Jehová; acuérdate de mí, y visítame, y véngame de mis enemigos. No me reproches en la prolongación de tu enojo; sabes que por amor de ti sufro afrenta. 
15:16 Fueron halladas tus palabras, y yo las comí; y tu palabra me fue por gozo y por alegría de mi corazón; porque tu nombre se invocó sobre mí, oh Jehová Dios de los ejércitos. 
15:17 No me senté en compañía de burladores, ni me engreí a causa de tu profecía; me senté solo, porque me llenaste de indignación. 
15:18 ¿Por qué fue perpetuo mi dolor, y mi herida desahuciada no admitió curación? ¿Serás para mí como cosa ilusoria, como aguas que no son estables? 
15:19 Por tanto, así dijo Jehová: Si te convirtieres, yo te restauraré, y delante de mí estarás; y si entresacares lo precioso de lo vil, serás como mi boca. Conviértanse ellos a ti, y tú no te conviertas a ellos. 
15:20 Y te pondré en este pueblo por muro fortificado de bronce, y pelearán contra ti, pero no te vencerán; porque yo estoy contigo para guardarte y para defenderte, dice Jehová. 
15:21 Y te libraré de la mano de los malos, y te redimiré de la mano de los fuertes. 
\section*{Capítulo 16 }
Juicio de Jehová contra Judá 
 
16:1 Vino a mí palabra de Jehová, diciendo: 
16:2 No tomarás para ti mujer, ni tendrás hijos ni hijas en este lugar. 
16:3 Porque así ha dicho Jehová acerca de los hijos y de las hijas que nazcan en este lugar, de sus madres que los den a luz y de los padres que los engendren en esta tierra: 
16:4 De dolorosas enfermedades morirán; no serán plañidos ni enterrados; serán como estiércol sobre la faz de la tierra; con espada y con hambre serán consumidos, y sus cuerpos servirán de comida a las aves del cielo y a las bestias de la tierra. 
16:5 Porque así ha dicho Jehová: No entres en casa de luto, ni vayas a lamentar, ni los consueles; porque yo he quitado mi paz de este pueblo, dice Jehová, mi misericordia y mis piedades. 
16:6 Morirán en esta tierra grandes y pequeños; no se enterrarán, ni los plañirán, ni se rasgarán ni se raerán los cabellos por ellos; 
16:7 ni partirán pan por ellos en el luto para consolarlos de sus muertos; ni les darán a beber vaso de consolaciones por su padre o por su madre. 
16:8 Asimismo no entres en casa de banquete, para sentarte con ellos a comer o a beber. 
16:9 Porque así ha dicho Jehová de los ejércitos, Dios de Israel: He aquí que yo haré cesar en este lugar, delante de vuestros ojos y en vuestros días, toda voz de gozo y toda voz de alegría, y toda voz de esposo y toda voz de esposa. 
16:10 Y acontecerá que cuando anuncies a este pueblo todas estas cosas, te dirán ellos: ¿Por qué anuncia Jehová contra nosotros todo este mal tan grande? ¿Qué maldad es la nuestra, o qué pecado es el nuestro, que hemos cometido contra Jehová nuestro Dios? 
16:11 Entonces les dirás: Porque vuestros padres me dejaron, dice Jehová, y anduvieron en pos de dioses ajenos, y los sirvieron, y ante ellos se postraron, y me dejaron a mí y no guardaron mi ley; 
16:12 y vosotros habéis hecho peor que vuestros padres; porque he aquí que vosotros camináis cada uno tras la imaginación de su malvado corazón, no oyéndome a mí. 
16:13 Por tanto, yo os arrojaré de esta tierra a una tierra que ni vosotros ni vuestros padres habéis conocido, y allá serviréis a dioses ajenos de día y de noche; porque no os mostraré clemencia. 
16:14 No obstante, he aquí vienen días, dice Jehová, en que no se dirá más: Vive Jehová, que hizo subir a los hijos de Israel de tierra de Egipto; 
16:15 sino: Vive Jehová, que hizo subir a los hijos de Israel de la tierra del norte, y de todas las tierras adonde los había arrojado; y los volveré a su tierra, la cual di a sus padres. 
16:16 He aquí que yo envío muchos pescadores, dice Jehová, y los pescarán, y después enviaré muchos cazadores, y los cazarán por todo monte y por todo collado, y por las cavernas de los peñascos. 
16:17 Porque mis ojos están sobre todos sus caminos, los cuales no se me ocultaron, ni su maldad se esconde de la presencia de mis ojos. 
16:18 Pero primero pagaré al doble su iniquidad y su pecado; porque contaminaron mi tierra con los cadáveres de sus ídolos, y de sus abominaciones llenaron mi heredad. 
16:19 Oh Jehová, fortaleza mía y fuerza mía, y refugio mío en el tiempo de la aflicción, a ti vendrán naciones desde los extremos de la tierra, y dirán: Ciertamente mentira poseyeron nuestros padres, vanidad, y no hay en ellos provecho. 
16:20 ¿Hará acaso el hombre dioses para sí? Mas ellos no son dioses. 
16:21 Por tanto, he aquí les enseñaré esta vez, les haré conocer mi mano y mi poder, y sabrán que mi nombre es Jehová. 
\section*{Capítulo 17 }
El pecado escrito en el corazón de Judá 
 
17:1 El pecado de Judá escrito está con cincel de hierro y con punta de diamante; esculpido está en la tabla de su corazón, y en los cuernos de sus altares, 
17:2 mientras sus hijos se acuerdan de sus altares y de sus imágenes de Asera, que están junto a los árboles frondosos y en los collados altos, 
17:3 sobre las montañas y sobre el campo. Todos tus tesoros entregaré al pillaje por el pecado de tus lugares altos en todo tu territorio. 
17:4 Y perderás la heredad que yo te di, y te haré servir a tus enemigos en tierra que no conociste; porque fuego habéis encendido en mi furor, que para siempre arderá. 
17:5 Así ha dicho Jehová: Maldito el varón que confía en el hombre, y pone carne por su brazo, y su corazón se aparta de Jehová. 
17:6 Será como la retama en el desierto, y no verá cuando viene el bien, sino que morará en los sequedales en el desierto, en tierra despoblada y deshabitada. 
17:7 Bendito el varón que confía en Jehová, y cuya confianza es Jehová. 
17:8 Porque será como el árbol plantado junto a las aguas, que junto a la corriente echará sus raíces, y no verá cuando viene el calor, sino que su hoja estará verde; y en el año de sequía no se fatigará, ni dejará de dar fruto. 
17:9 Engañoso es el corazón más que todas las cosas, y perverso; ¿quién lo conocerá? 
17:10 Yo Jehová, que escudriño la mente, que pruebo el corazón, para dar a cada uno según su camino, según el fruto de sus obras. 
17:11 Como la perdiz que cubre lo que no puso, es el que injustamente amontona riquezas; en la mitad de sus días las dejará, y en su postrimería será insensato. 
17:12 Trono de gloria, excelso desde el principio, es el lugar de nuestro santuario. 
17:13 ¡Oh Jehová, esperanza de Israel! todos los que te dejan serán avergonzados; y los que se apartan de mí serán escritos en el polvo, porque dejaron a Jehová, manantial de aguas vivas. 
17:14 Sáname, oh Jehová, y seré sano; sálvame, y seré salvo; porque tú eres mi alabanza. 
17:15 He aquí que ellos me dicen: ¿Dónde está la palabra de Jehová? ¡Que se cumpla ahora! 
17:16 Mas yo no he ido en pos de ti para incitarte a su castigo, ni deseé día de calamidad, tú lo sabes. Lo que de mi boca ha salido, fue en tu presencia. 
17:17 No me seas tú por espanto, pues mi refugio eres tú en el día malo. 
17:18 Avergüéncense los que me persiguen, y no me avergüence yo; asómbrense ellos, y yo no me asombre; trae sobre ellos día malo, y quebrántalos con doble quebrantamiento. 
Observancia del día de reposo 
17:19 Así me ha dicho Jehová: Ve y ponte a la puerta de los hijos del pueblo, por la cual entran y salen los reyes de Judá, y ponte en todas las puertas de Jerusalén, 
17:20 y diles: Oíd la palabra de Jehová, reyes de Judá, y todo Judá y todos los moradores de Jerusalén que entráis por estas puertas. 
17:21 Así ha dicho Jehová: Guardaos por vuestra vida de llevar carga en el día de reposo, y de meterla por las puertas de Jerusalén. 
17:22 Ni saquéis carga de vuestras casas en el día de reposo, ni hagáis trabajo alguno, sino santificad el día de reposo, como mandé a vuestros padres. 
17:23 Pero ellos no oyeron, ni inclinaron su oído, sino endurecieron su cerviz para no oír, ni recibir corrección. 
17:24 No obstante, si vosotros me obedeciereis, dice Jehová, no metiendo carga por las puertas de esta ciudad en el día de reposo, sino que santificareis el día de reposo, no haciendo en él ningún trabajo, 
17:25 entrarán por las puertas de esta ciudad, en carros y en caballos, los reyes y los príncipes que se sientan sobre el trono de David, ellos y sus príncipes, los varones de Judá y los moradores de Jerusalén; y esta ciudad será habitada para siempre. 
17:26 Y vendrán de las ciudades de Judá, de los alrededores de Jerusalén, de tierra de Benjamín, de la Sefela, de los montes y del Neguev, trayendo holocausto y sacrificio, y ofrenda e incienso, y trayendo sacrificio de alabanza a la casa de Jehová. 
17:27 Pero si no me oyereis para santificar el día de reposo, y para no traer carga ni meterla por las puertas de Jerusalén en día de reposo, yo haré descender fuego en sus puertas, y consumirá los palacios de Jerusalén, y no se apagará. 

\section*{Capítulo 18}
La señal del alfarero y el barro 
 
18:1 Palabra de Jehová que vino a Jeremías, diciendo: 
18:2 Levántate y vete a casa del alfarero, y allí te haré oír mis palabras. 
18:3 Y descendí a casa del alfarero, y he aquí que él trabajaba sobre la rueda. 
18:4 Y la vasija de barro que él hacía se echó a perder en su mano; y volvió y la hizo otra vasija, según le pareció mejor hacerla. 
18:5 Entonces vino a mí palabra de Jehová, diciendo: 
18:6 ¿No podré yo hacer de vosotros como este alfarero, oh casa de Israel? dice Jehová. He aquí que como el barro en la mano del alfarero, así sois vosotros en mi mano, oh casa de Israel. 
18:7 En un instante hablaré contra pueblos y contra reinos, para arrancar, y derribar, y destruir. 
18:8 Pero si esos pueblos se convirtieren de su maldad contra la cual hablé, yo me arrepentiré del mal que había pensado hacerles, 
18:9 y en un instante hablaré de la gente y del reino, para edificar y para plantar. 
18:10 Pero si hiciere lo malo delante de mis ojos, no oyendo mi voz, me arrepentiré del bien que había determinado hacerle. 
18:11 Ahora, pues, habla luego a todo hombre de Judá y a los moradores de Jerusalén, diciendo: Así ha dicho Jehová: He aquí que yo dispongo mal contra vosotros, y trazo contra vosotros designios; conviértase ahora cada uno de su mal camino, y mejore sus caminos y sus obras. 
18:12 Y dijeron: Es en vano; porque en pos de nuestros ídolos iremos, y haremos cada uno el pensamiento de nuestro malvado corazón. 
18:13 Por tanto, así dijo Jehová: Preguntad ahora a las naciones, quién ha oído cosa semejante. Gran fealdad ha hecho la virgen de Israel. 
18:14 ¿Faltará la nieve del Líbano de la piedra del campo? ¿Faltarán las aguas frías que corren de lejanas tierras? 
18:15 Porque mi pueblo me ha olvidado, incensando a lo que es vanidad, y ha tropezado en sus caminos, en las sendas antiguas, para que camine por sendas y no por camino transitado, 
18:16 para poner su tierra en desolación, objeto de burla perpetua; todo aquel que pasare por ella se asombrará, y meneará la cabeza. 
18:17 Como viento solano los esparciré delante del enemigo; les mostraré las espaldas y no el rostro, en el día de su perdición. 
Conspiración del pueblo y oración de Jeremías 
18:18 Y dijeron: Venid y maquinemos contra Jeremías; porque la ley no faltará al sacerdote, ni el consejo al sabio, ni la palabra al profeta. Venid e hirámoslo de lengua, y no atendamos a ninguna de sus palabras. 
18:19 Oh Jehová, mira por mí, y oye la voz de los que contienden conmigo. 
18:20 ¿Se da mal por bien, para que hayan cavado hoyo a mi alma? Acuérdate que me puse delante de ti para hablar bien por ellos, para apartar de ellos tu ira. 
18:21 Por tanto, entrega sus hijos a hambre, dispérsalos por medio de la espada, y queden sus mujeres sin hijos, y viudas; y sus maridos sean puestos a muerte, y sus jóvenes heridos a espada en la guerra. 
18:22 Oigase clamor de sus casas, cuando traigas sobre ellos ejército de repente; porque cavaron hoyo para prenderme, y a mis pies han escondido lazos. 
18:23 Pero tú, oh Jehová, conoces todo su consejo contra mí para muerte; no perdones su maldad, ni borres su pecado de delante de tu rostro; y tropiecen delante de ti; haz así con ellos en el tiempo de tu enojo. 
\section*{Capítulo 19 }
La señal de la vasija rota 
 
19:1 Así dijo Jehová: Ve y compra una vasija de barro del alfarero, y lleva contigo de los ancianos del pueblo, y de los ancianos de los sacerdotes; 
19:2 y saldrás al valle del hijo de Hinom, que está a la entrada de la puerta oriental, y proclamarás allí las palabras que yo te hablaré. 
19:3 Dirás, pues: Oíd palabra de Jehová, oh reyes de Judá, y moradores de Jerusalén. Así dice Jehová de los ejércitos, Dios de Israel: He aquí que yo traigo mal sobre este lugar, tal que a todo el que lo oyere, le retiñan los oídos. 
19:4 Porque me dejaron, y enajenaron este lugar, y ofrecieron en él incienso a dioses ajenos, los cuales no habían conocido ellos, ni sus padres, ni los reyes de Judá; y llenaron este lugar de sangre de inocentes. 
19:5 Y edificaron lugares altos a Baal, para quemar con fuego a sus hijos en holocaustos al mismo Baal; cosa que no les mandé, ni hablé, ni me vino al pensamiento. 
19:6 Por tanto, he aquí vienen días, dice Jehová, que este lugar no se llamará más Tofet, ni valle del hijo de Hinom, sino Valle de la Matanza. 
19:7 Y desvaneceré el consejo de Judá y de Jerusalén en este lugar, y les haré caer a espada delante de sus enemigos, y en las manos de los que buscan sus vidas; y daré sus cuerpos para comida a las aves del cielo y a las bestias de la tierra. 
19:8 Pondré a esta ciudad por espanto y burla; todo aquel que pasare por ella se asombrará, y se burlará sobre toda su destrucción. 
19:9 Y les haré comer la carne de sus hijos y la carne de sus hijas, y cada uno comerá la carne de su amigo, en el asedio y en el apuro con que los estrecharán sus enemigos y los que buscan sus vidas. 
19:10 Entonces quebrarás la vasija ante los ojos de los varones que van contigo, 
19:11 y les dirás: Así ha dicho Jehová de los ejércitos: Así quebrantaré a este pueblo y a esta ciudad, como quien quiebra una vasija de barro, que no se puede restaurar más; y en Tofet se enterrarán, porque no habrá otro lugar para enterrar. 
19:12 Así haré a este lugar, dice Jehová, y a sus moradores, poniendo esta ciudad como Tofet. 
19:13 Las casas de Jerusalén, y las casas de los reyes de Judá, serán como el lugar de Tofet, inmundas, por todas las casas sobre cuyos tejados ofrecieron incienso a todo el ejército del cielo, y vertieron libaciones a dioses ajenos. 
19:14 Y volvió Jeremías de Tofet, adonde le envió Jehová a profetizar, y se paró en el atrio de la casa de Jehová y dijo a todo el pueblo: 
19:15 Así ha dicho Jehová de los ejércitos, Dios de Israel: He aquí, yo traigo sobre esta ciudad y sobre todas sus villas todo el mal que hablé contra ella; porque han endurecido su cerviz para no oír mis palabras. 
\section*{Capítulo 20 }
Profecía contra Pasur 
 
20:1 El sacerdote Pasur hijo de Imer, que presidía como príncipe en la casa de Jehová, oyó a Jeremías que profetizaba estas palabras. 
20:2 Y azotó Pasur al profeta Jeremías, y lo puso en el cepo que estaba en la puerta superior de Benjamín, la cual conducía a la casa de Jehová. 
20:3 Y el día siguiente Pasur sacó a Jeremías del cepo. Le dijo entonces Jeremías: Jehová no ha llamado tu nombre Pasur, sino Magor-misabib. 
20:4 Porque así ha dicho Jehová: He aquí, haré que seas un terror a ti mismo y a todos los que bien te quieren, y caerán por la espada de sus enemigos, y tus ojos lo verán; y a todo Judá entregaré en manos del rey de Babilonia, y los llevará cautivos a Babilonia, y los matará a espada. 
20:5 Entregaré asimismo toda la riqueza de esta ciudad, todo su trabajo y todas sus cosas preciosas; y daré todos los tesoros de los reyes de Judá en manos de sus enemigos, y los saquearán, y los tomarán y los llevarán a Babilonia. 
20:6 Y tú, Pasur, y todos los moradores de tu casa iréis cautivos; entrarás en Babilonia, y allí morirás, y allí serás enterrado tú, y todos los que bien te quieren, a los cuales has profetizado con mentira. 
Lamento de Jeremías 
20:7 Me sedujiste, oh Jehová, y fui seducido; más fuerte fuiste que yo, y me venciste; cada día he sido escarnecido, cada cual se burla de mí. 
20:8 Porque cuantas veces hablo, doy voces, grito: Violencia y destrucción; porque la palabra de Jehová me ha sido para afrenta y escarnio cada día. 
20:9 Y dije: No me acordaré más de él, ni hablaré más en su nombre; no obstante, había en mi corazón como un fuego ardiente metido en mis huesos; traté de sufrirlo, y no pude. 
20:10 Porque oí la murmuración de muchos, temor de todas partes: Denunciad, denunciémosle. Todos mis amigos miraban si claudicaría. Quizá se engañará, decían, y prevaleceremos contra él, y tomaremos de él nuestra venganza. 
20:11 Mas Jehová está conmigo como poderoso gigante; por tanto, los que me persiguen tropezarán, y no prevalecerán; serán avergonzados en gran manera, porque no prosperarán; tendrán perpetua confusión que jamás será olvidada. 
20:12 Oh Jehová de los ejércitos, que pruebas a los justos, que ves los pensamientos y el corazón, vea yo tu venganza de ellos; porque a ti he encomendado mi causa. 
20:13 Cantad a Jehová, load a Jehová; porque ha librado el alma del pobre de mano de los malignos. 
20:14 Maldito el día en que nací; el día en que mi madre me dio a luz no sea bendito. 
20:15 Maldito el hombre que dio nuevas a mi padre, diciendo: Hijo varón te ha nacido, haciéndole alegrarse así mucho. 
20:16 Y sea el tal hombre como las ciudades que asoló Jehová, y no se arrepintió; oiga gritos de mañana, y voces a mediodía, 
20:17 porque no me mató en el vientre, y mi madre me hubiera sido mi sepulcro, y su vientre embarazado para siempre. 
20:18 ¿Para qué salí del vientre? ¿Para ver trabajo y dolor, y que mis días se gastasen en afrenta? 
\section*{Capítulo 21 }
Jerusalén será destruida 
 
21:1 Palabra de Jehová que vino a Jeremías, cuando el rey Sedequías envió a él a Pasur hijo de Malquías y al sacerdote Sofonías hijo de Maasías, para que le dijesen: 
21:2 Consulta ahora acerca de nosotros a Jehová, porque Nabucodonosor rey de Babilonia hace guerra contra nosotros; quizá Jehová hará con nosotros según todas sus maravillas, y aquél se irá de sobre nosotros. 
21:3 Y Jeremías les dijo: Diréis así a Sedequías: 
21:4 Así ha dicho Jehová Dios de Israel: He aquí yo vuelvo atrás las armas de guerra que están en vuestras manos, con que vosotros peleáis contra el rey de Babilonia; y a los caldeos que están fuera de la muralla y os tienen sitiados, yo los reuniré en medio de esta ciudad. 
21:5 Pelearé contra vosotros con mano alzada y con brazo fuerte, con furor y enojo e ira grande. 
21:6 Y heriré a los moradores de esta ciudad, y los hombres y las bestias morirán de pestilencia grande. 
21:7 Después, dice Jehová, entregaré a Sedequías rey de Judá, a sus criados, al pueblo y a los que queden de la pestilencia, de la espada y del hambre en la ciudad, en mano de Nabucodonosor rey de Babilonia, en mano de sus enemigos y de los que buscan sus vidas, y él los herirá a filo de espada; no los perdonará, ni tendrá compasión de ellos, ni tendrá de ellos misericordia. 
21:8 Y a este pueblo dirás: Así ha dicho Jehová: He aquí pongo delante de vosotros camino de vida y camino de muerte. 
21:9 El que quedare en esta ciudad morirá a espada, de hambre o de pestilencia; mas el que saliere y se pasare a los caldeos que os tienen sitiados, vivirá, y su vida le será por despojo. 
21:10 Porque mi rostro he puesto contra esta ciudad para mal, y no para bien, dice Jehová; en mano del rey de Babilonia será entregada, y la quemará a fuego. 
21:11 Y a la casa del rey de Judá dirás: Oíd palabra de Jehová: 
21:12 Casa de David, así dijo Jehová: Haced de mañana juicio, y librad al oprimido de mano del opresor, para que mi ira no salga como fuego, y se encienda y no haya quien lo apague, por la maldad de vuestras obras. 
21:13 He aquí yo estoy contra ti, moradora del valle, y de la piedra de la llanura, dice Jehová; los que decís: ¿Quién subirá contra nosotros, y quién entrará en nuestras moradas? 
21:14 Yo os castigaré conforme al fruto de vuestras obras, dice Jehová, y haré encender fuego en su bosque, y consumirá todo lo que está alrededor de él. 
\section*{Capítulo 22 }
Profecías contra los reyes de Judá 
 
22:1 Así dijo Jehová: Desciende a la casa del rey de Judá, y habla allí esta palabra, 
22:2 y di: Oye palabra de Jehová, oh rey de Judá que estás sentado sobre el trono de David, tú, y tus siervos, y tu pueblo que entra por estas puertas. 
22:3 Así ha dicho Jehová: Haced juicio y justicia, y librad al oprimido de mano del opresor, y no engañéis ni robéis al extranjero, ni al huérfano ni a la viuda, ni derraméis sangre inocente en este lugar. 
22:4 Porque si efectivamente obedeciereis esta palabra, los reyes que en lugar de David se sientan sobre su trono, entrarán montados en carros y en caballos por las puertas de esta casa; ellos, y sus criados y su pueblo. 
22:5 Mas si no oyereis estas palabras, por mí mismo he jurado, dice Jehová, que esta casa será desierta. 
22:6 Porque así ha dicho Jehová acerca de la casa del rey de Judá: Como Galaad eres tú para mí, y como la cima del Líbano; sin embargo, te convertiré en soledad, y como ciudades deshabitadas. 
22:7 Prepararé contra ti destruidores, cada uno con sus armas, y cortarán tus cedros escogidos y los echarán en el fuego. 
22:8 Y muchas gentes pasarán junto a esta ciudad, y dirán cada uno a su compañero: ¿Por qué hizo así Jehová con esta gran ciudad? 
22:9 Y se les responderá: Porque dejaron el pacto de Jehová su Dios, y adoraron dioses ajenos y les sirvieron. 
22:10 No lloréis al muerto, ni de él os condoláis; llorad amargamente por el que se va, porque no volverá jamás, ni verá la tierra donde nació. 
22:11 Porque así ha dicho Jehová acerca de Salum hijo de Josías, rey de Judá, el cual reinó en lugar de Josías su padre, y que salió de este lugar: No volverá más aquí, 
22:12 sino que morirá en el lugar adonde lo llevaron cautivo, y no verá más esta tierra. 
22:13 ¡Ay del que edifica su casa sin justicia, y sus salas sin equidad, sirviéndose de su prójimo de balde, y no dándole el salario de su trabajo! 
22:14 Que dice: Edificaré para mí casa espaciosa, y salas airosas; y le abre ventanas, y la cubre de cedro, y la pinta de bermellón. 
22:15 ¿Reinarás, porque te rodeas de cedro? ¿No comió y bebió tu padre, e hizo juicio y justicia, y entonces le fue bien? 
22:16 El juzgó la causa del afligido y del menesteroso, y entonces estuvo bien. ¿No es esto conocerme a mí? dice Jehová. 
22:17 Mas tus ojos y tu corazón no son sino para tu avaricia, y para derramar sangre inocente, y para opresión y para hacer agravio. 
22:18 Por tanto, así ha dicho Jehová acerca de Joacim hijo de Josías, rey de Judá: No lo llorarán, diciendo: ¡Ay, hermano mío! y ¡Ay, hermana! ni lo lamentarán, diciendo: ¡Ay, señor! ¡Ay, su grandeza! 
22:19 En sepultura de asno será enterrado, arrastrándole y echándole fuera de las puertas de Jerusalén. 
22:20 Sube al Líbano y clama, y en Basán da tu voz, y grita hacia todas partes; porque todos tus enamorados son destruidos. 
22:21 Te he hablado en tus prosperidades, mas dijiste: No oiré. Este fue tu camino desde tu juventud, que nunca oíste mi voz. 
22:22 A todos tus pastores pastoreará el viento, y tus enamorados irán en cautiverio; entonces te avergonzarás y te confundirás a causa de toda tu maldad. 
22:23 Habitaste en el Líbano, hiciste tu nido en los cedros. ¡Cómo gemirás cuando te vinieren dolores, dolor como de mujer que está de parto! 
22:24 Vivo yo, dice Jehová, que si Conías hijo de Joacim rey de Judá fuera anillo en mi mano derecha, aun de allí te arrancaría. 
22:25 Te entregaré en mano de los que buscan tu vida, y en mano de aquellos cuya vista temes; sí, en mano de Nabucodonosor rey de Babilonia, y en mano de los caldeos. 
22:26 Te haré llevar cautivo a ti y a tu madre que te dio a luz, a tierra ajena en que no nacisteis; y allá moriréis. 
22:27 Y a la tierra a la cual ellos con toda el alma anhelan volver, allá no volverán. 
22:28 ¿Es este hombre Conías una vasija despreciada y quebrada? ¿Es un trasto que nadie estima? ¿Por qué fueron arrojados él y su generación, y echados a tierra que no habían conocido? 
22:29 ¡Tierra, tierra, tierra! oye palabra de Jehová. 
22:30 Así ha dicho Jehová: Escribid lo que sucederá a este hombre privado de descendencia, hombre a quien nada próspero sucederá en todos los días de su vida; porque ninguno de su descendencia logrará sentarse sobre el trono de David, ni reinar sobre Judá. 
\section*{Capítulo 23 }
Regreso del remanente 
 
23:1 ¡Ay de los pastores que destruyen y dispersan las ovejas de mi rebaño! dice Jehová. 
23:2 Por tanto, así ha dicho Jehová Dios de Israel a los pastores que apacientan mi pueblo: Vosotros dispersasteis mis ovejas, y las espantasteis, y no las habéis cuidado. He aquí que yo castigo la maldad de vuestras obras, dice Jehová. 
23:3 Y yo mismo recogeré el remanente de mis ovejas de todas las tierras adonde las eché, y las haré volver a sus moradas; y crecerán y se multiplicarán. 
23:4 Y pondré sobre ellas pastores que las apacienten; y no temerán más, ni se amedrentarán, ni serán menoscabadas, dice Jehová. 
23:5 He aquí que vienen días, dice Jehová, en que levantaré a David renuevo justo, y reinará como Rey, el cual será dichoso, y hará juicio y justicia en la tierra. 
23:6 En sus días será salvo Judá, e Israel habitará confiado; y este será su nombre con el cual le llamarán: Jehová, justicia nuestra. 
23:7 Por tanto, he aquí que vienen días, dice Jehová, en que no dirán más: Vive Jehová que hizo subir a los hijos de Israel de la tierra de Egipto, 
23:8 sino: Vive Jehová que hizo subir y trajo la descendencia de la casa de Israel de tierra del norte, y de todas las tierras adonde yo los había echado; y habitarán en su tierra. 
Denunciación de los falsos profetas 
23:9 A causa de los profetas mi corazón está quebrantado dentro de mí, todos mis huesos tiemblan; estoy como un ebrio, y como hombre a quien dominó el vino, delante de Jehová, y delante de sus santas palabras. 
23:10 Porque la tierra está llena de adúlteros; a causa de la maldición la tierra está desierta; los pastizales del desierto se secaron; la carrera de ellos fue mala, y su valentía no es recta. 
23:11 Porque tanto el profeta como el sacerdote son impíos; aun en mi casa hallé su maldad, dice Jehová. 
23:12 Por tanto, su camino será como resbaladeros en oscuridad; serán empujados, y caerán en él; porque yo traeré mal sobre ellos en el año de su castigo, dice Jehová. 
23:13 En los profetas de Samaria he visto desatinos; profetizaban en nombre de Baal, e hicieron errar a mi pueblo de Israel. 
23:14 Y en los profetas de Jerusalén he visto torpezas; cometían adulterios, y andaban en mentiras, y fortalecían las manos de los malos, para que ninguno se convirtiese de su maldad; me fueron todos ellos como Sodoma, y sus moradores como Gomorra. 
23:15 Por tanto, así ha dicho Jehová de los ejércitos contra aquellos profetas: He aquí que yo les hago comer ajenjos, y les haré beber agua de hiel; porque de los profetas de Jerusalén salió la hipocresía sobre toda la tierra. 
23:16 Así ha dicho Jehová de los ejércitos: No escuchéis las palabras de los profetas que os profetizan; os alimentan con vanas esperanzas; hablan visión de su propio corazón, no de la boca de Jehová. 
23:17 Dicen atrevidamente a los que me irritan: Jehová dijo: Paz tendréis; y a cualquiera que anda tras la obstinación de su corazón, dicen: No vendrá mal sobre vosotros. 
23:18 Porque ¿quién estuvo en el secreto de Jehová, y vio, y oyó su palabra? ¿Quién estuvo atento a su palabra, y la oyó? 
23:19 He aquí que la tempestad de Jehová saldrá con furor; y la tempestad que está preparada caerá sobre la cabeza de los malos. 
23:20 No se apartará el furor de Jehová hasta que lo haya hecho, y hasta que haya cumplido los pensamientos de su corazón; en los postreros días lo entenderéis cumplidamente. 
23:21 No envié yo aquellos profetas, pero ellos corrían; yo no les hablé, mas ellos profetizaban. 
23:22 Pero si ellos hubieran estado en mi secreto, habrían hecho oír mis palabras a mi pueblo, y lo habrían hecho volver de su mal camino, y de la maldad de sus obras. 
23:23 ¿Soy yo Dios de cerca solamente, dice Jehová, y no Dios desde muy lejos? 
23:24 ¿Se ocultará alguno, dice Jehová, en escondrijos que yo no lo vea? ¿No lleno yo, dice Jehová, el cielo y la tierra? 
23:25 Yo he oído lo que aquellos profetas dijeron, profetizando mentira en mi nombre, diciendo: Soñé, soñé. 
23:26 ¿Hasta cuándo estará esto en el corazón de los profetas que profetizan mentira, y que profetizan el engaño de su corazón? 
23:27 ¿No piensan cómo hacen que mi pueblo se olvide de mi nombre con sus sueños que cada uno cuenta a su compañero, al modo que sus padres se olvidaron de mi nombre por Baal? 
23:28 El profeta que tuviere un sueño, cuente el sueño; y aquel a quien fuere mi palabra, cuente mi palabra verdadera. ¿Qué tiene que ver la paja con el trigo? dice Jehová. 
23:29 ¿No es mi palabra como fuego, dice Jehová, y como martillo que quebranta la piedra? 
23:30 Por tanto, he aquí que yo estoy contra los profetas, dice Jehová, que hurtan mis palabras cada uno de su más cercano. 
23:31 Dice Jehová: He aquí que yo estoy contra los profetas que endulzan sus lenguas y dicen: El ha dicho. 
23:32 He aquí, dice Jehová, yo estoy contra los que profetizan sueños mentirosos, y los cuentan, y hacen errar a mi pueblo con sus mentiras y con sus lisonjas, y yo no los envié ni les mandé; y ningún provecho hicieron a este pueblo, dice Jehová. 
23:33 Y cuando te preguntare este pueblo, o el profeta, o el sacerdote, diciendo: ¿Cuál es la profecía de Jehová? les dirás: Esta es la profecía: Os dejaré, ha dicho Jehová. 
23:34 Y al profeta, al sacerdote o al pueblo que dijere: Profecía de Jehová, yo enviaré castigo sobre tal hombre y sobre su casa. 
23:35 Así diréis cada cual a su compañero, y cada cual a su hermano: ¿Qué ha respondido Jehová, y qué habló Jehová? 
23:36 Y nunca más os vendrá a la memoria decir: Profecía de Jehová; porque la palabra de cada uno le será por profecía; pues pervertisteis las palabras del Dios viviente, de Jehová de los ejércitos, Dios nuestro. 
23:37 Así dirás al profeta: ¿Qué te respondió Jehová, y qué habló Jehová? 
23:38 Mas si dijereis: Profecía de Jehová; por eso Jehová dice así: Porque dijisteis esta palabra, Profecía de Jehová, habiendo yo enviado a deciros: No digáis: Profecía de Jehová, 
23:39 por tanto, he aquí que yo os echaré en olvido, y arrancaré de mi presencia a vosotros y a la ciudad que di a vosotros y a vuestros padres; 
23:40 y pondré sobre vosotros afrenta perpetua, y eterna confusión que nunca borrará el olvido. 
\section*{Capítulo 24 }
La señal de los higos buenos y malos 
 
24:1 Después de haber transportado Nabucodonosor rey de Babilonia a Jeconías hijo de Joacim, rey de Judá, a los príncipes de Judá y los artesanos y herreros de Jerusalén, y haberlos llevado a Babilonia, me mostró Jehová dos cestas de higos puestas delante del templo de Jehová. 
24:2 Una cesta tenía higos muy buenos, como brevas; y la otra cesta tenía higos muy malos, que de malos no se podían comer. 
24:3 Y me dijo Jehová: ¿Qué ves tú, Jeremías? Y dije: Higos; higos buenos, muy buenos; y malos, muy malos, que de malos no se pueden comer. 
24:4 Y vino a mí palabra de Jehová, diciendo: 
24:5 Así ha dicho Jehová Dios de Israel: Como a estos higos buenos, así miraré a los transportados de Judá, a los cuales eché de este lugar a la tierra de los caldeos, para bien. 
24:6 Porque pondré mis ojos sobre ellos para bien, y los volveré a esta tierra, y los edificaré, y no los destruiré; los plantaré y no los arrancaré. 
24:7 Y les daré corazón para que me conozcan que yo soy Jehová; y me serán por pueblo, y yo les seré a ellos por Dios; porque se volverán a mí de todo su corazón. 
24:8 Y como los higos malos, que de malos no se pueden comer, así ha dicho Jehová, pondré a Sedequías rey de Judá, a sus príncipes y al resto de Jerusalén que quedó en esta tierra, y a los que moran en la tierra de Egipto. 
24:9 Y los daré por escarnio y por mal a todos los reinos de la tierra; por infamia, por ejemplo, por refrán y por maldición a todos los lugares adonde yo los arroje. 
24:10 Y enviaré sobre ellos espada, hambre y pestilencia, hasta que sean exterminados de la tierra que les di a ellos y a sus padres. 
\section*{Capítulo 25 }
Setenta años de desolación 
 
25:1 Palabra que vino a Jeremías acerca de todo el pueblo de Judá en el año cuarto de Joacim hijo de Josías, rey de Judá, el cual era el año primero de Nabucodonosor rey de Babilonia; 
25:2 la cual habló el profeta Jeremías a todo el pueblo de Judá y a todos los moradores de Jerusalén, diciendo: 
25:3 Desde el año trece de Josías hijo de Amón, rey de Judá, hasta este día, que son vientitrés años, ha venido a mí palabra de Jehová, y he hablado desde temprano y sin cesar; pero no oísteis. 
25:4 Y envió Jehová a vosotros todos sus siervos los profetas, enviándoles desde temprano y sin cesar; pero no oísteis, ni inclinasteis vuestro oído para escuchar 
25:5 cuando decían: Volveos ahora de vuestro mal camino y de la maldad de vuestras obras, y moraréis en la tierra que os dio Jehová a vosotros y a vuestros padres para siempre; 
25:6 y no vayáis en pos de dioses ajenos, sirviéndoles y adorándoles, ni me provoquéis a ira con la obra de vuestras manos; y no os haré mal. 
25:7 Pero no me habéis oído, dice Jehová, para provocarme a ira con la obra de vuestras manos para mal vuestro. 
25:8 Por tanto, así ha dicho Jehová de los ejércitos: Por cuanto no habéis oído mis palabras, 
25:9 he aquí enviaré y tomaré a todas las tribus del norte, dice Jehová, y a Nabucodonosor rey de Babilonia, mi siervo, y los traeré contra esta tierra y contra sus moradores, y contra todas estas naciones en derredor; y los destruiré, y los pondré por escarnio y por burla y en desolación perpetua. 
25:10 Y haré que desaparezca de entre ellos la voz de gozo y la voz de alegría, la voz de desposado y la voz de desposada, ruido de molino y luz de lámpara. 
25:11 Toda esta tierra será puesta en ruinas y en espanto; y servirán estas naciones al rey de Babilonia setenta años. 
25:12 Y cuando sean cumplidos los setenta años, castigaré al rey de Babilonia y a aquella nación por su maldad, ha dicho Jehová, y a la tierra de los caldeos; y la convertiré en desiertos para siempre. 
25:13 Y traeré sobre aquella tierra todas mis palabras que he hablado contra ella, con todo lo que está escrito en este libro, profetizado por Jeremías contra todas las naciones. 
25:14 Porque también ellas serán sojuzgadas por muchas naciones y grandes reyes; y yo les pagaré conforme a sus hechos, y conforme a la obra de sus manos. 
La copa de ira para las naciones 
25:15 Porque así me dijo Jehová Dios de Israel: Toma de mi mano la copa del vino de este furor, y da a beber de él a todas las naciones a las cuales yo te envío. 
25:16 Y beberán, y temblarán y enloquecerán, a causa de la espada que yo envío entre ellas. 
25:17 Y tomé la copa de la mano de Jehová, y di de beber a todas las naciones, a las cuales me envió Jehová: 
25:18 a Jerusalén, a las ciudades de Judá y a sus reyes, y a sus príncipes, para ponerlos en ruinas, en escarnio y en burla y en maldición, como hasta hoy; 
25:19 a Faraón rey de Egipto, a sus siervos, a sus príncipes y a todo su pueblo; 
25:20 y a toda la mezcla de naciones, a todos los reyes de tierra de Uz, y a todos los reyes de la tierra de Filistea, a Ascalón, a Gaza, a Ecrón y al remanente de Asdod; 
25:21 a Edom, a Moab y a los hijos de Amón; 
25:22 a todos los reyes de Tiro, a todos los reyes de Sidón, a los reyes de las costas que están de ese lado del mar; 
25:23 a Dedán, a Tema y a Buz, y a todos los que se rapan las sienes; 
25:24 a todos los reyes de Arabia, a todos los reyes de pueblos mezclados que habitan en el desierto; 
25:25 a todos los reyes de Zimri, a todos los reyes de Elam, a todos los reyes de Media; 
25:26 a todos los reyes del norte, los de cerca y los de lejos, los unos con los otros, y a todos los reinos del mundo que están sobre la faz de la tierra; y el rey de Babilonia beberá después de ellos. 
25:27 Les dirás, pues: Así ha dicho Jehová de los ejércitos, Dios de Israel: Bebed, y embriagaos, y vomitad, y caed, y no os levantéis, a causa de la espada que yo envío entre vosotros. 
25:28 Y si no quieren tomar la copa de tu mano para beber, les dirás tú: Así ha dicho Jehová de los ejércitos: Tenéis que beber. 
25:29 Porque he aquí que a la ciudad en la cual es invocado mi nombre yo comienzo a hacer mal; ¿y vosotros seréis absueltos? No seréis absueltos; porque espada traigo sobre todos los moradores de la tierra, dice Jehová de los ejércitos. 
25:30 Tú, pues, profetizarás contra ellos todas estas palabras y les dirás: Jehová rugirá desde lo alto, y desde su morada santa dará su voz; rugirá fuertemente contra su morada; canción de lagareros cantará contra todos los moradores de la tierra. 
25:31 Llegará el estruendo hasta el fin de la tierra, porque Jehová tiene juicio contra las naciones; él es el Juez de toda carne; entregará los impíos a espada, dice Jehová. 
25:32 Así ha dicho Jehová de los ejércitos: He aquí que el mal irá de nación en nación, y grande tempestad se levantará de los fines de la tierra. 
25:33 Y yacerán los muertos de Jehová en aquel día desde un extremo de la tierra hasta el otro; no se endecharán ni se recogerán ni serán enterrados; como estiércol quedarán sobre la faz de la tierra. 
25:34 Aullad, pastores, y clamad; revolcaos en el polvo, mayorales del rebaño; porque cumplidos son vuestros días para que seáis degollados y esparcidos, y caeréis como vaso precioso. 
25:35 Y se acabará la huida de los pastores, y el escape de los mayorales del rebaño. 
25:36 ¡Voz de la gritería de los pastores, y aullido de los mayorales del rebaño! porque Jehová asoló sus pastos. 
25:37 Y los pastos delicados serán destruidos por el ardor de la ira de Jehová. 
25:38 Dejó cual leoncillo su guarida; pues asolada fue la tierra de ellos por la ira del opresor, y por el furor de su saña. 
\section*{Capítulo 26 }
Jeremías es amenazado de muerte 
 
26:1 En el principio del reinado de Joacim hijo de Josías, rey de Judá, vino esta palabra de Jehová, diciendo: 
26:2 Así ha dicho Jehová: Ponte en el atrio de la casa de Jehová, y habla a todas las ciudades de Judá, que vienen para adorar en la casa de Jehová, todas las palabras que yo te mandé hablarles; no retengas palabra. 
26:3 Quizá oigan, y se vuelvan cada uno de su mal camino, y me arrepentiré yo del mal que pienso hacerles por la maldad de sus obras. 
26:4 Les dirás, pues: Así ha dicho Jehová: Si no me oyereis para andar en mi ley, la cual puse ante vosotros, 
26:5 para atender a las palabras de mis siervos los profetas, que yo os envío desde temprano y sin cesar, a los cuales no habéis oído, 
26:6 yo pondré esta casa como Silo, y esta ciudad la pondré por maldición a todas las naciones de la tierra. 
26:7 Y los sacerdotes, los profetas y todo el pueblo oyeron a Jeremías hablar estas palabras en la casa de Jehová. 
26:8 Y cuando terminó de hablar Jeremías todo lo que Jehová le había mandado que hablase a todo el pueblo, los sacerdotes y los profetas y todo el pueblo le echaron mano, diciendo: De cierto morirás. 
26:9 ¿Por qué has profetizado en nombre de Jehová, diciendo: Esta casa será como Silo, y esta ciudad será asolada hasta no quedar morador? Y todo el pueblo se juntó contra Jeremías en la casa de Jehová. 
26:10 Y los príncipes de Judá oyeron estas cosas, y subieron de la casa del rey a la casa de Jehová, y se sentaron en la entrada de la puerta nueva de la casa de Jehová. 
26:11 Entonces hablaron los sacerdotes y los profetas a los príncipes y a todo el pueblo, diciendo: En pena de muerte ha incurrido este hombre; porque profetizó contra esta ciudad, como vosotros habéis oído con vuestros oídos. 
26:12 Y habló Jeremías a todos los príncipes y a todo el pueblo, diciendo: Jehová me envió a profetizar contra esta casa y contra esta ciudad, todas las palabras que habéis oído. 
26:13 Mejorad ahora vuestros caminos y vuestras obras, y oíd la voz de Jehová vuestro Dios, y se arrepentirá Jehová del mal que ha hablado contra vosotros. 
26:14 En lo que a mí toca, he aquí estoy en vuestras manos; haced de mí como mejor y más recto os parezca. 
26:15 Mas sabed de cierto que si me matáis, sangre inocente echaréis sobre vosotros, y sobre esta ciudad y sobre sus moradores; porque en verdad Jehová me envió a vosotros para que dijese todas estas palabras en vuestros oídos. 
26:16 Y dijeron los príncipes y todo el pueblo a los sacerdotes y profetas: No ha incurrido este hombre en pena de muerte, porque en nombre de Jehová nuestro Dios nos ha hablado. 
26:17 Entonces se levantaron algunos de los ancianos de la tierra y hablaron a toda la reunión del pueblo, diciendo: 
26:18 Miqueas de Moreset profetizó en tiempo de Ezequías rey de Judá, y habló a todo el pueblo de Judá, diciendo: Así ha dicho Jehová de los ejércitos: Sion será arada como campo, y Jerusalén vendrá a ser montones de ruinas, y el monte de la casa como cumbres de bosque. 
26:19 ¿Acaso lo mataron Ezequías rey de Judá y todo Judá? ¿No temió a Jehová, y oró en presencia de Jehová, y Jehová se arrepintió del mal que había hablado contra ellos? ¿Haremos, pues, nosotros tan gran mal contra nuestras almas? 
26:20 Hubo también un hombre que profetizaba en nombre de Jehová, Urías hijo de Semaías, de Quiriat-jearim, el cual profetizó contra esta ciudad y contra esta tierra, conforme a todas las palabras de Jeremías; 
26:21 y oyeron sus palabras el rey Joacim y todos sus grandes, y todos sus príncipes, y el rey procuró matarle; entendiendo lo cual Urías, tuvo temor, y huyó a Egipto. 
26:22 Y el rey Joacim envió hombres a Egipto, a Elnatán hijo de Acbor y otros hombres con él, a Egipto; 
26:23 los cuales sacaron a Urías de Egipto y lo trajeron al rey Joacim, el cual lo mató a espada, y echó su cuerpo en los sepulcros del vulgo. 
26:24 Pero la mano de Ahicam hijo de Safán estaba a favor de Jeremías, para que no lo entregasen en las manos del pueblo para matarlo. 

\section*{Capítulo 27 }
La señal de los yugos 
 
27:1 En el principio del reinado de Joacim hijo de Josías, rey de Judá, vino esta palabra de Jehová a Jeremías, diciendo: 
27:2 Jehová me ha dicho así: Hazte coyundas y yugos, y ponlos sobre tu cuello; 
27:3 y los enviarás al rey de Edom, y al rey de Moab, y al rey de los hijos de Amón, y al rey de Tiro, y al rey de Sidón, por mano de los mensajeros que vienen a Jerusalén a Sedequías rey de Judá. 
27:4 Y les mandarás que digan a sus señores: Así ha dicho Jehová de los ejércitos, Dios de Israel: Así habéis de decir a vuestros señores: 
27:5 Yo hice la tierra, el hombre y las bestias que están sobre la faz de la tierra, con mi gran poder y con mi brazo extendido, y la di a quien yo quise. 
27:6 Y ahora yo he puesto todas estas tierras en mano de Nabucodonosor rey de Babilonia, mi siervo, y aun las bestias del campo le he dado para que le sirvan. 
27:7 Y todas las naciones le servirán a él, a su hijo, y al hijo de su hijo, hasta que venga también el tiempo de su misma tierra, y la reduzcan a servidumbre muchas naciones y grandes reyes. 
27:8 Y a la nación y al reino que no sirviere a Nabucodonosor rey de Babilonia, y que no pusiere su cuello debajo del yugo del rey de Babilonia, castigaré a tal nación con espada y con hambre y con pestilencia, dice Jehová, hasta que la acabe yo por su mano. 
27:9 Y vosotros no prestéis oído a vuestros profetas, ni a vuestros adivinos, ni a vuestros soñadores, ni a vuestros agoreros, ni a vuestros encantadores, que os hablan diciendo: No serviréis al rey de Babilonia. 
27:10 Porque ellos os profetizan mentira, para haceros alejar de vuestra tierra, y para que yo os arroje y perezcáis. 
27:11 Mas a la nación que sometiere su cuello al yugo del rey de Babilonia y le sirviere, la dejaré en su tierra, dice Jehová, y la labrará y morará en ella. 
27:12 Hablé también a Sedequías rey de Judá conforme a todas estas palabras, diciendo: Someted vuestros cuellos al yugo del rey de Babilonia, y servidle a él y a su pueblo, y vivid. 
27:13 ¿Por qué moriréis tú y tu pueblo a espada, de hambre y de pestilencia, según ha dicho Jehová de la nación que no sirviere al rey de Babilonia? 
27:14 No oigáis las palabras de los profetas que os hablan diciendo: No serviréis al rey de Babilonia; porque os profetizan mentira. 
27:15 Porque yo no los envié, dice Jehová, y ellos profetizan falsamente en mi nombre, para que yo os arroje y perezcáis vosotros y los profetas que os profetizan. 
27:16 También a los sacerdotes y a todo este pueblo hablé diciendo: Así ha dicho Jehová: No oigáis las palabras de vuestros profetas que os profetizan diciendo: He aquí que los utensilios de la casa de Jehová volverán de Babilonia ahora pronto; porque os profetizan mentira. 
27:17 No los oigáis; servid al rey de Babilonia y vivid; ¿por qué ha de ser desolada esta ciudad? 
27:18 Y si ellos son profetas, y si está con ellos la palabra de Jehová, oren ahora a Jehová de los ejércitos para que los utensilios que han quedado en la casa de Jehová y en la casa del rey de Judá y en Jerusalén, no vayan a Babilonia. 
27:19 Porque así ha dicho Jehová de los ejércitos acerca de aquellas columnas, del estanque, de las basas y del resto de los utensilios que quedan en esta ciudad, 
27:20 que no quitó Nabucodonosor rey de Babilonia cuando transportó de Jerusalén a Babilonia a Jeconías hijo de Joacim, rey de Judá, y a todos los nobles de Judá y de Jerusalén; 
27:21 así, pues, ha dicho Jehová de los ejércitos, Dios de Israel, acerca de los utensilios que quedaron en la casa de Jehová, y en la casa del rey de Judá, y en Jerusalén: 
27:22 A Babilonia serán transportados, y allí estarán hasta el día en que yo los visite, dice Jehová; y después los traeré y los restauraré a este lugar. 
\section*{Capítulo 28 }
Falsa profecía de Hananías 
 
28:1 Aconteció en el mismo año, en el principio del reinado de Sedequías rey de Judá, en el año cuarto, en el quinto mes, que Hananías hijo de Azur, profeta que era de Gabaón, me habló en la casa de Jehová delante de los sacerdotes y de todo el pueblo, diciendo: 
28:2 Así habló Jehová de los ejércitos, Dios de Israel, diciendo: Quebranté el yugo del rey de Babilonia. 
28:3 Dentro de dos años haré volver a este lugar todos los utensilios de la casa de Jehová, que Nabucodonosor rey de Babilonia tomó de este lugar para llevarlos a Babilonia, 
28:4 y yo haré volver a este lugar a Jeconías hijo de Joacim, rey de Judá, y a todos los transportados de Judá que entraron en Babilonia, dice Jehová; porque yo quebrantaré el yugo del rey de Babilonia. 
28:5 Entonces respondió el profeta Jeremías al profeta Hananías, delante de los sacerdotes y delante de todo el pueblo que estaba en la casa de Jehová. 
28:6 Y dijo el profeta Jeremías: Amén, así lo haga Jehová. Confirme Jehová tus palabras, con las cuales profetizaste que los utensilios de la casa de Jehová, y todos los transportados, han de ser devueltos de Babilonia a este lugar. 
28:7 Con todo eso, oye ahora esta palabra que yo hablo en tus oídos y en los oídos de todo el pueblo: 
28:8 Los profetas que fueron antes de mí y antes de ti en tiempos pasados, profetizaron guerra, aflicción y pestilencia contra muchas tierras y contra grandes reinos. 
28:9 El profeta que profetiza de paz, cuando se cumpla la palabra del profeta, será conocido como el profeta que Jehová en verdad envió. 
28:10 Entonces el profeta Hananías quitó el yugo del cuello del profeta Jeremías, y lo quebró. 
28:11 Y habló Hananías en presencia de todo el pueblo, diciendo: Así ha dicho Jehová: De esta manera romperé el yugo de Nabucodonosor rey de Babilonia, del cuello de todas las naciones, dentro de dos años. Y siguió Jeremías su camino. 
28:12 Y después que el profeta Hananías rompió el yugo del cuello del profeta Jeremías, vino palabra de Jehová a Jeremías, diciendo: 
28:13 Ve y habla a Hananías, diciendo: Así ha dicho Jehová: Yugos de madera quebraste, mas en vez de ellos harás yugos de hierro. 
28:14 Porque así ha dicho Jehová de los ejércitos, Dios de Israel: Yugo de hierro puse sobre el cuello de todas estas naciones, para que sirvan a Nabucodonosor rey de Babilonia, y han de servirle; y aun también le he dado las bestias del campo. 
28:15 Entonces dijo el profeta Jeremías al profeta Hananías: Ahora oye, Hananías: Jehová no te envió, y tú has hecho confiar en mentira a este pueblo. 
28:16 Por tanto, así ha dicho Jehová: He aquí que yo te quito de sobre la faz de la tierra; morirás en este año, porque hablaste rebelión contra Jehová. 
28:17 Y en el mismo año murió Hananías, en el mes séptimo. 
\section*{Capítulo 29 }
Carta de Jeremías a los cautivos 
 
29:1 Estas son las palabras de la carta que el profeta Jeremías envió de Jerusalén a los ancianos que habían quedado de los que fueron transportados, y a los sacerdotes y profetas y a todo el pueblo que Nabucodonosor llevó cautivo de Jerusalén a Babilonia 
29:2 (después que salió el rey Jeconías, la reina, los del palacio, los príncipes de Judá y de Jerusalén, los artífices y los ingenieros de Jerusalén), 
29:3 por mano de Elasa hijo de Safán y de Gemarías hijo de Hilcías, a quienes envió Sedequías rey de Judá a Babilonia, a Nabucodonosor rey de Babilonia. Decía: 
29:4 Así ha dicho Jehová de los ejércitos, Dios de Israel, a todos los de la cautividad que hice transportar de Jerusalén a Babilonia: 
29:5 Edificad casas, y habitadlas; y plantad huertos, y comed del fruto de ellos. 
29:6 Casaos, y engendrad hijos e hijas; dad mujeres a vuestros hijos, y dad maridos a vuestras hijas, para que tengan hijos e hijas; y multiplicaos ahí, y no os disminuyáis. 
29:7 Y procurad la paz de la ciudad a la cual os hice transportar, y rogad por ella a Jehová; porque en su paz tendréis vosotros paz. 
29:8 Porque así ha dicho Jehová de los ejércitos, Dios de Israel: No os engañen vuestros profetas que están entre vosotros, ni vuestros adivinos; ni atendáis a los sueños que soñáis. 
29:9 Porque falsamente os profetizan ellos en mi nombre; no los envié, ha dicho Jehová. 
29:10 Porque así dijo Jehová: Cuando en Babilonia se cumplan los setenta años, yo os visitaré, y despertaré sobre vosotros mi buena palabra, para haceros volver a este lugar. 
29:11 Porque yo sé los pensamientos que tengo acerca de vosotros, dice Jehová, pensamientos de paz, y no de mal, para daros el fin que esperáis. 
29:12 Entonces me invocaréis, y vendréis y oraréis a mí, y yo os oiré; 
29:13 y me buscaréis y me hallaréis, porque me buscaréis de todo vuestro corazón. 
29:14 Y seré hallado por vosotros, dice Jehová, y haré volver vuestra cautividad, y os reuniré de todas las naciones y de todos los lugares adonde os arrojé, dice Jehová; y os haré volver al lugar de donde os hice llevar. 
29:15 Mas habéis dicho: Jehová nos ha levantado profetas en Babilonia. 
29:16 Pero así ha dicho Jehová acerca del rey que está sentado sobre el trono de David, y de todo el pueblo que mora en esta ciudad, de vuestros hermanos que no salieron con vosotros en cautiverio; 
29:17 así ha dicho Jehová de los ejércitos: He aquí envío yo contra ellos espada, hambre y pestilencia, y los pondré como los higos malos, que de tan malos no se pueden comer. 
29:18 Los perseguiré con espada, con hambre y con pestilencia, y los daré por escarnio a todos los reinos de la tierra, por maldición y por espanto, y por burla y por afrenta para todas las naciones entre las cuales los he arrojado; 
29:19 por cuanto no oyeron mis palabras, dice Jehová, que les envié por mis siervos los profetas, desde temprano y sin cesar; y no habéis escuchado, dice Jehová. 
29:20 Oíd, pues, palabra de Jehová, vosotros todos los transportados que envié de Jerusalén a Babilonia. 
29:21 Así ha dicho Jehová de los ejércitos, Dios de Israel, acerca de Acab hijo de Colaías, y acerca de Sedequías hijo de Maasías, que os profetizan falsamente en mi nombre: He aquí los entrego yo en mano de Nabucodonosor rey de Babilonia, y él los matará delante de vuestros ojos. 
29:22 Y todos los transportados de Judá que están en Babilonia harán de ellos una maldición, diciendo: Póngate Jehová como a Sedequías y como a Acab, a quienes asó al fuego el rey de Babilonia. 
29:23 Porque hicieron maldad en Israel, y cometieron adulterio con las mujeres de sus prójimos, y falsamente hablaron en mi nombre palabra que no les mandé; lo cual yo sé y testifico, dice Jehová. 
29:24 Y a Semaías de Nehelam hablarás, diciendo: 
29:25 Así habló Jehová de los ejércitos, Dios de Israel, diciendo: Tú enviaste cartas en tu nombre a todo el pueblo que está en Jerusalén, y al sacerdote Sofonías hijo de Maasías, y a todos los sacerdotes, diciendo: 
29:26 Jehová te ha puesto por sacerdote en lugar del sacerdote Joiada, para que te encargues en la casa de Jehová de todo hombre loco que profetice, poniéndolo en el calabozo y en el cepo. 
29:27 ¿Por qué, pues, no has reprendido ahora a Jeremías de Anatot, que os profetiza? 
29:28 Porque él nos envió a decir en Babilonia: Largo será el cautiverio; edificad casas, y habitadlas; plantad huertos, y comed el fruto de ellos. 
29:29 Y el sacerdote Sofonías había leído esta carta a oídos del profeta Jeremías. 
29:30 Y vino palabra de Jehová a Jeremías, diciendo: 
29:31 Envía a decir a todos los cautivos: Así ha dicho Jehová de Semaías de Nehelam: Porque os profetizó Semaías, y yo no lo envié, y os hizo confiar en mentira; 
29:32 por tanto, así ha dicho Jehová: He aquí que yo castigaré a Semaías de Nehelam y a su descendencia; no tendrá varón que more entre este pueblo, ni verá el bien que haré yo a mi pueblo, dice Jehová; porque contra Jehová ha hablado rebelión. 
\section*{Capítulo 30 }
Dios promete que los cautivos volverán 
 
30:1 Palabra de Jehová que vino a Jeremías, diciendo: 
30:2 Así habló Jehová Dios de Israel, diciendo: Escríbete en un libro todas las palabras que te he hablado. 
30:3 Porque he aquí que vienen días, dice Jehová, en que haré volver a los cautivos de mi pueblo Israel y Judá, ha dicho Jehová, y los traeré a la tierra que di a sus padres, y la disfrutarán. 
30:4 Estas, pues, son las palabras que habló Jehová acerca de Israel y de Judá. 
30:5 Porque así ha dicho Jehová: Hemos oído voz de temblor; de espanto, y no de paz. 
30:6 Inquirid ahora, y mirad si el varón da a luz; porque he visto que todo hombre tenía las manos sobre sus lomos, como mujer que está de parto, y se han vuelto pálidos todos los rostros. 
30:7 ¡Ah, cuán grande es aquel día! tanto, que no hay otro semejante a él; tiempo de angustia para Jacob; pero de ella será librado. 
30:8 En aquel día, dice Jehová de los ejércitos, yo quebraré su yugo de tu cuello, y romperé tus coyundas, y extranjeros no lo volverán más a poner en servidumbre, 
30:9 sino que servirán a Jehová su Dios y a David su rey, a quien yo les levantaré. 
30:10 Tú, pues, siervo mío Jacob, no temas, dice Jehová, ni te atemorices, Israel; porque he aquí que yo soy el que te salvo de lejos a ti y a tu descendencia de la tierra de cautividad; y Jacob volverá, descansará y vivirá tranquilo, y no habrá quien le espante. 
30:11 Porque yo estoy contigo para salvarte, dice Jehová, y destruiré a todas las naciones entre las cuales te esparcí; pero a ti no te destruiré, sino que te castigaré con justicia; de ninguna manera te dejaré sin castigo. 
30:12 Porque así ha dicho Jehová: Incurable es tu quebrantamiento, y dolorosa tu llaga. 
30:13 No hay quien juzgue tu causa para sanarte; no hay para ti medicamentos eficaces. 
30:14 Todos tus enamorados te olvidaron; no te buscan; porque como hiere un enemigo te herí, con azote de adversario cruel, a causa de la magnitud de tu maldad y de la multitud de tus pecados. 
30:15 ¿Por qué gritas a causa de tu quebrantamiento? Incurable es tu dolor, porque por la grandeza de tu iniquidad y por tus muchos pecados te he hecho esto. 
30:16 Pero serán consumidos todos los que te consumen; y todos tus adversarios, todos irán en cautiverio; hollados serán los que te hollaron, y a todos los que hicieron presa de ti daré en presa. 
30:17 Mas yo haré venir sanidad para ti, y sanaré tus heridas, dice Jehová; porque desechada te llamaron, diciendo: Esta es Sion, de la que nadie se acuerda. 
30:18 Así ha dicho Jehová: He aquí yo hago volver los cautivos de las tiendas de Jacob, y de sus tiendas tendré misericordia, y la ciudad será edificada sobre su colina, y el templo será asentado según su forma. 
30:19 Y saldrá de ellos acción de gracias, y voz de nación que está en regocijo, y los multiplicaré, y no serán disminuidos; los multiplicaré, y no serán menoscabados. 
30:20 Y serán sus hijos como antes, y su congregación delante de mí será confirmada; y castigaré a todos sus opresores. 
30:21 De ella saldrá su príncipe, y de en medio de ella saldrá su señoreador; y le haré llegar cerca, y él se acercará a mí; porque ¿quién es aquel que se atreve a acercarse a mí? dice Jehová. 
30:22 Y me seréis por pueblo, y yo seré vuestro Dios. 
30:23 He aquí, la tempestad de Jehová sale con furor; la tempestad que se prepara, sobre la cabeza de los impíos reposará. 
30:24 No se calmará el ardor de la ira de Jehová, hasta que haya hecho y cumplido los pensamientos de su corazón; en el fin de los días entenderéis esto. 
\section*{Capítulo 31 }
 
31:1 En aquel tiempo, dice Jehová, yo seré por Dios a todas las familias de Israel, y ellas me serán a mí por pueblo. 
31:2 Así ha dicho Jehová: El pueblo que escapó de la espada halló gracia en el desierto, cuando Israel iba en busca de reposo. 
31:3 Jehová se manifestó a mí hace ya mucho tiempo, diciendo: Con amor eterno te he amado; por tanto, te prolongué mi misericordia. 
31:4 Aún te edificaré, y serás edificada, oh virgen de Israel; todavía serás adornada con tus panderos, y saldrás en alegres danzas. 
31:5 Aún plantarás viñas en los montes de Samaria; plantarán los que plantan, y disfrutarán de ellas. 
31:6 Porque habrá día en que clamarán los guardas en el monte de Efraín: Levantaos, y subamos a Sion, a Jehová nuestro Dios. 
31:7 Porque así ha dicho Jehová: Regocijaos en Jacob con alegría, y dad voces de júbilo a la cabeza de naciones; haced oír, alabad, y decid: Oh Jehová, salva a tu pueblo, el remanente de Israel. 
31:8 He aquí yo los hago volver de la tierra del norte, y los reuniré de los fines de la tierra, y entre ellos ciegos y cojos, la mujer que está encinta y la que dio a luz juntamente; en gran compañía volverán acá. 
31:9 Irán con lloro, mas con misericordia los haré volver, y los haré andar junto a arroyos de aguas, por camino derecho en el cual no tropezarán; porque soy a Israel por padre, y Efraín es mi primogénito. 
31:10 Oíd palabra de Jehová, oh naciones, y hacedlo saber en las costas que están lejos, y decid: El que esparció a Israel lo reunirá y guardará, como el pastor a su rebaño. 
31:11 Porque Jehová redimió a Jacob, lo redimió de mano del más fuerte que él. 
31:12 Y vendrán con gritos de gozo en lo alto de Sion, y correrán al bien de Jehová, al pan, al vino, al aceite, y al ganado de las ovejas y de las vacas; y su alma será como huerto de riego, y nunca más tendrán dolor. 
31:13 Entonces la virgen se alegrará en la danza, los jóvenes y los viejos juntamente; y cambiaré su lloro en gozo, y los consolaré, y los alegraré de su dolor. 
31:14 Y el alma del sacerdote satisfaré con abundancia, y mi pueblo será saciado de mi bien, dice Jehová. 
31:15 Así ha dicho Jehová: Voz fue oída en Ramá, llanto y lloro amargo; Raquel que lamenta por sus hijos, y no quiso ser consolada acerca de sus hijos, porque perecieron. 
31:16 Así ha dicho Jehová: Reprime del llanto tu voz, y de las lágrimas tus ojos; porque salario hay para tu trabajo, dice Jehová, y volverán de la tierra del enemigo. 
31:17 Esperanza hay también para tu porvenir, dice Jehová, y los hijos volverán a su propia tierra. 
31:18 Escuchando, he oído a Efraín que se lamentaba: Me azotaste, y fui castigado como novillo indómito; conviérteme, y seré convertido, porque tú eres Jehová mi Dios. 
31:19 Porque después que me aparté tuve arrepentimiento, y después que reconocí mi falta, herí mi muslo; me avergoncé y me confundí, porque llevé la afrenta de mi juventud. 
31:20 ¿No es Efraín hijo precioso para mí? ¿no es niño en quien me deleito? pues desde que hablé de él, me he acordado de él constantemente. Por eso mis entrañas se conmovieron por él; ciertamente tendré de él misericordia, dice Jehová. 
31:21 Establécete señales, ponte majanos altos, nota atentamente la calzada; vuélvete por el camino por donde fuiste, virgen de Israel, vuelve a estas tus ciudades. 
31:22 ¿Hasta cuándo andarás errante, oh hija contumaz? Porque Jehová creará una cosa nueva sobre la tierra: la mujer rodeará al varón. 
31:23 Así ha dicho Jehová de los ejércitos, Dios de Israel: Aún dirán esta palabra en la tierra de Judá y en sus ciudades, cuando yo haga volver sus cautivos: Jehová te bendiga, oh morada de justicia, oh monte santo. 
31:24 Y habitará allí Judá, y también en todas sus ciudades labradores, y los que van con rebaño. 
31:25 Porque satisfaré al alma cansada, y saciaré a toda alma entristecida. 
31:26 En esto me desperté, y vi, y mi sueño me fue agradable. 
El nuevo pacto 
31:27 He aquí vienen días, dice Jehová, en que sembraré la casa de Israel y la casa de Judá de simiente de hombre y de simiente de animal. 
31:28 Y así como tuve cuidado de ellos para arrancar y derribar, y trastornar y perder y afligir, tendré cuidado de ellos para edificar y plantar, dice Jehová. 
31:29 En aquellos días no dirán más: Los padres comieron las uvas agrias y los dientes de los hijos tienen la dentera, 
31:30 sino que cada cual morirá por su propia maldad; los dientes de todo hombre que comiere las uvas agrias, tendrán la dentera. 
31:31 He aquí que vienen días, dice Jehová, en los cuales haré nuevo pacto con la casa de Israel y con la casa de Judá. 
31:32 No como el pacto que hice con sus padres el día que tomé su mano para sacarlos de la tierra de Egipto; porque ellos invalidaron mi pacto, aunque fui yo un marido para ellos, dice Jehová. 
31:33 Pero este es el pacto que haré con la casa de Israel después de aquellos días, dice Jehová: Daré mi ley en su mente, y la escribiré en su corazón; y yo seré a ellos por Dios, y ellos me serán por pueblo. 
31:34 Y no enseñará más ninguno a su prójimo, ni ninguno a su hermano, diciendo: Conoce a Jehová; porque todos me conocerán, desde el más pequeño de ellos hasta el más grande, dice Jehová; porque perdonaré la maldad de ellos, y no me acordaré más de su pecado. 
31:35 Así ha dicho Jehová, que da el sol para luz del día, las leyes de la luna y de las estrellas para luz de la noche, que parte el mar, y braman sus ondas; Jehová de los ejércitos es su nombre: 
31:36 Si faltaren estas leyes delante de mí, dice Jehová, también la descendencia de Israel faltará para no ser nación delante de mí eternamente. 
31:37 Así ha dicho Jehová: Si los cielos arriba se pueden medir, y explorarse abajo los fundamentos de la tierra, también yo desecharé toda la descendencia de Israel por todo lo que hicieron, dice Jehová. 
31:38 He aquí que vienen días, dice Jehová, en que la ciudad será edificada a Jehová, desde la torre de Hananeel hasta la puerta del Angulo. 
31:39 Y saldrá más allá el cordel de la medida delante de él sobre el collado de Gareb, y rodeará a Goa. 
31:40 Y todo el valle de los cuerpos muertos y de la ceniza, y todas las llanuras hasta el arroyo de Cedrón, hasta la esquina de la puerta de los caballos al oriente, será santo a Jehová; no será arrancada ni destruida más para siempre. 
\section*{Capítulo 32 }
Jeremías compra la heredad de Hanameel 
 
32:1 Palabra de Jehová que vino a Jeremías, el año décimo de Sedequías rey de Judá, que fue el año decimoctavo de Nabucodonosor. 
32:2 Entonces el ejército del rey de Babilonia tenía sitiada a Jerusalén, y el profeta Jeremías estaba preso en el patio de la cárcel que estaba en la casa del rey de Judá. 
32:3 Porque Sedequías rey de Judá lo había puesto preso, diciendo: ¿Por qué profetizas tú diciendo: Así ha dicho Jehová: He aquí yo entrego esta ciudad en mano del rey de Babilonia, y la tomará; 
32:4 y Sedequías rey de Judá no escapará de la mano de los caldeos, sino que de cierto será entregado en mano del rey de Babilonia, y hablará con él boca a boca, y sus ojos verán sus ojos, 
32:5 y hará llevar a Sedequías a Babilonia, y allá estará hasta que yo le visite; y si peleareis contra los caldeos, no os irá bien, dice Jehová? 
32:6 Dijo Jeremías: Palabra de Jehová vino a mí, diciendo: 
32:7 He aquí que Hanameel hijo de Salum tu tío viene a ti, diciendo: Cómprame mi heredad que está en Anatot; porque tú tienes derecho a ella para comprarla. 
32:8 Y vino a mí Hanameel hijo de mi tío, conforme a la palabra de Jehová, al patio de la cárcel, y me dijo: Compra ahora mi heredad, que está en Anatot en tierra de Benjamín, porque tuyo es el derecho de la herencia, y a ti corresponde el rescate; cómprala para ti. Entonces conocí que era palabra de Jehová. 
32:9 Y compré la heredad de Hanameel, hijo de mi tío, la cual estaba en Anatot, y le pesé el dinero; diecisiete siclos de plata. 
32:10 Y escribí la carta y la sellé, y la hice certificar con testigos, y pesé el dinero en balanza. 
32:11 Tomé luego la carta de venta, sellada según el derecho y costumbre, y la copia abierta. 
32:12 Y di la carta de venta a Baruc hijo de Nerías, hijo de Maasías, delante de Hanameel el hijo de mi tío, y delante de los testigos que habían suscrito la carta de venta, delante de todos los judíos que estaban en el patio de la cárcel. 
32:13 Y di orden a Baruc delante de ellos, diciendo: 
32:14 Así ha dicho Jehová de los ejércitos, Dios de Israel: Toma estas cartas, esta carta de venta sellada, y esta carta abierta, y ponlas en una vasija de barro, para que se conserven muchos días. 
32:15 Porque así ha dicho Jehová de los ejércitos, Dios de Israel: Aún se comprarán casas, heredades y viñas en esta tierra. 
32:16 Y después que di la carta de venta a Baruc hijo de Nerías, oré a Jehová, diciendo: 
32:17 ¡Oh Señor Jehová! he aquí que tú hiciste el cielo y la tierra con tu gran poder, y con tu brazo extendido, ni hay nada que sea difícil para ti; 
32:18 que haces misericordia a millares, y castigas la maldad de los padres en sus hijos después de ellos; Dios grande, poderoso, Jehová de los ejércitos es su nombre; 
32:19 grande en consejo, y magnífico en hechos; porque tus ojos están abiertos sobre todos los caminos de los hijos de los hombres, para dar a cada uno según sus caminos, y según el fruto de sus obras. 
32:20 Tú hiciste señales y portentos en tierra de Egipto hasta este día, y en Israel, y entre los hombres; y te has hecho nombre, como se ve en el día de hoy. 
32:21 Y sacaste a tu pueblo Israel de la tierra de Egipto con señales y portentos, con mano fuerte y brazo extendido, y con terror grande; 
32:22 y les diste esta tierra, de la cual juraste a sus padres que se la darías, la tierra que fluye leche y miel; 
32:23 y entraron, y la disfrutaron; pero no oyeron tu voz, ni anduvieron en tu ley; nada hicieron de lo que les mandaste hacer; por tanto, has hecho venir sobre ellos todo este mal. 
32:24 He aquí que con arietes han acometido la ciudad para tomarla, y la ciudad va a ser entregada en mano de los caldeos que pelean contra ella, a causa de la espada, del hambre y de la pestilencia; ha venido, pues, a suceder lo que tú dijiste, y he aquí lo estás viendo. 
32:25 ¡Oh Señor Jehová! ¿y tú me has dicho: Cómprate la heredad por dinero, y pon testigos; aunque la ciudad sea entregada en manos de los caldeos? 
32:26 Y vino palabra de Jehová a Jeremías, diciendo: 
32:27 He aquí que yo soy Jehová, Dios de toda carne; ¿habrá algo que sea difícil para mí? 
32:28 Por tanto, así ha dicho Jehová: He aquí voy a entregar esta ciudad en mano de los caldeos, y en mano de Nabucodonosor rey de Babilonia, y la tomará. 
32:29 Y vendrán los caldeos que atacan esta ciudad, y la pondrán a fuego y la quemarán, asimismo las casas sobre cuyas azoteas ofrecieron incienso a Baal y derramaron libaciones a dioses ajenos, para provocarme a ira. 
32:30 Porque los hijos de Israel y los hijos de Judá no han hecho sino lo malo delante de mis ojos desde su juventud; porque los hijos de Israel no han hecho más que provocarme a ira con la obra de sus manos, dice Jehová. 
32:31 De tal manera que para enojo mío y para ira mía me ha sido esta ciudad desde el día que la edificaron hasta hoy, para que la haga quitar de mi presencia, 
32:32 por toda la maldad de los hijos de Israel y de los hijos de Judá, que han hecho para enojarme, ellos, sus reyes, sus príncipes, sus sacerdotes y sus profetas, y los varones de Judá y los moradores de Jerusalén. 
32:33 Y me volvieron la cerviz, y no el rostro; y cuando los enseñaba desde temprano y sin cesar, no escucharon para recibir corrección. 
32:34 Antes pusieron sus abominaciones en la casa en la cual es invocado mi nombre, contaminándola. 
32:35 Y edificaron lugares altos a Baal, los cuales están en el valle del hijo de Hinom,   para hacer pasar por el fuego sus hijos y sus hijas a Moloc; lo cual no les mandé, ni me vino al pensamiento que hiciesen esta abominación, para hacer pecar a Judá. 
32:36 Y con todo, ahora así dice Jehová Dios de Israel a esta ciudad, de la cual decís vosotros: Entregada será en mano del rey de Babilonia a espada, a hambre y a pestilencia: 
32:37 He aquí que yo los reuniré de todas las tierras a las cuales los eché con mi furor, y con mi enojo e indignación grande; y los haré volver a este lugar, y los haré habitar seguramente; 
32:38 y me serán por pueblo, y yo seré a ellos por Dios. 
32:39 Y les daré un corazón, y un camino, para que me teman perpetuamente, para que tengan bien ellos, y sus hijos después de ellos. 
32:40 Y haré con ellos pacto eterno, que no me volveré atrás de hacerles bien, y pondré mi temor en el corazón de ellos, para que no se aparten de mí. 
32:41 Y me alegraré con ellos haciéndoles bien, y los plantaré en esta tierra en verdad, de todo mi corazón y de toda mi alma. 
32:42 Porque así ha dicho Jehová: Como traje sobre este pueblo todo este gran mal, así traeré sobre ellos todo el bien que acerca de ellos hablo. 
32:43 Y poseerán heredad en esta tierra de la cual vosotros decís: Está desierta, sin hombres y sin animales, es entregada en manos de los caldeos. 
32:44 Heredades comprarán por dinero, y harán escritura y la sellarán y pondrán testigos, en tierra de Benjamín y en los contornos de Jerusalén, y en las ciudades de Judá; y en las ciudades de las montañas, y en las ciudades de la Sefela, y en las ciudades del Neguev; porque yo haré regresar sus cautivos, dice Jehová. 
\section*{Capítulo 33 }
Restauración de la prosperidad de Jerusalén 
 
33:1 Vino palabra de Jehová a Jeremías la segunda vez, estando él aún preso en el patio de la cárcel, diciendo: 
33:2 Así ha dicho Jehová, que hizo la tierra, Jehová que la formó para afirmarla; Jehová es su nombre: 
33:3 Clama a mí, y yo te responderé, y te enseñaré cosas grandes y ocultas que tú no conoces. 
33:4 Porque así ha dicho Jehová Dios de Israel acerca de las casas de esta ciudad, y de las casas de los reyes de Judá, derribadas con arietes y con hachas 
33:5 (porque vinieron para pelear contra los caldeos, para llenarlas de cuerpos de hombres muertos, a los cuales herí yo con mi furor y con mi ira, pues escondí mi rostro de esta ciudad a causa de toda su maldad): 
33:6 He aquí que yo les traeré sanidad y medicina; y los curaré, y les revelaré abundancia de paz y de verdad. 
33:7 Y haré volver los cautivos de Judá y los cautivos de Israel, y los restableceré como al principio. 
33:8 Y los limpiaré de toda su maldad con que pecaron contra mí; y perdonaré todos sus pecados con que contra mí pecaron, y con que contra mí se rebelaron. 
33:9 Y me será a mí por nombre de gozo, de alabanza y de gloria, entre todas las naciones de la tierra, que habrán oído todo el bien que yo les hago; y temerán y temblarán de todo el bien y de toda la paz que yo les haré. 
33:10 Así ha dicho Jehová: En este lugar, del cual decís que está desierto sin hombres y sin animales, en las ciudades de Judá y en las calles de Jerusalén, que están asoladas, sin hombre y sin morador y sin animal, 
33:11 ha de oírse aún voz de gozo y de alegría, voz de desposado y voz de desposada, voz de los que digan: Alabad a Jehová de los ejércitos, porque Jehová es bueno, porque para siempre es su misericordia; voz de los que traigan ofrendas de acción de gracias a la casa de Jehová. Porque volveré a traer los cautivos de la tierra como al principio, ha dicho Jehová. 
33:12 Así dice Jehová de los ejércitos: En este lugar desierto, sin hombre y sin animal, y en todas sus ciudades, aún habrá cabañas de pastores que hagan pastar sus ganados. 
33:13 En las ciudades de las montañas, en las ciudades de la Sefela, en las ciudades del Neguev, en la tierra de Benjamín, y alrededor de Jerusalén y en las ciudades de Judá, aún pasarán ganados por las manos del que los cuente, ha dicho Jehová. 
33:14 He aquí vienen días, dice Jehová, en que yo confirmaré la buena palabra que he hablado a la casa de Israel y a la casa de Judá. 
33:15 En aquellos días y en aquel tiempo haré brotar a David un Renuevo de justicia, y hará juicio y justicia en la tierra. 
33:16 En aquellos días Judá será salvo, y Jerusalén habitará segura, y se le llamará: Jehová, justicia nuestra. 
33:17 Porque así ha dicho Jehová: No faltará a David varón que se siente sobre el trono de la casa de Israel. 
33:18 Ni a los sacerdotes y levitas faltará varón que delante de mí ofrezca holocausto y encienda ofrenda, y que haga sacrificio todos los días. 
33:19 Vino palabra de Jehová a Jeremías, diciendo: 
33:20 Así ha dicho Jehová: Si pudiereis invalidar mi pacto con el día y mi pacto con la noche, de tal manera que no haya día ni noche a su tiempo, 
33:21 podrá también invalidarse mi pacto con mi siervo David, para que deje de tener hijo que reine sobre su trono, y mi pacto con los levitas y sacerdotes, mis ministros. 
33:22 Como no puede ser contado el ejército del cielo, ni la arena del mar se puede medir, así multiplicaré la descendencia de David mi siervo, y los levitas que me sirven. 
33:23 Vino palabra de Jehová a Jeremías, diciendo: 
33:24 ¿No has echado de ver lo que habla este pueblo, diciendo: Dos familias que Jehová escogiera ha desechado? Y han tenido en poco a mi pueblo, hasta no tenerlo más por nación. 
33:25 Así ha dicho Jehová: Si no permanece mi pacto con el día y la noche, si yo no he puesto las leyes del cielo y la tierra, 
33:26 también desecharé la descendencia de Jacob, y de David mi siervo, para no tomar de su descendencia quien sea señor sobre la posteridad de Abraham, de Isaac y de Jacob. Porque haré volver sus cautivos, y tendré de ellos misericordia. 
\section*{Capítulo 34 }
Jeremías amonesta a Sedequías 
 
34:1 Palabra de Jehová que vino a Jeremías cuando Nabucodonosor rey de Babilonia y todo su ejército, y todos los reinos de la tierra bajo el señorío de su mano, y todos los pueblos, peleaban contra Jerusalén y contra todas sus ciudades, la cual dijo: 
34:2 Así ha dicho Jehová Dios de Israel: Ve y habla a Sedequías rey de Judá, y dile: Así ha dicho Jehová: He aquí yo entregaré esta ciudad al rey de Babilonia, y la quemará con fuego; 
34:3 y no escaparás tú de su mano, sino que ciertamente serás apresado, y en su mano serás entregado; y tus ojos verán los ojos del rey de Babilonia, y te hablará boca a boca, y en Babilonia entrarás. 
34:4 Con todo eso, oye palabra de Jehová, Sedequías rey de Judá: Así ha dicho Jehová acerca de ti: No morirás a espada. 
34:5 En paz morirás, y así como quemaron especias por tus padres, los reyes primeros que fueron antes de ti, las quemarán por ti, y te endecharán, diciendo, ¡Ay, señor! Porque yo he hablado la palabra, dice Jehová. 
34:6 Y habló el profeta Jeremías a Sedequías rey de Judá todas estas palabras en Jerusalén. 
34:7 Y el ejército del rey de Babilonia peleaba contra Jerusalén, y contra todas las ciudades de Judá que habían quedado, contra Laquis y contra Azeca; porque de las ciudades fortificadas de Judá éstas habían quedado. 
Violación del pacto de libertar a los siervos hebreos 
34:8 Palabra de Jehová que vino a Jeremías, después que Sedequías hizo pacto con todo el pueblo en Jerusalén para promulgarles libertad; 
34:9 que cada uno dejase libre a su siervo y a su sierva, hebreo y hebrea; que ninguno usase a los judíos, sus hermanos, como siervos. 
34:10 Y cuando oyeron todos los príncipes, y todo el pueblo que había convenido en el pacto de dejar libre cada uno a su siervo y cada uno a su sierva, que ninguno los usase más como siervos, obedecieron, y los dejaron. 
34:11 Pero después se arrepintieron, e hicieron volver a los siervos y a las siervas que habían dejado libres, y los sujetaron como siervos y siervas. 
34:12 Vino, pues, palabra de Jehová a Jeremías, diciendo: 
34:13 Así dice Jehová Dios de Israel: Yo hice pacto con vuestros padres el día que los saqué de tierra de Egipto, de casa de servidumbre, diciendo: 
34:14 Al cabo de siete años dejará cada uno a su hermano hebreo que le fuere vendido; le servirá seis años, y lo enviará libre; pero vuestros padres no me oyeron, ni inclinaron su oído. 
34:15 Y vosotros os habíais hoy convertido, y hecho lo recto delante de mis ojos, anunciando cada uno libertad a su prójimo; y habíais hecho pacto en mi presencia, en la casa en la cual es invocado mi nombre. 
34:16 Pero os habéis vuelto y profanado mi nombre, y habéis vuelto a tomar cada uno a su siervo y cada uno a su sierva, que habíais dejado libres a su voluntad; y los habéis sujetado para que os sean siervos y siervas. 
34:17 Por tanto, así ha dicho Jehová: Vosotros no me habéis oído para promulgar cada uno libertad a su hermano, y cada uno a su compañero; he aquí que yo promulgo libertad, dice Jehová, a la espada y a la pestilencia y al hambre; y os pondré por afrenta ante todos los reinos de la tierra. 
34:18 Y entregaré a los hombres que traspasaron mi pacto, que no han llevado a efecto las palabras del pacto que celebraron en mi presencia, dividiendo en dos partes el becerro y pasando por medio de ellas; 
34:19 a los príncipes de Judá y a los príncipes de Jerusalén, a los oficiales y a los sacerdotes y a todo el pueblo de la tierra, que pasaron entre las partes del becerro, 
34:20 los entregaré en mano de sus enemigos y en mano de los que buscan su vida; y sus cuerpos muertos serán comida de las aves del cielo, y de las bestias de la tierra. 
34:21 Y a Sedequías rey de Judá y a sus príncipes los entregaré en mano de sus enemigos, y en mano de los que buscan su vida, y en mano del ejército del rey de Babilonia, que se ha ido de vosotros. 
34:22 He aquí, mandaré yo, dice Jehová, y los haré volver a esta ciudad, y pelearán contra ella y la tomarán, y la quemarán con fuego; y reduciré a soledad las ciudades de Judá, hasta no quedar morador. 

\section*{Capítulo 35}
Obediencia de los recabitas 
 
35:1 Palabra de Jehová que vino a Jeremías en días de Joacim hijo de Josías, rey de Judá, diciendo: 
35:2 Ve a casa de los recabitas y habla con ellos, e introdúcelos en la casa de Jehová, en uno de los aposentos, y dales a beber vino. 
35:3 Tomé entonces a Jaazanías hijo de Jeremías, hijo de Habasinías, a sus hermanos, a todos sus hijos, y a toda la familia de los recabitas; 
35:4 y los llevé a la casa de Jehová, al aposento de los hijos de Hanán hijo de Igdalías, varón de Dios, el cual estaba junto al aposento de los príncipes, que estaba sobre el aposento de Maasías hijo de Salum, guarda de la puerta. 
35:5 Y puse delante de los hijos de la familia de los recabitas tazas y copas llenas de vino, y les dije: Bebed vino. 
35:6 Mas ellos dijeron: No beberemos vino; porque Jonadab hijo de Recab nuestro padre nos ordenó diciendo: No beberéis jamás vino vosotros ni vuestros hijos; 
35:7 ni edificaréis casa, ni sembraréis sementera, ni plantaréis viña, ni la retendréis; sino que moraréis en tiendas todos vuestros días, para que viváis muchos días sobre la faz de la tierra donde vosotros habitáis. 
35:8 Y nosotros hemos obedecido a la voz de nuestro padre Jonadab hijo de Recab en todas las cosas que nos mandó, de no beber vino en todos nuestros días, ni nosotros, ni nuestras mujeres, ni nuestros hijos ni nuestras hijas; 
35:9 y de no edificar casas para nuestra morada, y de no tener viña, ni heredad, ni sementera. 
35:10 Moramos, pues, en tiendas, y hemos obedecido y hecho conforme a todas las cosas que nos mandó Jonadab nuestro padre. 
35:11 Sucedió, no obstante, que cuando Nabucodonosor rey de Babilonia subió a la tierra, dijimos: Venid, y ocultémonos en Jerusalén, de la presencia del ejército de los caldeos y de la presencia del ejército de los de Siria; y en Jerusalén nos quedamos. 
35:12 Y vino palabra de Jehová a Jeremías, diciendo: 
35:13 Así ha dicho Jehová de los ejércitos, Dios de Israel: Ve y di a los varones de Judá, y a los moradores de Jerusalén: ¿No aprenderéis a obedecer mis palabras? dice Jehová. 
35:14 Fue firme la palabra de Jonadab hijo de Recab, el cual mandó a sus hijos que no bebiesen vino, y no lo han bebido hasta hoy, por obedecer al mandamiento de su padre; y yo os he hablado a vosotros desde temprano y sin cesar, y no me habéis oído. 
35:15 Y envié a vosotros todos mis siervos los profetas, desde temprano y sin cesar, para deciros: Volveos ahora cada uno de vuestro mal camino, y enmendad vuestras obras, y no vayáis tras dioses ajenos para servirles, y viviréis en la tierra que di a vosotros y a vuestros padres; mas no inclinasteis vuestro oído, ni me oísteis. 
35:16 Ciertamente los hijos de Jonadab hijo de Recab tuvieron por firme el mandamiento que les dio su padre; pero este pueblo no me ha obedecido. 
35:17 Por tanto, así ha dicho Jehová Dios de los ejércitos, Dios de Israel: He aquí traeré yo sobre Judá y sobre todos los moradores de Jerusalén todo el mal que contra ellos he hablado; porque les hablé, y no oyeron; los llamé, y no han respondido. 
35:18 Y dijo Jeremías a la familia de los recabitas: Así ha dicho Jehová de los ejércitos, Dios de Israel: Por cuanto obedecisteis al mandamiento de Jonadab vuestro padre, y guardasteis todos sus mandamientos, e hicisteis conforme a todas las cosas que os mandó; 
35:19 por tanto, así ha dicho Jehová de los ejércitos, Dios de Israel: No faltará de Jonadab hijo de Recab un varón que esté en mi presencia todos los días. 
\section*{Capítulo 36 }
El rey quema el rollo 
 
36:1 Aconteció en el cuarto año de Joacim hijo de Josías, rey de Judá, que vino esta palabra de Jehová a Jeremías, diciendo: 
36:2 Toma un rollo de libro, y escribe en él todas las palabras que te he hablado contra Israel y contra Judá, y contra todas las naciones, desde el día que comencé a hablarte, desde los días de Josías hasta hoy. 
36:3 Quizá oiga la casa de Judá todo el mal que yo pienso hacerles, y se arrepienta cada uno de su mal camino, y yo perdonaré su maldad y su pecado. 
36:4 Y llamó Jeremías a Baruc hijo de Nerías, y escribió Baruc de boca de Jeremías, en un rollo de libro, todas las palabras que Jehová le había hablado. 
36:5 Después mandó Jeremías a Baruc, diciendo: A mí se me ha prohibido entrar en la casa de Jehová. 
36:6 Entra tú, pues, y lee de este rollo que escribiste de mi boca, las palabras de Jehová a los oídos del pueblo, en la casa de Jehová, el día del ayuno; y las leerás también a oídos de todos los de Judá que vienen de sus ciudades. 
36:7 Quizá llegue la oración de ellos a la presencia de Jehová, y se vuelva cada uno de su mal camino; porque grande es el furor y la ira que ha expresado Jehová contra este pueblo. 
36:8 Y Baruc hijo de Nerías hizo conforme a todas las cosas que le mandó Jeremías profeta, leyendo en el libro las palabras de Jehová en la casa de Jehová. 
36:9 Y aconteció en el año quinto de Joacim hijo de Josías, rey de Judá, en el mes noveno, que promulgaron ayuno en la presencia de Jehová a todo el pueblo de Jerusalén y a todo el pueblo que venía de las ciudades de Judá a Jerusalén. 
36:10 Y Baruc leyó en el libro las palabras de Jeremías en la casa de Jehová, en el aposento de Gemarías hijo de Safán escriba, en el atrio de arriba, a la entrada de la puerta nueva de la casa de Jehová, a oídos del pueblo. 
36:11 Y Micaías hijo de Gemarías, hijo de Safán, habiendo oído del libro todas las palabras de Jehová, 
36:12 descendió a la casa del rey, al aposento del secretario, y he aquí que todos los príncipes estaban allí sentados, esto es: Elisama secretario, Delaía hijo de Semaías, Elnatán hijo de Acbor, Gemarías hijo de Safán, Sedequías hijo de Ananías, y todos los príncipes. 
36:13 Y les contó Micaías todas las palabras que había oído cuando Baruc leyó en el libro a oídos del pueblo. 
36:14 Entonces enviaron todos los príncipes a Jehudí hijo de Netanías, hijo de Selemías, hijo de Cusi, para que dijese a Baruc: Toma el rollo en el que leíste a oídos del pueblo, y ven. Y Baruc hijo de Nerías tomó el rollo en su mano y vino a ellos. 
36:15 Y le dijeron: Siéntate ahora, y léelo a nosotros. Y se lo leyó Baruc. 
36:16 Cuando oyeron todas aquellas palabras, cada uno se volvió espantado a su compañero, y dijeron a Baruc: Sin duda contaremos al rey todas estas palabras. 
36:17 Preguntaron luego a Baruc, diciendo: Cuéntanos ahora cómo escribiste de boca de Jeremías todas estas palabras. 
36:18 Y Baruc les dijo: El me dictaba de su boca todas estas palabras, y yo escribía con tinta en el libro. 
36:19 Entonces dijeron los príncipes a Baruc: Ve y escóndete, tú y Jeremías, y nadie sepa dónde estáis. 
36:20 Y entraron a donde estaba el rey, al atrio, habiendo depositado el rollo en el aposento de Elisama secretario; y contaron a oídos del rey todas estas palabras. 
36:21 Y envió el rey a Jehudí a que tomase el rollo, el cual lo tomó del aposento de Elisama secretario, y leyó en él Jehudí a oídos del rey, y a oídos de todos los príncipes que junto al rey estaban. 
36:22 Y el rey estaba en la casa de invierno en el mes noveno, y había un brasero ardiendo delante de él. 
36:23 Cuando Jehudí había leído tres o cuatro planas, lo rasgó el rey con un cortaplumas de escriba, y lo echó en el fuego que había en el brasero, hasta que todo el rollo se consumió sobre el fuego que en el brasero había. 
36:24 Y no tuvieron temor ni rasgaron sus vestidos el rey y todos sus siervos que oyeron todas estas palabras. 
36:25 Y aunque Elnatán y Delaía y Gemarías rogaron al rey que no quemase aquel rollo, no los quiso oír. 
36:26 También mandó el rey a Jerameel hijo de Hamelec, a Seraías hijo de Azriel y a Selemías hijo de Abdeel, para que prendiesen a Baruc el escribiente y al profeta Jeremías; pero Jehová los escondió. 
36:27 Y vino palabra de Jehová a Jeremías, después que el rey quemó el rollo, las palabras que Baruc había escrito de boca de Jeremías, diciendo: 
36:28 Vuelve a tomar otro rollo, y escribe en él todas las palabras primeras que estaban en el primer rollo que quemó Joacim rey de Judá. 
36:29 Y dirás a Joacim rey de Judá: Así ha dicho Jehová: Tú quemaste este rollo, diciendo: ¿Por qué escribiste en él, diciendo: De cierto vendrá el rey de Babilonia, y destruirá esta tierra, y hará que no queden en ella ni hombres ni animales? 
36:30 Por tanto, así ha dicho Jehová acerca de Joacim rey de Judá: No tendrá quien se siente sobre el trono de David; y su cuerpo será echado al calor del día y al hielo de la noche. 
36:31 Y castigaré su maldad en él, y en su descendencia y en sus siervos; y traeré sobre ellos, y sobre los moradores de Jerusalén y sobre los varones de Judá, todo el mal que les he anunciado y no escucharon. 
36:32 Y tomó Jeremías otro rollo y lo dio a Baruc hijo de Nerías escriba; y escribió en él de boca de Jeremías todas las palabras del libro que quemó en el fuego Joacim rey de Judá; y aun fueron añadidas sobre ellas muchas otras palabras semejantes. 
\section*{Capítulo 37 }
Encarcelamiento de Jeremías 
 
37:1 En lugar de Conías hijo de Joacim reinó el rey Sedequías hijo de Josías, al cual Nabucodonosor rey de Babilonia constituyó por rey en la tierra de Judá. 
37:2 Pero no obedeció él ni sus siervos ni el pueblo de la tierra a las palabras de Jehová, las cuales dijo por el profeta Jeremías. 
37:3 Y envió el rey Sedequías a Jucal hijo de Selemías, y al sacerdote Sofonías hijo de Maasías, para que dijesen al profeta Jeremías: Ruega ahora por nosotros a Jehová nuestro Dios. 
37:4 Y Jeremías entraba y salía en medio del pueblo; porque todavía no lo habían puesto en la cárcel. 
37:5 Y cuando el ejército de Faraón había salido de Egipto, y llegó noticia de ello a oídos de los caldeos que tenían sitiada a Jerusalén, se retiraron de Jerusalén. 
37:6 Entonces vino palabra de Jehová al profeta Jeremías, diciendo: 
37:7 Así ha dicho Jehová Dios de Israel: Diréis así al rey de Judá, que os envió a mí para que me consultaseis: He aquí que el ejército de Faraón que había salido en vuestro socorro, se volvió a su tierra en Egipto. 
37:8 Y volverán los caldeos y atacarán esta ciudad, y la tomarán y la pondrán a fuego. 
37:9 Así ha dicho Jehová: No os engañéis a vosotros mismos, diciendo: Sin duda ya los caldeos se apartarán de nosotros; porque no se apartarán. 
37:10 Porque aun cuando hirieseis a todo el ejército de los caldeos que pelean contra vosotros, y quedasen de ellos solamente hombres heridos, cada uno se levantará de su tienda, y pondrán esta ciudad a fuego. 
37:11 Y aconteció que cuando el ejército de los caldeos se retiró de Jerusalén a causa del ejército de Faraón, 
37:12 salía Jeremías de Jerusalén para irse a tierra de Benjamín, para apartarse de en medio del pueblo. 
37:13 Y cuando fue a la puerta de Benjamín, estaba allí un capitán que se llamaba Irías hijo de Selemías, hijo de Hananías, el cual apresó al profeta Jeremías, diciendo: Tú te pasas a los caldeos. 
37:14 Y Jeremías dijo: Falso; no me paso a los caldeos. Pero él no lo escuchó, sino prendió Irías a Jeremías, y lo llevó delante de los príncipes. 
37:15 Y los príncipes se airaron contra Jeremías, y le azotaron y le pusieron en prisión en la casa del escriba Jonatán, porque la habían convertido en cárcel. 
37:16 Entró, pues, Jeremías en la casa de la cisterna, y en las bóvedas. Y habiendo estado allá Jeremías por muchos días, 
37:17 el rey Sedequías envió y le sacó; y le preguntó el rey secretamente en su casa, y dijo: ¿Hay palabra de Jehová? Y Jeremías dijo: Hay. Y dijo más: En mano del rey de Babilonia serás entregado. 
37:18 Dijo también Jeremías al rey Sedequías: ¿En qué pequé contra ti, y contra tus siervos, y contra este pueblo, para que me pusieseis en la cárcel? 
37:19 ¿Y dónde están vuestros profetas que os profetizaban diciendo: No vendrá el rey de Babilonia contra vosotros, ni contra esta tierra? 
37:20 Ahora pues, oye, te ruego, oh rey mi señor; caiga ahora mi súplica delante de ti, y no me hagas volver a casa del escriba Jonatán, para que no muera allí. 
37:21 Entonces dio orden el rey Sedequías, y custodiaron a Jeremías en el patio de la cárcel, haciéndole dar una torta de pan al día, de la calle de los Panaderos, hasta que todo el pan de la ciudad se gastase. Y quedó Jeremías en el patio de la cárcel. 
\section*{Capítulo 38} 
Jeremías en la cisterna 
 
38:1 Oyeron Sefatías hijo de Matán, Gedalías hijo de Pasur, Jucal hijo de Selemías, y Pasur hijo de Malquías, las palabras que Jeremías hablaba a todo el pueblo, diciendo: 
38:2 Así ha dicho Jehová: El que se quedare en esta ciudad morirá a espada, o de hambre, o de pestilencia; mas el que se pasare a los caldeos vivirá, pues su vida le será por botín, y vivirá. 
38:3 Así ha dicho Jehová: De cierto será entregada esta ciudad en manos del ejército del rey de Babilonia, y la tomará. 
38:4 Y dijeron los príncipes al rey: Muera ahora este hombre; porque de esta manera hace desmayar las manos de los hombres de guerra que han quedado en esta ciudad, y las manos de todo el pueblo, hablándoles tales palabras; porque este hombre no busca la paz de este pueblo, sino el mal. 
38:5 Y dijo el rey Sedequías: He aquí que él está en vuestras manos; pues el rey nada puede hacer contra vosotros. 
38:6 Entonces tomaron ellos a Jeremías y lo hicieron echar en la cisterna de Malquías hijo de Hamelec, que estaba en el patio de la cárcel; y metieron a Jeremías con sogas. Y en la cisterna no había agua, sino cieno, y se hundió Jeremías en el cieno. 
38:7 Y oyendo Ebed-melec, hombre etíope, eunuco de la casa real, que habían puesto a Jeremías en la cisterna, y estando sentado el rey a la puerta de Benjamín, 
38:8 Ebed-melec salió de la casa del rey y habló al rey, diciendo: 
38:9 Mi señor el rey, mal hicieron estos varones en todo lo que han hecho con el profeta Jeremías, al cual hicieron echar en la cisterna; porque allí morirá de hambre, pues no hay más pan en la ciudad. 
38:10 Entonces mandó el rey al mismo etíope Ebed-melec, diciendo: Toma en tu poder treinta hombres de aquí, y haz sacar al profeta Jeremías de la cisterna, antes que muera. 
38:11 Y tomó Ebed-melec en su poder a los hombres, y entró a la casa del rey debajo de la tesorería, y tomó de allí trapos viejos y ropas raídas y andrajosas, y los echó a Jeremías con sogas en la cisterna. 
38:12 Y dijo el etíope Ebed-melec a Jeremías: Pon ahora esos trapos viejos y ropas raídas y andrajosas, bajo los sobacos, debajo de las sogas. Y lo hizo así Jeremías. 
38:13 De este modo sacaron a Jeremías con sogas, y lo subieron de la cisterna; y quedó Jeremías en el patio de la cárcel. 
Sedequías consulta secretamente a Jeremías 
38:14 Después envió el rey Sedequías, e hizo traer al profeta Jeremías a su presencia, en la tercera entrada de la casa de Jehová. Y dijo el rey a Jeremías: Te haré una pregunta; no me encubras ninguna cosa. 
38:15 Y Jeremías dijo a Sedequías: Si te lo declarare, ¿no es verdad que me matarás? y si te diere consejo, no me escucharás. 
38:16 Y juró el rey Sedequías en secreto a Jeremías, diciendo: Vive Jehová que nos hizo esta alma, que no te mataré, ni te entregaré en mano de estos varones que buscan tu vida. 
38:17 Entonces dijo Jeremías a Sedequías: Así ha dicho Jehová Dios de los ejércitos, Dios de Israel: Si te entregas en seguida a los príncipes del rey de Babilonia, tu alma vivirá, y esta ciudad no será puesta a fuego, y vivirás tú y tu casa. 
38:18 Pero si no te entregas a los príncipes del rey de Babilonia, esta ciudad será entregada en mano de los caldeos, y la pondrán a fuego, y tú no escaparás de sus manos. 
38:19 Y dijo el rey Sedequías a Jeremías: Tengo temor de los judíos que se han pasado a los caldeos, no sea que me entreguen en sus manos y me escarnezcan. 
38:20 Y dijo Jeremías: No te entregarán. Oye ahora la voz de Jehová que yo te hablo, y te irá bien y vivirás. 
38:21 Pero si no quieres entregarte, esta es la palabra que me ha mostrado Jehová: 
38:22 He aquí que todas las mujeres que han quedado en casa del rey de Judá serán sacadas a los príncipes del rey de Babilonia; y ellas mismas dirán: Te han engañado, y han prevalecido contra ti tus amigos; hundieron en el cieno tus pies, se volvieron atrás. 
38:23 Sacarán, pues, todas tus mujeres y tus hijos a los caldeos, y tú no escaparás de sus manos, sino que por mano del rey de Babilonia serás apresado, y a esta ciudad quemará a fuego. 
38:24 Y dijo Sedequías a Jeremías: Nadie sepa estas palabras, y no morirás. 
38:25 Y si los príncipes oyeren que yo he hablado contigo, y vinieren a ti y te dijeren: Decláranos ahora qué hablaste con el rey, no nos lo encubras, y no te mataremos; asimismo qué te dijo el rey; 
38:26 les dirás: Supliqué al rey que no me hiciese volver a casa de Jonatán para que no me muriese allí. 
38:27 Y vinieron luego todos los príncipes a Jeremías, y le preguntaron; y él les respondió conforme a todo lo que el rey le había mandado. Con esto se alejaron de él, porque el asunto no se había oído. 
38:28 Y quedó Jeremías en el patio de la cárcel hasta el día que fue tomada Jerusalén; y allí estaba cuando Jerusalén fue tomada. 
\section*{Capítulo 39 }
Caída de Jerusalén 
 
39:1 En el noveno año de Sedequías rey de Judá, en el mes décimo, vino Nabucodonosor rey de Babilonia con todo su ejército contra Jerusalén, y la sitiaron. 
39:2 Y en el undécimo año de Sedequías, en el mes cuarto, a los nueve días del mes se abrió brecha en el muro de la ciudad. 
39:3 Y entraron todos los príncipes del rey de Babilonia, y acamparon a la puerta de en medio: Nergal-sarezer, Samgar-nebo, Sarsequim el Rabsaris, Nergal-sarezer el Rabmag y todos los demás príncipes del rey de Babilonia. 
39:4 Y viéndolos Sedequías rey de Judá y todos los hombres de guerra, huyeron y salieron de noche de la ciudad por el camino del huerto del rey, por la puerta entre los dos muros; y salió el rey por el camino del Arabá. 
39:5 Pero el ejército de los caldeos los siguió, y alcanzaron a Sedequías en los llanos de Jericó; y le tomaron, y le hicieron subir a Ribla en tierra de Hamat, donde estaba Nabucodonosor rey de Babilonia, y le sentenció. 
39:6 Y degolló el rey de Babilonia a los hijos de Sedequías en presencia de éste en Ribla, haciendo asimismo degollar el rey de Babilonia a todos los nobles de Judá. 
39:7 Y sacó los ojos del rey Sedequías, y le aprisionó con grillos para llevarle a Babilonia. 
39:8 Y los caldeos pusieron a fuego la casa del rey y las casas del pueblo, y derribaron los muros de Jerusalén. 
39:9 Y al resto del pueblo que había quedado en la ciudad, y a los que se habían adherido a él, con todo el resto del pueblo que había quedado, Nabuzaradán capitán de la guardia los transportó a Babilonia. 
39:10 Pero Nabuzaradán capitán de la guardia hizo quedar en tierra de Judá a los pobres del pueblo que no tenían nada, y les dio viñas y heredades. 
Nabucodonosor cuida de Jeremías 
39:11 Y Nabucodonosor había ordenado a Nabuzaradán capitán de la guardia acerca de Jeremías, diciendo: 
39:12 Tómale y vela por él, y no le hagas mal alguno, sino que harás con él como él te dijere. 
39:13 Envió, por tanto, Nabuzaradán capitán de la guardia, y Nabusazbán el Rabsaris, Nergal-sarezer el Rabmag y todos los príncipes del rey de Babilonia; 
39:14 enviaron entonces y tomaron a Jeremías del patio de la cárcel, y lo entregaron a Gedalías hijo de Ahicam, hijo de Safán, para que lo sacase a casa; y vivió entre el pueblo. 
Dios promete librar a Ebed-melec 
39:15 Y había venido palabra de Jehová a Jeremías, estando preso en el patio de la cárcel, diciendo; 
39:16 Ve y habla a Ebed-melec etíope, diciendo: Así ha dicho Jehová de los ejércitos, Dios de Israel: He aquí yo traigo mis palabras sobre esta ciudad para mal, y no para bien; y sucederá esto en aquel día en presencia tuya. 
39:17 Pero en aquel día yo te libraré, dice Jehová, y no serás entregado en manos de aquellos a quienes tú temes. 
39:18 Porque ciertamente te libraré, y no caerás a espada, sino que tu vida te será por botín, porque tuviste confianza en mí, dice Jehová. 
\section*{Capítulo 40 }
Jeremías y el remanente con Gedalías 
 
40:1 Palabra de Jehová que vino a Jeremías, después que Nabuzaradán capitán de la guardia le envió desde Ramá, cuando le tomó estando atado con cadenas entre todos los cautivos de Jerusalén y de Judá que iban deportados a Babilonia. 
40:2 Tomó, pues, el capitán de la guardia a Jeremías y le dijo: Jehová tu Dios habló este mal contra este lugar; 
40:3 y lo ha traído y hecho Jehová según lo había dicho; porque pecasteis contra Jehová, y no oísteis su voz, por eso os ha venido esto. 
40:4 Y ahora yo te he soltado hoy de las cadenas que tenías en tus manos. Si te parece bien venir conmigo a Babilonia, ven, y yo velaré por ti; pero si no te parece bien venir conmigo a Babilonia, déjalo. Mira, toda la tierra está delante de ti; vé a donde mejor y más cómodo te parezca ir. 
40:5 Si prefieres quedarte, vuélvete a Gedalías hijo de Ahicam, hijo de Safán, al cual el rey de Babilonia ha puesto sobre todas las ciudades de Judá, y vive con él en medio del pueblo; o ve a donde te parezca más cómodo ir. Y le dio el capitán de la guardia provisiones y un presente, y le despidió. 
40:6 Se fue entonces Jeremías a Gedalías hijo de Ahicam, a Mizpa, y habitó con él en medio del pueblo que había quedado en la tierra. 
40:7 Cuando todos los jefes del ejército que estaban por el campo, ellos y sus hombres, oyeron que el rey de Babilonia había puesto a Gedalías hijo de Ahicam para gobernar la tierra, y que le había encomendado los hombres y las mujeres y los niños, y los pobres de la tierra que no fueron transportados a Babilonia, 
40:8 vinieron luego a Gedalías en Mizpa; esto es, Ismael hijo de Netanías, Johanán y Jonatán hijos de Carea, Seraías hijo de Tanhumet, los hijos de Efai netofatita, y Jezanías hijo de un maacateo, ellos y sus hombres. 
40:9 Y les juró Gedalías hijo de Ahicam, hijo de Safán, a ellos y a sus hombres, diciendo: No tengáis temor de servir a los caldeos; habitad en la tierra, y servid al rey de Babilonia, y os irá bien. 
40:10 Y he aquí que yo habito en Mizpa, para estar delante de los caldeos que vendrán a nosotros; mas vosotros tomad el vino, los frutos del verano y el aceite, y ponedlos en vuestros almacenes, y quedaos en vuestras ciudades que habéis tomado. 
40:11 Asimismo todos los judíos que estaban en Moab, y entre los hijos de Amón, y en Edom, y los que estaban en todas las tierras, cuando oyeron decir que el rey de Babilonia había dejado a algunos en Judá, y que había puesto sobre ellos a Gedalías hijo de Ahicam, hijo de Safán, 
40:12 todos estos judíos regresaron entonces de todos los lugares adonde habían sido echados, y vinieron a tierra de Judá, a Gedalías en Mizpa; y recogieron vino y abundantes frutos. 
Conspiración de Ismael contra Gedalías 
40:13 Y Johanán hijo de Carea y todos los príncipes de la gente de guerra que estaban en el campo, vinieron a Gedalías en Mizpa, 
40:14 Y le dijeron: ¿No sabes que Baalis rey de los hijos de Amón ha enviado a Ismael hijo de Netanías para matarte? Mas Gedalías hijo de Ahicam no les creyó. 
40:15 Entonces Johanán hijo de Carea habló a Gedalías en secreto en Mizpa, diciendo: Yo iré ahora y mataré a Ismael hijo de Netanías, y ningún hombre lo sabrá. ¿Por qué te ha de matar, y todos los judíos que se han reunido a ti se dispersarán, y perecerá el resto de Judá? 
40:16 Pero Gedalías hijo de Ahicam dijo a Johanán hijo de Carea: No hagas esto, porque es falso lo que tú dices de Ismael. 
\section*{Capítulo 41 }
 
41:1 Aconteció en el mes séptimo que vino Ismael hijo de Netanías, hijo de Elisama, de la descendencia real, y algunos príncipes del rey y diez hombres con él, a Gedalías hijo de Ahicam en Mizpa; y comieron pan juntos allí en Mizpa. 
41:2 Y se levantó Ismael hijo de Netanías y los diez hombres que con él estaban, e hirieron a espada a Gedalías hijo de Ahicam, hijo de Safán, matando así a aquel a quien el rey de Babilonia había puesto para gobernar la tierra. 
41:3 Asimismo mató Ismael a todos los judíos que estaban con Gedalías en Mizpa, y a los soldados caldeos que allí estaban. 
41:4 Sucedió además, un día después que mató a Gedalías, cuando nadie lo sabía aún, 
41:5 que venían unos hombres de Siquem, de Silo y de Samaria, ochenta hombres, raída la barba y rotas las ropas, y rasguñados, y traían en sus manos ofrenda e incienso para llevar a la casa de Jehová. 
41:6 Y de Mizpa les salió al encuentro, llorando, Ismael el hijo de Netanías. Y aconteció que cuando los encontró, les dijo: Venid a Gedalías hijo de Ahicam. 
41:7 Y cuando llegaron dentro de la ciudad, Ismael hijo de Netanías los degolló, y los echó dentro de una cisterna, él y los hombres que con él estaban. 
41:8 Mas entre aquéllos fueron hallados diez hombres que dijeron a Ismael: No nos mates; porque tenemos en el campo tesoros de trigos y cebadas y aceites y miel. Y los dejó, y no los mató entre sus hermanos. 
41:9 Y la cisterna en que echó Ismael todos los cuerpos de los hombres que mató a causa de Gedalías, era la misma que había hecho el rey Asa a causa de Baasa rey de Israel; Ismael hijo de Netanías la llenó de muertos. 
41:10 Después llevó Ismael cautivo a todo el resto del pueblo que estaba en Mizpa, a las hijas del rey y a todo el pueblo que en Mizpa había quedado, el cual había encargado Nabuzaradán capitán de la guardia a Gedalías hijo de Ahicam. Los llevó, pues, cautivos Ismael hijo de Netanías, y se fue para pasarse a los hijos de Amón. 
41:11 Y oyeron Johanán hijo de Carea y todos los príncipes de la gente de guerra que estaban con él, todo el mal que había hecho Ismael hijo de Netanías. 
41:12 Entonces tomaron a todos los hombres y fueron a pelear contra Ismael hijo de Netanías, y lo hallaron junto al gran estanque que está en Gabaón. 
41:13 Y aconteció que cuando todo el pueblo que estaba con Ismael vio a Johanán hijo de Carea y a todos los capitanes de la gente de guerra que estaban con él, se alegraron. 
41:14 Y todo el pueblo que Ismael había traído cautivo de Mizpa se volvió y fue con Johanán hijo de Carea. 
41:15 Pero Ismael hijo de Netanías escapó delante de Johanán con ocho hombres, y se fue a los hijos de Amón. 
41:16 Y Johanán hijo de Carea y todos los capitanes de la gente de guerra que con él estaban tomaron a todo el resto del pueblo que había recobrado de Ismael hijo de Netanías, a quienes llevó de Mizpa después que mató a Gedalías hijo de Ahicam; hombres de guerra, mujeres, niños y eunucos, que Johanán había traído de Gabaón; 
41:17 y fueron y habitaron en Gerutquimam, que está cerca de Belén, a fin de ir y meterse en Egipto, 
41:18 a causa de los caldeos; porque los temían, por haber dado muerte Ismael hijo de Netanías a Gedalías hijo de Ahicam, al cual el rey de Babilonia había puesto para gobernar la tierra. 
\section*{Capítulo 42 }
Mensaje a Johanán 
 
42:1 Vinieron todos los oficiales de la gente de guerra, y Johanán hijo de Carea, Jezanías hijo de Osaías, y todo el pueblo desde el menor hasta el mayor, 
42:2 y dijeron al profeta Jeremías: Acepta ahora nuestro ruego delante de ti, y ruega por nosotros a Jehová tu Dios por todo este resto (pues de muchos hemos quedado unos pocos, como nos ven tus ojos), 
42:3 para que Jehová tu Dios nos enseñe el camino por donde vayamos, y lo que hemos de hacer. 
42:4 Y el profeta Jeremías les dijo: He oído. He aquí que voy a orar a Jehová vuestro Dios, como habéis dicho, y todo lo que Jehová os respondiere, os enseñaré; no os reservaré palabra. 
42:5 Y ellos dijeron a Jeremías: Jehová sea entre nosotros testigo de la verdad y de la lealtad, si no hiciéremos conforme a todo aquello para lo cual Jehová tu Dios te enviare a nosotros. 
42:6 Sea bueno, sea malo, a la voz de Jehová nuestro Dios al cual te enviamos, obedeceremos, para que obedeciendo a la voz de Jehová nuestro Dios nos vaya bien. 
42:7 Aconteció que al cabo de diez días vino palabra de Jehová a Jeremías. 
42:8 Y llamó a Johanán hijo de Carea y a todos los oficiales de la gente de guerra que con él estaban, y a todo el pueblo desde el menor hasta el mayor; 
42:9 y les dijo: Así ha dicho Jehová Dios de Israel, al cual me enviasteis para presentar vuestros ruegos en su presencia: 
42:10 Si os quedareis quietos en esta tierra, os edificaré, y no os destruiré; os plantaré, y no os arrancaré; porque estoy arrepentido del mal que os he hecho. 
42:11 No temáis de la presencia del rey de Babilonia, del cual tenéis temor; no temáis de su presencia, ha dicho Jehová, porque con vosotros estoy yo para salvaros y libraros de su mano; 
42:12 y tendré de vosotros misericordia, y él tendrá misericordia de vosotros y os hará regresar a vuestra tierra. 
42:13 Mas si dijereis: No moraremos en esta tierra, no obedeciendo así a la voz de Jehová vuestro Dios, 
42:14 diciendo: No, sino que entraremos en la tierra de Egipto, en la cual no veremos guerra, ni oiremos sonido de trompeta, ni padeceremos hambre, y allá moraremos; 
42:15 ahora por eso, oíd la palabra de Jehová, remanente de Judá: Así ha dicho Jehová de los ejércitos, Dios de Israel: Si vosotros volviereis vuestros rostros para entrar en Egipto, y entrareis para morar allá, 
42:16 sucederá que la espada que teméis, os alcanzará allí en la tierra de Egipto, y el hambre de que tenéis temor, allá en Egipto os perseguirá; y allí moriréis. 
42:17 Todos los hombres que volvieren sus rostros para entrar en Egipto para morar allí, morirán a espada, de hambre y de pestilencia; no habrá de ellos quien quede vivo, ni quien escape delante del mal que traeré yo sobre ellos. 
42:18 Porque así ha dicho Jehová de los ejércitos, Dios de Israel: Como se derramó mi enojo y mi ira sobre los moradores de Jerusalén, así se derramará mi ira sobre vosotros cuando entrareis en Egipto; y seréis objeto de execración y de espanto, y de maldición y de afrenta; y no veréis más este lugar. 
42:19 Jehová habló sobre vosotros, oh remanente de Judá: No vayáis a Egipto; sabed ciertamente que os lo aviso hoy. 
42:20 ¿Por qué hicisteis errar vuestras almas? Pues vosotros me enviasteis a Jehová vuestro Dios, diciendo: Ora por nosotros a Jehová nuestro Dios, y haznos saber todas las cosas que Jehová nuestro Dios dijere, y lo haremos. 
42:21 Y os lo he declarado hoy, y no habéis obedecido a la voz de Jehová vuestro Dios, ni a todas las cosas por las cuales me envió a vosotros. 
42:22 Ahora, pues, sabed de cierto que a espada, de hambre y de pestilencia moriréis en el lugar donde deseasteis entrar para morar allí. 
\section*{Capítulo 43 }
La emigración a Egipto 
 
43:1 Aconteció que cuando Jeremías acabó de hablar a todo el pueblo todas las palabras de Jehová Dios de ellos, todas estas palabras por las cuales Jehová Dios de ellos le había enviado a ellos mismos, 
43:2 dijo Azarías hijo de Osaías y Johanán hijo de Carea, y todos los varones soberbios dijeron a Jeremías: Mentira dices; no te ha enviado Jehová nuestro Dios para decir: No vayáis a Egipto para morar allí, 
43:3 sino que Baruc hijo de Nerías te incita contra nosotros, para entregarnos en manos de los caldeos, para matarnos y hacernos transportar a Babilonia. 
43:4 No obedeció, pues, Johanán hijo de Carea y todos los oficiales de la gente de guerra y todo el pueblo, a la voz de Jehová para quedarse en tierra de Judá, 
43:5 sino que tomó Johanán hijo de Carea y todos los oficiales de la gente de guerra, a todo el remanente de Judá que se había vuelto de todas las naciones donde había sido echado, para morar en tierra de Judá; 
43:6 a hombres y mujeres y niños, y a las hijas del rey y a toda persona que había dejado Nabuzaradán capitán de la guardia con Gedalías hijo de Ahicam, hijo de Safán, y al profeta Jeremías y a Baruc hijo de Nerías, 
43:7 y entraron en tierra de Egipto, porque no obedecieron a la voz de Jehová; y llegaron hasta Tafnes. 
43:8 Y vino palabra de Jehová a Jeremías en Tafnes, diciendo: 
43:9 Toma con tu mano piedras grandes, y cúbrelas de barro en el enladrillado que está a la puerta de la casa de Faraón en Tafnes, a vista de los hombres de Judá; 
43:10 y diles: Así ha dicho Jehová de los ejércitos, Dios de Israel: He aquí yo enviaré y tomaré a Nabucodonosor rey de Babilonia, mi siervo, y pondré su trono sobre estas piedras que he escondido, y extenderá su pabellón sobre ellas. 
43:11 Y vendrá y asolará la tierra de Egipto; los que a muerte, a muerte, y los que a cautiverio, a cautiverio, y los que a espada, a espada. 
43:12 Y pondrá fuego a los templos de los dioses de Egipto y los quemará, y a ellos los llevará cautivos; y limpiará la tierra de Egipto, como el pastor limpia su capa, y saldrá de allá en paz. 
43:13 Además quebrará las estatuas de Bet-semes, que está en tierra de Egipto, y los templos de los dioses de Egipto quemará a fuego. 
\section*{Capítulo 44 }
Jeremías profetiza a los judíos en Egipto 
 
44:1 Palabra que vino a Jeremías acerca de todos los judíos que moraban en la tierra de Egipto, que vivían en Migdol, en Tafnes, en Menfis y en tierra de Patros, diciendo: 
44:2 Así ha dicho Jehová de los ejércitos, Dios de Israel: Vosotros habéis visto todo el mal que traje sobre Jerusalén y sobre todas las ciudades de Judá; y he aquí que ellas están el día de hoy asoladas; no hay quien more en ellas, 
44:3 a causa de la maldad que ellos cometieron para enojarme, yendo a ofrecer incienso, honrando a dioses ajenos que ellos no habían conocido, ni vosotros ni vuestros padres. 
44:4 Y envié a vosotros todos mis siervos los profetas, desde temprano y sin cesar, para deciros: No hagáis esta cosa abominable que yo aborrezco. 
44:5 Pero no oyeron ni inclinaron su oído para convertirse de su maldad, para dejar de ofrecer incienso a dioses ajenos. 
44:6 Se derramó, por tanto, mi ira y mi furor, y se encendió en las ciudades de Judá y en las calles de Jerusalén, y fueron puestas en soledad y en destrucción, como están hoy. 
44:7 Ahora, pues, así ha dicho Jehová de los ejércitos, Dios de Israel: ¿Por qué hacéis tan grande mal contra vosotros mismos, para ser destruidos el hombre y la mujer, el muchacho y el niño de pecho de en medio de Judá, sin que os quede remanente alguno, 
44:8 haciéndome enojar con las obras de vuestras manos, ofreciendo incienso a dioses ajenos en la tierra de Egipto, adonde habéis entrado para vivir, de suerte que os acabéis, y seáis por maldición y por oprobio a todas las naciones de la tierra? 
44:9 ¿Os habéis olvidado de las maldades de vuestros padres, de las maldades de los reyes de Judá, de las maldades de sus mujeres, de vuestras maldades y de las maldades de vuestras mujeres, que hicieron en la tierra de Judá y en las calles de Jerusalén? 
44:10 No se han humillado hasta el día de hoy, ni han tenido temor, ni han caminado en mi ley ni en mis estatutos, los cuales puse delante de vosotros y delante de vuestros padres. 
44:11 Por tanto, así ha dicho Jehová de los ejércitos, Dios de Israel: He aquí que yo vuelvo mi rostro contra vosotros para mal, y para destruir a todo Judá. 
44:12 Y tomaré el resto de Judá que volvieron sus rostros para ir a tierra de Egipto para morar allí, y en tierra de Egipto serán todos consumidos; caerán a espada, y serán consumidos de hambre; a espada y de hambre morirán desde el menor hasta el mayor, y serán objeto de execración, de espanto, de maldición y de oprobio. 
44:13 Pues castigaré a los que moran en tierra de Egipto como castigué a Jerusalén, con espada, con hambre y con pestilencia. 
44:14 Y del resto de los de Judá que entraron en la tierra de Egipto para habitar allí, no habrá quien escape, ni quien quede vivo para volver a la tierra de Judá, por volver a la cual suspiran ellos para habitar allí; porque no volverán sino algunos fugitivos. 
44:15 Entonces todos los que sabían que sus mujeres habían ofrecido incienso a dioses ajenos, y todas las mujeres que estaban presentes, una gran concurrencia, y todo el pueblo que habitaba en tierra de Egipto, en Patros, respondieron a Jeremías, diciendo: 
44:16 La palabra que nos has hablado en nombre de Jehová, no la oiremos de ti; 
44:17 sino que ciertamente pondremos por obra toda palabra que ha salido de nuestra boca, para ofrecer incienso a la reina del cielo, derramándole libaciones, como hemos hecho nosotros y nuestros padres, nuestros reyes y nuestros príncipes, en las ciudades de Judá y en las plazas de Jerusalén, y tuvimos abundancia de pan, y estuvimos alegres, y no vimos mal alguno. 
44:18 Mas desde que dejamos de ofrecer incienso a la reina del cielo y de derramarle libaciones, nos falta todo, y a espada y de hambre somos consumidos. 
44:19 Y cuando ofrecimos incienso a la reina del cielo, y le derramamos libaciones, ¿acaso le hicimos nosotras tortas para tributarle culto, y le derramamos libaciones, sin consentimiento de nuestros maridos? 
44:20 Y habló Jeremías a todo el pueblo, a los hombres y a las mujeres y a todo el pueblo que le había respondido esto, diciendo: 
44:21 ¿No se ha acordado Jehová, y no ha venido a su memoria el incienso que ofrecisteis en las ciudades de Judá, y en las calles de Jerusalén, vosotros y vuestros padres, vuestros reyes y vuestros príncipes y el pueblo de la tierra? 
44:22 Y no pudo sufrirlo más Jehová, a causa de la maldad de vuestras obras, a causa de las abominaciones que habíais hecho; por tanto, vuestra tierra fue puesta en asolamiento, en espanto y en maldición, hasta quedar sin morador, como está hoy. 
44:23 Porque ofrecisteis incienso y pecasteis contra Jehová, y no obedecisteis a la voz de Jehová, ni anduvisteis en su ley ni en sus estatutos ni en sus testimonios; por tanto, ha venido sobre vosotros este mal, como hasta hoy. 
44:24 Y dijo Jeremías a todo el pueblo, y a todas las mujeres: Oíd palabra de Jehová, todos los de Judá que estáis en tierra de Egipto. 
44:25 Así ha hablado Jehová de los ejércitos, Dios de Israel, diciendo: Vosotros y vuestras mujeres hablasteis con vuestras bocas, y con vuestras manos lo ejecutasteis, diciendo: Cumpliremos efectivamente nuestros votos que hicimos, de ofrecer incienso a la reina del cielo y derramarle libaciones; confirmáis a la verdad vuestros votos, y ponéis vuestros votos por obra. 
44:26 Por tanto, oíd palabra de Jehová, todo Judá que habitáis en tierra de Egipto: He aquí he jurado por mi grande nombre, dice Jehová, que mi nombre no será invocado más en toda la tierra de Egipto por boca de ningún hombre de Judá, diciendo: Vive Jehová el Señor. 
44:27 He aquí que yo velo sobre ellos para mal, y no para bien; y todos los hombres de Judá que están en tierra de Egipto serán consumidos a espada y de hambre, hasta que perezcan del todo. 
44:28 Y los que escapen de la espada volverán de la tierra de Egipto a la tierra de Judá, pocos hombres; sabrá, pues, todo el resto de Judá que ha entrado en Egipto a morar allí, la palabra de quién ha de permanecer: si la mía, o la suya. 
44:29 Y esto tendréis por señal, dice Jehová, de que en este lugar os castigo, para que sepáis que de cierto permanecerán mis palabras para mal sobre vosotros. 
44:30 Así ha dicho Jehová: He aquí que yo entrego a Faraón Hofra rey de Egipto en mano de sus enemigos, y en mano de los que buscan su vida, así como entregué a Sedequías rey de Judá en mano de Nabucodonosor rey de Babilonia, su enemigo que buscaba su vida. 
\section*{Capítulo 45 }
Mensaje a Baruc 
 
45:1 Palabra que habló el profeta Jeremías a Baruc hijo de Nerías, cuando escribía en el libro estas palabras de boca de Jeremías, en el año cuarto de Joacim hijo de Josías rey de Judá, diciendo: 
45:2 Así ha dicho Jehová Dios de Israel a ti, oh Baruc: 
45:3 Tú dijiste: ¡Ay de mí ahora! porque ha añadido Jehová tristeza a mi dolor; fatigado estoy de gemir, y no he hallado descanso. 
45:4 Así le dirás: Ha dicho Jehová: He aquí que yo destruyo a los que edifiqué, y arranco a los que planté, y a toda esta tierra. 
45:5 ¿Y tú buscas para ti grandezas? No las busques; porque he aquí que yo traigo mal sobre toda carne, ha dicho Jehová; pero a ti te daré tu vida por botín en todos los lugares adonde fueres. 
\section*{Capítulo 46 }
Profecías acerca de Egipto 
 
46:1 Palabra de Jehová que vino al profeta Jeremías, contra las naciones. 
46:2 Con respecto a Egipto: contra el ejército de Faraón Necao rey de Egipto, que estaba cerca del río Eufrates en Carquemis, a quien destruyó Nabucodonosor rey de Babilonia, en el año cuarto de Joacim hijo de Josías, rey de Judá. 
46:3 Preparad escudo y pavés, y venid a la guerra. 
46:4 Uncid caballos y subid, vosotros los jinetes, y poneos con yelmos; limpiad las lanzas, vestíos las corazas. 
46:5 ¿Por qué los vi medrosos, retrocediendo? Sus valientes fueron deshechos, y huyeron sin volver a mirar atrás; miedo de todas partes, dice Jehová. 
46:6 No huya el ligero, ni el valiente escape; al norte junto a la ribera del Eufrates tropezaron y cayeron. 
46:7 ¿Quién es éste que sube como río, y cuyas aguas se mueven como ríos? 
46:8 Egipto como río se ensancha, y las aguas se mueven como ríos, y dijo: Subiré, cubriré la tierra, destruiré a la ciudad y a los que en ella moran. 
46:9 Subid, caballos, y alborotaos, carros, y salgan los valientes; los etíopes y los de Put que toman escudo, y los de Lud que toman y entesan arco. 
46:10 Mas ese día será para Jehová Dios de los ejércitos día de retribución, para vengarse de sus enemigos; y la espada devorará y se saciará, y se embriagará de la sangre de ellos; porque sacrificio será para Jehová Dios de los ejércitos, en tierra del norte junto al río Eufrates. 
46:11 Sube a Galaad, y toma bálsamo, virgen hija de Egipto; por demás multiplicarás las medicinas; no hay curación para ti. 
46:12 Las naciones oyeron tu afrenta, y tu clamor llenó la tierra; porque valiente tropezó contra valiente, y cayeron ambos juntos. 
46:13 Palabra que habló Jehová al profeta Jeremías acerca de la venida de Nabucodonosor rey de Babilonia, para asolar la tierra de Egipto: 
46:14 Anunciad en Egipto, y haced saber en Migdol; haced saber también en Menfis y en Tafnes; decid: Ponte en pie y prepárate, porque espada devorará tu comarca. 
46:15 ¿Por qué ha sido derribada tu fortaleza? No pudo mantenerse firme, porque Jehová la empujó. 
46:16 Multiplicó los caídos, y cada uno cayó sobre su compañero; y dijeron: Levántate y volvámonos a nuestro pueblo, y a la tierra de nuestro nacimiento, huyamos ante la espada vencedora. 
46:17 Allí gritaron: Faraón rey de Egipto es destruido; dejó pasar el tiempo señalado. 
46:18 Vivo yo, dice el Rey, cuyo nombre es Jehová de los ejércitos, que como Tabor entre los montes, y como Carmelo junto al mar, así vendrá. 
46:19 Hazte enseres de cautiverio, moradora hija de Egipto; porque Menfis será desierto, y será asolada hasta no quedar morador. 
46:20 Becerra hermosa es Egipto; mas viene destrucción, del norte viene. 
46:21 Sus soldados mercenarios también en medio de ella como becerros engordados; porque también ellos volvieron atrás, huyeron todos sin pararse, porque vino sobre ellos el día de su quebrantamiento, el tiempo de su castigo. 
46:22 Su voz saldrá como de serpiente; porque vendrán los enemigos, y con hachas vendrán a ella como cortadores de leña. 
46:23 Cortarán sus bosques, dice Jehová, aunque sean impenetrables; porque serán más numerosos que langostas, no tendrán número. 
46:24 Se avergonzará la hija de Egipto; entregada será en manos del pueblo del norte. 
46:25 Jehová de los ejércitos, Dios de Israel, ha dicho: He aquí que yo castigo a Amón dios de Tebas, a Faraón, a Egipto, y a sus dioses y a sus reyes; así a Faraón como a los que en él confían. 
46:26 Y los entregaré en mano de los que buscan su vida, en mano de Nabucodonosor rey de Babilonia y en mano de sus siervos; pero después será habitado como en los días pasados, dice Jehová. 
46:27 Y tú no temas, siervo mío Jacob, ni desmayes, Israel; porque he aquí yo te salvaré de lejos, y a tu descendencia de la tierra de su cautividad. Y volverá Jacob, y descansará y será prosperado, y no habrá quién lo atemorice. 
46:28 Tú, siervo mío Jacob, no temas, dice Jehová, porque yo estoy contigo; porque destruiré a todas las naciones entre las cuales te he dispersado; pero a ti no te destruiré del todo, sino que te castigaré con justicia; de ninguna manera te dejaré sin castigo.   
\section*{Capítulo 47 }
Profecía sobre los filisteos 
 
47:1 Palabra de Jehová que vino al profeta Jeremías acerca de los filisteos, antes que Faraón destruyese a Gaza. 
47:2 Así ha dicho Jehová: He aquí que suben aguas del norte, y se harán torrente; inundarán la tierra y su plenitud, la ciudad y los moradores de ella; y los hombres clamarán, y lamentará todo morador de la tierra. 
47:3 Por el sonido de los cascos de sus caballos, por el alboroto de sus carros, por el estruendo de sus ruedas, los padres no cuidaron a los hijos por la debilidad de sus manos; 
47:4 a causa del día que viene para destrucción de todos los filisteos, para destruir a Tiro y a Sidón todo aliado que les queda todavía; porque Jehová destruirá a los filisteos, al resto de la costa de Caftor. 
47:5 Gaza fue rapada, Ascalón ha perecido, y el resto de su valle; ¿hasta cuándo te sajarás? 
47:6 Oh espada de Jehová, ¿hasta cuándo reposarás? Vuelve a tu vaina, reposa y sosiégate. 
47:7 ¿Cómo reposarás? pues Jehová te ha enviado contra Ascalón, y contra la costa del mar, allí te puso. 
\section*{Capítulo 48 }
Profecía sobre Moab 
 
48:1 Acerca de Moab. Así ha dicho Jehová de los ejércitos, Dios de Israel: ¡Ay de Nebo! porque fue destruida y avergonzada: Quiriataim fue tomada; fue confundida Misgab, y desmayó. 
48:2 No se alabará ya más Moab; en Hesbón maquinaron mal contra ella, diciendo: Venid, y quitémosla de entre las naciones. También tú, Madmena, serás cortada; espada irá en pos de ti. 
48:3 ¡Voz de clamor de Horonaim, destrucción y gran quebrantamiento! 
48:4 Moab fue quebrantada; hicieron que se oyese el clamor de sus pequeños. 
48:5 Porque a la subida de Luhit con llanto subirá el que llora; porque a la bajada de Horonaim los enemigos oyeron clamor de quebranto. 
48:6 Huid, salvad vuestra vida, y sed como retama en el desierto. 
48:7 Pues por cuanto confiaste en tus bienes y en tus tesoros, tú también serás tomada; y Quemos será llevado en cautiverio, sus sacerdotes y sus príncipes juntamente. 
48:8 Y vendrá destruidor a cada una de las ciudades, y ninguna ciudad escapará; se arruinará también el valle, y será destruida la llanura, como ha dicho Jehová. 
48:9 Dad alas a Moab, para que se vaya volando; pues serán desiertas sus ciudades hasta no quedar en ellas morador. 
48:10 Maldito el que hiciere indolentemente la obra de Jehová, y maldito el que detuviere de la sangre su espada. 
48:11 Quieto estuvo Moab desde su juventud, y sobre su sedimento ha estado reposado, y no fue vaciado de vasija en vasija, ni nunca estuvo en cautiverio; por tanto, quedó su sabor en él, y su olor no se ha cambiado. 
48:12 Por eso vienen días, ha dicho Jehová, en que yo le enviaré trasvasadores que le trasvasarán; y vaciarán sus vasijas, y romperán sus odres. 
48:13 Y se avergonzará Moab de Quemos, como la casa de Israel se avergonzó de Bet-el, su confianza. 
48:14 ¿Cómo, pues, diréis: Somos hombres valientes, y robustos para la guerra? 
48:15 Destruido fue Moab, y sus ciudades asoladas, y sus jóvenes escogidos descendieron al degolladero, ha dicho el Rey, cuyo nombre es Jehová de los ejércitos. 
48:16 Cercano está el quebrantamiento de Moab para venir, y su mal se apresura mucho. 
48:17 Compadeceos de él todos los que estáis alrededor suyo; y todos los que sabéis su nombre, decid: ¡Cómo se quebró la vara fuerte, el báculo hermoso! 
48:18 Desciende de la gloria, siéntate en tierra seca, moradora hija de Dibón; porque el destruidor de Moab subió contra ti, destruyó tus fortalezas. 
48:19 Párate en el camino, y mira, oh moradora de Aroer; pregunta a la que va huyendo, y a la que escapó; dile: ¿Qué ha acontecido? 
48:20 Se avergonzó Moab, porque fue quebrantado; lamentad y clamad; anunciad en Arnón que Moab es destruido. 
48:21 Vino juicio sobre la tierra de la llanura; sobre Holón, sobre Jahaza, sobre Mefaat, 
48:22 sobre Dibón, sobre Nebo, sobre Bet-diblataim, 
48:23 sobre Quiriataim, sobre Bet-gamul, sobre Bet-meón, 
48:24 sobre Queriot, sobre Bosra y sobre todas las ciudades de tierra de Moab, las de lejos y las de cerca. 
48:25 Cortado es el poder de Moab, y su brazo quebrantado, dice Jehová. 
48:26 Embriagadle, porque contra Jehová se engrandeció; y revuélquese Moab sobre su vómito, y sea también él por motivo de escarnio. 
48:27 ¿Y no te fue a ti Israel por motivo de escarnio, como si lo tomaran entre ladrones? Porque cuando de él hablaste, tú te has burlado. 
48:28 Abandonad las ciudades y habitad en peñascos, oh moradores de Moab, y sed como la paloma que hace nido en la boca de la caverna. 
48:29 Hemos oído la soberbia de Moab, que es muy soberbio, arrogante, orgulloso, altivo y altanero de corazón. 
48:30 Yo conozco, dice Jehová, su cólera, pero no tendrá efecto; sus jactancias no le aprovecharán. 
48:31 Por tanto, yo aullaré sobre Moab; sobre todo Moab haré clamor, y sobre los hombres de Kir-hares gemiré. 
48:32 Con llanto de Jazer lloraré por ti, oh vid de Sibma; tus sarmientos pasaron el mar, llegaron hasta el mar de Jazer; sobre tu cosecha y sobre tu vendimia vino el destruidor. 
48:33 Y será cortada la alegría y el regocijo de los campos fértiles, de la tierra de Moab; y de los lagares haré que falte el vino; no pisarán con canción; la canción no será canción. 
48:34 El clamor de Hesbón llega hasta Eleale; hasta Jahaza dieron su voz; desde Zoar hasta Horonaim, becerra de tres años; porque también las aguas de Nimrim serán destruidas. 
48:35 Y exterminaré de Moab, dice Jehová, a quien sacrifique sobre los lugares altos, y a quien ofrezca incienso a sus dioses. 
48:36 Por tanto, mi corazón resonará como flautas por causa de Moab, asimismo resonará mi corazón a modo de flautas por los hombres de Kir-hares; porque perecieron las riquezas que habían hecho. 
48:37 Porque toda cabeza será rapada, y toda barba raída; sobre toda mano habrá rasguños, y cilicio sobre todo lomo. 
48:38 Sobre todos los terrados de Moab, y en sus calles, todo él será llanto; porque yo quebranté a Moab como a vasija que no agrada, dice Jehová. 
48:39 ¡Lamentad! ¡Cómo ha sido quebrantado! ¡Cómo volvió la espalda Moab, y fue avergonzado! Fue Moab objeto de escarnio y de espanto a todos los que están en sus alrededores. 
48:40 Porque así ha dicho Jehová: He aquí que como águila volará, y extenderá sus alas contra Moab. 
48:41 Tomadas serán las ciudades, y tomadas serán las fortalezas; y será aquel día el corazón de los valientes de Moab como el corazón de mujer en angustias. 
48:42 Y Moab será destruido hasta dejar de ser pueblo, porque se engrandeció contra Jehová. 
48:43 Miedo y hoyo y lazo contra ti, oh morador de Moab, dice Jehová. 
48:44 El que huyere del miedo caerá en el hoyo, y el que saliere del hoyo será preso en el lazo; porque yo traeré sobre él, sobre Moab, el año de su castigo, dice Jehová. 
48:45 A la sombra de Hesbón se pararon sin fuerzas los que huían; mas salió fuego de Hesbón, y llama de en medio de Sehón, y quemó el rincón de Moab, y la coronilla de los hijos revoltosos. 
48:46 ¡Ay de ti, Moab! pereció el pueblo de Quemos; porque tus hijos fueron puestos presos para cautividad, y tus hijas para cautiverio. 
48:47 Pero haré volver a los cautivos de Moab en lo postrero de los tiempos, dice Jehová. Hasta aquí es el juicio de Moab.  
\section*{Capítulo 49 }
Profecía sobre los amonitas 
 
49:1 Acerca de los hijos de Amón.  Así ha dicho Jehová: ¿No tiene hijos Israel? ¿No tiene heredero? ¿Por qué Milcom ha desposeído a Gad, y su pueblo se ha establecido en sus ciudades? 
49:2 Por tanto, vienen días, ha dicho Jehová, en que haré oír clamor de guerra en Rabá de los hijos de Amón; y será convertida en montón de ruinas, y sus ciudades serán puestas a fuego, e Israel tomará por heredad a los que los tomaron a ellos, ha dicho Jehová. 
49:3 Lamenta, oh Hesbón, porque destruida es Hai; clamad, hijas de Rabá, vestíos de cilicio, endechad, y rodead los vallados, porque Milcom fue llevado en cautiverio, sus sacerdotes y sus príncipes juntamente. 
49:4 ¿Por qué te glorías de los valles? Tu valle se deshizo, oh hija contumaz, la que confía en sus tesoros, la que dice: ¿Quién vendrá contra mí? 
49:5 He aquí yo traigo sobre ti espanto, dice el Señor, Jehová de los ejércitos, de todos tus alrededores; y seréis lanzados cada uno derecho hacia adelante, y no habrá quien recoja a los fugitivos. 
49:6 Y después de esto haré volver a los cautivos de los hijos de Amón, dice Jehová. 
Profecía sobre Edom 
49:7 Acerca de Edom. Así ha dicho Jehová de los ejércitos: ¿No hay más sabiduría en Temán? ¿Se ha acabado el consejo en los sabios? ¿Se corrompió su sabiduría? 
49:8 Huid, volveos atrás, habitad en lugares profundos, oh moradores de Dedán; porque el quebrantamiento de Esaú traeré sobre él en el tiempo en que lo castigue. 
49:9 Si vendimiadores hubieran venido contra ti, ¿no habrían dejado rebuscos? Si ladrones de noche, ¿no habrían tomado lo que les bastase? 
49:10 Mas yo desnudaré a Esaú, descubriré sus escondrijos, y no podrá esconderse; será destruida su descendencia, sus hermanos y sus vecinos, y dejará de ser. 
49:11 Deja tus huérfanos, yo los criaré; y en mí confiarán tus viudas. 
49:12 Porque así ha dicho Jehová: He aquí que los que no estaban condenados a beber el cáliz, beberán ciertamente; ¿y serás tú absuelto del todo? No serás absuelto, sino que ciertamente beberás. 
49:13 Porque por mí he jurado, dice Jehová, que asolamiento, oprobio, soledad y maldición será Bosra, y todas sus ciudades serán desolaciones perpetuas. 
49:14 La noticia oí, que de Jehová había sido enviado mensajero a las naciones, diciendo: Juntaos y venid contra ella, y subid a la batalla. 
49:15 He aquí que te haré pequeño entre las naciones, menospreciado entre los hombres. 
49:16 Tu arrogancia te engañó, y la soberbia de tu corazón. Tú que habitas en cavernas de peñas, que tienes la altura del monte, aunque alces como águila tu nido, de allí te haré descender, dice Jehová. 
49:17 Y se convertirá Edom en desolación; todo aquel que pasare por ella se asombrará, y se burlará de todas sus calamidades. 
49:18 Como sucedió en la destrucción de Sodoma y de Gomorra y de sus ciudades vecinas, dice Jehová, así no morará allí nadie, ni la habitará hijo de hombre. 
49:19 He aquí que como león subirá de la espesura del Jordán contra la bella y robusta; porque muy pronto le haré huir de ella, y al que fuere escogido la encargaré; porque ¿quién es semejante a mí, y quién me emplazará? ¿Quién será aquel pastor que me podrá resistir? 
49:20 Por tanto, oíd el consejo que Jehová ha acordado sobre Edom, y sus pensamientos que ha resuelto sobre los moradores de Temán. Ciertamente a los más pequeños de su rebaño los arrastrarán, y destruirán sus moradas con ellos. 
49:21 Del estruendo de la caída de ellos la tierra temblará, y el grito de su voz se oirá en el Mar Rojo. 
49:22 He aquí que como águila subirá y volará, y extenderá sus alas contra Bosra; y el corazón de los valientes de Edom será en aquel día como el corazón de mujer en angustias. 
Profecía sobre Damasco 
49:23 Acerca de Damasco. Se confundieron Hamat y Arfad, porque oyeron malas nuevas; se derritieron en aguas de desmayo, no pueden sosegarse. 
49:24 Se desmayó Damasco, se volvió para huir, y le tomó temblor y angustia, y dolores le tomaron, como de mujer que está de parto. 
49:25 ¡Cómo dejaron a la ciudad tan alabada, la ciudad de mi gozo! 
49:26 Por tanto, sus jóvenes caerán en sus plazas, y todos los hombres de guerra morirán en aquel día, ha dicho Jehová de los ejércitos. 
49:27 Y haré encender fuego en el muro de Damasco, y consumirá las casas de Ben-adad. 
Profecía sobre Cedar y Hazor 
49:28 Acerca de Cedar y de los reinos de Hazor, los cuales asoló Nabucodonosor rey de Babilonia. Así ha dicho Jehová: Levantaos, subid contra Cedar, y destruid a los hijos del oriente. 
49:29 Sus tiendas y sus ganados tomarán; sus cortinas y todos sus utensilios y sus camellos tomarán para sí, y clamarán contra ellos: Miedo alrededor. 
49:30 Huid, idos muy lejos, habitad en lugares profundos, oh moradores de Hazor, dice Jehová; porque tomó consejo contra vosotros Nabucodonosor rey de Babilonia, y contra vosotros ha formado un designio. 
49:31 Levantaos, subid contra una nación pacífica que vive confiadamente, dice Jehová, que ni tiene puertas ni cerrojos, que vive solitaria. 
49:32 Serán sus camellos por botín, y la multitud de sus ganados por despojo; y los esparciré por todos los vientos, arrojados hasta el último rincón; y de todos lados les traeré su ruina, dice Jehová. 
49:33 Hazor será morada de chacales, soledad para siempre; ninguno morará allí, ni la habitará hijo de hombre. 
Profecía sobre Elam 
49:34 Palabra de Jehová que vino al profeta Jeremías acerca de Elam, en el principio del reinado de Sedequías rey de Judá, diciendo: 
49:35 Así ha dicho Jehová de los ejércitos: He aquí que yo quiebro el arco de Elam, parte principal de su fortaleza. 
49:36 Traeré sobre Elam los cuatro vientos de los cuatro puntos del cielo, y los aventaré a todos estos vientos; y no habrá nación a donde no vayan fugitivos de Elam. 
49:37 Y haré que Elam se intimide delante de sus enemigos, y delante de los que buscan su vida; y traeré sobre ellos mal, y el ardor de mi ira, dice Jehová; y enviaré en pos de ellos espada hasta que los acabe. 
49:38 Y pondré mi trono en Elam, y destruiré a su rey y a su príncipe, dice Jehová. 
49:39 Pero acontecerá en los últimos días, que haré volver a los cautivos de Elam, dice Jehová. 
\section*{Capítulo 50 }
Profecía sobre Babilonia 
 
50:1 Palabra que habló Jehová contra Babilonia,  contra la tierra de los caldeos, por medio del profeta Jeremías. 
50:2 Anunciad en las naciones, y haced saber; levantad también bandera, publicad, y no encubráis; decid: Tomada es Babilonia, Bel es confundido, deshecho es Merodac; destruidas son sus esculturas, quebrados son sus ídolos. 
50:3 Porque subió contra ella una nación del norte, la cual pondrá su tierra en asolamiento, y no habrá ni hombre ni animal que en ella more; huyeron, y se fueron. 
50:4 En aquellos días y en aquel tiempo, dice Jehová, vendrán los hijos de Israel, ellos y los hijos de Judá juntamente; e irán andando y llorando, y buscarán a Jehová su Dios. 
50:5 Preguntarán por el camino de Sion, hacia donde volverán sus rostros, diciendo: Venid, y juntémonos a Jehová con pacto eterno que jamás se ponga en olvido. 
50:6 Ovejas perdidas fueron mi pueblo; sus pastores las hicieron errar, por los montes las descarriaron; anduvieron de monte en collado, y se olvidaron de sus rediles. 
50:7 Todos los que los hallaban, los devoraban; y decían sus enemigos: No pecaremos, porque ellos pecaron contra Jehová morada de justicia, contra Jehová esperanza de sus padres. 
50:8 Huid de en medio de Babilonia, y salid de la tierra de los caldeos, y sed como los machos cabríos que van delante del rebaño. 
50:9 Porque yo levanto y hago subir contra Babilonia reunión de grandes pueblos de la tierra del norte; desde allí se prepararán contra ella, y será tomada; sus flechas son como de valiente diestro, que no volverá vacío. 
50:10 Y Caldea será para botín; todos los que la saquearen se saciarán, dice Jehová. 
50:11 Porque os alegrasteis, porque os gozasteis destruyendo mi heredad, porque os llenasteis como novilla sobre la hierba, y relinchasteis como caballos. 
50:12 Vuestra madre se avergonzó mucho, se afrentó la que os dio a luz; he aquí será la última de las naciones; desierto, sequedal y páramo. 
50:13 Por la ira de Jehová no será habitada, sino será asolada toda ella; todo hombre que pasare por Babilonia se asombrará, y se burlará de sus calamidades. 
50:14 Poneos en orden contra Babilonia alrededor, todos los que entesáis arco; tirad contra ella, no escatiméis las saetas, porque pecó contra Jehová. 
50:15 Gritad contra ella en derredor; se rindió; han caído sus cimientos, derribados son sus muros, porque es venganza de Jehová. Tomad venganza de ella; haced con ella como ella hizo. 
50:16 Destruid en Babilonia al que siembra, y al que mete hoz en tiempo de la siega; delante de la espada destructora cada uno volverá el rostro hacia su pueblo, cada uno huirá hacia su tierra. 
50:17 Rebaño descarriado es Israel; leones lo dispersaron; el rey de Asiria lo devoró primero, Nabucodonosor rey de Babilonia lo deshuesó después. 
50:18 Por tanto, así ha dicho Jehová de los ejércitos, Dios de Israel: Yo castigo al rey de Babilonia y a su tierra, como castigué al rey de Asiria. 
50:19 Y volveré a traer a Israel a su morada, y pacerá en el Carmelo y en Basán; y en el monte de Efraín y en Galaad se saciará su alma. 
50:20 En aquellos días y en aquel tiempo, dice Jehová, la maldad de Israel será buscada, y no aparecerá; y los pecados de Judá, y no se hallarán; porque perdonaré a los que yo hubiere dejado. 
50:21 Sube contra la tierra de Merataim, contra ella y contra los moradores de Pecod; destruye y mata en pos de ellos, dice Jehová, y haz conforme a todo lo que yo te he mandado. 
50:22 Estruendo de guerra en la tierra, y quebrantamiento grande. 
50:23 ¡Cómo fue cortado y quebrado el martillo de toda la tierra! ¡cómo se convirtió Babilonia en desolación entre las naciones! 
50:24 Te puse lazos, y fuiste tomada, oh Babilonia, y tú no lo supiste; fuiste hallada, y aun presa, porque provocaste a Jehová. 
50:25 Abrió Jehová su tesoro, y sacó los instrumentos de su furor; porque esta es obra de Jehová, Dios de los ejércitos, en la tierra de los caldeos. 
50:26 Venid contra ella desde el extremo de la tierra; abrid sus almacenes, convertidla en montón de ruinas, y destruidla; que no le quede nada. 
50:27 Matad a todos sus novillos; que vayan al matadero. ¡Ay de ellos! pues ha venido su día, el tiempo de su castigo. 
50:28 Voz de los que huyen y escapan de la tierra de Babilonia, para dar en Sion las nuevas de la retribución de Jehová nuestro Dios, de la venganza de su templo. 
50:29 Haced juntar contra Babilonia flecheros, a todos los que entesan arco; acampad contra ella alrededor; no escape de ella ninguno; pagadle según su obra; conforme a todo lo que ella hizo, haced con ella; porque contra Jehová se ensoberbeció, contra el Santo de Israel. 
50:30 Por tanto, sus jóvenes caerán en sus plazas, y todos sus hombres de guerra serán destruidos en aquel día, dice Jehová. 
50:31 He aquí yo estoy contra ti, oh soberbio, dice el Señor, Jehová de los ejércitos; porque tu día ha venido, el tiempo en que te castigaré. 
50:32 Y el soberbio tropezará y caerá, y no tendrá quien lo levante; y encenderé fuego en sus ciudades, y quemaré todos sus alrededores. 
50:33 Así ha dicho Jehová de los ejércitos: Oprimidos fueron los hijos de Israel y los hijos de Judá juntamente; y todos los que los tomaron cautivos los retuvieron; no los quisieron soltar. 
50:34 El redentor de ellos es el Fuerte; Jehová de los ejércitos es su nombre; de cierto abogará la causa de ellos para hacer reposar la tierra, y turbar a los moradores de Babilonia. 
50:35 Espada contra los caldeos, dice Jehová, y contra los moradores de Babilonia, contra sus príncipes y contra sus sabios. 
50:36 Espada contra los adivinos, y se entontecerán; espada contra sus valientes, y serán quebrantados. 
50:37 Espada contra sus caballos, contra sus carros, y contra todo el pueblo que está en medio de ella, y serán como mujeres; espada contra sus tesoros, y serán saqueados. 
50:38 Sequedad sobre sus aguas, y se secarán; porque es tierra de ídolos, y se entontecen con imágenes. 
50:39 Por tanto, allí morarán fieras del desierto y chacales,  morarán también en ella polluelos de avestruz; nunca más será poblada ni se habitará por generaciones y generaciones. 
50:40 Como en la destrucción que Dios hizo de Sodoma y de Gomorra y de sus ciudades vecinas, dice Jehová, así no morará allí hombre, ni hijo de hombre la habitará. 
50:41 He aquí viene un pueblo del norte, y una nación grande y muchos reyes se levantarán de los extremos de la tierra. 
50:42 Arco y lanza manejarán; serán crueles, y no tendrán compasión; su voz rugirá como el mar, y montarán sobre caballos; se prepararán contra ti como hombres a la pelea, oh hija de Babilonia. 
50:43 Oyó la noticia el rey de Babilonia, y sus manos se debilitaron; angustia le tomó, dolor como de mujer de parto. 
50:44 He aquí que como león subirá de la espesura del Jordán a la morada fortificada; porque muy pronto le haré huir de ella, y al que yo escoja la encargaré; porque ¿quién es semejante a mí? ¿y quién me emplazará? ¿o quién será aquel pastor que podrá resistirme? 
50:45 Por tanto, oíd la determinación que Jehová ha acordado contra Babilonia, y los pensamientos que ha formado contra la tierra de los caldeos: Ciertamente a los más pequeños de su rebaño los arrastrarán, y destruirán sus moradas con ellos. 
50:46 Al grito de la toma de Babilonia la tierra tembló, y el clamor se oyó entre las naciones. 
\section*{Capítulo 51 }
Juicios de Jehová contra Babilonia 
 
51:1 Así ha dicho Jehová: He aquí que yo levanto un viento destruidor contra Babilonia, y contra sus moradores que se levantan contra mí. 
51:2 Y enviaré a Babilonia aventadores que la avienten, y vaciarán su tierra; porque se pondrán contra ella de todas partes en el día del mal. 
51:3 Diré al flechero que entesa su arco, y al que se enorgullece de su coraza: No perdonéis a sus jóvenes, destruid todo su ejército. 
51:4 Y caerán muertos en la tierra de los caldeos, y alanceados en sus calles. 
51:5 Porque Israel y Judá no han enviudado de su Dios, Jehová de los ejércitos, aunque su tierra fue llena de pecado contra el Santo de Israel. 
51:6 Huid de en medio de Babilonia, y librad cada uno su vida, para que no perezcáis a causa de su maldad; porque el tiempo es de venganza de Jehová; le dará su pago. 
51:7 Copa de oro fue Babilonia en la mano de Jehová, que embriagó a toda la tierra; de su vino bebieron los pueblos; se aturdieron, por tanto, las naciones. 
51:8 En un momento cayó Babilonia, y se despedazó; gemid sobre ella; tomad bálsamo para su dolor, quizá sane. 
51:9 Curamos a Babilonia, y no ha sanado; dejadla, y vámonos cada uno a su tierra; porque ha llegado hasta el cielo su juicio, y se ha alzado hasta las nubes. 
51:10 Jehová sacó a luz nuestras justicias; venid, y contemos en Sion la obra de Jehová nuestro Dios. 
51:11 Limpiad las saetas, embrazad los escudos; ha despertado Jehová el espíritu de los reyes de Media; porque contra Babilonia es su pensamiento para destruirla; porque venganza es de Jehová, y venganza de su templo. 
51:12 Levantad bandera sobre los muros de Babilonia, reforzad la guardia, poned centinelas, disponed celadas; porque deliberó Jehová, y aun pondrá en efecto lo que ha dicho contra los moradores de Babilonia. 
51:13 Tú, la que moras entre muchas aguas,  rica en tesoros, ha venido tu fin, la medida de tu codicia. 
51:14 Jehová de los ejércitos juró por sí mismo, diciendo: Yo te llenaré de hombres como de langostas, y levantarán contra ti gritería. 
51:15 El es el que hizo la tierra con su poder, el que afirmó el mundo con su sabiduría, y extendió los cielos con su inteligencia. 
51:16 A su voz se producen tumultos de aguas en los cielos, y hace subir las nubes de lo último de la tierra; él hace relámpagos con la lluvia, y saca el viento de sus depósitos. 
51:17 Todo hombre se ha infatuado, y no tiene ciencia; se avergüenza todo artífice de su escultura, porque mentira es su ídolo, no tiene espíritu. 
51:18 Vanidad son, obra digna de burla; en el tiempo del castigo perecerán. 
51:19 No es como ellos la porción de Jacob; porque él es el Formador de todo, e Israel es el cetro de su herencia; Jehová de los ejércitos es su nombre. 
51:20 Martillo me sois, y armas de guerra; y por medio de ti quebrantaré naciones, y por medio de ti destruiré reinos. 
51:21 Por tu medio quebrantaré caballos y a sus jinetes, y por medio de ti quebrantaré carros y a los que en ellos suben. 
51:22 Asimismo por tu medio quebrantaré hombres y mujeres, y por medio de ti quebrantaré viejos y jóvenes, y por tu medio quebrantaré jóvenes y vírgenes. 
51:23 También quebrantaré por medio de ti al pastor y a su rebaño; quebrantaré por tu medio a labradores y a sus yuntas; a jefes y a príncipes quebrantaré por medio de ti. 
51:24 Y pagaré a Babilonia y a todos los moradores de Caldea, todo el mal que ellos hicieron en Sion delante de vuestros ojos, dice Jehová. 
51:25 He aquí yo estoy contra ti, oh monte destruidor, dice Jehová, que destruiste toda la tierra; y extenderé mi mano contra ti, y te haré rodar de las peñas, y te reduciré a monte quemado. 
51:26 Y nadie tomará de ti piedra para esquina, ni piedra para cimiento; porque perpetuo asolamiento serás, ha dicho Jehová. 
51:27 Alzad bandera en la tierra, tocad trompeta en las naciones, preparad pueblos contra ella; juntad contra ella los reinos de Ararat, de Mini y de Askenaz; señalad contra ella capitán, haced subir caballos como langostas erizadas. 
51:28 Preparad contra ella naciones; los reyes de Media, sus capitanes y todos sus príncipes, y todo territorio de su dominio. 
51:29 Temblará la tierra, y se afligirá; porque es confirmado contra Babilonia todo el pensamiento de Jehová, para poner la tierra de Babilonia en soledad, para que no haya morador en ella. 
51:30 Los valientes de Babilonia dejaron de pelear, se encerraron en sus fortalezas; les faltaron las fuerzas, se volvieron como mujeres; incendiadas están sus casas, rotos sus cerrojos. 
51:31 Correo se encontrará con correo, mensajero se encontrará con mensajero, para anunciar al rey de Babilonia que su ciudad es tomada por todas partes. 
51:32 Los vados fueron tomados, y los baluartes quemados a fuego, y se consternaron los hombres de guerra. 
51:33 Porque así ha dicho Jehová de los ejércitos, Dios de Israel: La hija de Babilonia es como una era cuando está de trillar; de aquí a poco le vendrá el tiempo de la siega. 
51:34 Me devoró, me desmenuzó Nabucodonosor rey de Babilonia, y me dejó como vaso vacío; me tragó como dragón, llenó su vientre de mis delicadezas, y me echó fuera. 
51:35 Sobre Babilonia caiga la violencia hecha a mí y a mi carne, dirá la moradora de Sion; y mi sangre caiga sobre los moradores de Caldea, dirá Jerusalén. 
51:36 Por tanto, así ha dicho Jehová: He aquí que yo juzgo tu causa y haré tu venganza; y secaré su mar, y haré que su corriente quede seca. 
51:37 Y será Babilonia montones de ruinas, morada de chacales, espanto y burla, sin morador. 
51:38 Todos a una rugirán como leones; como cachorros de leones gruñirán. 
51:39 En medio de su calor les pondré banquetes, y haré que se embriaguen, para que se alegren, y duerman eterno sueño y no despierten, dice Jehová. 
51:40 Los haré traer como corderos al matadero, como carneros y machos cabríos. 
51:41 ¡Cómo fue apresada Babilonia, y fue tomada la que era alabada por toda la tierra! ¡Cómo vino a ser Babilonia objeto de espanto entre las naciones! 
51:42 Subió el mar sobre Babilonia; de la multitud de sus olas fue cubierta. 
51:43 Sus ciudades fueron asoladas, la tierra seca y desierta, tierra en que no morará nadie, ni pasará por ella hijo de hombre. 
51:44 Y juzgaré a Bel en Babilonia, y sacaré de su boca lo que se ha tragado; y no vendrán más naciones a él, y el muro de Babilonia caerá. 
51:45 Salid de en medio de ella, pueblo mío, y salvad cada uno su vida del ardor de la ira de Jehová. 
51:46 Y no desmaye vuestro corazón, ni temáis a causa del rumor que se oirá por la tierra; en un año vendrá el rumor, y después en otro año rumor, y habrá violencia en la tierra, dominador contra dominador. 
51:47 Por tanto, he aquí vienen días en que yo destruiré los ídolos de Babilonia, y toda su tierra será avergonzada, y todos sus muertos caerán en medio de ella. 
51:48 Los cielos y la tierra y todo lo que está en ellos cantarán de gozo sobre Babilonia;  porque del norte vendrán contra ella destruidores, dice Jehová. 
51:49 Por los muertos de Israel caerá Babilonia, como por Babilonia cayeron los muertos de toda la tierra.  
51:50 Los que escapasteis de la espada, andad, no os detengáis; acordaos por muchos días de Jehová, y acordaos de Jerusalén. 
51:51 Estamos avergonzados, porque oímos la afrenta; la confusión cubrió nuestros rostros, porque vinieron extranjeros contra los santuarios de la casa de Jehová. 
51:52 Por tanto, vienen días, dice Jehová, en que yo destruiré sus ídolos, y en toda su tierra gemirán los heridos. 
51:53 Aunque suba Babilonia hasta el cielo, y se fortifique en las alturas, de mí vendrán a ella destruidores, dice Jehová. 
51:54 ¡Oyese el clamor de Babilonia, y el gran quebrantamiento de la tierra de los caldeos! 
51:55 Porque Jehová destruirá a Babilonia, y quitará de ella la mucha jactancia; y bramarán sus olas, y como sonido de muchas aguas será la voz de ellos. 
51:56 Porque vino destruidor contra ella, contra Babilonia, y sus valientes fueron apresados; el arco de ellos fue quebrado; porque Jehová, Dios de retribuciones, dará la paga. 
51:57 Y embriagaré a sus príncipes y a sus sabios, a sus capitanes, a sus nobles y a sus fuertes; y dormirán sueño eterno y no despertarán, dice el Rey, cuyo nombre es Jehová de los ejércitos. 
51:58 Así ha dicho Jehová de los ejércitos: El muro ancho de Babilonia será derribado enteramente, y sus altas puertas serán quemadas a fuego; en vano trabajaron los pueblos, y las naciones se cansaron sólo para el fuego. 
51:59 Palabra que envió el profeta Jeremías a Seraías hijo de Nerías, hijo de Maasías, cuando iba con Sedequías rey de Judá a Babilonia, en el cuarto año de su reinado. Y era Seraías el principal camarero. 
51:60 Escribió, pues, Jeremías en un libro todo el mal que había de venir sobre Babilonia, todas las palabras que están escritas contra Babilonia. 
51:61 Y dijo Jeremías a Seraías: Cuando llegues a Babilonia, y veas y leas todas estas cosas, 
51:62 dirás: Oh Jehová, tú has dicho contra este lugar que lo habías de destruir, hasta no quedar en él morador, ni hombre ni animal, sino que para siempre ha de ser asolado. 
51:63 Y cuando acabes de leer este libro, le atarás una piedra, y lo echarás en medio del Eufrates, 
51:64 y dirás: Así se hundirá Babilonia, y no se levantará del mal que yo traigo sobre ella; y serán rendidos. Hasta aquí son las palabras de Jeremías. 
\section*{Capítulo 52 }
Reinado de Sedequías 
 
52:1 Era Sedequías de edad de veintiún años cuando comenzó a reinar, y reinó once años en Jerusalén. Su madre se llamaba Hamutal, hija de Jeremías de Libna. 
52:2 E hizo lo malo ante los ojos de Jehová, conforme a todo lo que hizo Joacim. 
52:3 Y a causa de la ira de Jehová contra Jerusalén y Judá, llegó a echarlos de su presencia. Y se rebeló Sedequías contra el rey de Babilonia. 
Caída de Jerusalén 

52:4 Aconteció, por tanto, a los nueve años de su reinado, en el mes décimo, a los diez días del mes, que vino Nabucodonosor rey de Babilonia, él y todo su ejército, contra Jerusalén, y acamparon contra ella, y de todas partes edificaron contra ella baluartes. 
52:5 Y estuvo sitiada la ciudad hasta el undécimo año del rey Sedequías. 
52:6 En el mes cuarto, a los nueve días del mes, prevaleció el hambre en la ciudad, hasta no haber pan para el pueblo. 
52:7 Y fue abierta una brecha en el muro de la ciudad, y todos los hombres de guerra huyeron, y salieron de la ciudad de noche por el camino de la puerta entre los dos muros que había cerca del jardín del rey, y se fueron por el camino del Arabá, estando aún los caldeos junto a la ciudad alrededor. 
52:8 Y el ejército de los caldeos siguió al rey, y alcanzaron a Sedequías en los llanos de Jericó; y lo abandonó todo su ejército. 
52:9 Entonces prendieron al rey, y le hicieron venir al rey de Babilonia, a Ribla en tierra de Hamat, donde pronunció sentencia contra él. 
52:10 Y degolló el rey de Babilonia a los hijos de Sedequías delante de sus ojos, y también degolló en Ribla a todos los príncipes de Judá. 
52:11 No obstante, el rey de Babilonia sólo le sacó los ojos a Sedequías, y le ató con grillos, y lo hizo llevar a Babilonia; y lo puso en la cárcel hasta el día en que murió. 
Cautividad de Judá 

52:12 Y en el mes quinto, a los diez días del mes, que era el año diecinueve del reinado de Nabucodonosor rey de Babilonia, vino a Jerusalén Nabuzaradán capitán de la guardia, que solía estar delante del rey de Babilonia. 
52:13 Y quemó la casa de Jehová, y la casa del rey, y todas las casas de Jerusalén; y destruyó con fuego todo edificio grande. 
52:14 Y todo el ejército de los caldeos, que venía con el capitán de la guardia, destruyó todos los muros en derredor de Jerusalén. 
52:15 E hizo transportar Nabuzaradán capitán de la guardia a los pobres del pueblo, y a toda la otra gente del pueblo que había quedado en la ciudad, a los desertores que se habían pasado al rey de Babilonia, y a todo el resto de la multitud del pueblo. 
52:16 Mas de los pobres del país dejó Nabuzaradán capitán de la guardia para viñadores y labradores. 
52:17 Y los caldeos quebraron las columnas de bronce que estaban en la casa de Jehová, y las basas, y el mar de bronce que estaba en la casa de Jehová, y llevaron todo el bronce a Babilonia. 
52:18 Se llevaron también los calderos, las palas, las despabiladeras, los tazones, las cucharas, y todos los utensilios de bronce con que se ministraba, 
52:19 y los incensarios, tazones, copas, ollas, candeleros, escudillas y tazas; lo de oro por oro, y lo de plata por plata, se llevó el capitán de la guardia. 
52:20 Las dos columnas, un mar, y los doce bueyes de bronce que estaban debajo de las basas, que había hecho el rey Salomón en la casa de Jehová; el peso del bronce de todo esto era incalculable. 
52:21 En cuanto a las columnas, la altura de cada columna era de dieciocho codos, y un cordón de doce codos la rodeaba; y su espesor era de cuatro dedos, y eran huecas. 
52:22 Y el capitel de bronce que había sobre ella era de una altura de cinco codos, con una red y granadas alrededor del capitel, todo de bronce; y lo mismo era lo de la segunda columna con sus granadas. 
52:23 Había noventa y seis granadas en cada hilera; todas ellas eran ciento sobre la red alrededor.  
52:24 Tomó también el capitán de la guardia a Seraías el principal sacerdote, a Sofonías el segundo sacerdote, y tres guardas del atrio. 
52:25 Y de la ciudad tomó a un oficial que era capitán de los hombres de guerra, a siete hombres de los consejeros íntimos del rey, que estaban en la ciudad, y al principal secretario de la milicia, que pasaba revista al pueblo de la tierra para la guerra, y sesenta hombres del pueblo que se hallaron dentro de la ciudad. 
52:26 Los tomó, pues, Nabuzaradán capitán de la guardia, y los llevó al rey de Babilonia en Ribla. 
52:27 Y el rey de Babilonia los hirió, y los mató en Ribla en tierra de Hamat. Así Judá fue transportada de su tierra. 
52:28 Este es el pueblo que Nabucodonosor llevó cautivo: En el año séptimo, a tres mil veintitrés hombres de Judá. 
52:29 En el año dieciocho de Nabucodonosor él llevó cautivas de Jerusalén a ochocientas treinta y dos personas. 
52:30 El año veintitrés de Nabucodonosor, Nabuzaradán capitán de la guardia llevó cautivas a setecientas cuarenta y cinco personas de los hombres de Judá; todas las personas en total fueron cuatro mil seiscientas. 
Joaquín es libertado y recibe honores en Babilonia 

52:31 Y sucedió que en el año treinta y siete del cautiverio de Joaquín rey de Judá, en el mes duodécimo, a los veinticinco días del mes, Evil-merodac rey de Babilonia, en el año primero de su reinado, alzó la cabeza de Joaquín rey de Judá y lo sacó de la cárcel. 
52:32 Y habló con él amigablemente, e hizo poner su trono sobre los tronos de los reyes que estaban con él en Babilonia. 
52:33 Le hizo mudar también los vestidos de prisionero, y comía pan en la mesa del rey siempre todos los días de su vida. 
52:34 Y continuamente se le daba una ración de parte del rey de Babilonia, cada día durante todos los días de su vida, hasta el día de su muerte.